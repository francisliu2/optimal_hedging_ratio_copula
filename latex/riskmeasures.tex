\subsection{Risk Measures}\label{subsec:spectral-risk-measures}
We consider a variety of risk measures: variance, Value-at-Risk (VaR), Expected Shortfall (ES), and Exponential Risk Measure (ERM).
%They are used in many literature about hedging, e.g. ;
%The risk measures are also used by regulatory bodies,
%for example Basel III ....
A summary of risk measures being used in portfolio selection problem can be found in \citet{hardle2008applied}.\medskip
\medskip


Let $Z$ be a random variable of distribution $F_Z$.
\begin{enumerate}
	\item Variance is $\text{Var}(F_Z)$
	\item VaR of a given confidence level $\alpha$ is $\text{VaR}(F_Z) = -F_{Z}^{(-1)}(1-\alpha)$
	\item ES with parameter $\alpha$ is $\text{ES}(F_Z) = -\frac{1}{1-\alpha}\int_0^{1-\alpha}F_Z^{(-1)}(p)dp$
	\item ERM with Arrow-Pratt coeficient of absolute risk aversion $k$ is $\text{ERM}_k(F_Z) = \int_0^{1-\alpha}\phi(p) F_Z^{(-1)}(p)dp$ where $\phi$ is a weight function described in (\ref{eq:phi}) below.
	\end{enumerate}\medskip

VaR, ES, and ERM fall into the class of Spectral Risk Measure (SRM).
SRM has the from \citep{Acerbi2002}%, adam2008spectral,dowd2008spectral}
\begin{align}
	\rho_\phi(r^h) = - \int_0^1 \phi(p) F_{Z}^{(-1)}(p)d p,
	\end{align}

where $p$ is the loss quantile and $\phi(p)$ is a user-defined weighting function defined over $[0,1]$. \medskip
We consider only admissible risk spectra $\phi(p)$ %(named by \citet{Acerbi2002})
\begin{enumerate}[label=\roman*]
	\item $\phi$ is positive
	\item $\phi$ is decreasing
	\item integrates to one.
	\end{enumerate}\medskip

The VaR's $\phi(p)$ gives all its weight on the $1-\alpha$ quantile of $Z$ and zero elsewhere,
i.e. the weighting function is a Dirac delta function, hence violates the ii property of admissible risk spectra.
The ES' $\phi(p)$ gives all tail quantiles the same weight of $\frac{1}{1-\alpha}$ and non-tail quantiles zero weight.
ERM assumes investor's risk preference is in a form of exponential utility function $U(x)=1-e^{kx}$,
its corresponding risk spectrum is defined as

\begin{align}
	\phi(p) =\frac{k e^{-k(1-p)}}{1-e^{-k}} , \label{eq:phi}
	\end{align}
where $k$ is the Arrow-Pratt coefficient of absolute risk aversion. \medskip

The parameter $k$ has an economic interpretation of being the ratio between the second derivative and first derivative
of investor's utility function on an risky asset

\begin{align}
	k = -\frac{U''(x)}{U'(x)},
	\end{align}
for $x$ in all possible outcomes.
In case of the exponential utility, $k$ is the the constant absolute risk aversion (CARA).



%Spectral risk measures can also be written as
%\begin{align}
%	\rho_\phi(r^h) = - \int_\mathbb{R} x f_{r^h}(x) \phi\{F_{r^h}(x)\} d x.
%	\end{align}