\documentclass[11pt,a4paper,english]{article}
\usepackage[titletoc, title]{appendix}
\usepackage{amsmath}
\usepackage{amssymb}
\usepackage{bm}
\usepackage{array}
\usepackage{babel}
\usepackage{bbding}
\usepackage{color}
\usepackage[normal]{caption}
\usepackage{subcaption}
\usepackage{epsfig}
\usepackage{graphicx}
\usepackage{pdflscape}
\usepackage{lipsum}
%\usepackage{multirow}
\usepackage{psfrag}
\usepackage{proofapnd}
\usepackage[round]{natbib}
%\usepackage{bbm}
%\usepackage[T1]{fontenc}
%\usepackage[normal]{caption2} % for caption
\usepackage{rotating}
\usepackage[margin=2cm]{geometry} % for the same margin
\usepackage{latexsym}
%\usepackage{subfig}
\usepackage{float}
\usepackage{setspace}
\usepackage{slashbox}
\usepackage{enumitem}
\usepackage{booktabs}
\usepackage{tabularx,ragged2e}
\newcolumntype{C}{>{\Centering\arraybackslash}X}
\newcolumntype{s}{>{\hsize=.2\hsize \Centering\arraybackslash}X}

\usepackage{longtable,booktabs}
\usepackage{authblk}
\usepackage{hyperref}
\usepackage{indentfirst} % Macht eine Einrückung nach der Section
\bibliographystyle{ecta}

\definecolor{markergreen}{rgb}{0.6, 1.0, 0}
\definecolor{darkgreen}{rgb}{0, .5, 0}
\definecolor{darkred}{rgb}{.7,0,0}
\definecolor{markergreen}{rgb}{0.6, 1.0, 0}
\definecolor{darkgreen}{rgb}{0, .5, 0}
\definecolor{darkorange}{rgb}{1,0.3,0}
\definecolor{darkred}{RGB}{.7,0,0}
\definecolor{darkblue}{RGB}{0,29,245}
\definecolor{orange}{RGB}{239, 133, 54}
\definecolor{lightblue}{RGB}{59, 188, 175}

\providecommand{\marker}[1]{\fcolorbox{markergreen}{markergreen}{{#1}}}
\providecommand{\mj}[1]{\textcolor{darkred}{#1}}
\providecommand{\francis}[1]{\textcolor{darkgreen}{#1}}
\providecommand{\natp}[1]{\textcolor{darkorange}{#1}}

\setlist[itemize]{leftmargin=*}
\setlist[description]{leftmargin=*}

%\setlength{\topmargin}{0.0 in} \setlength{\textwidth}{6in}
%\setlength{\oddsidemargin}{0.5in}
%\setlength{\evensidemargin}{-0.01in} \setlength{\textheight}{9in}
\captionsetup{font={onehalfspacing,small}, labelfont=bf}

\title{\LARGE \bf Hedging Cryptos with Bitcoin Futures}
%\title{\natp{\large\bf Hedging Cryptocurrencies with Bitcoin Futures}}
\author{
	\begin{tabular}[t]{ccc}
		\and
        Francis Liu\thanks{
			Department of Business and Economics, Berlin School of Economics and Law, Badensche Str. 52, 10825 Berlin, Germany.
            Blockchain Research Center, Humboldt-Universität zu Berlin, Germany.
            International Research Training Group
1792, Humboldt-Universität zu Berlin, Germany.
     E-mail: \texttt{Francis.Liu@hwr-berlin.de}.}
        \and
		Meng-Jou Lu
        \thanks{
             Department of Finance, Asia University, 500, Lioufeng Rd., Wufeng, Taichung 41354, Taiwan
             Department of Finance, Asia University, 500, Lioufeng Rd., Wufeng, Taichung 41354, Taiwan
     E-mail: \texttt{mangrou@gmail.com}.}

		 \and
        Natalie Packham\thanks{
			Department of Business and Economics, Berlin School of Economics and Law, Badensche Str. 52, 10825 Berlin, Germany.
            International Research Training Group 1792, Humboldt-Universität zu Berlin, Germany.
     E-mail: \texttt{packham@hwr-berlin.de}.}
		 \and
         Wolfgang Karl H\"ardle\thanks{
			Blockchain Research Center, Humboldt-Universit\"at zu Berlin, Germany. Wang Yanan Institute for Studies in Economics, Xiamen University, China. Sim Kee Boon Institute for Financial Economics, Singapore Management University, Singapore. Faculty of Mathematics and Physics, Charles University, Czech Republic. National Yang Ming Chiao Tung University, Taiwan.
     E-mail: \texttt{haerdle@wiwi.hu-berlin.de}.}
        \thanks{ Financial support of the European Union's Horizon 2020 research and innovation program ``FIN-TECH: A Financial supervision and Technology compliance training programme" under the grant agreement No 825215 (Topic: ICT-35-2018, Type of action: CSA), the European Cooperation in Science \& Technology COST Action grant CA19130 - Fintech and Artificial Intelligence in Finance - Towards a transparent financial industry, the Deutsche Forschungsgemeinschaft's IRTG 1792 grant, the Yushan Scholar Program of Taiwan, the Czech Science Foundation's grant no. 19-28231X / CAS: XDA 23020303, as well as support by Ansar Aynetdinov (\texttt{ansar.aynetdinov@hu-berlin.de}) are greatly acknowledged.
     }
	\end{tabular}
}
\date{This version: \today}
%%%%%%%%%%%%%%%%%%%%%%%%%%%%%%%%%%%%%%%%%%%%%%%%%%%%%%%%%%%%%%%%%%%%%%%%%%%%%%%%%%%%%%%%%%%%%%%%
\renewcommand{\baselinestretch}{1.2}
%\newcommand{\indicator}{$1{\hskip -2.5 pt}\hbox{I}$}
\newcommand{\indicator}{I}
%%
%% $Id: Definitions.tex,v 1.6 2008/07/26 14:55:50 natalie Exp $
%% $Source: /Users/natalie/cvs/tex/dynamics/Definitions.tex,v $
%% $Date: 2008/07/26 14:55:50 $
%% $Revision: 1.6 $
%%

\usepackage{mathrsfs}

%% GENERAL DEFINITIONS
\unitlength1cm

%% COMMAND DEFINITIONS
\newcommand{\E}{{\mathbb{E}}}
%%\renewcommand{\E}{{\mathds E}}
%%\renewcommand{\E}{{\varmathbb{E}}}
%%\renewcommand{\E}{{\mathrm{I\!E}}}
\providecommand{\R}{{\mathbb{R}}}
\newcommand{\T}{{\mathbb{T}}}
\newcommand{\Fb}{{\mathbb{F}}}
\newcommand{\Eqn}{{\mathbb{E}}_{{\bf Q}_N}}
\newcommand{\Eq}{{\mathbb{E}}_{{\bf Q}}}
\newcommand{\Eqm}{{\mathbb{E}}_{{\bf Q}_M}}
\newcommand{\EqT}{{\mathbb{E}}_{{\bf Q}_T}}
\newcommand{\EqTz}{{\mathbb{E}}_{{\bf Q}_{T_2}}}
\newcommand{\EqTe}{{\mathbb{E}}_{{\bf Q}_{T_1}}}
\newcommand{\EqSe}{{\mathbb{E}}_{{\bf Q}_{S^1}}}
\newcommand{\EqSz}{{\mathbb{E}}_{{\bf Q}_{S^2}}}
\newcommand{\p}{{\bf P}}
%%\renewcommand{\p}{{\mathds{P}}}
%%\renewcommand{\p}{{\varmathbb{P}}}
%%\renewcommand{\p}{{\mathrm{I\!P}}}
\newcommand{\pas}{\text{{\bf P}--a.s.}}
\newcommand{\paa}{\text{{\bf P}--a.a.}}
\newcommand{\qas}{\text{{\bf Q}--a.s.}}
\newcommand{\e}{{\bf e}}
\newcommand{\q}{{\bf Q}}
\newcommand{\qn}{{\bf Q}_N}
\newcommand{\qm}{{\bf Q}_M}
\newcommand{\qT}{{\bf Q}_T}
\newcommand{\qTz}{{\bf Q}_{T_2}}
\newcommand{\qTe}{{\bf Q}_{T_1}}
\newcommand{\qS}{{\bf Q}_S}
\newcommand{\qSe}{{\bf Q}_{S^1}}
\newcommand{\qSz}{{\bf Q}_{S^2}}
\newcommand{\F}{{\cal F}}
\newcommand{\G}{{\cal G}}
\newcommand{\A}{{\cal A}}
\newcommand{\Hc}{{\cal H}}
\newcommand{\dP}{{\rm d}{\bf P}}
\newcommand{\du}{{\rm d}u}
%%\newcommand{\dt}{{\rm d}t}
\newcommand{\dd}{{\rm d}}
\newcommand{\df}{{\rm \bf DF}}
\providecommand{\N}{{\mathbb N}}
\providecommand{\Ncdf}{{\rm N}}
%\renewcommand{\Ncdf}{{\Phi}}
\newcommand{\n}{{\rm n}}
\newcommand{\emb}{\bf \em}
\newcommand{\1}{\textbf{1}}
\newcommand{\qs}{{\q_{\rm Swap}}}
\newcommand{\fx}{{\rm fx}}
\newcommand{\V}{{\rm Var}}
%\newcommand{\C}{{\bf C}}
\newcommand{\Om}{{\Omega}}
\providecommand{\limn}{\ensuremath{\lim_{n\rightarrow\infty}}}
\providecommand{\qv}[2]{\ensuremath{\langle #1,#1\rangle_{#2}}}

%% ENVIRONMENT DEFINITIONS
%\newtheorem{prop}{Proposition}[section]
%\newtheorem{theo}{Theorem}[section]
%\newtheorem{lem}{Lemma}[section]
%\newtheorem{ass}{Assumption}[section]
%\newtheorem{cor}{Corollary}[section]
%\newtheorem{aufg}{Exercise}[section]
%\newtheorem{defi}{Definition}[section]

\ifx\prop\undefined
\newtheorem{prop}{Proposition}[section]
\fi
\newtheorem{theo}[prop]{Theorem}
\newtheorem{lem}[prop]{Lemma}
\newtheorem{cor}[prop]{Corollary}
\newtheorem{defi}[prop]{Definition}

%% enumeration in lists
\providecommand{\labelenumi}{{\rm (\roman{enumi})}}
   %\setlength{\topsep}{0cm}
    \setlength{\labelsep}{0.3cm}
    %\setlength{\itemindent}{0cm}
   \setlength{\leftmargin}{10cm}
    \setlength{\labelwidth}{5cm}

\providecommand{\cadlag}{c\`adl\`ag }
\providecommand{\cadlagns}{c\`adl\`ag}
\providecommand{\caglad}{c\`agl\`ad }
\providecommand{\cad}{c\`ad}
\providecommand{\cag}{c\`ag}
\providecommand{\levy}{L\'evy\ }
\providecommand{\levyns}{L\'evy}
\providecommand{\levyito}{L\'evy-It\^o\ } 
\providecommand{\levykhinchin}{L\'evy-Khinchin\ }
\providecommand{\D}{\ensuremath{D(\R_+,\R)}}
\providecommand{\Dsig}{\ensuremath{D(\R_+, \R_+\setminus\{0\}})}
\providecommand{\Dd}{\ensuremath{D(\R_+,\R^d)}}
\providecommand{\C}{\ensuremath{C(\R_+,\R)}}
\providecommand{\Cd}{\ensuremath{C(\R_+,\R^d)}}
\providecommand{\rpos}{\ensuremath{{[0,\infty)}}}

\def\Z{{\mathbb Z}}
%\def\N{{\mathbb N}}
%\def\R{{\mathbb R}}
%\def\C{{\mathbb C}}
%\def\H{{\mathbb H}}
\def\P{{\mathbb P}}
\def\Q{{\mathbb Q}}
%\def\E{{\mathbb E}}
\def\I{{\mathbb I}}
%\def\T{{\mathbb T}}
%\def\F{{\mathbb F}}
\def\M{{\mathbb M}}
%\def\Hc{{\mathcal H}}
\def\Mc{{\mathcal M}}
\def\filtration#1{{\ensuremath\mathcal{#1}}}
%\def\filt{{\mathcal F}}
\def\tp{\tilde{\p}}
\providecommand{\vec}[1]{\ensuremath{\bm #1}}
\providecommand{\vecb}[1]{\ensuremath{\bm #1}}
\providecommand{\abs}[1]{\ensuremath{\lvert#1\rvert}}
\providecommand{\norm}[1]{\ensuremath{\lVert#1\rVert}}
\providecommand{\var}{\ensuremath{\text{Var}}}
\providecommand{\cov}{\ensuremath{\text{Cov}}}
\providecommand{\borel}[0]{\ensuremath{\mathcal{B}}}
\providecommand{\intinf}[0]{\ensuremath{\int_{-\infty}^\infty}}
\providecommand{\intpos}[0]{\ensuremath{\int_0^\infty}}
\providecommand{\intneg}[0]{\ensuremath{\int_{-\infty}^0}}
\providecommand{\todo}[1]{\footnote{#1}}
\providecommand{\dynkin}[0]{\ensuremath{\mathcal D}}
\providecommand{\ce}[2]{\ensuremath{\E(#1|\filtration{#2})}}
\providecommand{\inv}[1]{\ensuremath{#1^{(-1)}}}
\providecommand{\os}[2]{\ensuremath{#1^{(#2)}}}
\providecommand{\pos}[2]{\ensuremath{h_{#1}(#2)}}
%\providecommand{\poslong}[2]{\ensuremath{h(#1, #2)}}
\providecommand{\poslong}[3]{\ensuremath{h_{#1, #2}(#3)}}

%% Class of finite variation processes
\providecommand{\classfv}{\ensuremath{\mathscr V}}
\providecommand{\classv}{\ensuremath{\mathscr V}}
%% Stochastic integral operator
\providecommand{\stint}{\ensuremath{\cdotp}}
\providecommand{\classh}{\ensuremath{\mathscr H^2}}
\providecommand{\classhloc}{\ensuremath{\mathscr H^2_{\rm loc}}}
\providecommand{\classm}{\ensuremath{\mathscr M}}
\providecommand{\classmloc}{\ensuremath{\mathscr M_{\rm loc}}}
\providecommand{\classl}{\ensuremath{L^2}}
\providecommand{\classlloc}{\ensuremath{L^2_{\rm loc}}}
\providecommand{\classa}{\ensuremath{\mathscr A}}
\providecommand{\classaloc}{\ensuremath{\mathscr A_{\rm loc}}}
\providecommand{\classalocpos}{\ensuremath{\mathscr A_{\rm loc}^+}}
\providecommand{\classp}{\ensuremath{\mathscr P}}
\providecommand{\classo}{\ensuremath{\mathscr O}}
\providecommand{\classs}{\ensuremath{\mathscr S}}
\providecommand{\classsp}{\ensuremath{\mathscr S_p}}
\providecommand{\nullset}{\ensuremath{\mathscr N}}

\providecommand{\ito}{It\^o }
\providecommand{\itos}{It\^o's\, }

\providecommand{\variation}[2]{\ensuremath{\rm V_{#1}(#2)}}
\renewcommand{\H}{\ensuremath{\mathcal H}}
%% CPO distribution
\providecommand{\cpo}{\ensuremath{{\rm CPO}}}
\providecommand{\Fsigma}{\ensuremath{\mathcal \F_\infty^\sigma}}
\providecommand{\sigd}{\ensuremath{\mathscr D}}

%% Credit spreads
\providecommand{\s}{{\bf s}}
\providecommand{\classu}{\ensuremath{\mathscr U}}

\providecommand{\sX}{\ensuremath{\mathcal X}}
\providecommand{\sY}{\ensuremath{\mathcal Y}}
\providecommand{\dx}{\ensuremath{\frac{\partial}{\partial x}}} %%
\providecommand{\dt}{\ensuremath{\frac{\partial}{\partial t}}} %%
\providecommand{\dy}{\ensuremath{\frac{\partial}{\partial y}}} %%
%\newcommand{\argmax}{\operatornamewithlimits{argmax}}
%\newcommand{\argmin}{\operatornamewithlimits{argmin}}


\begin{document}

\newtheorem{lemma}{Lemma}
\newtheorem {proposition}[lemma]{Proposition}
\newtheorem {corollary}{Corollary}
\newtheorem {theorem}{Theorem}
\newtheorem{claim}[lemma]{Claim}
\newtheorem{comment}[lemma]{Comment}
\newtheorem{example}[lemma]{Example}
\newtheorem{fact}[lemma]{Fact}
\newtheorem{defn}[lemma]{Definition}
\newtheorem{exercise}{Exercise}[section]

\newtheorem{programming}[exercise]{Programming assignment}
\newenvironment{proof}{{\flushleft\textbf{\textsl{Proof.\ \ }}}}{\hfill{\hfill\rule{2mm}{2mm}}}
\pagenumbering{arabic}

\maketitle

%%%%%%%%%%%%%%%%%%%%%%%%%%%%%%%%%%%%%%%%%%%%%%%%%%%%%%%%%%%%%%%%%%%%%%%%%%%%%%%%%%%%%%%%%%%%%%%
\begin{abstract}
\footnotesize{
The introduction of derivatives on Bitcoin, in particular the launch of futures contracts on CME in December 2017 and introduction of cryptocurrency index (CRIX) \citep{trimborn2018crix},
enables investors to hedge risk exposures of Bitcoin by futures or a cryptocurrency index.
We investigate methods of finding the optimal hedge ratio $h^*$ under different dependence structures modeled by copulae and optimality definition described by a range of risk measures.
Because of volatility swings and jumps in Bitcoin prices, the traditional variance-based approach to obtain the hedge ratios is infeasible.
The approach is therefore generalised  to various risk measures, such as Value-at-Risk, Expected Shortfall and Spectral Risk Measures,
and to different copulae for capturing the dependency between spot and future returns, such as the Gaussian, Student-$t$,
NIG and Archimedean copulae. Various measures of hedge effectiveness in out-of-sample tests give insights in the practice of hedging Bitcoin and the CRIX,
a cryptocurrency index.\\

\noindent {\bf JEL classification:}  \\
\noindent {\bf Keywords:} Portfolio Selection, Spectral Risk Measurement,  Coherent Risk}\pagestyle{empty}\\
\end{abstract}

%\natp{{\bf TODO}:
%  \begin{itemize}
%  \item Please generate all graphics as pdf and possibly eps. pdf is a
%    vector graphic format, so it scales well. eps may be
%    required during the publishing process.
%  \item Plackett copula: This is a bivariate copula only, which is
%    probably one of the reasons it is not commonly found in finance
%    applications. It does not have tail dependence, which is one of
%    the things we typically look for in finance. We need a compelling
%    reason why it is of interest, otherwise I suggest to remove it.
%  \item We need to discuss if we use copulas or copulae. I have a
%    preference for copulas, which is the terminology used by Nelsen,
%    McNeil et al., Joe.
%  \item The introduction needs to be revised, see comments
%    below. Think about what needs to go into the introduction and what
%    does not, and then stick to this in a concise way. Also, re-read
%    every sentence 2-3 times and think carefully about what it says
%    and what it is supposed to say.
%  \end{itemize}
%}

%\tableofcontents

\clearpage
%%%%%%%%%%%%%%%%%%%%%%%%%%%%%%%%%%%%%%%%%%%%%%%%%%%%%%%%%%%%%%%%%%%%%%%%%%%%%%%%%%%%%%%%%%%%%%%%%%%%%%%%%%%%%%%%%%%%%%%%%%%%%%%%%%%%%%%%%%%%%%%%%%%%%%
%%%%%%%%%%%%%%%%%%%%%%%%%%%%%%%%%%%%%%%%%%%%%%%%%%%%%%%%%%%%%%%%%%%%%%%%%%%%%%%%%%%%%%%%%%%%%%%%%%%%%%%%%%%%%%%%%%%%%%%%%%%%%%%%%%%%%%%%%%%%%%%%%%%%%%
\section{Introduction}\label{sec:introduction}

Cryptocurrencies (CCs) are a growing asset class.
Many more CCs are now available on the market since the first
cryptocurrency Bitcoin (BTC) surfaced, Nakemoto (2009). %\natp{\em [BTC was introduced in 2009.]}
As the network effect weighs in, the prices of bitcoin and its variants have risen in tandem. These innovations and the perceived investment potential have led to rapid growth in the number of altcoins and the market size of CC. The price of bitcoin even surged to USD 20,000 at beginning of 2021. 
Bitcoin is popular with the techno tribe, the currency is regarded as being beyond the reach of government regulation- the anonymous founder of Bitcoin introduced the idea of a distributed block chain to prevent the counterfeiting of Bitcoin \citep{oet2015evaluating}.
In response to the rapid development of the CC market, the CME group launched a BTC future contract in Dec 2017. Trading volume in BTC futures surpassed \$2 trillion in 2020 (CryptoCompare, 2020).
While more and more investors (individuals and institution) are adding
CCs and their derivatives into their portfolios. %\natp{portfolios (was: portfolii)}
we see the need to understand the downside risk and find a suitable way to hedge and are interested in resisting extreme risks and improve their profits.
This paper analyse modern techniques for the choice of the hedge ratio of the CC portfolios with various copulae and risk measures. \medskip

Optimal hedge ratio is the appropriate size of futures contracts which should be held such that the movements in future price cancel that of BTC.
The task of determining an optimal hedge ratio is not easy.
It relies on the dependence between the BTC price and future price.
%In this paper, we investigate the performance of different copulae and
%risk measures in hedging Bitcoin and CRIX with Bitcoin
%futures. {\color{blue} only this portfolio?} \natp{\em [$\leftarrow$ Make this the first sentence of the paper!]}
Copulae provide the flexibility to model multivariate random variable
separately by its margins and dependence structure.
The concept of copulae was originated by Wassily Hoeffding \citep{hoeffding2012collected} and, despite the slight difference, popularised by the work of Abe Sklar \citep{Sklar1959}.
Different risk measures account for investors' risk attitude.
They serve as loss functions in the searching process of optimal hedge ratio.
Vast literature discussed the relationship between risk measures and investor's risk attitude, we refer readers to
\citet{artzner1999coherent} for an axiomatic, economic reasoning approach of risk measure construction;
\citet{embrechts2002correlation} for reasoning of using Expected Shortfall (ES) and Spectral Risk Measure (SRM) in addition to VaR;
\citet{Acerbi2002} for direct linkage between risk measures and investor's risk attitude using the concept of "risk aversion function". \medskip

Financial asset return is known to be non Gaussian \citep{fama1963mandelbrot}.
In particular, Gaussian models cannot produce so-called fat tails and asymmetry of observed probability densities,
which leads to underestimate financial risks.
%\natp{\cite{Cont2001}}. \natp{\em [non-Gaussian is not
  %very specific, a uniform r.v.\ is also non-Gaussian. Try to express
 % this in a more meaningful way, e.g.\ Financial data are known to
  %exhibit more extreme events than a normal distribution can capture.]}
Therefore, one cannot solely rely on $2^\text{nd}$ order moment calculations in order to
minimize downside risk. Variance as a risk measure doesn't consider the variety of investors' utility functions. However, the investors are tail-risk averse.
\citet{bollerslev2015tail} find that the jump tail risk is more closely associated with changes in risk-aversion.
It is important to link the investor utility's functions as hedging the tail risk.
Significant tail risks lead to the need to investigate even static hedge with more refined methods than minimum-variance based \citep{ederington2008minimum}. \medskip
%One may turn to VaR to monitor tail risk. Hence, the VaR as a sole risk measure has two disadvantages.
%First, it reflects only tail probability and not tail loss, and next it is not coherent;

In order to capture the risk preferences of investors, in addition to variance, we include other risk measures.
We consider also Value-at-Risk (VaR), Expected Shortfall (ES), and Spectral Risk Measure (SRM).
%Coherency is a very natural property that suggest diversification will reduce risk.
VaR is widely used by the industry and easy to understand.
ES and SRM are chosen because of their coherence property, in particular, they encourage diversification.
SRM is also directly related to individual's utility function.
Popular examples are the exponential SRM and power SRM introduced by
\citet{dowd2008spectral}.\medskip
%\natp{\em [Careful with wording! The
 %hedge is not optimal. The optimal hedge ratio that minimises a risk measure is chosen.]}
%In particular, the SRM proposed by \citet{Acerbi2002} accounts for investors' utility (i.e. risk preference).
%SRM is a weighted average of the quantities of a loss distribution, the weights of which depend on the investor's risk-aversion. In other words, 

% \natp{\em [This needs not go in the introduction. Just briefly mention the risk measures, and possibly that ES and SRM are chosen because of their coherence.]}
%\medskip
%
%\natp{The introduction should go along the lines:
%  \begin{itemize}
%  \item 1-2 sentence: Which problem is solved in this paper?
%  \item Background Bitcoin: Growth, but roller-coaster ride, institutional investment,
%    exchange-traded futures (the exchange is important!) (5 sentences,
%    with references!)
%  \item Significant tail risks and basis risk lead to the need to
%    investigate even static hedges (=futures) with more refined
%    methods than minimum-variance based (reference to Ederington
%    here; this uses variance as risk measure and correlation as
%    dependence measure).
%  \item To capture empirical properties, extend to other risk measures
%    and dependence models. 
%  \end{itemize}
%  }

This paper considers hedging Bitcoin using its future. % and an aggregated index of cryptocurrencies CRIX,
i.e. to find an optimal hedge ratio $h^*$ such that the risk of a hedged portfolio $r^h = r^S - h^*r^F$ has
minimal risk.
Here $r^S$ as the log return of BTC spot price, $r^F$ the log return of Bitcoin future.
The leptokurtic properties mentioned above leads us to deploy a comprehensive way of modelling dependency namely copulae together with various risk measures as loss function to find optimal hedge ratio.
In this paper, we first calibrate the log returns of BTC and CME future by copulae,
then find the optimal quantity of assets in the hedged portfolio according to a range of risk measures.
%By Sklar's theorem, we can model the margin and the dependence structure separately using copulae.
%This gives us enormous flexibility to model financial data.
\citet{barbi2014copula} use the C-convolution operator introduced by \citet{cherubini2011copula} to derive the distribution
of linear combination of margins with copula as their dependence structure.
Our main result shows that the distribution function of a linear combination of random variables can be expressed via the copula and margins. \medskip
%We propose a corrected expression the \citet{barbi2014copula}'s equation and propose a general expression for the density of the linear combination.
 %\natp{\em [No need to mention in the introduction that
 % the version is corrected. The main result -- that the distribution
 % function of a linear combination of random variables can be
  %expressed via the copula and margins -- remains valid.]}\medskip
%Another advantage of copulae is that they \natp{capture (was: describe)} the whole dependence structure of random variables.
%Figure \ref{fig:density illustration} illustrate the samples drawn
%from different copulae but with the same Spearman's $\rho$. \natp{\em
%  [Move this out of introduction.]}
%
%The distribution of the linear combination of margins $Z$ is also affected by the copulae.
%One can see the $Z$ of Gumbel and Clayton copula are skewed to the right and left respectively due to the
%asymmetry (radial symmetry \citet{Nelsen1999}) of copula.\natp{\em
%  [Move this out of the introduction.]}
%\medskip
%
%\begin{figure}[h]
%\includegraphics[width=\textwidth]{_pics/density illustration1.png}
%\includegraphics[width=\textwidth]{_pics/density illustration2.png}
%  \caption{Upper Panel: Density of $Z= X - hY$ of different copulae with
%  $X, Y \sim N(0,1)$,
%  $0.75$ Spearman's rho between $X$ and $Y$, and $h=0.5$;
%  Lower Panel: Scatter plot of samples from copulae.
%  This illustration shows how dependence structure modelled by different copulae affects the density of the linear combination
%  of margins.
%  Notice that the $Z$ modelled by the asymmetric copulae, namely the Clayton and Gumbel copulae, are skewed to right
%  and left respectively. \href{http://www.quantlet.com/}{\includegraphics[width=15pt]{_pics/qletlogo_tr.png}}}
%\label{fig:density illustration}
%\end{figure}

%Even though they reveal that SRM have some properties which cause problems when applying to practical risk management,
%they show that exponential utility function might be plausible in some circumstances under weak conditions \citealp{buhlmann1980economic}.

%However, it still causes some problems to capture the behaviors of investors when the value of absolute risk aversion (ARA) parameter beyond a threshold \citealp{markowitz2014mean}.
%If the relative risk aversion coefficients (RRA) are less than 1, \citet{dowd2008spectral} address that the weighting of lower risk-averse is higher than the higher risk-averse as the loss of portfolio increases.
%On the other hand, the power SRM proposed by \citet{dowd2008spectral} when the relative risk aversion coefficients (RRA) are larger than 1, has also proper features to give a higher weight as loss increase.
%Note that the selection of the utility function and the value of risk aversion parameter would be the matters of solving specific financial problem.
%By contrast, the estimation of the VaR and the ES are conditional on the confidence level which is not easy to determine.
%Since SRM is capable of reflecting the investor's attitude toward risk and has been applied to various fields of financial decision making, this paper apply to the determination of the optimal hedge ratio.
%It is important for the hedger who should choose a proper value for the hedge ratio in order to minimize the risk of portfolio.\\
%
%A joint distribution of spot and futures has been specified in terms of a copula function to embody the tail behaviors of the spot and the futures \citealp{barbi2014copula}.
%Copulae enable us to build the flexible multivariate distributions of dependence structure.
%This paper conducts four types of copulae (Gaussian, t, Frank, Clayton, and Normal Inverse Gaussian) to derive the copula representation of quantities to reach copula-based SRM of the hedged portfolio.
%It should be noted that the Clayton copula can be also used to construct the joint distribution with right tail dependence. Frank copula is symmetric and appropriate for data that exhibit weak or no tail dependence. Normal Inverse Gaussian (NIG) copula is a flexible system of joint distribution that includes fat-tailed and skewed distributions. However, there is still no evidence yet for selecting an exclusive copula in applications of hedging.\\
%
%An optimal hedge ratio represents the investor's subjective marginal rate of substitution between risk and return. \citet{cecchetti1988estimation} found that the optimal hedge ratios increases when an investor with a greater risk aversion by maximizing the expected value of the logarithm of wealth.
%It is understandable if a investor's attitude is more risk-averse, they will increase the position of futures contracts to hedging the uncertain risk which they may take in the future.
%On the contrary, \citet{brandtner2015decision} address that the theoretical result predictions for the subset of exponential and power SRMs are not reasonable but may be counter-intuitive if the corresponding parameter of risk aversion is large enough.
%Different from \citet{brandtner2015decision}, we consider the joint distribution of financial assets to choose the optimal hedge ratio by minimizing SRM.
%However, the empirical result shows the direction of optimal hedge ratio is increasing as the parameters represents the investors' attitude increases. \\

%This paper has two main contributions to the existing literature. First, we reveal the quantiles of loss function built by different copulae. Second, by minimizing exponential SRM (ESRM) to determine the optimal hedge ratio, we give a guidance on the choice of risk aversion function to assist the risk manager in choosing an optimal risk aversion function for a portfolio. To our knowledge, these have received no attention so far in the published literatures.\\

This research proposes the model techniques for the analysis of the hedging strategy on the CC's tail risk in five aspects.
The remainder of the article is organized as follows. Section 2 methodology. Section 3 data, and Section 4 empirical result. Section 5 concludes. \medskip

All calculations in this work can be reproduced.
The codes are available on \href{http://www.quantlet.com/}{www.quantlet.com}.
%%%%%%%%%%%%%%%%%%%%%%%%%%%%%%%%%%%%%%%%%%%%%%%%%%%%%%%%%%%%%%%%%%%%%%%%%%%%%%

\section{Optimal hedge ratio}\label{sec:optimal-hedge-ratio}

\subsection{Distribution of hedge portfolio}\label{subsec:DHP}
We form a portfolio with two assets, consisting of one unit in the
spot asset and a short position of $h$ units of a futures contract,
for example one Bitcoin and a short position in a CME Bitcoin
futures contract. 
The objective is to minimize the risk of the exposure in the spot. 
Let $R^S$ and $R^F$ be the (discrete) returns of the spot and
futures price. The (discrete) return of the portfolio is\footnote{%
In practice, as the nominal investment in the futures is zero, $R^F$
is understood as the return on the notional amount underlying the
futures contract. In other words, if both the spot price $S_{t-1}$
and the futures price $F_{t-1}$ are 
normalised to $1$, then the portfolio return will be identical to the
portfolio value change $\Delta V = \Delta S - h \Delta F$, where $\Delta S =
S_t-S_{t-1}$, etc.}
\begin{equation*}
R^h = R^S -h R^F.
\end{equation*}
%\natp{\em [I fixed this, please check.] [We need to discuss the
%  footnote. Generally, the portfolio return is $R_p = \sum_{i=1}^n w_i
%  R_i$. With the futures contract, the notional investment in the
%  futures is zero, so the portfolio return is $(S_0 (1+R^S) -h F_0
%  R^F)/S_0-1 = R^S-h R^F$, if $S_0=F_0$.]}

To measure risk, we define a risk measure $\rho$ to be a mapping from
a financial position or its return, such as $R^h$, to a real number, which is often
interpreted as the amount of money to make the position acceptable
(e.g.\ to a regulator), see e.g.\ \citep{Foellmer2002}.
For example, a widely used risk measure is value-at-risk (VaR), which,
at the confidence level $\alpha$,
is derived from the $1-\alpha$ quantile of the return distribution. %  at the confidence level $\alpha$ is
% the absolute value of the $1-\alpha$-quantile of $R^h$, i.e., $\text{VaR}_{1-\alpha} =
% -F_{R^h}^{(-1)}(1-\alpha) = -\inf\{x \in \mathbb{R}: 1-\alpha \leq
% F_{R^h}(x) \}$, where $F_{R^h}$ is the distribution function of
% $R^h$.

If the portfolio reduces the risk of the spot position, then
we call this a hedge portfolio.
An optimal hedge ratio $h^*$ is a parameter that
minimizes the risk of the aforementioned portfolio
\begin{equation*}
h^* = \argmin_h \rho(R^h).
\end{equation*}

Obviously the cdf and pdf of $R^h$ and the risk measure depend on the
joint distribution of $R^S$ and $-hR^F$. However, optimising $h$
according to $f_{R^S,-hR^F}$ is unfavorable since one would need to
calibrate the joint pdf $f_{R^S,-hR^F}$ whenever updating $h$.
Another problem of using the joint pdf is that one lacks the
flexibility to model the margins separately from the dependence
structure. Copulae allow to overcome both of these problems. 

The advantage of using copulae is two-fold.
First, copulae are invariant under strictly
monotone increasing function \citep{schweizer1981nonparametric}, a
property used in Lemma \ref{lemma:copula} below. 
Second, copulae allow us to model the margins and dependence structure 
separately, a result known as Sklar's Theorem \citep{Sklar1959}, which
is given as Theorem \ref{theorem:sklar} below. 
See also \citep{Nelsen1999, joe1997multivariate, McNeil2005} for
Sklar's Theorem and more properties of copulae.

We adapt the definition of a two-dimensional copula from \citep{Nelsen1999} as follows.

\begin{defi} [A two-dimensional copula]
  A two-dimensional copula is a function $C: [0,1]^2 \mapsto [0,1]$ with following properties:
  \begin{enumerate}
    \item For every $u,v$ in $[0,1]$,
      \[C(u,0)= C(0,v)=0, \]
    \[C(u,1)= u \text{, and}\]
    \[C(1,v)= v;\]
    \item For every $u_1,u_2, v_1, v_2$ in $[0,1]$ such that $u_1 \leq u_2$ and $v_1 \leq v_2$,
    \[C(u_2,v_2)-C(u_2,v_1)-C(u_1, v_2)+C(u_1,v_1) \geq 0\].
    \end{enumerate}
  \end{defi}

The second property is called 2-non-decreasing.
In other words, the two-dimensional copula is a joint cdf of a two-dimensional random vector
on a unit square with uniform marginals.

The following Hoeffding-Sklar-Theorem (usually known as Sklar theorem) ensures the existence of copula.

\begin{theorem}[Hoeffding-Sklar-Theorem]
  \label{theorem:sklar}
  Let $F$ be a joint distribution function with marginal distributions
  $F_X$ and $F_Y$. Then, there exists a copula $C:[0,1]^2 \mapsto
  [0,1]$ such that, for all $x,y\in \R$
  \begin{equation}
    \label{eq:4}
    F(x,y)=C\{F_X(x), F_Y(y)\}.
  \end{equation}
  If the margins are continuous, then $C$ is unique; otherwise $C$ is
  unique on the range of the margins.

  Conversely, if $C$ is a copula and $F_X, F_Y$ are univariate
  distribution functions, then the function $F$ defined by (\ref{eq:4})
  is a joint distribution function with margins $F_X, F_Y$.
\end{theorem}

Indeed, many basic results about copulae can be traced back to early
works of Wassily Hoeffding \citep{hoedffding1940, hoedffding1941}. 
The works aimed to derive a measure of relationship of variables,
which is invariant under change of scale. 
See also \citet{hoeffding2012collected} for English translations of
the original papers written in German. 
%The following Lemma is not hard to prove.

\begin{lemma}
  \label{lemma:copula}
  Let $h>0$ and let $X$ and $Y$ be continuous random variables. Then,
  the joint distribution of the portfolio positions 
  can be expressed via the joint distribution of the securities as
  follows:
  \begin{align}
  C_{X, hY}\left(F_X(s),F_{hY}(t)\right) = C_{X,
    Y}\left(F_X(s),F_{Y}(t/h)\right), \quad s,t\in \R.
    \end{align}
  \end{lemma}

\begin{proof}
  Since copulae are invariant under strictly monotone increasing
  function \cite[Theorem 3 (i)]{schweizer1981nonparametric} or
  \cite[Theorem 2.4.3]{Nelsen1999}, 
  \begin{equation*}
    C_{X, hY}\left(F_X(s),F_{hY}(t)\right) = C_{X, Y}\left(F_X(s),F_{hY}(t)\right).
    \end{equation*}
Re-writing the second argument of the copula gives
\begin{equation*}
  F_{hY}(t) = \mathbb{P}(hY \leq t)
  = \mathbb{P}(Y \leq t/h)
  = F_Y(t/h).
\end{equation*}
\end{proof}

%The optimal hedge ratio is $h^\ast = \argmin_h \rho(Z)$, that is the best ratio that can minimize the risk of a hedged portfolio measured in terms of $\rho$ .
Leveraging these two features of copulae, \citet{barbi2014copula}
introduce the distribution of linear combinations of random variables
using copulae. 
We slightly edit their Corollary 2.1 of their work and yield the 
following expression of the distribution. 

\begin{proposition}
  \label{prop:dfrh}
  Let $X$ and $Y$ be two real-valued continuous random
  variables on a
  probability space $(\Omega, \F, \p)$ with
  absolutely continuous copula $C_{X, Y}$ and marginal distribution functions $F_{X}$
  and $F_{Y}$. Then, the distribution function of $Z=X-hY$, $h >0$,  is given by
  \begin{equation}
    \label{eq:3}
    F_{Z}(z) = 1- \int^1_0 D_1 C_{X, Y}
    \left[ u, F_{Y} \left\{ \frac{F^{(-1)}_{X}(u)-z}{h} \right\}
    \right]\, d u,   
  \end{equation}
  where, $F^{(-1)}$ denotes the inverse of $F$, i.e., the quantile
function.
\end{proposition}
Here, $D_1 C(u,v)=\displaystyle \frac{\partial}{\partial u}
C(u,v)$ and, see e.g.\ Equation (5.15) of \citep{McNeil2005},
\begin{equation}
  \label{eq:1}
  D_1 C_{X,Y}\{F_X(x), F_Y(y)\} = \p(Y\leq y|X=x).
\end{equation}
\begin{proof}
  Using the identity (\ref{eq:1}) gives
  \begin{align*}
    F_{Z}(z) &= \p(X - h Y\leq z) %
                 = \E\left\{\p\left(Y\geq \frac{X-z}{h}\Big|
                 X\right)\right\}\\[10pt]
               &= 1-\E\left\{\p\left(Y\leq \frac{X-z}{h}\Big|
                 X\right)\right\}% \\[10pt]
               = 1- \int_0^1 D_1 C_{X, Y}\left[u,
                 F_{Y}\left\{\frac{F^{(-1)}_{X}(u) -
                 z}{h}\right\}\right]\, d u.
  \end{align*}
  \end{proof}

%In addition to~\cite{barbi2014copula} we propose a more handy
%expression for the pdf of $Z$.
%\natp{\em [Please double-check the ``+'' signs in the second equation.]}\ \francis{ \em [the + sign is correct.]}

\begin{corollary} The pdf of $Z$ can be written as
  \begin{align}
  f_{Z}(z) &= h^{-1}\int_0^1 c_{X, Y} \left[
  F_{Y}\left\{\frac{F^{(-1)}_{X}(u)-z}{h}\right\}, u
  \right]
   \cdot
  f_{Y}
  \left\{\frac{F^{(-1)}_{X}(u)-z}{h}\right\} du, \label{eq:density1}
  \end{align}
  \end{corollary}
Note that the pdf of $Z$ in the above proposition can be assessed via numerical integration
as long as we have the copula density and the marginal
densities.
A multivariate generalised of the expression above and its proof can be found in the
appendix \ref{sec:appendix}.

\subsection{Backtesting Procedure}\label{sec:empirical-procedure}
First, we take the earliest 300 data points from the dataset 
as training data to obtain the optimal hedge ratio via the following steps:

\begin{enumerate}
\item \textbf{Construct univariate kernel density function (KDE)}:
  Construct the spot and futures' univariate kernel density functions seperately
  using the Gaussian kernel. The bandwidths are determined seperately by the refined plug-in method \citep[section
  3.3.3]{hardle2004nonparametric}.
\item \textbf{Calibrate copulae}:
  Calibrate the copulae outlined in section \ref{subsec:copulae} by the
  method of moments described in section \ref{subsec:simulated-method-of-moments}.
\item \textbf{Select copula}:
  Compute the Akaike Information Criterion (AIC). The copula with the
  best (i.e., lowest) AIC is used for the next step. 
  A discussion of this step is found in \ref{subsec:copula-selection}.
\item \textbf{Determine optimal hedge ratio}:
  Determine the optimal hedge ratios with respect to different
  risk measures numerically. 
  To do so, we draw samples from the calibrated copulae and KDEs 
  and search for the hedge ratio that gives the lowest risk measure. 
  The risk measures are outlined in \ref{subsec:spectral-risk-measures} 
  The minimisation algorithm \textit{scipy.optimize.minimize} from a Python package Scipy \citep{2020SciPy-NMeth} is used for the search of optimal hedge ratio.
\end{enumerate}

Next, we apply the optimal hedge ratio to the test data to obtain out-of-sample hedged portfolio returns.
The test data is the 5 data points subsequent to the last training data point. 
The out-of-sample portfolio returns is also 5 data points in length.

Finally, we roll forward by 5 data points and repeat the steps until the test data reach the end of the dataset. 
The collection of out-of-sample portfolio returns forms a non-overlapping time serie (rolling step size is equal to test data length) that represents the performance of 
the hedging methodology. See Section \ref{subsec:HP2} for results and discussions.  

The backtesting procedure without the copula selection step is also carried out to examine the effects of deploying different copula. 
Section \ref{subsec:HP1} discusses the effects. 

\section{Copulae and risk measures}\label{sec:crm}

Recall the definitions given 

\subsection{Dependence measures in copula terms}
This section introduces the dependence measures in Copula terms that are relevant to this work, 
they are the Kendall's tau, Spearman's rho, and quantile dependence. 
The sample, population versions, as well as the version written in copula, 
of the depednece measures are introduced as they will be used in the method of moments calibration described in Section \ref{subsec:simulated-method-of-moments}. 

The following definitions are adapted from \cite{Nelsen1999}. 

\begin{defi}[Concordance]
  Let $(x_i, y_i)$ and $(x_j, y_j)$ denote two realisations of a
  vector $(X, Y)$ of continuous random variables. 
  A pair of observations is concordant if $x_i<x_j$ and $y_i < y_j$, discordant if
  $x_i>x_j$ and $y_i < y_j$ or if $x_i<x_j$ and $y_i>y_j$. 
\end{defi}

The index of observations $i$ and $j$ are interchangable, so the case
$x_i>x_j$ and $y_i>y_j$ is covered.

\begin{defi}[Sample version of Kendall's tau]
  Let $\{(x_1, y_1), ..., (x_n, y_n)\}$ be realisations of a random vector $(X, Y)$,
  let $c$ denote the number of concordant pairs, and $d$ the number of discordant pairs. 
  The {\em{sample version of Kendall's tau}} is defined as:
  \begin{equation*}
  \hat \tau_K = \frac{c-d}{c+d} = \frac{c-d}{\binom{n}{2}}. 
  \end{equation*}
  \end{defi} % put all endall's tau together

  The second equality holds because there are $\binom{n}{2}$ distinct pairs for $n$ observations of a bivariate random variable. 

\begin{defi}[Population Kendall's tau]
  Let $(X_1, Y_1)$ and $(X_2, Y_2)$ be independent and identically distributed random vectors, each with joint distribution function $H$.
  The population Kandall's tau is defined as the difference between probability of concordance and the probability of discordance. 
  That is  
  \begin{equation*}
  \tau_K = P\left\{(X_1- X_2)(Y_1-Y_2) >0\right\} - P\left\{(X_1- X_2)(Y_1-Y_2) < 0\right\}. 
  \end{equation*}
  \end{defi}

\begin{prop}
  Let $X$ and $Y$ be continuous random variables whose copula is $C$. 
  Then the population Kendall's tau for $X$ and $Y$ is 
  \[\tau_K = 1-4\int_{[0,1]^2}
  \frac{\partial C(u,v)}{\partial u}
  \frac{\partial C(u,v)}{\partial v}
  dudv. \]
\end{prop}

We refer readers to \cite{Nelsen1999}[section 5.1.1] for the proof. 

\begin{defi}[Rank]
  Let $x_1,...,x_n$ be realisations of a one dimensional random variable $X$. 
  The rank of $x_i$ is $r_i = k$ if $x_i$ is the $k$-th smallest among $x_1,...,x_n$.
  \end{defi}

\begin{defi}[Sample Spearman's rho]
  Let $\{(x_1, y_1),..., (x_n, y_n)\}$ be realisations of a vector $(X, Y)$ of random variables,
  $r_{ix}$ and $r_{iy}$ be the rank of $x_i$ and $y_i$ respectively, 
  $r_x = (r_{1x},..., r_{nx})$, and $r_y = (r_{1y},..., r_{ny})$. 

  The sample Spreaman's rho is defined as
  \begin{equation*}
  \hat \rho_S = \hat \rho(r_x, r_y), 
  \end{equation*}
  where $\hat \rho$ is the sample Pearson correlation.
  \end{defi}

\begin{defi}[Population Spearman's rho]
  Let $F_X$ and $F_Y$ be the cdfs of random variable $X$ and $Y$ respectively, 
  The population Spreaman's rho is defined as follows:
  \begin{equation*}
  \rho_S = \rho(F_X(X), F_Y(Y)),  
  \end{equation*}
  where $\rho$ is the population Pearson correlation. 
  \end{defi}

  \begin{theo}
    Let $X$ and $Y$ be continuous random variables whose copula is $C$. 
    Then the population Spearman's rho for $X$ and $Y$ is 
    \[\rho_S = 12\int_{[0,1]^2}C(u,v)dudv-3. \]
  \end{theo}

We refer readers to \cite{joe1997multivariate}[section 2.12.2] for the proof. 

Quantile dependence measures the probability of two variables that is higher or below a given quantile of their univariate distributions.

(simulated base paper from Oh and Patton)
\begin{defi}[Sample quantile dependence] 
  Let $\hat F_X$ and $\hat F_Y$ be the empirical cdfs of random variable $X$ and $Y$ respectively.
  Let $(x_1, y_1),...,(x_n, y_n)$ be $n$ realisations of $X$ and $Y$. 
  The sample quantile dependence of $X$ and $Y$ at the $q$-th quantile is 

  \begin{equation*}
    \lambda_q = \begin{cases}
      (nq)^{-1}\sum_{i=1}^n 1\left( 
        \hat F_X(x_i) \leq q, \hat F_Y(y_i) \leq q 
      \right) & q \leq 0.5 \\
      (n(1-q))^{-1}\sum_{i=1}^n 1\left( 
        \hat F_X(x_i) > q, \hat  F_Y(y_i) > q 
      \right) & q > 0.5 
    \end{cases},
  \end{equation*}
  where $1(\cdot)$ is the indicator function. 

\end{defi}

\subsection{Copulae}\label{sec:ellpitical-copulae}

To capture different aspects of the dependence structure, we consider
a set of different copulas, which are layed
  out in detail below. These are the Gaussian-, $t$-, Frank-,
Gumbel-, Clayton-, mixture, NIG factor, and Plackett-copula. 

Figure~\ref{fig:copulaeScatterPlot} shows scatter plots of random
samples of each of the copulae treated. 
\begin{figure}[t]
    \centering
  \includegraphics[width=\textwidth]{_pics/copulas_scatterplots.pdf}
  \caption{Scatterplots of samples drawn from various copulae. All
    copulae are calibrated to Spearman's $\rho$ of 0.75 before
    sampling.}\label{fig:copulaeScatterPlot} 
\end{figure}

As this hedging backtest concerns only portfolios with two assets, we
focus on the bivariate version of each copula. 

\subsubsection{Gaussian and $t$ Copulae}\label{sec:ellpitical-copulae}

The Gaussian and $t$ copulae are dervived from Gaussian and $t$
distributions. 

The bivariate Gaussian copula is defined as
\begin{align*}
  \bm{C}(u,v) &= \Phi_{2, \rho}\{\Phi^{(-1)}(u), \Phi^{(-1)}(v)\} \nonumber \\
              &= \int_{-\infty}^{\Phi^{(-1)}(u)}
                \int_{-\infty}^{\Phi^{(-1)}(v)}
                \frac{1}{2\pi\sqrt{1-\rho^2}}
                \exp{\left\{
                \frac{s^2-2\rho st+t^2}{2(1-\rho^2)}
                \right\}} \dd s\, \dd t,\quad, u,v\in [0,1],
\end{align*}
where $\Phi_{2, \rho}$ is the bivariate Normal cdf
with zero mean, unit variance, and correlation coefficient $\rho$, and
$\Phi^{(-1)}$ is the quantile function of the univariate standard normal
distribution.
The Gaussian copula is fully specified by the correlation parameter $\rho$. \footnote{
The symbol $\rho$ is used to denote both the correlation parameter as
well as a general risk measure. However, it will be clear from the
context, what $\rho$ refers to.}
It has no tail dependence, which, in a finance context, implies that
it often underestimates tail risk.  

% The Gaussian copula density is
% \begin{equation*}
%   \bm{c}_\rho(u,v) = \frac{\bm{\varphi}_{2,\rho}\{\Phi^{(-1)}(u), \Phi^{(-1)}(v)\}}
%                      {\varphi\{\Phi^{(-1)}(u)\} \cdot \varphi\{\Phi^{(-1)}(v)\}} 
%                   = \frac{1}{2\pi\sqrt{1-\rho^2}}\exp\left\{
%                      -\frac{u^2 - 2\rho uv + v^2}{2(1-\rho^2)}
%                      \right\},
% \end{equation*}
% where $\bm{\varphi}_{2,\rho}(\cdot)$ is the pdf corresponding to
% $\Phi_{2, \rho}$, and $\varphi(\cdot)$ the standard normal
% pdf. \natp{\em [I think the abbreviations cdf and pdf where not
%   introduced. Please double-check.]}

Kendall's $\tau_K$ and Spearman's $\rho_S$ of the bivariate Gaussian copula are
    \begin{align*}
        \tau_K(\rho) = \frac{2}{\pi}\arcsin\rho
        \end{align*}
    \begin{align*}
        \rho_S(\rho) = \frac{6}{\pi}\arcsin\frac{\rho}{2}.
        \end{align*}

The $t$-copula has the form
\begin{multline*}
        \bm{C}(u,v) = \bm{T}_{2, \rho, \nu}\{T^{(-1)}_\nu(u), T^{(-1)}_\nu(v)\}\\
        = \int_{-\infty}^{T^{(-1)}_\nu(u)}
               \int_{-\infty}^{T^{(-1)}_\nu(v)}
            \frac{\Gamma\left(\frac{\nu+2}{2}\right)}
            {\Gamma\left(\frac{\nu}{2}\right)\pi\nu\sqrt{1-\rho^2}}
             \left(
        1+\frac{s^2-2st\rho+t^2}{\nu}
        \right)^{-\frac{\nu+2}{2}}\, \dd s\, \dd t,
    \end{multline*}
where $\bm{T}_{2, \rho, \nu}$ denotes the 
bivariate $t$ cdf with dependence parameter $\rho$ and degrees of
freedom parameter $\nu$, $\nu>2$,
and where $T^{(-1)}_\nu(\cdot)$ is the quantile function of a standard
$t$ distribution with parameter $\nu$. 

The $t$-copula and Gaussian copula with parameter $\rho$ have equal Kendall's $\tau$, \citep[see][and references therein]{demarta2005t}.

On the other hand, the $t$-copula has a non-zero tail dependence coefficient,
 which makes it more appropriate for dependence modelling in finance. (ref)
% The copula density is
% \begin{align*}
%     \bm{c}(u,v) &= \frac{\bm{t}_{2, \rho, \nu}\{T^{(-1)}_\nu(u), T^{(-1)}_\nu(v)\}}
%     {t_\nu\{T^{(-1)}_\nu(u)\}\cdot t_\nu\{T^{(-1)}_\nu(v)\}},
%     \end{align*}
% where $\bm{t}_{2,\rho, \nu}$ is the pdf of $\bm{T}_{2, \rho, \nu}$
% and $t_\nu$ the density of standard $t$ distribution.

\subsubsection{Archimedean copulae}\label{sec:archimedean-copula}
The family of Archimedean copulae forms a large class of copulae with
many convenient features.
% Contrary to elliptical copulas, which are derived from
% elliptical distributions.
Archimedean copulas are determined via a simple parametric form of the
dependence structure. A prominent feature is the ability to model
asymmetric dependence structures.  

In general, an Archimedean copula takes the form
\begin{align*}
  \bm{C}_\theta(u,v) = \psi^{(-1)}\{\psi(u; \theta), \psi(v; \theta); \theta\},\quad u,v\in [0,1],
    \end{align*}
where $\psi:[0,1] \rightarrow [0,\infty)$ is a continuous, strictly
decreasing and convex function such that $\psi(1)=0$ for any
permissible dependence parameter $\theta$. The function $\psi$ is 
called the generator, with $\psi^{(-1)}$ its inverse.

The {\em Frank copula\/} (B3 in \citet{joe1997multivariate}) takes the form
\begin{align*}
    \bm{C}_{\theta}(u,v) &= \frac{1}{\theta}
    \log \left\{
    1 + \frac{(e^{-\theta u}-1)(e^{-\theta v}-1)}{e^{-\theta}-1}
    \right\}, \quad u,v\in [0,1],
    \end{align*}
    with $\theta \in [0, \infty]$ the dependence parameter. 
    It is a symmetric copula and cannot produce any tail
    dependence. The following parameters correspond perfect dependence
    and independence: $\bm{C}_{-\infty} = \bm{M}$, $\bm{C}_1 = \bm{\Pi}$,
    and $\bm{C}_\infty = \bm{W}$. 
    % The copula density is
    % \begin{align*}
    %   \bm{c}_{\theta}(u,v)
    %   &= \frac{\theta e^{\theta(u+v)(e^\theta-1)}}
    %     {\left\{e^\theta-e^{\theta u}-e^{\theta v}+e^{\theta (u+v)}\right\}^2}.
    % \end{align*}
    The Frank copula has Kendall's $\tau$ :
\begin{align*}
    \tau_K(\theta) = 1-4\frac{D_1\{-\log(\theta)\}}{\log(\theta)},
    \end{align*}
% and
% \begin{align*}
%     \rho_S(\theta) = 1-12\frac{D_2\{-\log(\theta)\} - D_1\{\log(\theta)\}}{\log(\theta)},
%     \end{align*}
where $D_1$ and $D_2$ are the Debye function of order 1 and 2, with
the Debye function defined as $D_n =
\frac{n}{x^n}\int_0^x\frac{t^n}{e^t-1}dt$.
We refer readers to \cite{abramowitz1972handbook}[p.998] for definition of the Debye function. 

The {\em Gumbel copula\/} (B6 in \citet{joe1997multivariate}) has
distribution function
\begin{equation*}
  \bm{C}_{\theta}(u,v) = \exp{-\{ (-\log(u))^\theta +(-\log(v))^\theta 
    \}^{\frac{1}{\theta}}},
\end{equation*}
where $\theta \in [1,\infty)$ is the dependence parameter.
Its  Kendall's tau takes the form \begin{equation*}
  \tau_K(\theta) =\frac{\theta-1}{\theta}. 
 \end{equation*}
It has upper tail dependence with dependence parameter $\lambda^U
= 2-2^{\frac{1}{\theta}}$ and displays no lower tail dependence. 
    
While the Gumbel copula cannot model perfect counter-dependence
\citep{Nelsen2002}, $\bm{C}_{1} = \bm{\Pi}$ models independence, 
and $\lim_{\theta\rightarrow\infty} \bm{C}_\theta = \bm{W}$ models
perfect dependence. 


The {\em Clayton copula\/} takes the form
\begin{equation*}
  \bm{C}_{\theta}(u,v) = \left\{
    \max(u^{-\theta}+v^{-\theta}-1,0)\right\}^{-\frac{1}{\theta}},
\end{equation*}
where $\theta \in (-\infty, \infty)$ is the dependence parameter.
The Clayton copula, by contrast to Gumbel copula,
generates lower tail dependence with $\lambda^L =
2^{-\frac{1}{\theta}}$, but cannot generate upper tail dependence.
Moreover, $\lim_{\theta\rightarrow -\infty} \bm{C}_\theta = \bm{M}$, $\bm{C}_0 =
\bm{\Pi}$, and $\lim_{\theta\rightarrow\infty} \bm{C}_\theta = \bm{W}$. 
Kendall's $\tau$ of the Clayton copula is given by 
\begin{align*}
    \tau_K(\theta) =\frac{\theta}{\theta+2}.
    \end{align*}

\subsubsection{Mixture Copula}\label{sec:mixture-copula}
The mixture copula is a linear combination of copulae. 
The distribution of a 2-dimensional random variable
$\bm{X}=(X_1,X_2)^\top$ is written as linear combination of $K$
copulae 
\begin{equation*} 
    \bm{C}(u,v)= \sum_{k=1}^K p^{(k)} \cdot \bm{C}^{(k)}\{F^{(-1)}_{X_1}(u),
    F^{(-1)}_{X_2}(v); \bm{\theta^{(k)}}\}, \quad u,v\in [0,1].
  \end{equation*}
  Here, $\bm{\theta^{(k)}}$ refers to the parameters of the
    $k$-th copula.
%     Likewise, the copula density is a linear
%     combination of copula densities 
% \begin{align*}
%     \bm{c}(u,v)= \sum_{k=1}^K p^{(k)} \cdot \bm{c}^{(k)}\{F^{(-1)}_{X_1}(u),
%     F^{(-1)}_{X_2}(v); \bm{\theta^{(k)}}\}.
%     \end{align*}
   
While Kendall's $\tau$ of the mixture copula is not known in closed form,
Spearman's $\rho$ is easily derived as 
\begin{equation*}
  \rho_S = \sum_{k=1}^K p^{(k)} \cdot \rho_S^{(k)}. 
\end{equation*}

% \natp{\em [Old text below.]}

% While Kendall's $\tau$ of the mixture copula is not known in closed form,
% Spearman's $\rho$ is specified by the following statement. 
% \begin{proposition}
%   Let $\rho_S^{(k)}$ be Spearman's $\rho$ of the $k$-th component
%   Spearman's $\rho$ of the mixture copula is given by 
%   \begin{align*}
%         \rho_S = \sum_{k=1}^K p^{(k)} \cdot \rho_S^{(k)}.
%         \end{align*}
%     \end{proposition}

% \begin{proof}
%   Since Spearman's $\rho$ is defined as \citep{Nelsen1999}
%   \begin{equation*}
%     \rho_S = 12 \int_{\mathbb{I}^2} \bm{C}(s,t) ds dt - 3,
%   \end{equation*}
%   Spearman's $\rho$ of the the mixture copula is given by summation
%   of the components 
%   \begin{align*}
%     \rho_S = 12 \int_{\mathbb{I}^2} \sum_{k=1}^K p^{(k)} \cdot
%     \bm{C}^{(k)}(s,t) ds dt - 3. 
%   \end{align*}
% \end{proof}
% \natp{\em [Continue here.]}

An example of a mixture copula is the Fr\'echet class of copulae, which
are given by convex combinations of $\bm{W}$, $\bm{\Pi}$, and $\bm{M}$
\citep{Nelsen1999}.  

We use a mixture of Gaussian and independence copulae in our analysis,
i.e., 
\begin{equation*}
  \bm{C}(u,v) = p\, \bm{C}^\text{Gaussian}(u,v) + (1-p)(uv),\quad p\in (0,1).
\end{equation*}
% with corresponding density 
% \begin{equation*}
%   \bm{c}(u,v) = p\, \bm{c}^\text{Gaussian}(u,v) + (1-p).
% \end{equation*}

This mixture models the amount of ``random noise'' that appears in the
off-diagonal region of the dependence structure where the Gaussian copula has no control.
In the hedging exercise, the structure of the off-diagonal ``random noise'' is not our main concern, 
but the amount of it might affect the hedging effectiveness. 

\subsubsection{NIG factor copula}

Normal Inverse Gaussian (NIG) distribution is a flexible and yet analytical tractable distribution introduced by
\citep{BarndorffNielsen1997}.
The {\em NIG factor copula} is constructed based on the characteristics of the NIG disribution. 
We present the reparameterised version of NIG factor copula in this section.

The NIG distribution has density function
\begin{equation*}
  g(x; \alpha,\beta, \mu, \delta) = \frac{\alpha}{\pi} \e^{\delta
    \sqrt{\alpha^2-\beta^2} -\beta\mu} \frac{1}{q((x-\mu)/\delta)}
  K_1\left[\delta \alpha q\left(\frac{x-\mu}{\delta}\right) \right]
  \e^{\beta x},\quad x>0,
\end{equation*}
where $q(x) = \sqrt{1+x^2}$ and where $K_1$ is the modified Bessel
function of third order and index $1$. The parameters satisfy $0\leq
|\beta|\leq \alpha$, $\mu\in \R$ and $\delta>0$. The parameters have
the following interpretation: $\mu$ and $\delta$ are location and
scale parameters, respectively, $\alpha$ determines the heaviness of
the tails and $\beta$ determines the degree of asymmetry. If
$\beta=0$, then the distribution is symmetric around $\mu$.

The cdf and quantile function of NIG distribution, denoted by $G(x; \alpha, \beta, \mu, \delta)$ and $G^{(-1)}(x; \alpha, \beta, \mu, \delta)$,
 have no known analytical form.
 In this work, they are computed via numerical integration of the density and by simulation.

The NIG distribution belongs to
the class of so-called {\em normal
variance-mean mixture distributions},  (see Section 3.2 of 
\citep{McNeil2005}): $X$ follows an
$\text{NIG}(\alpha,\beta,\mu,\delta)$ distribution if $X$ conditional
on $W$ follows a normal distribution with mean $\mu+\beta W$ and
variance $W$, i.e., 
\begin{equation*}
  X|W\stackrel{\mathcal L}\sim \Ncdf(\mu + \beta W, W),
\end{equation*}
where $W$ follows an {\em inverse Gaussian distribution}, denoted by
$\text{IG}(\delta, \sqrt{\alpha^2-\beta^2})$.

Simulation procedure of NIG$(\alpha, \beta, \mu, \delta)$ distribution is a natural result of the above decomposition. 
To simulate the NIG distribution, first simulate a random variable $w \sim IG(\delta, \sqrt{\alpha^2-\beta^2})$, 
then simluate $x \sim N(\mu+ \beta w, w)$ given $w$.

The moment-generating function of the NIG distribution is given by
\begin{equation*}
  M(u; \alpha, \beta, \mu, \delta) = \exp\left( \delta
    \left(\sqrt{\alpha^2-\beta^2} - \sqrt{\alpha^2 - (\beta +
        u)^2}\right) + \mu u\right). 
\end{equation*}
As a direct consequence, moments are easily calculated with the
expectation and variance of the NIG distribution being
\begin{align*}
  \mathbb E X &= \mu +
                \frac{\delta \beta}{\sqrt{\alpha^2-\beta^2}}
  \end{align*}
\begin{align} \label{eq:5}
  \text{Var}(X) &= \frac{\alpha^2\delta}{(\alpha^2-\beta^2)^{3/2}}.
\end{align}

It is easily seen from the moment-generating function that the NIG distribution is preserved under linear combinations, provided
the variables share the parameters $\alpha$ and $\beta$. 
\begin{proposition}
  \label{prop:NIG}
  Let $Z\sim \text{NIG}(\alpha, \beta, \mu, \delta)$ and
  $Z_i\sim \text{NIG}(\alpha, \beta, \mu_i, \delta_i)$,
  $i=1,\ldots, n$ be independent NIG-distributed random
  variables. Then:
  \begin{enumerate}
  \item  $X_i = Z + Z_i\sim \text{NIG}(\alpha,\beta,\mu+\mu_i,
  \delta+\delta_i)$,
\item and 
  \begin{align}
    \text{Cov}(X_i,X_j) &= \text{Var(Z)},\nonumber\\
    \text{Corr}(X_i,X_j) &= \frac{\delta}{\sqrt{(\delta+\delta_i)
                           (\delta+\delta_j)}}. \label{eq:6}
  \end{align}
\end{enumerate}
\end{proposition}
\begin{proof}
  \begin{enumerate}
  \item This follows directly from the moment-generating function. 
  \item For the covariance,
    \begin{equation*}
      \text{Cov}(X_i,X_j)
      = \E[(Z+Z_i) (Z+Z_j)] - \E[Z+Z_i] \E[Z+Z_j]
      = \E[Z^2] -(\E Z)^2.
    \end{equation*}
    The correlation is determined directly from \ref{eq:5}.
  \end{enumerate}
\end{proof}

The NIG distribution is popular in many areas of
financial modelling; for example, it gives rise 
to the normal inverse Gaussian L\'evy process, which may be represented
as a Brownian motion with a time change.
In the setting here, we consider the {\em NIG factor copula}, which is
not directly derived from the multivariate NIG distribution, but
determined through a factor structure instead. \footnote{The factor structure,
which was applied e.g.\ in \citep{Kalemanova2007} for calibrating CDO's,
gives additionaly flexibility as it does not force the components to
have a mixing variable $W$.}

Denote 
\begin{align*}
  X &= Z + Z_1 \\ 
  Y &= Z + Z_2,
  \end{align*}
where $Z \sim \text{NIG}(\alpha, \beta, \mu, \delta)$, $Z_1 \sim \text{NIG}(\alpha, \beta, \mu_1, \delta_1)$, 
$Z_2 \sim \text{NIG}(\alpha, \beta, \mu_2, \delta_2)$, and $Z, Z_1, Z_2$ are mutually independent. 

The following reparameterisation steps reduce the number of parameters to three:
\begin{enumerate}
  \item Set $\mu = \mu_1= \mu_2 = 0$ . Location parameter does not affect the correlation structure.
  \item Set $\delta = \frac{(\alpha^2-\beta^2)^{3/2}}{\alpha^2}$, $\delta_1 = \delta_2$, $\tilde \delta = \delta_1 = \delta_2$. 
  The correlation between X and Y is fully captured by $\alpha, \beta$, and $\tilde \delta$.  
\end{enumerate}
  % Denote the margins by $u,v \sim U(0,1)$. 
% The NIG factor model is obtained by transforming the unifrom margins to standardised NIG distriutions.

\begin{prop}
  Let $u,v \in [0,1]$, $f(\cdot) = g\left(\cdot; \alpha, \beta, 0, \frac{(\alpha^2-\beta^2)^{3/2}}{\alpha^2}
  \right)$ and $F (\cdot) = G(\cdot; \alpha, \beta, 0, \tilde \delta)$, 
  the {\em NIG factor copula} is 
  \begin{equation*}
    C(u,v) = \int_\mathbb{R} F(u-z)F(v-z)f(z)dz,
  \end{equation*}
  where $\alpha, \beta \in \mathbb{R}$ satisfying $0 \leq |\beta| \leq \alpha$, and $\tilde \delta >0 $.
\end{prop}

The parameters $\alpha, \beta, \tilde \delta$ fully control the dependence between $u$ and $v$, $u, v \sim U(0,1)$, captured by the NIG factor copula.
We refer readers to \cite{krupskii2013factor} for the methodology of constructing a factor copula. 

The quantile dependence and Spearman rho of NIG factor copula have no known analystical form.
In this work, the quantile dependence is computed numerically; 
the Spearman rho is approximated by the Spearman rho of the bivariate Gaussian copula. 
When $\beta \rightarrow 0$ and $\alpha \rightarrow \infty$, the NIG distribution behaves similarly to Gaussian distribution, 
making the NIG factor copula (bivariate) behaves similarly to the Gaussian copula (bivariate), and therefore, 
the NIG factor copula's Spreaman rho is well approximated by the Spearman rho of the bivariate Gaussian copula.




% \natp{\em [Please clarify that $\circ$ refers to composition. Clean
%   up notation, e.g.\ marginals can be denoted $F_F$ and $F_S$, Use
%   just $C$ for the copula. What are $Z_1$ and $Z_2$? I don't find the
%   formula in the paper mentioned. Also, where is the formula for
%   Kendall's tau taken from?]}
%  The NIG factor copula is obtained by transforming the margins to
% uniforms (see Sklar's Theorem), giving (e.g.\
% \citep{krupskii2013factor}):
% \begin{equation*}
%   C_{r^S, r^F}(F_{r^S}(r^S), F_{r^F}(r^F)) = \int_\mathbb{R}
%   F_{Z_1}(F_{X_1}^{(-1)} \circ F_{r^S}(r^S) -z) \cdot
%   F_{Z_2}(F_{X_2}^{(-1)} \circ F_{r^F}(r^F) -z) \cdot
%   f_Z(z) dz.
%   \end{equation*}
% If the margins are continuous, then Spearman's rho of NIG factor
% copula is 
% \begin{equation*}
%   \rho_S = 12 \int \int \int_{\mathbb{R}^3}
%   F_{X_1}(x_1) \cdot
%   F_{X_2}(x_2) \cdot
%   f_{Z_1}(x_1-z) \cdot
%   f_{Z_2}(x_2-z) \cdot
%   f_Z(z) dx_1 dx_2 dz - \frac{1}{48}.x
%   \end{equation*}

% \begin{proof}
%   \begin{align}
%   \rho_S(r^S, r^F) &= \rho\{F_{r^S}(r^S), F_{r^F}(r^F)\} \\
%     &= \rho\{F_{X_1}(X_1), F_{X_2}(X_2)\} \\
%     &= 12 \cdot \mathbb{E}\{F_{X_1}(X_1) \cdot F_{X_2}(X_2) \} - \frac{1}{48}\\
%     &= 12 \cdot \int \int_{\mathbb{R}^2} F_{X_1}(X_1) \cdot F_{X_2}(X_2) dF_{X_1,X_2}(x_1,x_2)\\
%     \end{align}
%   Because
%   \begin{align}
%     F_{X_1,X_2}(x_1,x_2) &= \mathbb{P}(X_1 \leq x_1, X_2 \leq x_2)\\
%     &= \mathbb{P}(Z_1 \leq x_1 - Z, Z_2 \leq x_2 - Z) \\
%     &= \int_\mathbb{R}\mathbb{P}(Z_1 \leq x_1 - z) \cdot \mathbb{P}(Z_2 \leq x_2 - z) \cdot f_Z(z) dz,
%     \end{align}
%   so,
%   \begin{align}
%     \rho_S(r^S, r^F) = 12 \cdot \int \int \int_{\mathbb{R}^3} F_{X_1}(x_1) \cdot F_{X_2}(x_2) \cdot f_{Z_1}(x_1 -z) \cdot f_{Z_2}(x_2 -z) \cdot f_{Z}(z) dx_1 dx_2 dz -\frac{1}{48}
%     \end{align}
%   \end{proof}


\subsubsection{Plackett copula}\label{subsec:other-copula}
The Plackett copula has distribution function
\begin{align*}
    \bm{C}_{\theta}(u,v) &= \frac{1+(\theta-1)(u+v)}{2(\theta-1)}
                         - \frac{\sqrt{\{
    1+(\theta-1)(u+v)\}^2 - 4uv\theta(\theta-1)}}{2(\theta-1)},
\end{align*} where $0 \leq \theta < \infty$.
Spearman's Rho is given by 
\begin{align*}
    \rho_S(\theta) = \frac{\theta+1}{\theta-1} - \frac{2\theta \log
  \theta}{(\theta-1)^2}. 
    \end{align*}

The Placket copula possesses a special property:
the cross-product ratio is equal to the dependence parameter
\begin{equation} \label{eq:PlackettCrossProduct}
    % &\phantom{=}
    \frac{\p(U \leq u, V \leq v) \cdot \p(U > u, V > v)}
    {\p(U \leq u, V > v) \cdot \p(U > u, V \leq v)}\nonumber
    =
      \frac{\bm{C}_\theta(u,v)\{1-u-v+\bm{C}_\theta(u,v)\}}{\{u-\bm{C}_\theta(u,v)\}\{v-\bm{C}_\theta(u,v)}\nonumber 
    = \theta.
\end{equation}
In words, the dependence parameter is equal to the ratio of the 
number of concordance pairs and the number of discordance pairs of a 
bivariate random variable. 

%! Author = francis
%! Date = 30.10.20


\subsection{Simulated Method of Moments}\label{subsec:simulated-method-of-moments}
This method is suggested by Oh and Patton (2013).
In this setting, rank correlation e.g. Spearman's $\rho$ or Kendall's $\tau$,
and quantile dependence measures at different levels $\lambda_q$
are calibrated against their empirical counterparts.\medskip

Spearman's rho, Kendall's tau, and quantile dependence of a pair $(X,Y)$
with copula $C$ are defined as
\begin{align}
  \rho_S &= 12 \int\int_{I^2} C_{\bm{\theta}}(u,v)\, \dd u\, \dd v-3\label{eq:rho_S}\\
  \tau_K &= 4\mathbb{E}[C_{\bm{\theta}}\{F_X(x), F_Y(y)\}]-1,\\
  \lambda_q &=
  \begin{cases}
    \p(F_X(X)\leq q| F_Y(Y)\leq q) = \displaystyle \frac{C_{\bm{\theta}}(q,q)}{q},
    &\text{ if } q\in (0,0.5],\\
    \p(F_X(X)>q|F_Y(Y)>q) =\displaystyle \frac{1-2q+C_{\bm{\theta}}(q,q)} {1-q},
    &\text{ if } q\in (0.5,1).
  \end{cases}
\end{align}\medskip
The empirical counterparts are
\begin{align*}
  \hat\rho_S &= \frac{12}{n} \sum_{k=1}^n \hat F_X(x_k) \hat F_Y(y_k)
               -3,\\
  \hat\tau_K &= \frac{4}{n}\sum_{k=1}^n \hat{C}\{\hat{F}_X(x_i),\hat{F}_X(y_i)\} -1 ,\\
  \hat\lambda_q &=
                  \begin{cases}
                    \displaystyle\frac{1}{n} \sum_{k=1}^n \frac{\1_{\{\hat
                        F_X(x_k)\leq q, \hat F_Y(y_k)\leq q\}}} {q},
                    &\text { if } q\in (0, 0.5],\\
                    \displaystyle \frac{1}{n} \sum_{k=1}^n
                    \frac{\1_{\{\hat F_X(x_k)>q, \hat F_Y(y_k)>q\}}}
                    {1-q}, &\text { if } q\in (0.5,1).
                  \end{cases},
\end{align*}
where $\hat{F}(x) := \frac{1}{n}\sum_{k=1}^n 1_{\{x_i\leq x\}}$ and
$\hat{C}(u,v) := \frac{1}{n}\sum_{k=1}^n 1_{\{u_i\leq u, v_i\leq v\}}$.\medskip

We denote $\tilde{\bm{m}}(\bm{\theta})$ be a $m$-dimensional vector of dependence measures according the the
dependence parameters $\bm{\theta}$,and  $\hat{\bm{m}}$ be the corresponding empirical counterpart.
The difference between dependence measures and their counterpart is denoted by
\begin{align*}
    \bm{g}(\bm{\theta}) = \hat{\bm{m}} - \tilde{\bm{m}}(\bm{\theta}).
\end{align*}\medskip

The SMM estimator is
\begin{align*}
    \hat{\bm{\theta}} = \argmin_{\bm{\theta}\in \bm{\Theta}} \bm{g}(\bm{\theta})^\intercal
    \hat{\bm{W}}
     \bm{g}(\bm{\theta}),
\end{align*}
where $\hat{W}$ is some positive definite weigh matrix.\medskip

In this work, we use $\tilde{\bm{m}}(\bm{\theta}) = (\rho_S, \lambda_{0.05}, \lambda_{0.1},
\lambda_{0.9}, \lambda_{0.95})^\intercal$
for calibration of Bitcoin price and CME Bitcoin future.

\subsection{Maximum Likelihood Estimation}\label{subsec:maximum-likelihood-estimation}
By Sklar's theorem, the joint density of a $d$-dimensional random variable $\bm{X}$ with sample size $n$ can be written as
\begin{align}
    \bm{f}_{\bm{X}}(x_1, ..., x_d) = \bm{c}\{F_{X_1}(x_1), ..., F_{X_d}(x_d)\} \prod_{j=1}^d f_{X_i}(x_i).
    \end{align}
We follow the treatment of MLE documented in section 10.1 of \citet{joe1997multivariate}, namely the inference functions for margins or IFM method.
The log-likelihood $\sum^n_{i=1}f_{\bm{X}}(X_{i,1}, ..., X_{i,d})$ can be decomposed into dependence part and marginal part,
\begin{align}
    L(\bm{\theta}) &= \sum_{i=1}^n \bm{c}\{F_{X_1}(x_{i,1};\bm{\delta}_1), ..., F_{X_d}(x_{i,d}; \bm{\delta}_d);\bm{\gamma}\}
    + \sum_{i=1}^n \sum_{j=1}^d f_{X_j}(x_{i,j};\bm{\delta}_j)
    &= L_C(\bm{\delta}_1, ..., \bm{\delta}_d, \bm{\gamma}) + \sum_{j=1}^d L_j(\bm{\delta}_j)
    \end{align}
where $\bm{\delta}_j$ is the parameter of the $j$-th margin, $\bm{\gamma}$ is the parameter of the parametric copula, and
$\bm{\theta} = (\bm{\delta}_1,..., \bm{\delta}_d, \bm{\gamma})$.

Instead of searching the $\bm{\theta}$ is a high dimensional space, \citet{joe1997multivariate} suggests to
search for $\hat{\bm{\delta}_1},..., \hat{\bm{\delta}_d}$ that maximize $L_1(\bm{\delta}_1), ..., L_d(\bm{\delta}_d)$,
then search for $\hat{\bm{\gamma}}$ that maximize $L_C(\hat{\bm{\delta}_1},..., \hat{\bm{\delta}_d}, \bm{\gamma})$.

That is, under regularity conditions, $(\hat{\bm{\delta}_1},..., \hat{\bm{\delta}_d}, \hat{\bm{\gamma}})$ is the solution of
\begin{align}
    \left( \frac{\partial L_1}{\partial \bm{\delta}_1}, ..., \frac{\partial L_d}{\partial \bm{\delta}_d},
    \frac{\partial L_C}{\partial \bm{\gamma}}\right) = \bm{0}.
    \end{align}

However, the IFM requires making assumption to the distribution of of the margins.
\citet{genest1995semiparametric} suggests to replace the estimation of marginals parameters estimation by non-parameteric estimation.
Given non-parametric estimator $\hat{F}_i$ of the margins $F_i$, the estimator of the dependence parameters $\bm{\gamma}$ is
\begin{align}
    \hat{\bm{\gamma}} = \argmax_{\bm{\gamma}} \sum_{i=1}^n \bm{c}\{ \hat{F}_{X_1}(x_{i,1}), ..., \hat{F}_{X_d}(x_{i,d});\bm{\gamma}\}.
    \end{align}



%With the decomposition, the MLE estimator for a bivariate parametric copula is
%\begin{align}
%    \hat{\bm{\theta}} = \argmax_{\bm{\theta} \in \bm{\Theta}} l(X_1,X_2; \bm{\theta}), \label{eq:EMLE}
%    \end{align}
%where
%\begin{align}
%    l(X_1,X_2; \bm{\theta}) = \sum_{i=1}^n \log c(x_{i,1}, x_{i,2};\bm{\theta}). \label{eq:Likelihood}
%    \end{align}\medskip

%Procedure of maximising equation~\ref{eq:EMLE} as a whole is called exact maximum likelihood method.
%Leveraging the attractive feature of copula that one can model the dependence structure and marginals separately,
%we rewrite~\ref{eq:Likelihood} into canonical expression
%\begin{align}
%    l(X,Y; \bm{\theta}) = \sum_{k=1}^n \log c\{F_X(x_i; \delta_X), F_Y(y_i; \delta_Y); \bm{\gamma}\}
%    + \sum_{k=1}^n \log f_X(x_i; \bm{\delta}_X) + \sum_{k=1}^n \log f_X(y_i; \bm{\delta}_Y),
%    \end{align}
%where the $\bm{\gamma}$ is the dependence parameter in the copula and $\bm{\delta}$ is the parameters in the margins.\medskip
%
%The inference-functions for margins (IFM) approach by Joe is a two step procedure of maximising~\ref{eq:EMLE}.
%The approach calibrate first the $\bm{\delta}$s and then the  $\bm{\gamma}$.\medskip
%
%Similar to the IFM approach, pseudo-maximum likelihood approach by Genest and Rivest (1993) replace the parametric margins by
%empirical estimates, we rewrite \ref{eq:Likelihood} again with
%\begin{align}
%    l(X,Y; \bm{\theta}) = \sum_{k=1}^n \log c(u_i, v_i;\bm{\gamma}),
%    \end{align}
%where $u_i = \hat{F}_X(x_i)$ and $v_i = \hat{F}_Y(y_i)$.

\subsection{Comparison}
Both the simulated method of moments and the maximum likelihood estimation are unbiased and
proven to give good fits.
The problem remain is which procedure is more suitable for hedging.
%Cryptocurrencies are known to be very volatile.
Sample and fitted quantile dependence for Bitcoin and CME future.

%\begin{figure}[th]
%\includegraphics[width=\textwidth]{_pics/t Copula quantile dependence.png}
%\includegraphics[width=\textwidth]{_pics/Gumbel Copula quantile dependence.png}
%\includegraphics[width=\textwidth]{_pics/Clayton Copula quantile dependence.png}
%  \caption{}
%\label{fig:quantile dependence1}
%\end{figure}


The MM estimation perform just as we decided: match the upper and lower quantile dependence.




%
%
%\subsection{Two-Stage Estimation}\label{subsec:two-stage-estimation}
%~\cite{joe2005asymptotic} study the efficiency of a two-stage estimation procedure of copula estimation.
%The authors also call this method inference function for margins IFM.
%
%\textbf{Pros}
%\begin{enumerate}
%    \item Almost as efficient as MLE methods but easier to be implemented
%    \item Yields an asymptotically Gaussian, unbiased estimate
%\end{enumerate}
%
%\textbf{Cons}
%\begin{enumerate}
%    \item Subject to specification of marginals \cite{kim2007comparison}
%\end{enumerate}
%
%Our data
%\begin{align}
%    \pmb{y} = \begin{bmatrix}
%                  y_{11} & \cdots & y_{1i}\\
%                  \vdots & \ddots & \vdots \\
%                  y_{n1} & \cdots & y_{ni}
%                  \end{bmatrix}
%    \end{align}
%Let $F$ and $f$ be the joint cdf and joint density of $\pmb{y}$ with parameters $\pmb{\delta}$,
%and let $F_i$ and $f_i$ be the marginal cdf and marginal density for the $i^\text{th}$ random variable with parameters $\pmb{\theta}_i$, we have
%\begin{align}
%    f(\pmb{y}; \pmb{\theta}_1, \pmb{\theta}_2,\dots \pmb{\theta}_i, \pmb{\delta}) =
%    c\{F_1(\pmb{y}_1;\pmb{\theta}_1), F_2(\pmb{y}_2; \pmb{\theta}_2), \dots, F_i(\pmb{y}_1;\pmb{\theta}_i); \pmb{\delta}\}
%    \prod^i_{j=1}f_i(\pmb{y}_j;\pmb{\theta}_j)
%    \end{align}
%
%For a sample of size $n$, the log-likelihood of functions of the $i^\text{th}$ univariate margin is
%\begin{align}
%    L_i(\theta_i) = \sum^n_{m=1} \log f_i(y_{mi}; \theta_i),
%    \end{align}
%
%and the log-likelihood function for the joint distribution is
%\begin{align}
%    L(\delta, \theta_1, \theta_2, \dots, \theta_i) = \sum^n_{m=1}\sum^i_{j=1} \log f(y_{mj}; \delta, \theta_1, \theta_2, ..., \theta_i)
%    \end{align}
%
%In most cases, one does not have closed form estimators and numerical techniques are needed.
%Numerical ML estimation difficulty increase when the total number of parameters increases.
%The two-stage estimation is designed to overcome this problem.
%
%The two-stage procedure is
%\begin{enumerate}
%    \item estimate the univariate parameters from separate univariate likelihoods to get $\tilde{\pmb{\theta}_1}, ..., \tilde{\pmb{\theta}_i}$
%    \item maximize $L(\pmb{\delta}, \tilde{\pmb{\theta}_1}, \dots, \tilde{\pmb{\theta}_i})$ over $\pmb{\delta}$ to get $\tilde{\pmb{\delta}}$
%    \end{enumerate}
%
%Under regularity conditions
%\footnote{Regularity conditions include
%1. $\exists \frac{\partial \log f(x;\theta)}{\partial \theta}, \frac{\partial^2 \log f(x;\theta)}{\partial \theta^2}, \frac{\partial^3 \log f(x;\theta)}{\partial \theta^3}$ for all $x$;
%2. $\exists g(x), h(x) and H(x)$ such that for $\theta$ in a neighborhood $N(\theta_0)$ the relations
%$\left|\frac{\partial f(x;\theta)}{\partial theta}\right| \leq g(x)$,
%$\left|\frac{\partial^2 f(x;\theta)}{\partial \theta^2}\right| \leq h(x)$,
%$\left|\frac{\partial^3 f(x;\theta)}{\partial \theta^3}\right| \leq H(x)$ hold for all $x$, and
%$\int g(x) dx < \infty$, $\int h(x) dx < \infty$, $\mathbb{E}_\theta \{H(X)\} < \infty$ for $\theta \in N(\theta_0)$;
%3. For each $\theta \in \Theta$, $0< \mathbb{E}_\theta \left\{
%\left(
%\frac{\partial \log f(X;\theta)}{\partial \theta}
%\right)^2
%\right\}$. For detail see section 4.2.2 of~\cite{serfling2009approximation}}
%, $(\pmb{\tilde{\theta}}_1,\dots \pmb{\tilde{\theta}}_i, \pmb{\tilde{\delta}})$ is the solution of
%\begin{align}
%    (\partial L_1 / \partial \pmb{\theta}^\intercal_1,
%    \dots, \partial L_i / \partial \pmb{\theta}^\intercal_i, \partial L / \partial \pmb{\pmb{\delta}}^\intercal_1) = \pmb{0}
%    \end{align}
%
%For comparison, if we optimize $L$ directly without the two-stage procedure (i.e.~MLE), we solve for
%\begin{align}
%    (\partial L / \partial \pmb{\theta}^\intercal_1,
%    \dots, \partial L / \partial \pmb{\theta}^\intercal_i, \partial L / \partial \pmb{\pmb{\delta}}^\intercal_1) = \pmb{0}
%    \end{align}
%
%We denote the two solutions as
%$\tilde{\pmb{\eta}} = (\pmb{\tilde{\theta}}_1,\dots \pmb{\tilde{\theta}}_i, \pmb{\tilde{\delta}})$ for two-stage procedure;
%$\hat{\pmb{\eta}} =(\pmb{\hat{\theta}}_1,\dots \pmb{\hat{\theta}}_i, \pmb{\hat{\delta}})$ for MLE procedure.
%and compare the asymptotic relative efficiency of $\tilde{\pmb{\eta}}$ and $\hat{\pmb{\eta}}$.
%
%Asymptotics: yet to be done.\\
%~\cite{kim2007comparison} show the estimation of $\pmb{\theta}$ may be seriously affected.
%They compare the two-stage approach and Canonical Maximum Likelihood Method by simulation and
%conclude that Canonical Maximum Likelihood is prefered from a computational statistics and data analysis point of view.
%
%\subsection{Canonical Maximum Likelihood Method}\label{subsec:canonical-maximum-likelihood-method}
%This approach was studied by~\cite{genest1995semiparametric} and~\cite{shih1995inferences}.
%One estimates the margins using empirical CDF
%\begin{align}F_X(x)=\frac{1}{n+1}\sum_{i=1}^n 1(X_i \leq x)\end{align},
%
%we maximize the log-likelihood
%\begin{align}
%    L(\delta) = \sum_{i=1}^n \log [c_\delta \{F_X(X_i), F_Y(Y_i)\}]
%    \end{align}
%
%This procedure does not require specification of marginals.
%
%
%
%
%
%%also by Wang and Ding, 2000; Tsukahara, 2005
%%This is also known as pseudo maximum likelihood (PML) and as canonical maximum likelihood (see Cherubini et al., 2004)
%%
%%Genest and Werker (2002) obtained conditions under which the PMLE is asymptotically efficient.
%%
%%

% ----------------
% --- Estimation of Copula ---
% ----------------

\subsubsection{Copula selection}\label{subsec:copula-selection}
As the dependence structure of price data changes
across time, we allow for a flexible choice of the best-fitting
copula, by re-calibrating periodically and re-evaluating performance
of the various copulas. 
In each re-calibration, we select the best-fitting
copula, characterised by the lowest {\em Akaike Information Criterion
  (AIC)},
\begin{equation*}
 \text{AIC} = 2k- 2 \log(L),
\end{equation*}
where $k$ is the number of estimated
parameteres and $L$ is the likelihood \citep{Akaike1973}. 

% they tend to suggest the same copula as the best fitting one.
%Simulation studies has also been carried out to compare different copula selection methods, see \cite{}.
Other model selection criteria, such as the TIC~\citep{takeuchi1976distribution} or likelihood ratio test could be used instead.
For a survey of model selection and inference, see \cite{anderson1998comparison}.
Among various copula selection procedures, AIC is a popular choice for
its applicability, see e.g. \cite{breymann2003dependence}.
In our case, the AICs are calculated only with dependence likelihood
since the marginals are modelled via kernel density estimators.
The selected copula will then enter the calculation of the optimal
hedge ratio.
% We consider the copula with the lowest AIC for a particular set of data the best fitting one and use it to generate OHR.

\subsection{Risk measures}\label{subsec:spectral-risk-measures}
The optimal hedge ratio is determined for the following variety of risk measures: variance, Value-at-Risk (VaR), Expected Shortfall (ES), and Exponential Risk Measure (ERM).
A summary of risk measures being used in portfolio selection problem
can be found in \citet{hardle2008applied}. 
The risk measures here serve as risk minimization objectives, i.e. loss functions for searching the optimal hedge ratio. 
%They are used in many literature about hedging, e.g. ;
%The risk measures are also used by regulatory bodies,
%for example Basel III ....

The risk measures are defined as follows.
Let $Z$ be a random
variable with distribution function $F_Z$.
\begin{enumerate}
\item Variance: $\text{Var}(Z) = \E[(Z-\E Z)^2]$. 
\item VaR at confidence level $\alpha$: $\text{VaR}_\alpha(Z) = -F_{Z}^{(-1)}(1-\alpha)$
\item ES at confidence level $\alpha$: $\text{ES}(F_Z) = -\frac{1}{1-\alpha}\int_0^{1-\alpha}F_Z^{(-1)}(p)dp$
\item ERM with Arrow-Pratt coefficient of absolute risk
  aversion $k$:
  \begin{equation*}
    \text{ERM}_k(F_Z) = \int_0^{1-\alpha}\phi(p) F_Z^{(-1)}(p)dp,
  \end{equation*}
  where $\phi$ is a weight function described in (\ref{eq:phi}) below.
\end{enumerate}

VaR, ES, and ERM fall into the class of spectral risk measures (SRM),
which have the form \citep{Acerbi2002}%, adam2008spectral,dowd2008spectral}
\begin{equation*}
  \rho_\phi(r^h) = - \int_0^1 \phi(p) F_{Z}^{(-1)}(p)d p,
\end{equation*}
where $p$ is the loss quantile and $\phi(p)$ is a user-defined
weighting function defined on $[0,1]$.
We consider only so-called admissible risk spectra $\phi(p)$, i.e.,
fulfilling %(named by \citet{Acerbi2002})
\begin{enumerate}[label=(\roman*)]
\item $\phi$ is positive,
\item $\phi$ is decreasing,
\item and $\int\phi=1$. 
\end{enumerate}

The VaR's $\phi(p)$ gives all its weight on the $1-\alpha$ quantile of
$Z$ and zero elsewhere, i.e., the weighting function is a Dirac delta
function, and hence it violates the (ii) property of admissible risk
spectra.  
The ES' $\phi(p)$ gives all tail quantiles the same weight of
$\displaystyle\frac{1}{1-\alpha}$ and non-tail quantiles zero weight. 
The ERM assumes investors' risk preference are in the form of an
exponential utility function $U(x)=1-e^{kx}$, so its corresponding
risk spectrum is defined as
\begin{equation*}
  \phi(p) =\frac{k e^{-k(1-p)}}{1-e^{-k}} , \label{eq:phi}
\end{equation*}
where $k$ is the Arrow-Pratt coefficient of absolute risk aversion. 
The parameter $k$ has an economic interpretation as being the ratio
between the second derivative and first derivative 
of investor's utility function on an risky asset,
\begin{equation*}
  k = -\frac{U''(x)}{U'(x)},
\end{equation*}
for $x$ in all possible outcomes.
In case of the exponential utility, $k$ is the the constant absolute risk aversion (CARA).


% ----------------
% --- Describe the methodology of finding the optimal h ---
% ----------------

\subsection{Risk Measures}\label{subsec:spectral-risk-measures}
We consider a variety of risk measures: variance, Value-at-Risk (VaR), Expected Shortfall (ES), and Exponential Risk Measure (ERM).
%They are used in many literature about hedging, e.g. ;
%The risk measures are also used by regulatory bodies,
%for example Basel III ....
A summary of risk measures being used in portfolio selection problem can be found in \citet{hardle2008applied}.\medskip
\medskip


Let $Z$ be a random variable of distribution $F_Z$.
\begin{enumerate}
	\item Variance is $\text{Var}(F_Z)$
	\item VaR of a given confidence level $\alpha$ is $\text{VaR}(F_Z) = -F_{Z}^{(-1)}(1-\alpha)$
	\item ES with parameter $\alpha$ is $\text{ES}(F_Z) = -\frac{1}{1-\alpha}\int_0^{1-\alpha}F_Z^{(-1)}(p)dp$
	\item ERM with Arrow-Pratt coeficient of absolute risk aversion $k$ is $\text{ERM}_k(F_Z) = \int_0^{1-\alpha}\phi(p) F_Z^{(-1)}(p)dp$ where $\phi$ is a weight function described in (\ref{eq:phi}) below.
	\end{enumerate}\medskip

VaR, ES, and ERM fall into the class of Spectral Risk Measure (SRM).
SRM has the from \citep{Acerbi2002}%, adam2008spectral,dowd2008spectral}
\begin{align}
	\rho_\phi(r^h) = - \int_0^1 \phi(p) F_{Z}^{(-1)}(p)d p,
	\end{align}

where $p$ is the loss quantile and $\phi(p)$ is a user-defined weighting function defined over $[0,1]$. \medskip
We consider only admissible risk spectra $\phi(p)$ %(named by \citet{Acerbi2002})
\begin{enumerate}[label=\roman*]
	\item $\phi$ is positive
	\item $\phi$ is decreasing
	\item integrates to one.
	\end{enumerate}\medskip

The VaR's $\phi(p)$ gives all its weight on the $1-\alpha$ quantile of $Z$ and zero elsewhere,
i.e. the weighting function is a Dirac delta function, hence violates the ii property of admissible risk spectra.
The ES' $\phi(p)$ gives all tail quantiles the same weight of $\frac{1}{1-\alpha}$ and non-tail quantiles zero weight.
ERM assumes investor's risk preference is in a form of exponential utility function $U(x)=1-e^{kx}$,
its corresponding risk spectrum is defined as

\begin{align}
	\phi(p) =\frac{k e^{-k(1-p)}}{1-e^{-k}} , \label{eq:phi}
	\end{align}
where $k$ is the Arrow-Pratt coefficient of absolute risk aversion. \medskip

The parameter $k$ has an economic interpretation of being the ratio between the second derivative and first derivative
of investor's utility function on an risky asset

\begin{align}
	k = -\frac{U''(x)}{U'(x)},
	\end{align}
for $x$ in all possible outcomes.
In case of the exponential utility, $k$ is the the constant absolute risk aversion (CARA).



%Spectral risk measures can also be written as
%\begin{align}
%	\rho_\phi(r^h) = - \int_\mathbb{R} x f_{r^h}(x) \phi\{F_{r^h}(x)\} d x.
%	\end{align}
% ----------------
% --- Risk measures' definition and properties ---
% ----------------


\subsection{Copulae}
We test different copulae's ability to model crypto-currency data,
they include Gaussian-, $t$-, Frank-, Gumbel-, Clayton-, Plackett-, mixture, and factor copula.

%Copula is a function represents the multivariate structure of random variables.
%
%
%Frechet-Hoeffding lower bound $\bm{W}(u,v) = \min(u,v)$, Frechet-Hoeffding upper bound $\bm{M}(u,v) = \max(u+v-1,0)$,
%and product copula $\bm{\Pi}=uv$ are three important special instants of copulas.
%They describe the perfect counter-dependence, perfect dependence, and independence of two random variables, respectively.
%The inequality $\bm{W}(u,v) \leq \bm{C}(u,v) \leq \bm{M}(u,v)$ holds for every copula $\bm{C}$ ad every $(u,v) \in \mathbb{I}^2$ (Nelsen 2.2.5).

\subsubsection{Ellpitical Copulae}\label{sec:ellpitical-copulas}
Elliptical copulae are copulae of elliptical distributions.
Gaussian copula is the copula associated with multivariate normal distribution.
The Gaussian copula has a form
    \begin{align}
        \bm{C}(u,v) &= \Phi_{2, \rho}\{\Phi^{-1}(u), \Phi^{-1}(v)\} \nonumber \\
                    &= \int_{-\infty}^{\Phi^{-1}(u)}
                       \int_{-\infty}^{\Phi^{-1}(v)}
                       \frac{1}{2\pi\sqrt{1-\rho^2}}
                       \exp{\left(
                       \frac{s^2-2\rho st+t^2}{2(1-\rho^2)}
                       \right)} ds dt
        \end{align}
where $\Phi_{2, \rho}$ is the cdf of bivariate Normal distribution with zero mean, unit variance, and correlation $\rho$,
and $\Phi^{-1}$ is quantile function univariate standard normal distribution.
The Gaussian copula density is
\begin{align}
    \bm{c}_\rho(u,v) &= \frac{\varphi_{2,\rho}\{\Phi^{-1}(u), \Phi^{-1}(v)\}}
                        {\varphi\{\Phi^{-1}(u)\} \cdot \varphi\{\Phi^{-1}(v)\}} \nonumber \\
                &= \frac{1}{2\pi\sqrt{1-\rho^2}}\exp\left(
                   -\frac{u^2 - 2\rho uv + v^2}{2(1-\rho^2)}
                   \right),
    \end{align}
where $\phi_{2,\rho}(\cdot)$ is the density of bivariate Normal distribution with zero mean,
unit variance,
and correlation $\rho$,
and, $\phi(\cdot)$ the density of standard normal distribution.\medskip

The Kendall's $\tau_K$ and Spearman's $\rho_S$ of a bivariate Gaussian Copula are
    \begin{align}
        \tau_K(\rho) = \frac{2}{\pi}\arcsin\rho
        \end{align}
    \begin{align}
        \rho_S(\rho) = \frac{6}{\pi}\arcsin\frac{\rho}{2}
        \end{align}\medskip

The $t$-copula is associated with multivariate t distribution.
The $t$-Copula takes a form
\begin{align}
        \bm{C}(u,v) &= \bm{T}_{2, \rho, \nu}\{T^{-1}_\nu(u), T^{-1}_\nu(v)\} \nonumber \\
            &= \int_{-\infty}^{T^{-1}_\nu(u)}
               \int_{-\infty}^{T^{-1}_\nu(v)}
            \frac{\Gamma\left(\frac{\nu+2}{2}\right)}
            {\Gamma\left(\frac{\nu}{2}\right)\pi\nu\sqrt{1-\rho^2}}\\
           & \left(
        1+\frac{s^2-2st\rho+t^2}{\nu}
        \right)^{-\frac{\nu+2}{2}} ds dt,
    \end{align}
where $\bm{T}_{2, \rho, \nu}(\cdot, \cdot)$ denotes the cdf of bivariate t distribution with scale parameter $\rho$ and degree of free $\nu$,
$T^{-1}_\nu(\cdot)$ is the quantile function of a standard t distribution with degree of freedom $\rho$.

The copula density is
\begin{align}
    \bm{c}(u,v) &= \frac{\bm{t}_{2, \rho, \nu}\{T^{-1}_\nu(u), T^{-1}_\nu(v)\}}
    {t_\nu\{T^{-1}_\nu(u)\}\cdot t_\nu\{T^{-1}_\nu(v)\}},
    \end{align}
where $\bm{t}_{2,\rho, \nu}$ is the density of bivariate t distribution,
and $t_\nu$ the density of standard t distribution.\medskip

Like all the other elliptical copula, t copula's Kendall's $\tau$ is same to that of Gaussian copula (Demarta and reference therein).

\subsubsection{Archimedean Copulae}\label{sec:archimedean-copula}
The Archimedean copulae forms a large class of copulas with many convenient features.\medskip

In general, they take a form
\begin{align}
    \bm{C}(u,v)= \psi^{-1}\{\psi(u), \psi(v)\},
    \end{align}
where $\psi:[0,1] \rightarrow [0,\infty)$ is a continuous, strictly decreasing and convex function such that
$\psi(1)=0$ for any permissible dependence parameter $\theta$. $\psi$ is also called generator.
$\psi^{-1}$ is the inverse the generator.\medskip

The Frank copula (B3 in \citet{joe1997multivariate}) is a radial symmetric copula and cannot produce any tail dependence.
It takes the form
\begin{align}
    \bm{C}_{\theta}(u,v) &= \frac{1}{\log(\theta)}
    \log \left\{
    1 + \frac{(\theta^u-1)(\theta^v-1)}{\theta-1}
    \right\}
    \end{align}
where $\theta \in [0, \infty]$ is the dependency parameter.
$\bm{C}_1 = \bm{M}$, $\bm{C}_1 = \bm{\Pi}$, and $\bm{C}_\infty = \bm{W}$.

The Copula density is
\begin{align}
    \bm{c}_{\theta}(u,v) &= \frac{(\theta-1)\theta^{u+v}\log(\theta)}
    {\theta^{u+v}-\theta^u-\theta^v+\theta}
    \end{align}\medskip

Frank copula has Kendall's $\tau$ and Spearman's $\rho$ as follow:
\begin{align}
    \tau_K(\theta) = 1-4\frac{D_1\{-\log(\theta)\}}{\log(\theta)},
    \end{align}
and
\begin{align}
    \rho_S(\theta) = 1-12\frac{D_2\{-\log(\theta)\} - D_1\{\log(\theta)\}}{\log(\theta)},
    \end{align}
where $D_1$ and $D_2$ are the Debye function of order 1 and 2.
Debye function is $D_n = \frac{n}{x^n}\int_0^x\frac{t^n}{e^t-1}dt$.\medskip

Gumbel copula (B6 in \citet{joe1997multivariate}) has upper tail dependence with the dependence parameter
$\lambda^U = 2-2^{\frac{1}{\theta}}$ and displays no lower tail dependence.
\begin{align}
    \bm{C}_{\theta}(u,v) &= \exp{-\{
    (-\log(u))^\theta +(-\log(v))^\theta
    \}^{\frac{1}{\theta}}},
    \end{align}
where $\theta \in [1,\infty)$ is the dependence parameter.\medskip
While Gumbel copula cannot model perfect counter dependence (ref), $\bm{C}_{1} = \bm{\Pi}$ models the independence,
and $\lim\limits_{\theta \to \infty} \bm{C}_\theta = \bm{W}$ models the perfect dependence.

%The copula density takes the form
%\begin{align}
%        f
%    \end{align}

  \begin{align}
    \tau_K(\theta) =\frac{\theta-1}{\theta}
    \end{align}

The Clayton copula, by contrast to Gumbel copula,
generates lower tail dependence in a form $\lambda^L = 2^{-\frac{1}{\theta}}$,
but cannot generate upper tail dependence.\medskip

The Clayton copula takes a form
\begin{align}
    \bm{C}_{\theta}(u,v) &= \left[
    \max\{u^{-\theta}+v^{-\theta}-1,0\}\right]^{-\frac{1}{\theta}},
    \end{align}
where $\theta \in (-\infty, \infty)$ is the dependency parameter.
$\lim\limits_{\theta \to -\infty} \bm{C}_\theta = \bm{M}$, $\bm{C}_0 = \bm{\Pi}$, and $\lim\limits_{\theta \to \infty} \bm{C}_\theta = \bm{W}$.\medskip

Its Kendall's $\tau$ is
\begin{align}
    \tau_K(\theta) =\frac{\theta}{\theta+2}.
    \end{align}\medskip

    \natp{The Plackett copula is not an Archimedean copula, so it
      should be moved somwhere else.}
    
The Plackett copula has an expression
\begin{align}
    \bm{C}_{\theta}(u,v) &= \frac{1+(\theta-1)(u+v)}{2(\theta-1)}
                         - \frac{\sqrt{\{
    1+(\theta-1)(u+v)\}^2 - 4uv\theta(\theta-1)}}{2(\theta-1)}
    \end{align}
\begin{align}
    \rho_S(\theta) = \frac{\theta+1}{\theta-1} - \frac{2\theta \log \theta}{(\theta-1)^2}
    \end{align}\medskip

We include Placket copula in our analysis as it possesses a special property,
the cross-product ratio is equal to the dependence parameter
\begin{align}
    &\phantom{=} \frac{\p(U \leq u, V \leq v) \cdot \p(U > u, V > v)}
    {\p(U \leq u, V > v) \cdot \p(U > u, V \leq v)}\nonumber\\
    &= \frac{\bm{C}_\theta(u,v)\{1-u-v+\bm{C}_\theta(u,v)\}}{\{u-\bm{C}_\theta(u,v)\}\{v-\bm{C}_\theta(u,v)}\nonumber\\
    &= \theta.
    \end{align}\medskip
That is, the dependence parameter is equal to the ratio between number of concordence pairs and number of discordence pairs of a bivariate random variable.
\subsubsection{Mixture Copula}\label{sec:mixture-copula}
Mixture copula is a linear combination of copulas.
It allows us to model the dependence structure in a more flexible manner.

For a 2-dimensional random variable $\bm{X}=(X_1,X_2)^\top$,
its distribution can be written as linear combination $K$ copulas
\begin{align}
    \p(X_1 \leq x_1, X_2 \leq x_2) = \sum_{k=1}^K p^k \cdot \bm{C}^{(k)}\{F^{(k)}_{X_1}(x_1;\bm{\gamma}^{(k)}_1),
    F^{(k)}_{X_2}(x_2;\bm{\gamma}^{(k)}_2); \bm{\theta^{(k)}}\}
    \end{align}
where $p^{(k)} \in [0,1]$ is the weight of each component,
$\bm{\gamma}^{(k)}$ is the parameter of the marginal distribution in the $k^\text{th}$ component,
and $\bm{\theta^{(k)}}$ is the dependence parameter of the $k^\text{th}$ component.
We also restrict the weight so that $\sum_{k=1}^K p^{(k)}=1$.
Analysis of mixture copula with higher dimension can be found in Vrac et. al. (2011).

We deploy a simplified version of the above representation by assuming the maringals of $\bm{X}$ are not mixture.
By Sklar's theorem we write
\begin{align}
    \bm{C}(u,v)= \sum_{k=1}^K p^{(k)} \cdot \bm{C}^{(k)}\{F^{-1}_{X_1}(u),
    F^{-1}_{X_2}(v); \bm{\theta^{(k)}}\}.
    \end{align}\medskip

The copula density is again a linear combination of copula density
\begin{align}
    \bm{c}(u,v)= \sum_{k=1}^K p^{(k)} \cdot \bm{c}^{(k)}\{F^{-1}_{X_1}(u),
    F^{-1}_{X_2}(v); \bm{\theta^{(k)}}\}.
    \end{align}\medskip

While Kendall's $\tau$ of mixture copula is not known in close form,
the Spearman's $\rho$ is

\begin{proposition}
    Let $\rho_S^{(k)}$ be the Spearman's $\rho$ of the $k^\text{th}$ component and $\sum_{k=1}^K p^{(k)}=1$ holds,
    the Spearman's $\rho$ of a mixture copula is
    \begin{align}
        \rho_S = \sum_{k=1}^K p^{(k)} \cdot \rho_S^{(k)}
        \end{align}
    \end{proposition}

\begin{proof}
    Spearman's $\rho$ is defined as (Nelsen)
    \begin{align}
        \rho_S = 12 \int_{\mathbb{I}^2} \bm{C}(s,t) ds dt - 3.
        \end{align}
    Rewrite the mixture copula into sumation of components
       \begin{align}
        \rho_S = 12 \int_{\mathbb{I}^2} \sum_{k=1}^K p^{(k)} \cdot \bm{C}^{(k)}(s,t) ds dt - 3.
        \end{align}
    \end{proof}

\begin{example}
    Frechet class can be seen as an example of mixture copula.
    It is a convex combinations of $\bm{W}$, $\bm{\Pi}$, and $\bm{M}$ \citep{Nelsen2007}
    \begin{align}
        \bm{C}_{\alpha, \beta}(u,v)
        = \alpha \bm{M}(u,v) +
        (1-\alpha-\beta)\bm{\Pi}(u,v)
        +\beta \bm{W}(u,v),
        \end{align}
    where $\alpha$ and $\beta$ are the dependence parameters, with $\alpha, \beta \geq 0$ and
    $\alpha+\beta \leq 1$.
    Its Kendall's $\tau$ and Spearman's $\rho$ are
    \begin{align}
        \tau_K(\alpha, \beta) = \frac{(\alpha - \beta)(\alpha+\beta+2)}{3}
        \end{align}
    , and
    \begin{align}
        \rho_S(\alpha, \beta) = \alpha - \beta
        \end{align}
    \end{example}\medskip
%Example 2 Gumbel-Clayton mixture
%Example 3 Hu 2006.

We use a mixture of Gaussian and independent copula in our analysis.
We write the copula
\begin{align}
    \bm{C}(u,v) = p\cdot \bm{C}^\text{Gaussian}(u,v) + (1-p)(uv).
    \end{align}
The corresponding copula density is
\begin{align}
    \bm{c}(u,v) = p\cdot \bm{c}^\text{Gaussian}(u,v) + (1-p).
    \end{align}

This mixture allows us to model how much "random noise" appear in the dependency structure.
In this hedging exercise, the structure of the "random noise" is not of our concern nor we can
hedge the noise by a two-asset portfolio.
However, the proportion of the "random noise" does affect the distribution of $R^h$ (see figure),
so as the optimal hedging ratio $h^\ast$ (see figure).
One can consider the mixture copula as a handful tool for stress testing.
Similar to this Gaussian mix Independent copula,
t copula is also a two parameter copula allow us to model the noise,
but its interpretation of parameters is not as intuitive as that of a mixture.
The mixing variable $p$ is the proportion of a manageable (hedgable?) Gaussian copula,
while the remaining proportion $1-p$ cannot be managed.

% ----------------
% --- Copulae's definition and properties ---
% ----------------

%! Author = francis
%! Date = 30.10.20


\subsection{Simulated Method of Moments}\label{subsec:simulated-method-of-moments}
This method is suggested by Oh and Patton (2013).
In this setting, rank correlation e.g. Spearman's $\rho$ or Kendall's $\tau$,
and quantile dependence measures at different levels $\lambda_q$
are calibrated against their empirical counterparts.\medskip

Spearman's rho, Kendall's tau, and quantile dependence of a pair $(X,Y)$
with copula $C$ are defined as
\begin{align}
  \rho_S &= 12 \int\int_{I^2} C_{\bm{\theta}}(u,v)\, \dd u\, \dd v-3\label{eq:rho_S}\\
  \tau_K &= 4\mathbb{E}[C_{\bm{\theta}}\{F_X(x), F_Y(y)\}]-1,\\
  \lambda_q &=
  \begin{cases}
    \p(F_X(X)\leq q| F_Y(Y)\leq q) = \displaystyle \frac{C_{\bm{\theta}}(q,q)}{q},
    &\text{ if } q\in (0,0.5],\\
    \p(F_X(X)>q|F_Y(Y)>q) =\displaystyle \frac{1-2q+C_{\bm{\theta}}(q,q)} {1-q},
    &\text{ if } q\in (0.5,1).
  \end{cases}
\end{align}\medskip
The empirical counterparts are
\begin{align*}
  \hat\rho_S &= \frac{12}{n} \sum_{k=1}^n \hat F_X(x_k) \hat F_Y(y_k)
               -3,\\
  \hat\tau_K &= \frac{4}{n}\sum_{k=1}^n \hat{C}\{\hat{F}_X(x_i),\hat{F}_X(y_i)\} -1 ,\\
  \hat\lambda_q &=
                  \begin{cases}
                    \displaystyle\frac{1}{n} \sum_{k=1}^n \frac{\1_{\{\hat
                        F_X(x_k)\leq q, \hat F_Y(y_k)\leq q\}}} {q},
                    &\text { if } q\in (0, 0.5],\\
                    \displaystyle \frac{1}{n} \sum_{k=1}^n
                    \frac{\1_{\{\hat F_X(x_k)>q, \hat F_Y(y_k)>q\}}}
                    {1-q}, &\text { if } q\in (0.5,1).
                  \end{cases},
\end{align*}
where $\hat{F}(x) := \frac{1}{n}\sum_{k=1}^n 1_{\{x_i\leq x\}}$ and
$\hat{C}(u,v) := \frac{1}{n}\sum_{k=1}^n 1_{\{u_i\leq u, v_i\leq v\}}$.\medskip

We denote $\tilde{\bm{m}}(\bm{\theta})$ be a $m$-dimensional vector of dependence measures according the the
dependence parameters $\bm{\theta}$,and  $\hat{\bm{m}}$ be the corresponding empirical counterpart.
The difference between dependence measures and their counterpart is denoted by
\begin{align*}
    \bm{g}(\bm{\theta}) = \hat{\bm{m}} - \tilde{\bm{m}}(\bm{\theta}).
\end{align*}\medskip

The SMM estimator is
\begin{align*}
    \hat{\bm{\theta}} = \argmin_{\bm{\theta}\in \bm{\Theta}} \bm{g}(\bm{\theta})^\intercal
    \hat{\bm{W}}
     \bm{g}(\bm{\theta}),
\end{align*}
where $\hat{W}$ is some positive definite weigh matrix.\medskip

In this work, we use $\tilde{\bm{m}}(\bm{\theta}) = (\rho_S, \lambda_{0.05}, \lambda_{0.1},
\lambda_{0.9}, \lambda_{0.95})^\intercal$
for calibration of Bitcoin price and CME Bitcoin future.

\subsection{Maximum Likelihood Estimation}\label{subsec:maximum-likelihood-estimation}
By Sklar's theorem, the joint density of a $d$-dimensional random variable $\bm{X}$ with sample size $n$ can be written as
\begin{align}
    \bm{f}_{\bm{X}}(x_1, ..., x_d) = \bm{c}\{F_{X_1}(x_1), ..., F_{X_d}(x_d)\} \prod_{j=1}^d f_{X_i}(x_i).
    \end{align}
We follow the treatment of MLE documented in section 10.1 of \citet{joe1997multivariate}, namely the inference functions for margins or IFM method.
The log-likelihood $\sum^n_{i=1}f_{\bm{X}}(X_{i,1}, ..., X_{i,d})$ can be decomposed into dependence part and marginal part,
\begin{align}
    L(\bm{\theta}) &= \sum_{i=1}^n \bm{c}\{F_{X_1}(x_{i,1};\bm{\delta}_1), ..., F_{X_d}(x_{i,d}; \bm{\delta}_d);\bm{\gamma}\}
    + \sum_{i=1}^n \sum_{j=1}^d f_{X_j}(x_{i,j};\bm{\delta}_j)
    &= L_C(\bm{\delta}_1, ..., \bm{\delta}_d, \bm{\gamma}) + \sum_{j=1}^d L_j(\bm{\delta}_j)
    \end{align}
where $\bm{\delta}_j$ is the parameter of the $j$-th margin, $\bm{\gamma}$ is the parameter of the parametric copula, and
$\bm{\theta} = (\bm{\delta}_1,..., \bm{\delta}_d, \bm{\gamma})$.

Instead of searching the $\bm{\theta}$ is a high dimensional space, \citet{joe1997multivariate} suggests to
search for $\hat{\bm{\delta}_1},..., \hat{\bm{\delta}_d}$ that maximize $L_1(\bm{\delta}_1), ..., L_d(\bm{\delta}_d)$,
then search for $\hat{\bm{\gamma}}$ that maximize $L_C(\hat{\bm{\delta}_1},..., \hat{\bm{\delta}_d}, \bm{\gamma})$.

That is, under regularity conditions, $(\hat{\bm{\delta}_1},..., \hat{\bm{\delta}_d}, \hat{\bm{\gamma}})$ is the solution of
\begin{align}
    \left( \frac{\partial L_1}{\partial \bm{\delta}_1}, ..., \frac{\partial L_d}{\partial \bm{\delta}_d},
    \frac{\partial L_C}{\partial \bm{\gamma}}\right) = \bm{0}.
    \end{align}

However, the IFM requires making assumption to the distribution of of the margins.
\citet{genest1995semiparametric} suggests to replace the estimation of marginals parameters estimation by non-parameteric estimation.
Given non-parametric estimator $\hat{F}_i$ of the margins $F_i$, the estimator of the dependence parameters $\bm{\gamma}$ is
\begin{align}
    \hat{\bm{\gamma}} = \argmax_{\bm{\gamma}} \sum_{i=1}^n \bm{c}\{ \hat{F}_{X_1}(x_{i,1}), ..., \hat{F}_{X_d}(x_{i,d});\bm{\gamma}\}.
    \end{align}



%With the decomposition, the MLE estimator for a bivariate parametric copula is
%\begin{align}
%    \hat{\bm{\theta}} = \argmax_{\bm{\theta} \in \bm{\Theta}} l(X_1,X_2; \bm{\theta}), \label{eq:EMLE}
%    \end{align}
%where
%\begin{align}
%    l(X_1,X_2; \bm{\theta}) = \sum_{i=1}^n \log c(x_{i,1}, x_{i,2};\bm{\theta}). \label{eq:Likelihood}
%    \end{align}\medskip

%Procedure of maximising equation~\ref{eq:EMLE} as a whole is called exact maximum likelihood method.
%Leveraging the attractive feature of copula that one can model the dependence structure and marginals separately,
%we rewrite~\ref{eq:Likelihood} into canonical expression
%\begin{align}
%    l(X,Y; \bm{\theta}) = \sum_{k=1}^n \log c\{F_X(x_i; \delta_X), F_Y(y_i; \delta_Y); \bm{\gamma}\}
%    + \sum_{k=1}^n \log f_X(x_i; \bm{\delta}_X) + \sum_{k=1}^n \log f_X(y_i; \bm{\delta}_Y),
%    \end{align}
%where the $\bm{\gamma}$ is the dependence parameter in the copula and $\bm{\delta}$ is the parameters in the margins.\medskip
%
%The inference-functions for margins (IFM) approach by Joe is a two step procedure of maximising~\ref{eq:EMLE}.
%The approach calibrate first the $\bm{\delta}$s and then the  $\bm{\gamma}$.\medskip
%
%Similar to the IFM approach, pseudo-maximum likelihood approach by Genest and Rivest (1993) replace the parametric margins by
%empirical estimates, we rewrite \ref{eq:Likelihood} again with
%\begin{align}
%    l(X,Y; \bm{\theta}) = \sum_{k=1}^n \log c(u_i, v_i;\bm{\gamma}),
%    \end{align}
%where $u_i = \hat{F}_X(x_i)$ and $v_i = \hat{F}_Y(y_i)$.

\subsection{Comparison}
Both the simulated method of moments and the maximum likelihood estimation are unbiased and
proven to give good fits.
The problem remain is which procedure is more suitable for hedging.
%Cryptocurrencies are known to be very volatile.
Sample and fitted quantile dependence for Bitcoin and CME future.

%\begin{figure}[th]
%\includegraphics[width=\textwidth]{_pics/t Copula quantile dependence.png}
%\includegraphics[width=\textwidth]{_pics/Gumbel Copula quantile dependence.png}
%\includegraphics[width=\textwidth]{_pics/Clayton Copula quantile dependence.png}
%  \caption{}
%\label{fig:quantile dependence1}
%\end{figure}


The MM estimation perform just as we decided: match the upper and lower quantile dependence.




%
%
%\subsection{Two-Stage Estimation}\label{subsec:two-stage-estimation}
%~\cite{joe2005asymptotic} study the efficiency of a two-stage estimation procedure of copula estimation.
%The authors also call this method inference function for margins IFM.
%
%\textbf{Pros}
%\begin{enumerate}
%    \item Almost as efficient as MLE methods but easier to be implemented
%    \item Yields an asymptotically Gaussian, unbiased estimate
%\end{enumerate}
%
%\textbf{Cons}
%\begin{enumerate}
%    \item Subject to specification of marginals \cite{kim2007comparison}
%\end{enumerate}
%
%Our data
%\begin{align}
%    \pmb{y} = \begin{bmatrix}
%                  y_{11} & \cdots & y_{1i}\\
%                  \vdots & \ddots & \vdots \\
%                  y_{n1} & \cdots & y_{ni}
%                  \end{bmatrix}
%    \end{align}
%Let $F$ and $f$ be the joint cdf and joint density of $\pmb{y}$ with parameters $\pmb{\delta}$,
%and let $F_i$ and $f_i$ be the marginal cdf and marginal density for the $i^\text{th}$ random variable with parameters $\pmb{\theta}_i$, we have
%\begin{align}
%    f(\pmb{y}; \pmb{\theta}_1, \pmb{\theta}_2,\dots \pmb{\theta}_i, \pmb{\delta}) =
%    c\{F_1(\pmb{y}_1;\pmb{\theta}_1), F_2(\pmb{y}_2; \pmb{\theta}_2), \dots, F_i(\pmb{y}_1;\pmb{\theta}_i); \pmb{\delta}\}
%    \prod^i_{j=1}f_i(\pmb{y}_j;\pmb{\theta}_j)
%    \end{align}
%
%For a sample of size $n$, the log-likelihood of functions of the $i^\text{th}$ univariate margin is
%\begin{align}
%    L_i(\theta_i) = \sum^n_{m=1} \log f_i(y_{mi}; \theta_i),
%    \end{align}
%
%and the log-likelihood function for the joint distribution is
%\begin{align}
%    L(\delta, \theta_1, \theta_2, \dots, \theta_i) = \sum^n_{m=1}\sum^i_{j=1} \log f(y_{mj}; \delta, \theta_1, \theta_2, ..., \theta_i)
%    \end{align}
%
%In most cases, one does not have closed form estimators and numerical techniques are needed.
%Numerical ML estimation difficulty increase when the total number of parameters increases.
%The two-stage estimation is designed to overcome this problem.
%
%The two-stage procedure is
%\begin{enumerate}
%    \item estimate the univariate parameters from separate univariate likelihoods to get $\tilde{\pmb{\theta}_1}, ..., \tilde{\pmb{\theta}_i}$
%    \item maximize $L(\pmb{\delta}, \tilde{\pmb{\theta}_1}, \dots, \tilde{\pmb{\theta}_i})$ over $\pmb{\delta}$ to get $\tilde{\pmb{\delta}}$
%    \end{enumerate}
%
%Under regularity conditions
%\footnote{Regularity conditions include
%1. $\exists \frac{\partial \log f(x;\theta)}{\partial \theta}, \frac{\partial^2 \log f(x;\theta)}{\partial \theta^2}, \frac{\partial^3 \log f(x;\theta)}{\partial \theta^3}$ for all $x$;
%2. $\exists g(x), h(x) and H(x)$ such that for $\theta$ in a neighborhood $N(\theta_0)$ the relations
%$\left|\frac{\partial f(x;\theta)}{\partial theta}\right| \leq g(x)$,
%$\left|\frac{\partial^2 f(x;\theta)}{\partial \theta^2}\right| \leq h(x)$,
%$\left|\frac{\partial^3 f(x;\theta)}{\partial \theta^3}\right| \leq H(x)$ hold for all $x$, and
%$\int g(x) dx < \infty$, $\int h(x) dx < \infty$, $\mathbb{E}_\theta \{H(X)\} < \infty$ for $\theta \in N(\theta_0)$;
%3. For each $\theta \in \Theta$, $0< \mathbb{E}_\theta \left\{
%\left(
%\frac{\partial \log f(X;\theta)}{\partial \theta}
%\right)^2
%\right\}$. For detail see section 4.2.2 of~\cite{serfling2009approximation}}
%, $(\pmb{\tilde{\theta}}_1,\dots \pmb{\tilde{\theta}}_i, \pmb{\tilde{\delta}})$ is the solution of
%\begin{align}
%    (\partial L_1 / \partial \pmb{\theta}^\intercal_1,
%    \dots, \partial L_i / \partial \pmb{\theta}^\intercal_i, \partial L / \partial \pmb{\pmb{\delta}}^\intercal_1) = \pmb{0}
%    \end{align}
%
%For comparison, if we optimize $L$ directly without the two-stage procedure (i.e.~MLE), we solve for
%\begin{align}
%    (\partial L / \partial \pmb{\theta}^\intercal_1,
%    \dots, \partial L / \partial \pmb{\theta}^\intercal_i, \partial L / \partial \pmb{\pmb{\delta}}^\intercal_1) = \pmb{0}
%    \end{align}
%
%We denote the two solutions as
%$\tilde{\pmb{\eta}} = (\pmb{\tilde{\theta}}_1,\dots \pmb{\tilde{\theta}}_i, \pmb{\tilde{\delta}})$ for two-stage procedure;
%$\hat{\pmb{\eta}} =(\pmb{\hat{\theta}}_1,\dots \pmb{\hat{\theta}}_i, \pmb{\hat{\delta}})$ for MLE procedure.
%and compare the asymptotic relative efficiency of $\tilde{\pmb{\eta}}$ and $\hat{\pmb{\eta}}$.
%
%Asymptotics: yet to be done.\\
%~\cite{kim2007comparison} show the estimation of $\pmb{\theta}$ may be seriously affected.
%They compare the two-stage approach and Canonical Maximum Likelihood Method by simulation and
%conclude that Canonical Maximum Likelihood is prefered from a computational statistics and data analysis point of view.
%
%\subsection{Canonical Maximum Likelihood Method}\label{subsec:canonical-maximum-likelihood-method}
%This approach was studied by~\cite{genest1995semiparametric} and~\cite{shih1995inferences}.
%One estimates the margins using empirical CDF
%\begin{align}F_X(x)=\frac{1}{n+1}\sum_{i=1}^n 1(X_i \leq x)\end{align},
%
%we maximize the log-likelihood
%\begin{align}
%    L(\delta) = \sum_{i=1}^n \log [c_\delta \{F_X(X_i), F_Y(Y_i)\}]
%    \end{align}
%
%This procedure does not require specification of marginals.
%
%
%
%
%
%%also by Wang and Ding, 2000; Tsukahara, 2005
%%This is also known as pseudo maximum likelihood (PML) and as canonical maximum likelihood (see Cherubini et al., 2004)
%%
%%Genest and Werker (2002) obtained conditions under which the PMLE is asymptotically efficient.
%%
%%

% ----------------
% --- Estimation of Copula ---
% ----------------

\section{Results}
We discuss the results in three directions, hedging effectiveness,
ability of hedging extreme negative events in $R^S$, and the stability of $h^*$.

\subsection{Hedging Effectiveness}
The hedging effectiveness(HE) is defined as
\begin{align}
  1- \frac{\rho_\phi(R^h)}{\rho_\phi(R^S)}.
  \end{align}
The hedging effectiveness is the reduction of portfolio risk.
This way of evaluating of hedging performance is proposed by \cite{ederington1979hedging} in the context of, at that time, hedging the newly introduced
organized futures market.
He evaluates the extent of variance reduction by introducing another asset.
We measure the hedging effectiveness also in other risk measure mentioned in section \ref{subsec:spectral-risk-measures},
for example
\begin{align}
  1- \frac{\text{ES}_\alpha(R^h)}{\text{ES}_\alpha(R^S)}.
  \end{align}

The box-plots in figure \ref{fig:OOSHE} show the out-of-sample hedging effectiveness of different copulas under various risk
reduction objectives across testing datasets.
Observe that in most of the copulas perform well in most of the time.
The average HE of copulas and risk reduction objectives is higher than 60\% except for Frank-copula.
However, the HEs vary a lot in different testing data.
In some instances, the HE can be as low as 10\%.
This reflects the highly violate nature of cryptocurrencies:
the optimal hedge ratio in the training data deviates from that of testing data.
There is a large literature about structural break points and time changing dependence, to name a few
\citet{hafner2012dynamic}, \citet{patton2006modelling}, \citet{creal2008general},
\citet{engle2002dynamic}, and \citet{giacomini2009inhomogeneous}.
\citet{manner2012survey} gives a great survey about this issue.
The discussion is out of the scope of this study.\medskip

Frank-copula, in general, is not a good choice to model financial data.
Figure~\ref{fig:Frank}

\begin{figure}[th]
   \centering
   \includegraphics[width=\textwidth]{_pics/Frank.png}
   \caption{Comparison of Frank Copula Samples and Pseudo Observations of Bitcoin and CME Future Returns.
   }
   \label{fig:Frank}
\end{figure}

Aside from the Frank-copula, the HEs of various combination of copula and risk reduction objective are very similar.
This is an expected result as the portfolio consists only two assets.
In addition to hedging effectiveness, we observe the out-of-sample returns of the hedged portfolio.
Figure~\ref{fig:OOSRH} tabulates the time series of out-of-sample returns of hedged portfolio under various copulas and risk reduction objectives.

One can see all the combinations of copula and risk reduction objective generate a large fluctuation of returns in
25/09/2019 and 26/09/2019.
This large fluctuation is due to dependence break.
\medskip

\begin{figure}[th]
   \centering
   \includegraphics[width=\textwidth]{_pics/OOSBitcoin.png}
   \caption{Out of Sample Log-return of Bitcoin.
%   Lower Panel: Out of Sample Hedged Portfolio log-returns.
%   The $h^*$ is obtained from Gumbel copula aiming at reducing variance.
%   The red dots indicate the 30 most extreme negative returns in Bitcoin.
   }
   \label{fig:Gumbel}
\end{figure}

\newpage
\begin{landscape}
\begin{figure}[th]
   \centering
   \includegraphics[width=\linewidth]{_pics/Rhs.png}
   \caption{Out-of-Sample Returns of Hedged Portfolio of Copulas and Risk Reduction Objectives.
   }
   \label{fig:OOSRH}
\end{figure}
\end{landscape}
\newpage

\newpage
\begin{landscape}
\begin{figure}[th]
   \centering
   \includegraphics[width=\linewidth]{_pics/OHRs.png}
   \caption{Optimal Hedge Ratio Obtained from Combinations of Copula and Risk Reduction Objective.
   }
   \label{fig:OHRs}
\end{figure}
\end{landscape}
\newpage


%\begin{figure}[t]
%\includegraphics[width=\textwidth, height=\textheight]{_pics/Out of Sample Hedging Effectiveness.png}
%  \caption{}
%\label{out of Sample Hedging Effectieness}
%\end{figure}

\begin{figure}[!th]
   \centering
   \includegraphics[height=25cm]{_pics/Out of Sample Hedging Effectiveness.png}
   \caption{Out of Sample Hedging Effectiveness Box-plot.
   The HEs are obtained from a set of out-of-sample data,
   each set consists 30 days log returns of Bitcoin and CME future.
   }
   \label{fig:OOSHE}
\end{figure}

Figure~\ref{fig:Gumbel} shows the time series of out-of-sample $R^h$ using Gumbel copula with the
objective of reducing variance.
The red dots are the 30 most extreme negative returns in Bitcoin.
In the figure, we can see the downside risk of Bitcoin is well managed by the hedging procedure with Gumbel copula.
Most of the extreme losses of Bitcoin are greatly reduced by introducing the CME future in the hedged portfolio.
Two exceptions are found in 25/09/2019 and 26/09/2019, where the CME future failed to follow the large drop in Bitcoin. (TODO: drop reason)
One of the possible reason is that traders was performing rollover activities on 25-26/09/2019, which
27/09/2019 is the expiry day of the September future.
Another reason for Gumbel fail of capturing the loss is dependence break.
The Kendall's tau in the training data is 0.2 higher than that of the testing data.
Other copulas suffer from the break as well.



\subsection{Stability of $h^*$}
We measure the stability of $h^*$ by sum of absolute change
\begin{align}
    \sum_{t=1}^T|h_t - h_{t-1}|.
    \end{align}

Adjustment of portfolio weights induces price slippage (ref) and transaction cost.
From figure \ref{SAD} we know the NIG factor copula with variance as risk reduction objective generates the smallest
sum of absolute change in OHR.

\begin{figure}[!th]
   \centering
   \includegraphics[width=\textwidth]{_pics/Sum of absolute change in OHR.png}
   \caption{Sum of Absolute Change in OHR.
   }
   \label{fig:SAD}
\end{figure}

%\usepackage{fontspec}
%\newcommand{\smallest}[1]{\textcolor{Maroon}{\textbf{#1}}}
\newpage
\begin{table}
    \centering
\begin{tabular}{lrrrrrr}
\toprule
{} &  ERM k=10 &    ES 99\% &    ES 95\% &   VaR 99\% &   VaR 95\% &  Variance \\
\midrule
Gaussian        &  0.019985 &  \color{green}0.020802 &  0.020061 &  \color{green}0.020230 &  0.019983 &  \color{blue}0.019757 \\
t\_Copula        &  0.020097 &  0.021698 &  0.020381 &  0.020966 &  0.020071 &  \color{blue}0.019890 \\
t\_Copula\_Capped &  0.020048 &  0.021018 &  0.020202 &  0.020554 &  0.020059 &  \color{blue}0.019792 \\
Clayton         &  \color{green}0.019519 &  0.021341 &  \color{green}0.019789 &  0.021045 &  \color{red}0.019389 &  \color{green}0.019675 \\
Frank           &  0.029234 &  0.026240 &  0.030770 &  0.029157 &  \color{blue}0.023085 &  0.025928 \\
Gumbel          &  0.020014 &  0.021411 &  0.020511 &  0.021643 &  \color{blue}0.019557 &  0.019757 \\
Plackett        &  0.020010 &  0.021531 &  0.020363 &  0.020870 &  \color{blue}0.019755 &  0.019909 \\
Gauss Mix Indep &  0.019949 &  0.025390 &  0.020454 &  0.023283 &  \color{blue}0.019667 &  0.020006 \\
NIG\_factor      &  \color{blue}0.019720 &  0.023425 &  0.020706 &  0.022039 &  0.019950 &  0.019999 \\
\bottomrule
\end{tabular}
\caption{Exponential Risk Measure $k=10$}
\end{table}

\begin{table}
        \centering
\begin{tabular}{lrrrrrr}
\toprule
{} &  ERM k=10 &    ES 99\% &    ES 95\% &   VaR 99\% &   VaR 95\% &  Variance \\
\midrule
Gaussian        &  0.061084 &  0.062405 &  0.061201 &  0.062148 &  0.061712 &  \color{blue}0.059310 \\
t\_Copula        &  0.062148 &  0.068702 &  0.063339 &  0.063964 &  0.062067 & \color{blue}0.060735 \\
t\_Copula\_Capped &  0.061623 &  0.064114 &  0.062198 &  0.062466 &  0.062072 & \color{blue}0.059676 \\
Clayton         &  0.058495 &  0.069910 &  0.060812 &  0.064595 &  \color{blue}0.055962 &  \color{green}0.058318 \\
Frank           &  0.104185 &  0.096795 &  0.108713 &  0.105070 &  \color{blue}0.068457 &  0.091321 \\
Gumbel          &  \color{green}0.056513 &  \color{green}0.059574 &  \color{green}0.056035 &  \color{green}0.058162 &  \color{red}0.055492 &  0.059525 \\
Plackett        &  0.061167 &  0.068027 &  0.063426 &  0.064563 &  \color{blue}0.058491 &  0.061017 \\
Gauss Mix Indep &  0.061157 &  0.088023 &  0.063900 &  0.073316 &  \color{blue}0.057007 &  0.063081 \\
NIG\_factor      &  \color{blue}0.060878 &  0.078959 &  0.065270 &  0.070919 &  0.062097 &  0.062848 \\
\bottomrule
\end{tabular}
\caption{ES 99\%}
\end{table}

\begin{table}
        \centering
\begin{tabular}{lrrrrrr}
\toprule
{} &  ERM k=10 &    ES 99\% &    ES 95\% &   VaR 99\% &   VaR 95\% &  Variance \\
\midrule
Gaussian        &  0.034488 &  \color{green}0.035237 &  0.034548 &  \color{green}0.035123 &  0.034838 &  \color{blue}0.034248 \\
t\_Copula        &  0.034777 &  0.037100 &  0.035234 &  0.035634 &  0.035055 &  \color{blue}0.034494 \\
t\_Copula\_Capped &  0.034647 &  0.035679 &  0.034862 &  0.035282 &  0.034937 &  \color{blue}0.034322 \\
Clayton         &  \color{green}0.033714 &  0.037282 &  \color{green}0.034230 &  0.036089 &  \color{red}0.033445 &  \color{green}0.034046 \\
Frank           &  0.053661 &  0.047849 &  0.056299 &  0.053409 &  \color{blue}0.037638 &  0.046953 \\
Gumbel          &  0.034028 &  0.035965 &  0.034528 &  0.036353 &  \color{blue}0.033568 &  0.034293 \\
Plackett        &  0.034592 &  0.036831 &  0.035316 &  0.035752 &  \color{blue}0.034186 &  0.034558 \\
Gauss Mix Indep &  0.034439 &  0.045160 &  0.035120 &  0.040027 &  \color{blue}0.033756 &  0.034478 \\
NIG\_factor      &  \color{blue}0.033882 &  0.041001 &  0.035677 &  0.037975 &  0.034656 &  0.034453 \\
\bottomrule
\end{tabular}
\caption{ES 95\%}
\end{table}

\begin{table}
            \centering
\begin{tabular}{lrrrrrr}
\toprule
{} &  ERM k=10 &    ES 99\% &    ES 95\% &   VaR 99\% &   VaR 95\% &  Variance \\
\midrule
Gaussian        &  \color{blue}0.041327 &  0.044416 &  \color{green}0.041943 &  0.043399 &  0.042275 &  0.041981 \\
t\_Copula        &  \color{blue}0.041450 &  0.044830 &  0.042806 &  0.043789 &  0.041693 &  0.041969 \\
t\_Copula\_Capped &  \color{blue}0.041498 &  0.044169 &  0.042411 &  0.044051 &  0.042018 &  0.042056 \\
Clayton         &  \color{red}0.040022 &  0.044523 &  0.042878 &  0.044215 &  0.040913 &  0.041943 \\
Frank           &  0.076644 &  0.055387 &  0.081273 &  0.073433 &  \color{blue}0.046177 &  0.061056 \\
Gumbel          &  0.042079 &  \color{green}0.042139 &  0.042187 &  0.045340 &  \color{blue}0.040523 &  0.041937 \\
Plackett        &  \color{blue}0.041013 &  0.044971 &  0.042370 &  \color{green}0.042995 &  0.041574 &  \color{green}0.041731 \\
Gauss Mix Indep &  0.040998 &  0.048017 &  0.043249 &  0.044518 &  \color{blue}0.040749 &  0.043386 \\
NIG\_factor      &  \color{blue}0.040457 &  0.047201 &  0.043925 &  0.044230 &  0.043240 &  0.043138 \\
\bottomrule
\end{tabular}
\caption{VaR 99\%}
    \label{table:OOSRHVaR99}
\end{table}

\begin{table}
            \centering
\begin{tabular}{lrrrrrr}
\toprule
{} &  ERM k=10 &    ES 99\% &    ES 95\% &   VaR 99\% &   VaR 95\% &  Variance \\
\midrule
Gaussian        &  0.020385 &  0.020315 &  \color{green}0.020143 &  0.020412 &  0.020121 &  \color{blue}0.019579 \\
t\_Copula        &  0.020547 &  0.020428 &  0.020661 &  0.020611 &  0.020370 &  \color{blue}0.019820 \\
t\_Copula\_Capped &  0.020525 &  0.020544 &  0.020503 &  0.020486 &  0.020224 &  \color{blue}0.019656 \\
Clayton         &  \color{green}0.019702 &  0.021042 &  0.020143 &  0.020640 &  0.019990 &  \color{blue}0.019700 \\
Frank           &  0.026372 &  0.023529 &  0.027105 &  0.026212 &  \color{blue}0.023389 &  0.023594 \\
Gumbel          &  0.019781 &  0.021311 &  0.020716 &  0.020421 &  \color{red}0.019077 &  \color{green}0.019541 \\
Plackett        &  0.020459 &  \color{green}0.020257 &  0.020589 &  \color{green}0.020100 &  0.020237 &  \color{blue}0.020047 \\
Gauss Mix Indep &  0.020482 &  0.024753 &  0.020304 &  0.024158 &  \color{blue}0.019944 &  0.020723 \\
NIG\_factor      &  \color{blue}0.019923 &  0.023784 &  0.021009 &  0.022172 &  0.019980 &  0.020670 \\
\bottomrule
\end{tabular}
\caption{VaR 95\%}
\end{table}

\begin{table}
            \centering
\begin{tabular}{lrrrrrr}
\toprule
{} &  ERM k=10 &    ES 99\% &    ES 95\% &   VaR 99\% &   VaR 95\% &  Variance \\
\midrule
Gaussian        &  0.014387 &  \color{green}0.014380 &  0.014360 &  0.014530 &  0.014670 &  \color{blue}0.014294 \\
t\_Copula        &  0.014378 &  0.014626 &  0.014343 &  \color{green}0.014385 &  0.014627 &  \color{blue}0.014306 \\
t\_Copula\_Capped &  0.014375 &  0.014418 &  0.014332 &  0.014430 &  0.014643 &  \color{blue}0.014290 \\
Clayton         &  \color{green}0.014306 &  0.014870 &  \color{green}0.014332 &  0.014532 &  0.014493 &  \color{red}0.014267 \\
Frank           &  0.021495 &  0.018982 &  0.022736 &  0.021476 &  \color{blue}0.018142 &  0.018897 \\
Gumbel          &  0.014618 &  0.014971 &  0.014878 &  0.015438 &  0.014622 &  \color{blue}0.014321 \\
Plackett        &  0.014444 &  0.014560 &  0.014424 &  0.014423 &  0.014596 &  \color{blue}0.014353 \\
Gauss Mix Indep &  0.014404 &  0.017404 &  \color{blue}0.014341 &  0.015671 &  \color{green}0.014453 &  0.014408 \\
NIG\_factor      &  \color{blue}0.014362 &  0.015841 &  0.014484 &  0.015043 &  0.014474 &  0.014415 \\
\bottomrule
\end{tabular}
\caption{Standard Deviation}
\end{table}
\newpage

% ----------------
% --- Results ---
% ----------------

\newpage
%
\bibliography{finance} %
\newpage
\section{Appendix}\label{sec:appendix}

\begin{proposition}
   Let $\bm{X} = (X_1, ..., X_d)^\top$ be real-valued random variables with corresponding
   copula density $\bm{c}_{X_1, ..., X_d}$, and continuous marginals $F_{X_1}, ..., F_{X_d}$.
   Then,
   density of the linear combination of marginals $Z = n_1 \cdot X_1 + ... +  n_d \cdot X_d $ is

   \begin{align}
   f_Z(z) &= \left| n_1^{-1} \right| \int_{[0,1]^{d-1}} \left[ \bm{c}_{X_1,...,X_d}
      \{F_{X_1} \circ S(z), u_2, ..., u_d \} \cdot
      f_{X_1} \circ S(z) \right] du_2 ... du_d \label{density} \\
      S(z) &= \frac{1}{n_1}\cdot z - \frac{n_2}{n_1} \cdot F^{-1}_{X_2}(u_2) - ... -  \frac{n_d}{n_1} \cdot F^{-1}_{X_d}(u_d)
      \end{align}
   \end{proposition}

\begin{proof} \medskip
   Rewrite $Z = n_1 \cdot X_1 + ... +  n_d \cdot X_d $ in matrix form
   \begin{align}
      \begin{bmatrix}
      Y \\ X_2 \\ \vdots \\ X_d
      \end{bmatrix}
      = \begin{bmatrix}
      n_1    & n_2   & \cdots & n_d     \\
      0      & 1     &  \cdots & 0       \\
      \vdots &       & \ddots & \vdots \\
      0      & \cdots &       & 1  \\
      \end{bmatrix}
      \begin{bmatrix}
         X_1 \\ X_2 \\ \vdots \\ X_d
      \end{bmatrix}
      = \bm{A}
      \begin{bmatrix}
         X_1 \\ X_2 \\ \vdots \\ X_d
      \end{bmatrix}.
      \end{align} \medskip

   By transformation variables
   \begin{align}
      \bm{f}_{Z,X_2,...,X_d}(z, x_2, ...,x_d) &= \bm{f}_{X_1,...,X_d}\left( \bm{A}^{-1}
      \begin{bmatrix}
         z \\ x_2 \\ \vdots \\ x_d
      \end{bmatrix}
      \right)  \cdot |\det \bm{A}^{-1}| \\
      &= \left| n_1^{-1} \right| \bm{f}_{X_1,...,X_d}\{S(z), x_2,...,x_d\}
      \end{align} \medskip

   Let $u_i = F_{X_i}(x_i)$ and
   use the relationship
   \begin{align}
      \bm{c}_{X_1,...,X_d}(u_1, ..., u_d)=\frac{\bm{f}_{X_1,...,X_d}(x_1,...,x_d)}{\prod_{i=1}^d f_{X_i}(x_i)},
   \end{align}
   we have
   \begin{align}
     & \bm{f}_{Z,X_2,...,X_d}(z, x_2, ...,x_d) = \\
      & \left| n_1^{-1} \right| \cdot
      \bm{c}_{X_1,...,X_d}\{F_{X_1} \circ S(z), u_2, ...,  u_d\}  \cdot
      f_{X_1} \{ S(z) \} \cdot
      \prod_{i=2}^d f_{X_i}(x_i)
      \end{align}

   The claim \ref{density} is obtained by integrating out $x_2, ... x_d$ by substituting $dx_i = \frac{1}{f_{X_i}(x_i)}du_i$.
   \end{proof}
\end{document}
