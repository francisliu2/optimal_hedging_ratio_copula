\section{Conclusion and Outlook}\label{sec:conclusion-and-outlook}
We study the hedging effectiveness of Bitcoin futures to cryptos and crypto indices.
We consider commonly used risk measures with different configurations to cater different risk appetite and scenarios, they are variance, Value-at-Risk 95\% and 99\%,
Expected Shortfall 95\% and 99\%, and Exponential Risk Measure $k=10$. Dependence between cryptos and the futures plays an important role in hedging as it determines the distribution of the portfolio returns.
However, in the time of writing, the crypto market is a vibrant and fast-developing market, causing cryptos to have complex and, possibly,
time changing dependence structures with the Bitcoin futures.
We use copulae, a flexible statistical tool that model marginals and dependence separately to capture complexity of the dependence structure;
and periodically re-calibrate copulae and find the best-fitting copula via AIC to address the potential time changing dependence. \medskip

An extensive out-of-sample backtest suggests that the Bitcoin futures is consistently capable of hedging BTC and BTC-involved indices,
BITX, CRIX, and BITW100, under different risk minimisation objectives and copula models.
The MSEs and LSVs of the resulting portfolios are indistinguishably at a low level except the Frank copula.
On the other hand, the AIC procedure flavours t-copula because it captures the tail dependence feature of the data.
Unsurprisingly, the portfolios' out-of-sample maximum drawdowns are significantly reduced. \medskip

Contrarily, we observe a diverse result of the capability of BTC futures to hedge other cryptos and BTC-not-involved indices.
In general, the ES 95\% and VaR 95\% perform better than their 99\% counterparts.
In particular minimising ES 99\% leads to relatively high MSE and LSV disregard of which copula is in use.
The ES 99\% and VaR 99\% even result in out-of-sample maximum drawdowns which are higher than that of the 95\% counterparts in some portfolios,
for example in the ETH- and LTC-BTCF portfolio.
Therefore, we conclude that overly emphasising tail risks by choosing extreme tail risk measures does not lead to a promising hedge in a cross hedging setting. \medskip

The AIC procedure mainly flavours rotGumbel and NIG factor copula in modelling other cryptos and BTC-not-involved indices.
This reflects the idiosyncratic nature of downward movements in crypto market.
However, the best-fitting copula does not necessary leads to best performing portfolio in MSE or LSV, see e.g. ADA. \medskip

This is what we call the discrepancy between the copula selection result and the MSE LSV results.
We suspect the discrepancy is the result of our hedge being a linear, long and short only with assets and derivative with linear payoffs i.e. futures.
Although copulae are flexible to model complex dependence structures by emphasising numbers of important features
e.g. lower tail dependence and radial symmetry, the simple linear hedge is very limited in flexibility to address to complex dependence.
Including derivatives with non-linear payoffs e.g. options might be a way out. \medskip

%A possible research direction would be hedging the linear part with futures and non-linear part with options, e.g. the BTC options.