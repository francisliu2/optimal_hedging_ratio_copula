\section{Conclusion and Outlook}\label{sec:conclusion-and-outlook}
We study the effectiveness of hedging cryptos and crypto indices with
Bitcoin futures.
To accomodate different risk appetites and scenarios, a variety of
commonly used risk measures are considered to determine the optimal
hedge ratio. The risk measures comprise variance, value-at-risk at
the confidence levels 95\% and 99\%, expected shortfall 95\% and 99\%,
and the exponential risk measure with parameter $k=10$.

At the time of writing, the crypto market is a vibrant and
fast-developing market, causing cryptos to have complex and
time-changing dependence structures with the Bitcoin futures.
As a consequence, the dependence between the cryptos and the futures
contract plays an important role in hedging as it determines the
distribution of the portfolio returns. We therefore consider various
copulae, a flexible statistical tool that separates modelling of the
marginals and the dependence structure of multivariate random
vectors. To address the potential time-changing dependence, we
periodically re-calibrate the copula models and determine the
best-fitting copula via AIC. 

An extensive out-of-sample backtest suggests that the Bitcoin futures
are consistently capable of hedging BTC and BTC-involved indices,
i.e., BITX, CRIX, and BITW100, under different risk minimisation
objectives 
and copula models. The mean-square errors (MSEs) and lower
semi-variances (LSVs) of the resulting portfolios are
indistinguishably at a low level except for the Frank copula. 
On the other hand, the AIC procedure favours the $t$-copula because it 
captures the tail dependence feature of the data.
Compared to the unhedged cases, the
portfolios' out-of-sample maximum drawdowns are significantly reduced. 

Contrarily, we observe diverse results of the capability of BTC
futures to hedge other cryptos and crypto indices that exclude Bitcoin. 
In general, ES 95\% and VaR 95\% perform better than their 99\%
counterparts. In particular, minimising ES 99\% leads to relatively
high MSEs and LSVs regardless of the copula in use. The ES 99\% and
VaR 99\% even result in out-of-sample maximum drawdowns that are
higher than that of the 95\% counterparts in some portfolios, 
for example in the ETH- and LTC-BTCF portfolio.
Therefore, we conclude that overly emphasising tail risks by choosing
extreme tail risk measures does not lead to a promising hedge in a
cross-hedging setting. 

The AIC procedure mainly favours the rotated Gumbel and the NIG
factor copula in modelling non-BTC relate cryptos and indices. This
reflects the idiosyncratic nature of 
downward movements in the crypto market. Interestingly, the best-fitting
copula does not necessary lead to the best performing portfolio in
terms of MSE or LSV. For example, this is the case for ADA.
We suspect this discrepancy between the optimal copula selection and
MSE-LSV results can be attributed to the static linear nature of the
hedge, as the sole hedge instrument is a futures contract. 

Although copulae are flexible to model complex dependence structures
by emphasising a number of important features such as lower tail
dependence and radial symmetry, the simple linear hedge is very
limited in its flexibility to address this complex dependence.
Including liquidly traded derivatives with non-linear payoffs, such as
options, might be a possibility to improve the hedge quality for these
cryptos and portfolios.

\section{Discussion}\label{sec:discussion}
\textit{Cryptohedgers}: Our results and conclusion are drawn using a non-parametric way, kernel density estimation,
to fit the marginal density of the crypto and futures returns because we intend to leave the modelling of marginal distributions flexible and generic.
Copula provides the flexibility to model marginal distributions separately.
That means the model assumption and parametrisation for spot and futures can be different.
This empowers hedgers to make use of variety of statistical tools to model the marginal distributions according to asset specific features they are interested,
to enhance the model predictability, and hence, to improve the hedging effectiveness.
For example, one can use various types of GARCH to deal with clustered volatility observed from the spot while using structural time series to deal with seasonality in futures;
alternatively, using Hawkes processes to deal with clustered jumps in spots and futures without modelling the spillover or co-jumps in the marginal distribution.
Hedgers are allowed to be creative in this regard. In fact, many researchers have enriched the understanding of the crypto market dynamics,
see for example \citet{kaiser2019seasonality}, \citet{gyamerah2022modelling}, and \citet{mark2022quantifying}.
Although the hedging effectiveness of combining various statistical tools with copula remain an open question,
our analysis shows that hedgers have the freedom to choose the copula (except the Frank copula) to model the dependence structure,
whereas t, Plackett, Gaussian Mix independent, rotated Gumbel and NIG copula appear to be conservative choices to start the analysis.

\textit{Policy makers}:
Our results connect to the discussion of minimum capital requirements for crypto market risk exposure.
The Basel Committee on Banking Supervision (BCBS) has made proposals and released consultative documents for the prudential treatment of banks' crypto-asset exposures.
The Basel chair's stand on placing tough capital rules for crypto assets \footnote{See \url{https://www.risk.net/regulation/7948536/basel-chair-stands-by-tough-capital-rules-for-crypto-assets}} attracted discussion.
The Second Consultation proposes to limit banks' exposures to Group 2 cryptoassets to 1\% of the bank's Tier 1 capital,
calculated on a "double-gross" basis by adding the long and short positions without any hedging recognition
$\text{Net position} = \max \left(
\text{Long position}, |\text{Short position}|
\right)
- .65 \min\left(
\text{Long position}, |\text{Short position}|
\right)$ (60.66).
Respondents to the consultations argued that the proposed methodology of calculating exposure is an extremely restrictive quantitative measure that prohibits banks to manage limit utilisation to crypto assets as the addition of a hedge instrument;
a price increase of the underlying cryptoassets could make banks breaching the limit.
According to our results, BTC position's market risk measured in various risk measures,
even for VaR99\% and ES99\%, can be effectively reduced by an appropriate size of opposite position in CME BTC futures.
While the debates carry on at the time of writing, our results align and can be seen as an extension of the existing researches made by stakeholders,
for example, research done by CME \footnote{See the CME and KPMG's report on \url{https://www.cmegroup.com/education/files/basics-of-hedge-effectiveness.pdf}},
and International Swaps and Derivatives Association's report \footnote{See the ISDA report "Crypto-asset Risks and Hedging Analysis" on \url{https://www.isda.org/a/pMWgE/Crypto-asset-Risks-and-Hedging-Analysis.pdf }}.

On the other hand, our results support the cryptoassets categorisation criteria proposed by the Basel Committee.
In the Second Consultation, Group 2 cryptoassets (Group 1 CCs are digitally tokenised traditional assets) are divided into two subcategories,
Group 2a are the CCs that exists a derivative or exchange-traded fund (ETF)/exchange-traded note (ETN) that is traded on a regulated exchange that solely references the cryptoasset;
Group 2b are the CCs that fail to meet the requirement.
Our results suggest that the existence of derivative that solely references the cryptoasset empowers market participants to hedge their corresponding crypto position,
thus it is sensible to treat the aforementioned Group 2a and 2b differently.
On the contrary, we show that the hedging effectiveness significantly lower than 1 when the spot is different from what the CME futures is referencing, i.e. the cross hedge setting.
That means the risk of holding a CC cannot be perfectly cross hedged by including the CME futures.
Even worse, we observe negative HEs in some of the bootstrapped samples.
From the great difference in hedging effectiveness between the BTC-BTCF and other cross hedge portfolios,
we conclude that the existance of derivative that solely reference the cryptoasset is a sensible choice of crtyptoassets categorisation criteria
as it reflects assessibility of market participants to form risk reduction strategy.

%%% Local Variables:
%%% mode: latex
%%% TeX-master: "SRM"
%%% End:
