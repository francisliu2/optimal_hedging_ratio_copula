%\newpage
\appendix
% \section{Appendix}
\section{Density of linear combination of random variables}
\label{sec:appendix}
\begin{proposition}
   Let $\bm{X} = (X_1, ..., X_d)^\top$ be real-valued random variables with corresponding
   copula density $\bm{c}_{X_1, ..., X_d}$, and continuous marginals $F_{X_1}, ..., F_{X_d}$.
   Then, the
   pdf of the linear combination of marginals $Z = n_1 \cdot X_1 +
   ... +  n_d \cdot X_d $ is
   \begin{align}
   f_Z(z) &= \left| n_1^{-1} \right| \int_{[0,1]^{d-1}} \bm{c}_{X_1,...,X_d}
      \left(F_{X_1} (S(z)), u_2, ..., u_d \right) \cdot
      f_{X_1} (S(z)) \dd u_2 ... \dd u_d, \label{density}
   \end{align}
   \natp{with}
   \begin{align*}
      S(z) &= \frac{1}{n_1}\cdot z - \frac{n_2}{n_1} \cdot F^{(-1)}_{X_2}(u_2) - ... -  \frac{n_d}{n_1} \cdot F^{(-1)}_{X_d}(u_d).
      \end{align*}
   \end{proposition}

\begin{proof}
  \natp{(delete -- already introduced above: Let $Z= n_1 \cdot X_1 +
    ... +  n_d \cdot X_d $ and let) Let}
    $\mathbf A=\displaystyle \begin{bmatrix}
      n_1    & n_2   & \cdots & n_d     \\
      0      & 1     &  \cdots & 0       \\
      \vdots &       & \ddots & \vdots \\
      0      & \cdots &       & 1  \\
    \end{bmatrix}$.\natp{\em [The specification of the matrix is not
      clear. How does the row proceed after 1 in the second row, when
      it ends in 0?]} Then,
    \begin{equation*}
      \begin{bmatrix}
        Z \\ X_2 \\ \vdots \\ X_d
      \end{bmatrix}
      = \bm{A}
      \begin{bmatrix}
         X_1 \\ X_2 \\ \vdots \\ X_d
      \end{bmatrix}.
    \end{equation*}

   By transformation of the variables~\citep{hardle2019applied}
   \natp{\em [If you cite a book and have a particular formula or
     section in mind, then state it.]}
   \begin{align*}
      \bm{f}_{Z,X_2,...,X_d}(z, x_2, ...,x_d) &= \bm{f}_{X_1,...,X_d}\left( \bm{A}^{-1}
      \begin{bmatrix}
         z \\ x_2 \\ \vdots \\ x_d
      \end{bmatrix}
      \right)  \cdot |\det \bm{A}^{-1}| \\
      &= \left| n_1^{-1} \right| \bm{f}_{X_1,...,X_d}\left(S(z), x_2,...,x_d\right).
      \end{align*} \medskip

   Let $u_i = F_{X_i}(x_i)$ and by chain rule we have
   \begin{align*}
       \bm{f}_{X_1,...,X_d}(x_1,...,x_d) &= \frac{\partial^d F_{X_1,...,X_d}(x_1, ..., x_d)}{\partial x_1 ... \partial x_d}\\
                                         &= \bm{c}_{X_1,...,X_d}(u_1, ..., u_d) \cdot \prod_{i=1}^d f_{X_i}(x_i).
   \end{align*}

    Therefore,
   \begin{align*}
      \bm{f}_{Z,X_2,...,X_d}&(z, x_2, ...,x_d) = \\
       \left| n_1^{-1} \right| &\cdot
      \bm{c}_{X_1,...,X_d}\left(F_{X_1}( S(z)), u_2, ...,  u_d\right)  \cdot
      f_{X_1} \{ S(z) \} \cdot
      \prod_{i=2}^d f_{X_i}(x_i).
      \end{align*}

   The claim (\ref{density}) is obtained by integrating out $x_2, ... x_d$ by substituting $\dd x_i = \frac{1}{f_{X_i}(x_i)} \dd u_i$.
   \end{proof}

%\clearpage

%\section{Summary Statistics of Assets}\begin{table}[th] \centering \resizebox{\textwidth}{!}{%
\begin{tabular}{lrrrrrrrr} \toprule
         {} &    Mean \% &     Std \% &      Skew &       Kurt &         MD \% &     MD date & $\rho$ & $\tau$ \\
\midrule
     \multicolumn{9}{l}{Hedging Instrument} \\
\ \ \ BTCF    &  0.3906 &  4.6312 & -0.5060 &   4.4204 & -26.9920 &  2020-03-12 &  1.0000 &  1.0000 \\
     \multicolumn{9}{l}{Individual Cryptos}                                                                                 \\
\ \ \ BTC     &  0.3915 &  4.4023 & -0.5857 &   4.6565 & -25.9965 &  2020-03-12 &  0.9975 &  0.9507 \\
\ \ \ ETH     &  0.6819 &  6.0103 & -0.2557 &   5.2646 & -32.0144 &  2020-03-12 &  0.7712 &  0.5988 \\
\ \ \ ADA     &  0.9467 &  6.6990 &  0.1661 &   2.3086 & -26.8528 &  2020-03-12 &  0.6296 &  0.4825 \\
\ \ \ LTC     &  0.3227 &  6.4781 & -0.9935 &   5.3011 & -37.5913 &  2021-05-19 &  0.8080 &  0.6113 \\
\ \ \ XRP     &  0.2987 &  7.9843 &  0.5542 &  12.4882 & -52.7652 &  2020-12-23 &  0.4510 &  0.4939 \\
   \multicolumn{9}{l}{Crypto Indices with BTC Constituent}                                                                  \\
\ \ \ BITX    &  0.4308 &  4.5676 & -0.8842 &   4.7222 & -27.0220 &  2020-03-12 &  0.9769 &  0.8738 \\
\ \ \ CRIX    &  0.4602 &  4.5420 & -0.7952 &   4.7549 & -27.1385 &  2020-03-12 &  0.9799 &  0.8769 \\
\ \ \ BITW100 &  0.4683 &  4.6174 & -0.9864 &   4.9381 & -27.2694 &  2020-03-12 &  0.9674 &  0.8537 \\
    \multicolumn{9}{l}{Crypto Indices without BTC Constituent}                                                              \\
\ \ \ BITW20  &  0.6249 &  5.5021 & -1.1518 &   5.2203 & -31.0092 &  2020-03-12 &  0.7674 &  0.5883 \\
\ \ \ BITW70  &  0.6353 &  5.8155 & -1.1171 &   5.1926 & -32.3453 &  2021-05-19 &  0.7525 &  0.5459 \\
\bottomrule
\end{tabular}}
\caption{Summary statistics of assets' daily returns during the out-of-sample period, from 2019-10-21 to 2021-05-27.
        The first four columns are the first four moments of assets' daily returns.
        The fifth and sixth columns are the maximum drawdown (MD) and the date of the MD.
        The last two columns are Pearson's $\rho$s and Kendall's $\tau$s between the assets and BTCF. }
\label{tab:SSA}

\end{table}
\section{Summary Statistics of Assets}\begin{table}[th] \centering \resizebox{\textwidth}{!}{%
\begin{tabular}{lrrrrrrrr} \toprule
         {} &    Mean \% &     Std \% &      Skew &       Kurt &         MD \% &     MD date & $\rho$ & $\tau$ \\
\midrule
     \multicolumn{9}{l}{Hedging Instrument} \\
\ \ \ BTCF    &  0.3906 &  4.6312 & -0.5060 &   4.4204 & -26.9920 &  2020-03-12 &  1.0000 &  1.0000 \\
     \multicolumn{9}{l}{Individual Cryptos}                                                                                 \\
\ \ \ BTC     &  0.3915 &  4.4023 & -0.5857 &   4.6565 & -25.9965 &  2020-03-12 &  0.9975 &  0.9507 \\
\ \ \ ETH     &  0.6819 &  6.0103 & -0.2557 &   5.2646 & -32.0144 &  2020-03-12 &  0.7712 &  0.5988 \\
\ \ \ ADA     &  0.9467 &  6.6990 &  0.1661 &   2.3086 & -26.8528 &  2020-03-12 &  0.6296 &  0.4825 \\
\ \ \ LTC     &  0.3227 &  6.4781 & -0.9935 &   5.3011 & -37.5913 &  2021-05-19 &  0.8080 &  0.6113 \\
\ \ \ XRP     &  0.2987 &  7.9843 &  0.5542 &  12.4882 & -52.7652 &  2020-12-23 &  0.4510 &  0.4939 \\
   \multicolumn{9}{l}{Crypto Indices with BTC Constituent}                                                                  \\
\ \ \ BITX    &  0.4308 &  4.5676 & -0.8842 &   4.7222 & -27.0220 &  2020-03-12 &  0.9769 &  0.8738 \\
\ \ \ CRIX    &  0.4602 &  4.5420 & -0.7952 &   4.7549 & -27.1385 &  2020-03-12 &  0.9799 &  0.8769 \\
\ \ \ BITW100 &  0.4683 &  4.6174 & -0.9864 &   4.9381 & -27.2694 &  2020-03-12 &  0.9674 &  0.8537 \\
    \multicolumn{9}{l}{Crypto Indices without BTC Constituent}                                                              \\
\ \ \ BITW20  &  0.6249 &  5.5021 & -1.1518 &   5.2203 & -31.0092 &  2020-03-12 &  0.7674 &  0.5883 \\
\ \ \ BITW70  &  0.6353 &  5.8155 & -1.1171 &   5.1926 & -32.3453 &  2021-05-19 &  0.7525 &  0.5459 \\
\bottomrule
\end{tabular}}
\caption{Summary statistics of assets' daily returns during the out-of-sample period, from 2019-10-21 to 2021-05-27.
        The first four columns are the first four moments of assets' daily returns.
        The fifth and sixth columns are the maximum drawdown (MD) and the date of the MD.
        The last two columns are Pearson's $\rho$s and Kendall's $\tau$s between the assets and BTCF. }
\label{tab:SSA}

\end{table}
\section{Summary Statistics of Hedged Portfolios}\label{sec:SSHP}
\begin{table}[h] \centering \resizebox{\textwidth}{!}{%
\begin{tabular}{lrrrrrrr} \toprule
         {} &    Mean \% &     Std \% &      Skew &       Kurt &         MD \% &     MD date & Variance \\
\midrule
     \multicolumn{7}{l}{Individual Cryptos}                                                                                 \\
\ \ \ BTC     &  0.0215 &  0.3221 & -1.0119 &   3.1929 &  -1.4393 &  2020-11-30 &    0.0000 \\
\ \ \ ETH     &  0.2823 &  3.8741 &  0.9469 &   7.1064 & -17.7421 &  2021-05-19 &    0.0015 \\
\ \ \ ADA     &  0.5617 &  5.2722 &  1.3634 &   4.4818 & -13.8687 &  2021-01-08 &    0.0028 \\
\ \ \ LTC     & -0.0871 &  3.9052 & -0.3617 &   7.6239 & -28.3029 &  2021-05-19 &    0.0018 \\
\ \ \ XRP     & -0.0123 &  7.1537 &  1.1451 &  20.0236 & -52.5236 &  2020-12-23 &    0.0043 \\
   \multicolumn{7}{l}{Crypto Indices with BTC Constituent}                                                                  \\
\ \ \ BITX    &  0.0561 &  0.9954 & -0.4204 &  13.2487 &  -7.7567 &  2021-05-19 &    0.0001 \\
\ \ \ CRIX    &  0.0812 &  0.9183 & -0.0027 &  14.3136 &  -7.1025 &  2021-05-19 &    0.0001 \\
\ \ \ BITW100 &  0.0855 &  1.1986 & -1.7440 &  22.2644 & -11.3866 &  2021-05-19 &    0.0001 \\
    \multicolumn{7}{l}{Crypto Indices without BTC Constituent}                                                              \\
\ \ \ BITW20  &  0.2429 &  3.5846 & -0.3063 &   4.1622 & -21.4680 &  2021-05-19 &    0.0013 \\
\ \ \ BITW70  &  0.2706 &  3.8838 & -0.6490 &   4.6312 & -23.9984 &  2021-05-19 &    0.0015 \\
\bottomrule
\end{tabular}}
\caption{Summary statistics of out-of-sample daily returns of hedged portfolios that minimize variance.}
\label{tab:var_rh}

\end{table}\begin{table}[!] \centering \resizebox{\textwidth}{!}{%
\begin{tabular}{lrrrrrrr} \toprule
         {} &    Mean \% &     Std \% &      Skew &       Kurt &         MD \% &     MD date & VaR 5\% \\
\midrule
     \multicolumn{7}{l}{Individual Cryptos}                                                                                 \\
\ \ \ BTC     &  0.0253 &  0.3294 & -0.9725 &   3.4373 &  -1.5347 &  2020-11-30 &  0.0063 \\
\ \ \ ETH     &  0.3084 &  3.8944 &  1.0243 &   7.4297 & -19.1750 &  2021-05-19 &  0.0514 \\
\ \ \ ADA     &  0.5726 &  5.2204 &  1.2981 &   4.2544 & -14.6974 &  2021-05-19 &  0.0769 \\
\ \ \ LTC     & -0.0742 &  3.9145 & -0.3836 &   7.5384 & -28.3672 &  2021-05-19 &  0.0622 \\
\ \ \ XRP     &  0.0208 &  7.1520 &  1.1269 &  19.8930 & -52.5667 &  2020-12-23 &  0.0683 \\
   \multicolumn{7}{l}{Crypto Indices with BTC Constituent}                                                                  \\
\ \ \ BITX    &  0.0562 &  0.9930 & -0.3117 &  12.4780 &  -7.5639 &  2021-05-19 &  0.0128 \\
\ \ \ CRIX    &  0.0863 &  0.9151 &  0.0718 &  13.7915 &  -6.9744 &  2021-05-19 &  0.0092 \\
\ \ \ BITW100 &  0.0846 &  1.1980 & -1.6592 &  21.3725 & -11.2582 &  2021-05-19 &  0.0164 \\
    \multicolumn{7}{l}{Crypto Indices without BTC Constituent}                                                              \\
\ \ \ BITW20  &  0.2728 &  3.5940 & -0.3721 &   4.4896 & -22.0733 &  2021-05-19 &  0.0546 \\
\ \ \ BITW70  &  0.2847 &  3.9133 & -0.6580 &   4.7874 & -24.6513 &  2021-05-19 &  0.0626 \\
\bottomrule
\end{tabular}}
\caption{Summary statistics of out-of-sample daily returns of hedged portfolios that minimize VaR 5\%.}
\label{tab:VaR5_rh}

\end{table}\begin{table}[!] \centering \resizebox{\textwidth}{!}{%
\begin{tabular}{lrrrrrrr} \toprule
         {} &    Mean \% &     Std \% &      Skew &       Kurt &         MD \% &     MD date & VaR 1\% \\
\midrule
     \multicolumn{7}{l}{Individual Cryptos}                                                                                 \\
\ \ \ BTC     &  0.0176 &  0.3270 & -1.0405 &   3.3742 &  -1.5689 &  2020-11-30 &  0.0134 \\
\ \ \ ETH     &  0.2977 &  3.9132 &  0.9547 &   7.2414 & -18.6061 &  2021-05-19 &  0.1026 \\
\ \ \ ADA     &  0.5562 &  5.3466 &  1.1362 &   3.9334 & -15.4795 &  2021-05-19 &  0.1106 \\
\ \ \ LTC     & -0.0852 &  4.1503 & -0.7234 &   7.3208 & -29.0915 &  2021-05-19 &  0.1030 \\
\ \ \ XRP     &  0.0352 &  7.1658 &  1.1582 &  19.8506 & -52.5727 &  2020-12-23 &  0.1387 \\
   \multicolumn{7}{l}{Crypto Indices with BTC Constituent}                                                                  \\
\ \ \ BITX    &  0.0593 &  1.0178 & -0.5331 &  13.3100 &  -8.0299 &  2021-05-19 &  0.0247 \\
\ \ \ CRIX    &  0.0738 &  0.9695 & -0.4729 &  13.6500 &  -7.0185 &  2021-05-19 &  0.0245 \\
\ \ \ BITW100 &  0.0823 &  1.2338 & -1.9365 &  23.1938 & -11.8752 &  2021-05-19 &  0.0347 \\
    \multicolumn{7}{l}{Crypto Indices without BTC Constituent}                                                              \\
\ \ \ BITW20  &  0.2499 &  3.6210 & -0.3866 &   4.3396 & -21.6634 &  2021-05-19 &  0.0988 \\
\ \ \ BITW70  &  0.2788 &  3.9257 & -0.7635 &   5.1288 & -24.5294 &  2021-05-19 &  0.1147 \\
\bottomrule
\end{tabular}}
\caption{Summary statistics of out-of-sample daily returns of hedged portfolios that minimize VaR 1\%.}
\label{tab:VaR1_rh}

\end{table}\begin{table}[!] \centering \resizebox{\textwidth}{!}{%
\begin{tabular}{lrrrrrrr} \toprule
         {} &    Mean \% &     Std \% &      Skew &       Kurt &         MD \% &     MD date & ES 5\% \\
\midrule
     \multicolumn{7}{l}{Individual Cryptos}                                                                                 \\
\ \ \ BTC     &  0.0204 &  0.3234 & -1.0150 &   3.4423 &  -1.5629 &  2020-11-30 &  0.0101 \\
\ \ \ ETH     &  0.3082 &  3.8890 &  1.0119 &   7.4077 & -18.7819 &  2021-05-19 &  0.0782 \\
\ \ \ ADA     &  0.5525 &  5.2673 &  1.2557 &   4.2423 & -14.9647 &  2021-05-19 &  0.0984 \\
\ \ \ LTC     & -0.0808 &  3.9829 & -0.4957 &   7.2302 & -28.4608 &  2021-05-19 &  0.0962 \\
\ \ \ XRP     &  0.0176 &  7.1533 &  1.1411 &  19.9176 & -52.5698 &  2020-12-23 &  0.1354 \\
   \multicolumn{7}{l}{Crypto Indices with BTC Constituent}                                                                  \\
\ \ \ BITX    &  0.0591 &  1.0065 & -0.3453 &  12.1335 &  -7.6211 &  2021-05-19 &  0.0215 \\
\ \ \ CRIX    &  0.0777 &  0.9207 &  0.0164 &  13.5608 &  -6.9894 &  2021-05-19 &  0.0173 \\
\ \ \ BITW100 &  0.0848 &  1.2125 & -1.6397 &  19.7472 & -11.1357 &  2021-05-19 &  0.0274 \\
    \multicolumn{7}{l}{Crypto Indices without BTC Constituent}                                                              \\
\ \ \ BITW20  &  0.2608 &  3.6115 & -0.3555 &   4.2016 & -21.5430 &  2021-05-19 &  0.0804 \\
\ \ \ BITW70  &  0.2785 &  3.9157 & -0.6949 &   4.8047 & -24.3474 &  2021-05-19 &  0.0908 \\
\bottomrule
\end{tabular}}
\caption{Summary statistics of out-of-sample daily returns of hedged portfolios that minimize ES 5\%.}
\label{tab:ES5_rh}

\end{table}\begin{table}[!] \centering \resizebox{\textwidth}{!}{%
\begin{tabular}{lrrrrrrr} \toprule
         {} &    Mean \% &     Std \% &      Skew &       Kurt &         MD \% &     MD date & ES 1\% \\
\midrule
     \multicolumn{7}{l}{Individual Cryptos}                                                                                 \\
\ \ \ BTC     &  0.0148 &  0.3476 & -0.8354 &   3.3054 &  -1.6225 &  2020-11-30 &  0.0234 \\
\ \ \ ETH     &  0.3080 &  3.8954 &  0.9840 &   7.4947 & -18.7625 &  2021-05-19 &  0.1299 \\
\ \ \ ADA     &  0.5016 &  5.4040 &  1.1008 &   3.9607 & -15.4481 &  2021-05-19 &  0.1463 \\
\ \ \ LTC     & -0.1029 &  4.1581 & -0.7757 &   7.4375 & -29.1727 &  2021-05-19 &  0.1647 \\
\ \ \ XRP     & -0.0200 &  7.2887 &  1.1121 &  18.8732 & -52.5700 &  2020-12-23 &  0.2516 \\
   \multicolumn{7}{l}{Crypto Indices with BTC Constituent}                                                                  \\
\ \ \ BITX    &  0.0598 &  1.0312 & -0.4410 &  11.5863 &  -7.7424 &  2021-05-19 &  0.0411 \\
\ \ \ CRIX    &  0.0835 &  0.9461 & -0.0361 &  12.4047 &  -7.0203 &  2021-05-19 &  0.0350 \\
\ \ \ BITW100 &  0.0781 &  1.2640 & -1.9645 &  21.8836 & -11.9263 &  2021-05-19 &  0.0593 \\
    \multicolumn{7}{l}{Crypto Indices without BTC Constituent}                                                              \\
\ \ \ BITW20  &  0.2538 &  3.6323 & -0.4086 &   4.4462 & -21.9866 &  2021-05-19 &  0.1282 \\
\ \ \ BITW70  &  0.2660 &  3.9320 & -0.7598 &   5.0050 & -24.4764 &  2021-05-19 &  0.1535 \\
\bottomrule
\end{tabular}
}
\caption{Summary statistics of out-of-sample daily returns of hedged portfolios that minimize ES 1\%.}
\label{tab:ES1_rh}

\end{table}\begin{table}[!] \centering \resizebox{\textwidth}{!}{%
\begin{tabular}{lrrrrrrr} \toprule
         {} &    Mean \% &     Std \% &      Skew &       Kurt &         MD \% &     MD date & ERM k=10 \\
\midrule
     \multicolumn{7}{l}{Individual Cryptos}                                                                                 \\
\ \ \ BTC     &  0.0223 &  0.3221 & -1.0008 &   3.4153 &  -1.5242 &  2020-11-30 &    0.0057 \\
\ \ \ ETH     &  0.3117 &  3.8679 &  1.0345 &   7.5751 & -18.8729 &  2021-05-19 &    0.0491 \\
\ \ \ ADA     &  0.5722 &  5.3590 &  1.4203 &   4.6970 & -14.3885 &  2021-01-08 &    0.0700 \\
\ \ \ LTC     & -0.0512 &  3.8812 & -0.2929 &   7.7022 & -28.0879 &  2021-05-19 &    0.0616 \\
\ \ \ XRP     &  0.0155 &  7.1579 &  1.1244 &  19.8583 & -52.5689 &  2020-12-23 &    0.0787 \\
   \multicolumn{7}{l}{Crypto Indices with BTC Constituent}                                                                  \\
\ \ \ BITX    &  0.0590 &  1.0078 & -0.4427 &  13.0839 &  -7.8581 &  2021-05-19 &    0.0127 \\
\ \ \ CRIX    &  0.0840 &  0.9087 &  0.0488 &  14.5501 &  -7.0530 &  2021-05-19 &    0.0100 \\
\ \ \ BITW100 &  0.0853 &  1.2032 & -1.6522 &  20.5562 & -11.1846 &  2021-05-19 &    0.0153 \\
    \multicolumn{7}{l}{Crypto Indices without BTC Constituent}                                                              \\
\ \ \ BITW20  &  0.2564 &  3.6009 & -0.3446 &   4.2152 & -21.5920 &  2021-05-19 &    0.0503 \\
\ \ \ BITW70  &  0.2818 &  3.9074 & -0.6952 &   4.8745 & -24.5250 &  2021-05-19 &    0.0557 \\
\bottomrule
\end{tabular}}
\caption{Summary statistics of out-of-sample daily returns of hedged portfolios that minimize ERM $k=10$.}
\label{tab:ERM_rh}

\end{table}


%\newpage
%\begin{landscape}
%\begin{figure}[h]
%   \centering
%   \includegraphics[width=\linewidth]{_pics/Rhs.pdf}
%   \caption{Out-of-Sample Returns of Hedged Portfolio of Copulas and Risk Reduction Objectives.
%   \href{http://www.quantlet.com/}{\includegraphics[width=20pt]{_pics/qletlogo_tr.png}}}
%   \label{fig:OOSRH}
%\end{figure}
%\end{landscape}
%\newpage
%
%\newpage
%\begin{landscape}
%\begin{figure}[th]
%   \centering
%   \includegraphics[width=\linewidth]{_pics/OHRs.pdf}
%   \caption{Optimal Hedge Ratio Obtained from Combinations of Copula and Risk Reduction Objective.
%   \href{http://www.quantlet.com/}{\includegraphics[width=20pt]{_pics/qletlogo_tr.png}}}
%   \label{fig:OHRs}
%\end{figure}
%\end{landscape}
%\newpage


%\section{Data}\label{subsec:data}
In the empirical analysis, we consider the risk reduction capability of the BTC-future (BTCF) on five cryptos
, BTC, ETH, ADA, LTC, and XRP, and five crypto indexes, BITX, BITW100, CRIX, BITW20, and BITW70,
For each of the 10 hedging portfolios, a crypto or index is considered as the spot and held in a unit size long position,
and the BTCF is held in short position of OHR unit in order to reduce the risk of the spot.
All the hedging portfolios are cross asset hedging except the BTCF portfolio.
ETH, ADA, LTC, and XRP are popular cryptos tradable in various exchanges and have large market capitalization.
BITX, BITW100, and CRIX are market-cap weighted crypto indexes with BTC as constituent.
BITX and BITW100 tracks the total return of the 10 and 100 cryptos with largest market-cap respectively.
CRIX decides the number of constituents by AIC and track that number of cryptos with largest market-cap.
In our case, the number of constituents in CRIX is 5.
BITW20 is also a market-cap weighted crypto index but with 20 largest market-cap cryptos outside the constituents of
BITX.
BITW70 has the same construction as BITW20 but with 70 largest market-cap cryptos outside BITX and BITW20.
Therefore, BTC is excluded as constituent in BITW20 and BITW70. \medskip

We collect the spots' and BTCF's daily price at 15:00 US Central Time (CT).
The reason of choosing this particular time is that the CME group determines the daily settlements for BTCFs based on the trading activities on CME Globex between 14:59 and 15:00 CT.
15:00 CT is also the reporting time of the daily closing price by the Bloomberg Terminal (BBT).
Cryptos data are collected from a data provider called Tiingo.
Tiingo aggregates crypto OHLC (open, high, low, and close) prices fed by APIs from various exhcanges.
Tiingo covers major exchanges, e.g. Binance, Gemini, Poloniex etc., so Tiingo's aggregated OHLC price is a good representation a market tradable price.
For each crypto, we match the opening price at 15:00 CT from Tiingo with the daily closing price of BTCF from BBT.
Since CRIX is not available at 15:00 CT, we recalculate a hourly CRIX using the monthly constituents weights and the hourly OHLC price data collected from Tiingo.
BITX, BITW20, BITW70, and BITW100 are collected from the official website of their publisher Bitwise.com.
The daily reporting time of the Bitwise indexes is 15:00 CT. \medskip

At the time of writing, the CRIX' is undergoing the listing process on the S\&P Dow Jones Indices,
the official CRIX data will then be calculated with Lukka Prime Data and available to public via S\&P.



















%\francis{\em This section is under construction}
%Cryptocurrenices are traded around the clock, but CME future are traded from
%Sunday to Friday from 05:00 p.m. to 04:00 p.m. U.S. central time.
%We match the timestamps and timezones of different data sources.
%
%
%\begin{table}[htbp]
%    \centering
%    \begin{tabularx}{\textwidth}{s|CCCCCCCC}
%      \hline\hline
%     \# & Asset & Data Source & Type & Tradable at CT\footnotemark & Tradable at CET\footnotemark during CST\footnotemark & Tradable at CET during CDT\footnotemark & Tradable at UTC during CST & Tradable at UTC during CDT\\       \hline
%      1 & Bitcoin & Coingecko API & Hourly Close &  & 11:00pm D+0 & 11:00pm D+0 & 10:00pm D+0$^*$ &10:00pm D+0$^*$ \\\hline
%      2 & CME Future & Bloomberg & Daily Open & 05:00pm D-1 & 00:00am D+0$^*$ & 00:00am D+0$^*$ & 11:00pm D-1 & 10:00pm D-1 \\       \hline
%      3 & CME Future & Bloomberg & Daily Close & 04:00pm D+0& 11:00pm D+0$^*$ & 11:00pm D+0$^*$ & 10:00pm D+0 & 09:00pm D+0\\       \hline
%      4 & CRIX & IRTG (from Coingecko) & Index &  &  &  & & 00:00am D+0$^*$\\\hline
%    \end{tabularx}
%    \caption{$^*$ indicates the timestamp of raw data from data source. }
%    \label{tab:table}
%\end{table}
%
%\addtocounter{footnote}{-3}
%\footnotetext{CT stands for U.S. Central Time. It represents two observances of time, the Central Standard Time (CST) and the Central Daylight Time (CDT)}
%\addtocounter{footnote}{1}
%\footnotetext{CET stands for Central European Time. It is one hour ahead UTC.}
%\addtocounter{footnote}{1}
%\footnotetext{CST is six hours behind UTC.}
%\addtocounter{footnote}{1}
%\footnotetext{CDT is five hours behind UTC.}
%
%Hedging Pair 1 is hedging \#1 (Bitcoin Spot) with \#3 (CME future).
%The time difference between the two prices is zero.
%They are both adjusted to CET time:
%\#1 by pandas.Series.dt.tz\_convert; \#3 by retrieving data from Bloomberg Terminal located in Berlin. \medskip
%
%Hedging Pair 2 is hedging \#4 (CRIX) with \#2 (CME future).
%We observe \#2 two hours and one hour before \#4 during CST and CDT respectively.
%
%
%\subsection{Time Difference}\label{subsec:time-difference}
%\begin{table}[h]
%    \centering
%
%\begin{tabular}{lrrrr}
%\toprule
%{} &     Open &     High &      Low &    Close \\
%\midrule
%2021-02-02 23:00 &  36360.0 &  38155.0 &  36240.0 &  37790.0 \\
%2021-02-01 23:00 &  34205.0 &  36665.0 &  34070.0 &  36535.0 \\
%2021-01-31 23:00 &  33715.0 &  35280.0 &  32800.0 &  34265.0 \\
%2021-01-28 23:00 &  33995.0 &  39530.0 &  32590.0 &  35180.0 \\
%2021-01-27 23:00 &  31005.0 &  33710.0 &  30350.0 &  33085.0 \\
%\bottomrule
%\end{tabular}
%       \caption{CME Bitcoin Future Raw Data}
%    \label{tab:table0} \medskip
%
%    \begin{tabular}[width=\textwidth]{llrrrr}
%\toprule
% &                      date &           CRIX &   future &  log return CRIX &  log return future \\
%\midrule
%0 & 2021-02-04  &  104518.468839 &  38080.0 &         0.054757 &           0.046220 \\
%1 & 2021-02-03  &   98949.179255 &  36360.0 &         0.059741 &           0.061097 \\
%2 & 2021-02-02  &   93210.948461 &  34205.0 &         0.002204 &           0.014429 \\
%3 & 2021-02-01  &   93005.711051 &  33715.0 &         0.013628 &          -0.008271 \\
%4 & 2021-01-29  &   91746.863103 &  33995.0 &         0.081917 &           0.092065 \\
%\bottomrule
%    \end{tabular}
%    \caption{CRIX \#4 with Opening price of CME Bitcoin future \#2 and their log returns}
%    \label{tab:table2} \medskip
%
%\begin{tabular}{llrrrr}
%\toprule
%{} &                      date &           CRIX &   future &  log return CRIX &  log return future \\
%\midrule
%0 & 2021-02-05  &  103348.488555 &  38220.0 &        -0.011257 &           0.011314 \\
%1 & 2021-02-04  &  104518.468839 &  37790.0 &         0.054757 &           0.033774 \\
%2 & 2021-02-03  &   98949.179255 &  36535.0 &         0.059741 &           0.064146 \\
%3 & 2021-02-02  &   93210.948461 &  34265.0 &        -0.016175 &          -0.026353 \\
%4 & 2021-01-30  &   94730.919657 &  35180.0 &         0.032007 &           0.061398 \\
%\bottomrule
%\end{tabular}
%    \caption{CRIX \#4 with Closing price of CME Bitcoin future \#3 shifted for one day (D-1) and their log returns}
%    \label{tab:table3}
%\end{table}
%
%\clearpage
%\begin{figure}[ht]
%    \centering
%    \includegraphics[scale=.35]{_pics_notes/CRIX_future_Open_Close.pdf}
%    \end{figure}
%
%Kendall's tau between CRIX and future Close is 0.608429;\\
%Kendall's tau between CRIX and future Open is 0.673266; we pick this unless we have hourly CRIX.
%
%\subsection{Statistics of Percentage Difference Between CME Bitcoin future Open Price and Last Close Price}
%
%$$\text{diff} = \frac{\text{Open}_{t} - \text{Close}_{t-1}} {\text{Close}_{t-1}}$$
%
%Mean of diff = 0.00236\\
%Std of diff = 0.02206\\
%Max of diff = 0.16394 \\
%UQ of diff = 0.00814 \\
%Median of diff = 0.00132\\
%LQ of diff = -0.00412 \\
%Min of diff = -0.12190 \\




%%% Local Variables:
%%% mode: latex
%%% TeX-master: "SRM"
%%% End:

