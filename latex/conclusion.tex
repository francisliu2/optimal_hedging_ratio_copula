\section{Conclusion and Outlook}\label{sec:conclusion-and-outlook}
We study the effectiveness of hedging cryptos and crypto indices with
Bitcoin futures.
To accomodate different risk appetites and scenarios, a variety of
commonly used risk measures are considered to determine the optimal
hedge ratio. The risk measures comprise variance, value-at-risk at
the confidence levels 95\% and 99\%, expected shortfall 95\% and 99\%,
and the exponential risk measure with parameter $k=10$.

At the time of writing, the crypto market is a vibrant and
fast-developing market, causing cryptos to have complex and, possibly,
time  changing dependence structures with the Bitcoin futures.
As a consequence, the dependence between the cryptos and the futures
contract plays an important role in hedging as it determines the
distribution of the portfolio returns. We therefore consider various
copulae, a flexible statistical tool that separate modelling of the
marginals and the dependence structure of multivariate random
vectors. To address the potential time changing dependence, we
periodically re-calibrate the copula models and find the best-fitting
copula via  AIC. 

An extensive out-of-sample backtest suggests that the Bitcoin futures
is consistently capable of hedging BTC and BTC-involved indices, 
BITX, CRIX, and BITW100, under different risk minimisation objectives
and copula models. The mean-square errors (MSEs) and lower
semi-variances (LSVs) of the resulting portfolios are
indistinguishably at a low level except for the Frank copula. 
On the other hand, the AIC procedure favours the $t$-copula because it 
captures the tail dependence feature of the data. \natp{Compared to
  the unhedged cases (was: Unsurprisingly)}, the
portfolios' out-of-sample maximum drawdowns are significantly reduced. 

Contrarily, we observe diverse results of the capability of BTC
futures to hedge other cryptos and crypto indices that exclude Bitcoin. 
In general, ES 95\% and VaR 95\% perform better than their 99\%
counterparts. In particular, minimising ES 99\% leads to relatively
high MSEs and LSVs regardless of the copula in use. The ES 99\% and
VaR 99\% even result in out-of-sample maximum drawdowns that are
higher than that of the 95\% counterparts in some portfolios, 
for example in the ETH- and LTC-BTCF portfolio.
Therefore, we conclude that overly emphasising tail risks by choosing
extreme tail risk measures does not lead to a promising hedge in a
cross-hedging setting. 

The AIC procedure mainly flavours the rotated Gumbel and the NIG
factor copula in modelling cryptos other than BTC and 
indices excluding BTC. This reflects the idiosyncratic nature of
downward movements in the crypto market. Interestingly, the best-fitting
copula does not necessary leads to best performing portfolio in MSE or
LSV, as is the case for ADA.
This is what we call the discrepancy between the copula selection
result and the MSE-LSV results. 
We suspect the discrepancy is due to the static linear nature of the
hedge as the sole hedge instrument is a futures contract. 
Although copulae are flexible to model complex dependence structures
by emphasising a number of important features such as lower tail
dependence and radial symmetry, the simple linear hedge is very
limited in its flexibility to address this complex dependence.
Including liquidly traded derivatives with non-linear payoffs, such as
options, might be a possibility to improve the hedge quality for these
cryptos and portfolios. 

% A possible research direction would be hedging the linear part with futures and non-linear part with options, e.g. the BTC options.



%%% Local Variables:
%%% mode: latex
%%% TeX-master: "SRM"
%%% End:
