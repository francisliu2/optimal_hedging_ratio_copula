\section{Optimal hedge ratio}
\label{sec:optimal-hedge-ratio}
We form a portfolio with two assets, a spot asset and a future contract, for example Bitcoin spot and CME Bitcoin future.
Our objective is to minimize the risk of the exposure in the spot.
To keep a simple portfolio setting, we long one unit of the spot and short $h$ unit of the future with $h \in [0, \infty)$.
Let $r^S$ and $r^F$ be the log returns of the spot and future price, the log return of the portfolio is
\[r^h = r^S -h r^F.\]
We call this portfolio a hedged portfolio: the price movement of spot is hedged by the price movement of future.
\medskip

%\footnote{The log return is simply the log of geometric difference between price at time $t$ and $t-1$ $\log \frac{\text{Price}_t}{\text{Price}_{t-1}}$.
%The index $t$ follow the trading calendar.
%The handling of calendar is important, we will discuss that in the data section.}


Risk is measured by risk measures.
Assume the payoff $r^h$ of a hedge portfolio lives in a probability space, $r^h \in L(\Omega,\F, \P)$,
and there is a risk measure on $r^h$ $\rho: r^h \mapsto \mathbb{R}$.
We are looking for an optimal hedge ratio (OHR) $h^*$ which minimizes a risk measure:

\[h^* = \argmin_h \rho(r^h).\]

Most risk measures are defined as functionals of the portfolio loss distribution $F_{r^h}$, i.e. $\rho: F_{r^h} \mapsto \mathbb{R}$.
For example, Value-at-Risk (VaR) is simply the quantile of $r^h$ multiply with negative one $\text{VaR}_{1-\alpha} = -F_{r^h}^{(-1)}(1-\alpha) = -\inf\{x \in \mathbb{R}: 1-\alpha \leq F_{r^h}(x) \}$, where $\alpha$ is a parameter chosen by investor.
We need the knowledge of $F_{r^h}$ in order to measure risk.
By convolution of random variables~\citep{WKHMVA}, $f_{r^h}(z) = \int_{-\infty}^{\infty}f_{r^S, -hr^F}(x, z-x)dx$, where
$f_{r^S, -hr^F}$ is the joint pdf of $r^S$ and $-hr^F$.
Obviously the cdf of $r^h$ and the risk measure depend on the joint distribution of $r^S$ and $-hr^F$.\medskip

Optimising $h$ according to $f_{r^S,-hr^F}$ is unfavorable in a sense that one needs to calibrate a new joint pdf $f_{r^S, -hr^F}$ when updating $h$.
This is too time consuming and unnecessary.
Another problem of using the joint pdf is that one lacks the flexibility to model the margins.
A joint pdf completely determines the form of its marginals, for example, margins of a bivariate $t$-distribution are
univariate $t$-distributions.\medskip

To overcome such a problem, we use copulae.
The benefit of using copulae is two folded.
First, copulae allow us to model the margins and dependence structure separately, see Sklar's Theorem.
Second, copulae are invariant under strictly monotone increasing function \citep{schweizer1981nonparametric}, see lemma below.

\begin{theorem}[Hoeffding Sklar Theorem]
  Let $F$ be a joint distribution function with margins $F_X,
  F_Y$. Then, there exists a copula $C:[0,1]^2 \mapsto [0,1]$ such
  that, for all $x,y\in \R$
  \begin{equation}
    \label{eq:4}
    F(x,y)=C\{F_X(x), F_Y(y)\}.
  \end{equation}
  If the margins are continuous, then $C$ is unique; otherwise $C$ is
  unique on the range of the margins.

  Conversely, if $C$ is a copula and $F_X, F_Y$ are univariate
  distribution functions, then the function $F$ defined by (\ref{eq:4})
  is a joint distribution function with margins $F_X, F_Y$.
\end{theorem}

Indeed, many basic results about copulae can be traced back to early works of Wassily Hoeffding \citep{hoedffding1940, hoedffding1941}.
The works aimed to derive a measure of relationship of variables which is invariant under change of scale.
See also \citet{hoeffding2012collected} for English translations of the original papers written in German.
The following lemma is not hard to prove.

\begin{lemma}
  \begin{align}
  C_{X, hY}\{F_X(s),F_{hY}(t)\} = C_{X, Y}\{F_X(s),F_{Y}(t/h)\}.
    \end{align}
  \end{lemma}


%The optimal hedge ratio is $h^\ast = \argmin_h \rho(Z)$, that is the best ratio that can minimize the risk of a hedged portfolio measured in terms of $\rho$ . \medskip
Leveraging the two features of copulae, \citet{barbi2014copula} introduces the distribution of linear combination of random variables using copulae.
We slightly edit the Corollary 2.1 of their work and yield the following correct expression of the distribution.

\begin{proposition}
  \label{prop:dfrh}
  Let $X$ and $Y$ be two real-valued continuous random
  variables on a
  probability space $(\Omega, \F, \p)$ with
  absolutely continuous copula $C_{X, Y}$ and marginal distribution functions $F_{X}$
  and $F_{Y}$. Then, the distribution function of of $Z$ is given by 
  \begin{equation}
    \label{eq:3}
    F_{Z}(z) = 1- \int^1_0 D_1 C_{X, Y}
    \left[ u, F_{Y} \left\{ \frac{F^{(-1)}_{X}(u)-z}{h} \right\}
    \right]\, d u.
  \end{equation}
\end{proposition}
Here, $F^{(-1)}$ denotes the inverse of $F$, i.e., the quantile
function. \medskip

Here $D_1 C(u,v)=\displaystyle \frac{\partial}{\partial u} C(u,v)$ see e.g.\ Equation (5.15) of
\citep{McNeil2005}:
\begin{equation}
  \label{eq:1}
  D_1 C_{X,Y}\{F_X(x), F_Y(y)\} = \p(Y\leq y|X=x).
\end{equation}
\begin{proof}
  Using the identity (\ref{eq:1}) gives
  \begin{align*}
    F_{Z}(z) &= \p(X - h Y\leq z) %
                 = \mbox{\sf E}\left\{\p\left(Y\geq \frac{X-z}{h}\Big|
                 X\right)\right\}\\[10pt]
               &= 1-\mbox{\sf E}\left\{\p\left(Y\leq \frac{X-z}{h}\Big|
                 X\right)\right\}% \\[10pt]
               = 1- \int_0^1 D_1 C_{X, Y}\left[u,
                 F_{Y}\left\{\frac{F^{(-1)}_{X}(u) -
                 z}{h}\right\}\right]\, d u.
  \end{align*}
  \end{proof}

%In addition to~\cite{barbi2014copula} we propose a more handy
%expression for the pdf of $Z$.
%\natp{\em [Please double-check the ``+'' signs in the second equation.]}\ \francis{ \em [the + sign is correct.]}

\begin{corollary} Given the formulation of the above portfolio, the pdf of $Z$ can be written as
  \begin{align}
  f_{Z}(z) &= \left|\frac{1}{h}\right|\int_0^1 c_{X, Y} \left[
  F_{Y}\left\{\frac{F^{(-1)}_{X}(u)-z}{h}\right\}, u
  \right]
   \cdot
  f_{Y}
  \left\{\frac{F^{(-1)}_{X}(u)-z}{h}\right\} du \label{eq:density1}
  \end{align}, or
    \begin{align}
      f_{Z}(z)
      = \int_0^1 c_{X, Y} \left[
      F_{X}\left\{z + h F^{(-1)}_{Y}(u)\right\}, u
      \right]
       \cdot
      f_{X}
      \left\{
      z+ hF^{(-1)}_{Y}(u)
      \right\} du. \label{eq:density2}
  \end{align}
  \end{corollary}
The two expressions are equivalent.
Notice that the pdf of $Z$ in the above proposition is readily accessible as long as we have
the copula density and the marginal densities.
The proof and a generic expression can be found in the
appendix. \medskip

\section{Empirical Procedure}\label{sec:empirical-procedure}
We discuss the empirical procedure to obtain the OHR.
%to evaluate the BTCF hedging effectiveness to various crypto assets and indices. \medskip

First, we split the time series of spots and future into sets of training and testing data.
The training data is the first 300 observations and its corresponding testing data is the consecutive 5 observations.
We roll 5 observations forward to obtain the next training and test data and repeat this until the end of the time series.
Notice that the testing data are non-overlapping. \medskip

Next, we obtain the OHR as follows:

\begin{enumerate}
  \item \textbf{Construct Univariate Kernel Density Function (KDE)}.
  With the training data, we obtain the spot and future's univariate kernel density function using the Gaussian kernel
  with bandwidth determined by the refined plug-in method \citep[section 3.3.3]{hardle2004nonparametric}.
  \item \textbf{Calibrate Copulae}.
  We then calibrate copulae outlined in \ref{subsec:copulae} via the method of moments described in \ref{subsec:simulated-method-of-moments}.
  \item \textbf{Select Copula}.
  We compute the Akaike Information Criterion.
  The copula with the lowest AIC is used for the next step.
  Discussion of this step is in \ref{subsec:copula-selection}.
  \item \textbf{Search for OHR}.
  We search OHRs using different risk measures as loss function numerically by drawing samples from the selected copula and KDEs.
  Risk measures used as risk reduction objectives are outlined in \ref{subsec:spectral-risk-measures}
  \item \textbf{Obtain testing log-return of hedged portfolio}.
  We apply the OHRs to the testing data $r_h = r_s - h^* r_f$.
  \end{enumerate}



\subsection{Copulae}
We test different copulae's ability to model crypto-currency data,
they include Gaussian-, $t$-, Frank-, Gumbel-, Clayton-, Plackett-, mixture, and factor copula.

%Copula is a function represents the multivariate structure of random variables.
%
%
%Frechet-Hoeffding lower bound $\bm{W}(u,v) = \min(u,v)$, Frechet-Hoeffding upper bound $\bm{M}(u,v) = \max(u+v-1,0)$,
%and product copula $\bm{\Pi}=uv$ are three important special instants of copulas.
%They describe the perfect counter-dependence, perfect dependence, and independence of two random variables, respectively.
%The inequality $\bm{W}(u,v) \leq \bm{C}(u,v) \leq \bm{M}(u,v)$ holds for every copula $\bm{C}$ ad every $(u,v) \in \mathbb{I}^2$ (Nelsen 2.2.5).

\subsubsection{Ellpitical Copulae}\label{sec:ellpitical-copulas}
Elliptical copulae are copulae of elliptical distributions.
Gaussian copula is the copula associated with multivariate normal distribution.
The Gaussian copula has a form
    \begin{align}
        \bm{C}(u,v) &= \Phi_{2, \rho}\{\Phi^{-1}(u), \Phi^{-1}(v)\} \nonumber \\
                    &= \int_{-\infty}^{\Phi^{-1}(u)}
                       \int_{-\infty}^{\Phi^{-1}(v)}
                       \frac{1}{2\pi\sqrt{1-\rho^2}}
                       \exp{\left(
                       \frac{s^2-2\rho st+t^2}{2(1-\rho^2)}
                       \right)} ds dt
        \end{align}
where $\Phi_{2, \rho}$ is the cdf of bivariate Normal distribution with zero mean, unit variance, and correlation $\rho$,
and $\Phi^{-1}$ is quantile function univariate standard normal distribution.
The Gaussian copula density is
\begin{align}
    \bm{c}_\rho(u,v) &= \frac{\varphi_{2,\rho}\{\Phi^{-1}(u), \Phi^{-1}(v)\}}
                        {\varphi\{\Phi^{-1}(u)\} \cdot \varphi\{\Phi^{-1}(v)\}} \nonumber \\
                &= \frac{1}{2\pi\sqrt{1-\rho^2}}\exp\left(
                   -\frac{u^2 - 2\rho uv + v^2}{2(1-\rho^2)}
                   \right),
    \end{align}
where $\phi_{2,\rho}(\cdot)$ is the density of bivariate Normal distribution with zero mean,
unit variance,
and correlation $\rho$,
and, $\phi(\cdot)$ the density of standard normal distribution.\medskip

The Kendall's $\tau_K$ and Spearman's $\rho_S$ of a bivariate Gaussian Copula are
    \begin{align}
        \tau_K(\rho) = \frac{2}{\pi}\arcsin\rho
        \end{align}
    \begin{align}
        \rho_S(\rho) = \frac{6}{\pi}\arcsin\frac{\rho}{2}
        \end{align}\medskip

The $t$-copula is associated with multivariate t distribution.
The $t$-Copula takes a form
\begin{align}
        \bm{C}(u,v) &= \bm{T}_{2, \rho, \nu}\{T^{-1}_\nu(u), T^{-1}_\nu(v)\} \nonumber \\
            &= \int_{-\infty}^{T^{-1}_\nu(u)}
               \int_{-\infty}^{T^{-1}_\nu(v)}
            \frac{\Gamma\left(\frac{\nu+2}{2}\right)}
            {\Gamma\left(\frac{\nu}{2}\right)\pi\nu\sqrt{1-\rho^2}}\\
           & \left(
        1+\frac{s^2-2st\rho+t^2}{\nu}
        \right)^{-\frac{\nu+2}{2}} ds dt,
    \end{align}
where $\bm{T}_{2, \rho, \nu}(\cdot, \cdot)$ denotes the cdf of bivariate t distribution with scale parameter $\rho$ and degree of free $\nu$,
$T^{-1}_\nu(\cdot)$ is the quantile function of a standard t distribution with degree of freedom $\rho$.

The copula density is
\begin{align}
    \bm{c}(u,v) &= \frac{\bm{t}_{2, \rho, \nu}\{T^{-1}_\nu(u), T^{-1}_\nu(v)\}}
    {t_\nu\{T^{-1}_\nu(u)\}\cdot t_\nu\{T^{-1}_\nu(v)\}},
    \end{align}
where $\bm{t}_{2,\rho, \nu}$ is the density of bivariate t distribution,
and $t_\nu$ the density of standard t distribution.\medskip

Like all the other elliptical copula, t copula's Kendall's $\tau$ is same to that of Gaussian copula (Demarta and reference therein).

\subsubsection{Archimedean Copulae}\label{sec:archimedean-copula}
The Archimedean copulae forms a large class of copulas with many convenient features.\medskip

In general, they take a form
\begin{align}
    \bm{C}(u,v)= \psi^{-1}\{\psi(u), \psi(v)\},
    \end{align}
where $\psi:[0,1] \rightarrow [0,\infty)$ is a continuous, strictly decreasing and convex function such that
$\psi(1)=0$ for any permissible dependence parameter $\theta$. $\psi$ is also called generator.
$\psi^{-1}$ is the inverse the generator.\medskip

The Frank copula (B3 in \citet{joe1997multivariate}) is a radial symmetric copula and cannot produce any tail dependence.
It takes the form
\begin{align}
    \bm{C}_{\theta}(u,v) &= \frac{1}{\log(\theta)}
    \log \left\{
    1 + \frac{(\theta^u-1)(\theta^v-1)}{\theta-1}
    \right\}
    \end{align}
where $\theta \in [0, \infty]$ is the dependency parameter.
$\bm{C}_1 = \bm{M}$, $\bm{C}_1 = \bm{\Pi}$, and $\bm{C}_\infty = \bm{W}$.

The Copula density is
\begin{align}
    \bm{c}_{\theta}(u,v) &= \frac{(\theta-1)\theta^{u+v}\log(\theta)}
    {\theta^{u+v}-\theta^u-\theta^v+\theta}
    \end{align}\medskip

Frank copula has Kendall's $\tau$ and Spearman's $\rho$ as follow:
\begin{align}
    \tau_K(\theta) = 1-4\frac{D_1\{-\log(\theta)\}}{\log(\theta)},
    \end{align}
and
\begin{align}
    \rho_S(\theta) = 1-12\frac{D_2\{-\log(\theta)\} - D_1\{\log(\theta)\}}{\log(\theta)},
    \end{align}
where $D_1$ and $D_2$ are the Debye function of order 1 and 2.
Debye function is $D_n = \frac{n}{x^n}\int_0^x\frac{t^n}{e^t-1}dt$.\medskip

Gumbel copula (B6 in \citet{joe1997multivariate}) has upper tail dependence with the dependence parameter
$\lambda^U = 2-2^{\frac{1}{\theta}}$ and displays no lower tail dependence.
\begin{align}
    \bm{C}_{\theta}(u,v) &= \exp{-\{
    (-\log(u))^\theta +(-\log(v))^\theta
    \}^{\frac{1}{\theta}}},
    \end{align}
where $\theta \in [1,\infty)$ is the dependence parameter.\medskip
While Gumbel copula cannot model perfect counter dependence (ref), $\bm{C}_{1} = \bm{\Pi}$ models the independence,
and $\lim\limits_{\theta \to \infty} \bm{C}_\theta = \bm{W}$ models the perfect dependence.

%The copula density takes the form
%\begin{align}
%        f
%    \end{align}

  \begin{align}
    \tau_K(\theta) =\frac{\theta-1}{\theta}
    \end{align}

The Clayton copula, by contrast to Gumbel copula,
generates lower tail dependence in a form $\lambda^L = 2^{-\frac{1}{\theta}}$,
but cannot generate upper tail dependence.\medskip

The Clayton copula takes a form
\begin{align}
    \bm{C}_{\theta}(u,v) &= \left[
    \max\{u^{-\theta}+v^{-\theta}-1,0\}\right]^{-\frac{1}{\theta}},
    \end{align}
where $\theta \in (-\infty, \infty)$ is the dependency parameter.
$\lim\limits_{\theta \to -\infty} \bm{C}_\theta = \bm{M}$, $\bm{C}_0 = \bm{\Pi}$, and $\lim\limits_{\theta \to \infty} \bm{C}_\theta = \bm{W}$.\medskip

Its Kendall's $\tau$ is
\begin{align}
    \tau_K(\theta) =\frac{\theta}{\theta+2}.
    \end{align}\medskip

    \natp{The Plackett copula is not an Archimedean copula, so it
      should be moved somwhere else.}
    
The Plackett copula has an expression
\begin{align}
    \bm{C}_{\theta}(u,v) &= \frac{1+(\theta-1)(u+v)}{2(\theta-1)}
                         - \frac{\sqrt{\{
    1+(\theta-1)(u+v)\}^2 - 4uv\theta(\theta-1)}}{2(\theta-1)}
    \end{align}
\begin{align}
    \rho_S(\theta) = \frac{\theta+1}{\theta-1} - \frac{2\theta \log \theta}{(\theta-1)^2}
    \end{align}\medskip

We include Placket copula in our analysis as it possesses a special property,
the cross-product ratio is equal to the dependence parameter
\begin{align}
    &\phantom{=} \frac{\p(U \leq u, V \leq v) \cdot \p(U > u, V > v)}
    {\p(U \leq u, V > v) \cdot \p(U > u, V \leq v)}\nonumber\\
    &= \frac{\bm{C}_\theta(u,v)\{1-u-v+\bm{C}_\theta(u,v)\}}{\{u-\bm{C}_\theta(u,v)\}\{v-\bm{C}_\theta(u,v)}\nonumber\\
    &= \theta.
    \end{align}\medskip
That is, the dependence parameter is equal to the ratio between number of concordence pairs and number of discordence pairs of a bivariate random variable.
\subsubsection{Mixture Copula}\label{sec:mixture-copula}
Mixture copula is a linear combination of copulas.
It allows us to model the dependence structure in a more flexible manner.

For a 2-dimensional random variable $\bm{X}=(X_1,X_2)^\top$,
its distribution can be written as linear combination $K$ copulas
\begin{align}
    \p(X_1 \leq x_1, X_2 \leq x_2) = \sum_{k=1}^K p^k \cdot \bm{C}^{(k)}\{F^{(k)}_{X_1}(x_1;\bm{\gamma}^{(k)}_1),
    F^{(k)}_{X_2}(x_2;\bm{\gamma}^{(k)}_2); \bm{\theta^{(k)}}\}
    \end{align}
where $p^{(k)} \in [0,1]$ is the weight of each component,
$\bm{\gamma}^{(k)}$ is the parameter of the marginal distribution in the $k^\text{th}$ component,
and $\bm{\theta^{(k)}}$ is the dependence parameter of the $k^\text{th}$ component.
We also restrict the weight so that $\sum_{k=1}^K p^{(k)}=1$.
Analysis of mixture copula with higher dimension can be found in Vrac et. al. (2011).

We deploy a simplified version of the above representation by assuming the maringals of $\bm{X}$ are not mixture.
By Sklar's theorem we write
\begin{align}
    \bm{C}(u,v)= \sum_{k=1}^K p^{(k)} \cdot \bm{C}^{(k)}\{F^{-1}_{X_1}(u),
    F^{-1}_{X_2}(v); \bm{\theta^{(k)}}\}.
    \end{align}\medskip

The copula density is again a linear combination of copula density
\begin{align}
    \bm{c}(u,v)= \sum_{k=1}^K p^{(k)} \cdot \bm{c}^{(k)}\{F^{-1}_{X_1}(u),
    F^{-1}_{X_2}(v); \bm{\theta^{(k)}}\}.
    \end{align}\medskip

While Kendall's $\tau$ of mixture copula is not known in close form,
the Spearman's $\rho$ is

\begin{proposition}
    Let $\rho_S^{(k)}$ be the Spearman's $\rho$ of the $k^\text{th}$ component and $\sum_{k=1}^K p^{(k)}=1$ holds,
    the Spearman's $\rho$ of a mixture copula is
    \begin{align}
        \rho_S = \sum_{k=1}^K p^{(k)} \cdot \rho_S^{(k)}
        \end{align}
    \end{proposition}

\begin{proof}
    Spearman's $\rho$ is defined as (Nelsen)
    \begin{align}
        \rho_S = 12 \int_{\mathbb{I}^2} \bm{C}(s,t) ds dt - 3.
        \end{align}
    Rewrite the mixture copula into sumation of components
       \begin{align}
        \rho_S = 12 \int_{\mathbb{I}^2} \sum_{k=1}^K p^{(k)} \cdot \bm{C}^{(k)}(s,t) ds dt - 3.
        \end{align}
    \end{proof}

\begin{example}
    Frechet class can be seen as an example of mixture copula.
    It is a convex combinations of $\bm{W}$, $\bm{\Pi}$, and $\bm{M}$ \citep{Nelsen2007}
    \begin{align}
        \bm{C}_{\alpha, \beta}(u,v)
        = \alpha \bm{M}(u,v) +
        (1-\alpha-\beta)\bm{\Pi}(u,v)
        +\beta \bm{W}(u,v),
        \end{align}
    where $\alpha$ and $\beta$ are the dependence parameters, with $\alpha, \beta \geq 0$ and
    $\alpha+\beta \leq 1$.
    Its Kendall's $\tau$ and Spearman's $\rho$ are
    \begin{align}
        \tau_K(\alpha, \beta) = \frac{(\alpha - \beta)(\alpha+\beta+2)}{3}
        \end{align}
    , and
    \begin{align}
        \rho_S(\alpha, \beta) = \alpha - \beta
        \end{align}
    \end{example}\medskip
%Example 2 Gumbel-Clayton mixture
%Example 3 Hu 2006.

We use a mixture of Gaussian and independent copula in our analysis.
We write the copula
\begin{align}
    \bm{C}(u,v) = p\cdot \bm{C}^\text{Gaussian}(u,v) + (1-p)(uv).
    \end{align}
The corresponding copula density is
\begin{align}
    \bm{c}(u,v) = p\cdot \bm{c}^\text{Gaussian}(u,v) + (1-p).
    \end{align}

This mixture allows us to model how much "random noise" appear in the dependency structure.
In this hedging exercise, the structure of the "random noise" is not of our concern nor we can
hedge the noise by a two-asset portfolio.
However, the proportion of the "random noise" does affect the distribution of $R^h$ (see figure),
so as the optimal hedging ratio $h^\ast$ (see figure).
One can consider the mixture copula as a handful tool for stress testing.
Similar to this Gaussian mix Independent copula,
t copula is also a two parameter copula allow us to model the noise,
but its interpretation of parameters is not as intuitive as that of a mixture.
The mixing variable $p$ is the proportion of a manageable (hedgable?) Gaussian copula,
while the remaining proportion $1-p$ cannot be managed.

% ----------------
% --- Copulae's definition and properties ---
% ----------------

%! Author = francis
%! Date = 30.10.20


\subsection{Simulated Method of Moments}\label{subsec:simulated-method-of-moments}
This method is suggested by Oh and Patton (2013).
In this setting, rank correlation e.g. Spearman's $\rho$ or Kendall's $\tau$,
and quantile dependence measures at different levels $\lambda_q$
are calibrated against their empirical counterparts.\medskip

Spearman's rho, Kendall's tau, and quantile dependence of a pair $(X,Y)$
with copula $C$ are defined as
\begin{align}
  \rho_S &= 12 \int\int_{I^2} C_{\bm{\theta}}(u,v)\, \dd u\, \dd v-3\label{eq:rho_S}\\
  \tau_K &= 4\mathbb{E}[C_{\bm{\theta}}\{F_X(x), F_Y(y)\}]-1,\\
  \lambda_q &=
  \begin{cases}
    \p(F_X(X)\leq q| F_Y(Y)\leq q) = \displaystyle \frac{C_{\bm{\theta}}(q,q)}{q},
    &\text{ if } q\in (0,0.5],\\
    \p(F_X(X)>q|F_Y(Y)>q) =\displaystyle \frac{1-2q+C_{\bm{\theta}}(q,q)} {1-q},
    &\text{ if } q\in (0.5,1).
  \end{cases}
\end{align}\medskip
The empirical counterparts are
\begin{align*}
  \hat\rho_S &= \frac{12}{n} \sum_{k=1}^n \hat F_X(x_k) \hat F_Y(y_k)
               -3,\\
  \hat\tau_K &= \frac{4}{n}\sum_{k=1}^n \hat{C}\{\hat{F}_X(x_i),\hat{F}_X(y_i)\} -1 ,\\
  \hat\lambda_q &=
                  \begin{cases}
                    \displaystyle\frac{1}{n} \sum_{k=1}^n \frac{\1_{\{\hat
                        F_X(x_k)\leq q, \hat F_Y(y_k)\leq q\}}} {q},
                    &\text { if } q\in (0, 0.5],\\
                    \displaystyle \frac{1}{n} \sum_{k=1}^n
                    \frac{\1_{\{\hat F_X(x_k)>q, \hat F_Y(y_k)>q\}}}
                    {1-q}, &\text { if } q\in (0.5,1).
                  \end{cases},
\end{align*}
where $\hat{F}(x) := \frac{1}{n}\sum_{k=1}^n 1_{\{x_i\leq x\}}$ and
$\hat{C}(u,v) := \frac{1}{n}\sum_{k=1}^n 1_{\{u_i\leq u, v_i\leq v\}}$.\medskip

We denote $\tilde{\bm{m}}(\bm{\theta})$ be a $m$-dimensional vector of dependence measures according the the
dependence parameters $\bm{\theta}$,and  $\hat{\bm{m}}$ be the corresponding empirical counterpart.
The difference between dependence measures and their counterpart is denoted by
\begin{align*}
    \bm{g}(\bm{\theta}) = \hat{\bm{m}} - \tilde{\bm{m}}(\bm{\theta}).
\end{align*}\medskip

The SMM estimator is
\begin{align*}
    \hat{\bm{\theta}} = \argmin_{\bm{\theta}\in \bm{\Theta}} \bm{g}(\bm{\theta})^\intercal
    \hat{\bm{W}}
     \bm{g}(\bm{\theta}),
\end{align*}
where $\hat{W}$ is some positive definite weigh matrix.\medskip

In this work, we use $\tilde{\bm{m}}(\bm{\theta}) = (\rho_S, \lambda_{0.05}, \lambda_{0.1},
\lambda_{0.9}, \lambda_{0.95})^\intercal$
for calibration of Bitcoin price and CME Bitcoin future.

\subsection{Maximum Likelihood Estimation}\label{subsec:maximum-likelihood-estimation}
By Sklar's theorem, the joint density of a $d$-dimensional random variable $\bm{X}$ with sample size $n$ can be written as
\begin{align}
    \bm{f}_{\bm{X}}(x_1, ..., x_d) = \bm{c}\{F_{X_1}(x_1), ..., F_{X_d}(x_d)\} \prod_{j=1}^d f_{X_i}(x_i).
    \end{align}
We follow the treatment of MLE documented in section 10.1 of \citet{joe1997multivariate}, namely the inference functions for margins or IFM method.
The log-likelihood $\sum^n_{i=1}f_{\bm{X}}(X_{i,1}, ..., X_{i,d})$ can be decomposed into dependence part and marginal part,
\begin{align}
    L(\bm{\theta}) &= \sum_{i=1}^n \bm{c}\{F_{X_1}(x_{i,1};\bm{\delta}_1), ..., F_{X_d}(x_{i,d}; \bm{\delta}_d);\bm{\gamma}\}
    + \sum_{i=1}^n \sum_{j=1}^d f_{X_j}(x_{i,j};\bm{\delta}_j)
    &= L_C(\bm{\delta}_1, ..., \bm{\delta}_d, \bm{\gamma}) + \sum_{j=1}^d L_j(\bm{\delta}_j)
    \end{align}
where $\bm{\delta}_j$ is the parameter of the $j$-th margin, $\bm{\gamma}$ is the parameter of the parametric copula, and
$\bm{\theta} = (\bm{\delta}_1,..., \bm{\delta}_d, \bm{\gamma})$.

Instead of searching the $\bm{\theta}$ is a high dimensional space, \citet{joe1997multivariate} suggests to
search for $\hat{\bm{\delta}_1},..., \hat{\bm{\delta}_d}$ that maximize $L_1(\bm{\delta}_1), ..., L_d(\bm{\delta}_d)$,
then search for $\hat{\bm{\gamma}}$ that maximize $L_C(\hat{\bm{\delta}_1},..., \hat{\bm{\delta}_d}, \bm{\gamma})$.

That is, under regularity conditions, $(\hat{\bm{\delta}_1},..., \hat{\bm{\delta}_d}, \hat{\bm{\gamma}})$ is the solution of
\begin{align}
    \left( \frac{\partial L_1}{\partial \bm{\delta}_1}, ..., \frac{\partial L_d}{\partial \bm{\delta}_d},
    \frac{\partial L_C}{\partial \bm{\gamma}}\right) = \bm{0}.
    \end{align}

However, the IFM requires making assumption to the distribution of of the margins.
\citet{genest1995semiparametric} suggests to replace the estimation of marginals parameters estimation by non-parameteric estimation.
Given non-parametric estimator $\hat{F}_i$ of the margins $F_i$, the estimator of the dependence parameters $\bm{\gamma}$ is
\begin{align}
    \hat{\bm{\gamma}} = \argmax_{\bm{\gamma}} \sum_{i=1}^n \bm{c}\{ \hat{F}_{X_1}(x_{i,1}), ..., \hat{F}_{X_d}(x_{i,d});\bm{\gamma}\}.
    \end{align}



%With the decomposition, the MLE estimator for a bivariate parametric copula is
%\begin{align}
%    \hat{\bm{\theta}} = \argmax_{\bm{\theta} \in \bm{\Theta}} l(X_1,X_2; \bm{\theta}), \label{eq:EMLE}
%    \end{align}
%where
%\begin{align}
%    l(X_1,X_2; \bm{\theta}) = \sum_{i=1}^n \log c(x_{i,1}, x_{i,2};\bm{\theta}). \label{eq:Likelihood}
%    \end{align}\medskip

%Procedure of maximising equation~\ref{eq:EMLE} as a whole is called exact maximum likelihood method.
%Leveraging the attractive feature of copula that one can model the dependence structure and marginals separately,
%we rewrite~\ref{eq:Likelihood} into canonical expression
%\begin{align}
%    l(X,Y; \bm{\theta}) = \sum_{k=1}^n \log c\{F_X(x_i; \delta_X), F_Y(y_i; \delta_Y); \bm{\gamma}\}
%    + \sum_{k=1}^n \log f_X(x_i; \bm{\delta}_X) + \sum_{k=1}^n \log f_X(y_i; \bm{\delta}_Y),
%    \end{align}
%where the $\bm{\gamma}$ is the dependence parameter in the copula and $\bm{\delta}$ is the parameters in the margins.\medskip
%
%The inference-functions for margins (IFM) approach by Joe is a two step procedure of maximising~\ref{eq:EMLE}.
%The approach calibrate first the $\bm{\delta}$s and then the  $\bm{\gamma}$.\medskip
%
%Similar to the IFM approach, pseudo-maximum likelihood approach by Genest and Rivest (1993) replace the parametric margins by
%empirical estimates, we rewrite \ref{eq:Likelihood} again with
%\begin{align}
%    l(X,Y; \bm{\theta}) = \sum_{k=1}^n \log c(u_i, v_i;\bm{\gamma}),
%    \end{align}
%where $u_i = \hat{F}_X(x_i)$ and $v_i = \hat{F}_Y(y_i)$.

\subsection{Comparison}
Both the simulated method of moments and the maximum likelihood estimation are unbiased and
proven to give good fits.
The problem remain is which procedure is more suitable for hedging.
%Cryptocurrencies are known to be very volatile.
Sample and fitted quantile dependence for Bitcoin and CME future.

%\begin{figure}[th]
%\includegraphics[width=\textwidth]{_pics/t Copula quantile dependence.png}
%\includegraphics[width=\textwidth]{_pics/Gumbel Copula quantile dependence.png}
%\includegraphics[width=\textwidth]{_pics/Clayton Copula quantile dependence.png}
%  \caption{}
%\label{fig:quantile dependence1}
%\end{figure}


The MM estimation perform just as we decided: match the upper and lower quantile dependence.




%
%
%\subsection{Two-Stage Estimation}\label{subsec:two-stage-estimation}
%~\cite{joe2005asymptotic} study the efficiency of a two-stage estimation procedure of copula estimation.
%The authors also call this method inference function for margins IFM.
%
%\textbf{Pros}
%\begin{enumerate}
%    \item Almost as efficient as MLE methods but easier to be implemented
%    \item Yields an asymptotically Gaussian, unbiased estimate
%\end{enumerate}
%
%\textbf{Cons}
%\begin{enumerate}
%    \item Subject to specification of marginals \cite{kim2007comparison}
%\end{enumerate}
%
%Our data
%\begin{align}
%    \pmb{y} = \begin{bmatrix}
%                  y_{11} & \cdots & y_{1i}\\
%                  \vdots & \ddots & \vdots \\
%                  y_{n1} & \cdots & y_{ni}
%                  \end{bmatrix}
%    \end{align}
%Let $F$ and $f$ be the joint cdf and joint density of $\pmb{y}$ with parameters $\pmb{\delta}$,
%and let $F_i$ and $f_i$ be the marginal cdf and marginal density for the $i^\text{th}$ random variable with parameters $\pmb{\theta}_i$, we have
%\begin{align}
%    f(\pmb{y}; \pmb{\theta}_1, \pmb{\theta}_2,\dots \pmb{\theta}_i, \pmb{\delta}) =
%    c\{F_1(\pmb{y}_1;\pmb{\theta}_1), F_2(\pmb{y}_2; \pmb{\theta}_2), \dots, F_i(\pmb{y}_1;\pmb{\theta}_i); \pmb{\delta}\}
%    \prod^i_{j=1}f_i(\pmb{y}_j;\pmb{\theta}_j)
%    \end{align}
%
%For a sample of size $n$, the log-likelihood of functions of the $i^\text{th}$ univariate margin is
%\begin{align}
%    L_i(\theta_i) = \sum^n_{m=1} \log f_i(y_{mi}; \theta_i),
%    \end{align}
%
%and the log-likelihood function for the joint distribution is
%\begin{align}
%    L(\delta, \theta_1, \theta_2, \dots, \theta_i) = \sum^n_{m=1}\sum^i_{j=1} \log f(y_{mj}; \delta, \theta_1, \theta_2, ..., \theta_i)
%    \end{align}
%
%In most cases, one does not have closed form estimators and numerical techniques are needed.
%Numerical ML estimation difficulty increase when the total number of parameters increases.
%The two-stage estimation is designed to overcome this problem.
%
%The two-stage procedure is
%\begin{enumerate}
%    \item estimate the univariate parameters from separate univariate likelihoods to get $\tilde{\pmb{\theta}_1}, ..., \tilde{\pmb{\theta}_i}$
%    \item maximize $L(\pmb{\delta}, \tilde{\pmb{\theta}_1}, \dots, \tilde{\pmb{\theta}_i})$ over $\pmb{\delta}$ to get $\tilde{\pmb{\delta}}$
%    \end{enumerate}
%
%Under regularity conditions
%\footnote{Regularity conditions include
%1. $\exists \frac{\partial \log f(x;\theta)}{\partial \theta}, \frac{\partial^2 \log f(x;\theta)}{\partial \theta^2}, \frac{\partial^3 \log f(x;\theta)}{\partial \theta^3}$ for all $x$;
%2. $\exists g(x), h(x) and H(x)$ such that for $\theta$ in a neighborhood $N(\theta_0)$ the relations
%$\left|\frac{\partial f(x;\theta)}{\partial theta}\right| \leq g(x)$,
%$\left|\frac{\partial^2 f(x;\theta)}{\partial \theta^2}\right| \leq h(x)$,
%$\left|\frac{\partial^3 f(x;\theta)}{\partial \theta^3}\right| \leq H(x)$ hold for all $x$, and
%$\int g(x) dx < \infty$, $\int h(x) dx < \infty$, $\mathbb{E}_\theta \{H(X)\} < \infty$ for $\theta \in N(\theta_0)$;
%3. For each $\theta \in \Theta$, $0< \mathbb{E}_\theta \left\{
%\left(
%\frac{\partial \log f(X;\theta)}{\partial \theta}
%\right)^2
%\right\}$. For detail see section 4.2.2 of~\cite{serfling2009approximation}}
%, $(\pmb{\tilde{\theta}}_1,\dots \pmb{\tilde{\theta}}_i, \pmb{\tilde{\delta}})$ is the solution of
%\begin{align}
%    (\partial L_1 / \partial \pmb{\theta}^\intercal_1,
%    \dots, \partial L_i / \partial \pmb{\theta}^\intercal_i, \partial L / \partial \pmb{\pmb{\delta}}^\intercal_1) = \pmb{0}
%    \end{align}
%
%For comparison, if we optimize $L$ directly without the two-stage procedure (i.e.~MLE), we solve for
%\begin{align}
%    (\partial L / \partial \pmb{\theta}^\intercal_1,
%    \dots, \partial L / \partial \pmb{\theta}^\intercal_i, \partial L / \partial \pmb{\pmb{\delta}}^\intercal_1) = \pmb{0}
%    \end{align}
%
%We denote the two solutions as
%$\tilde{\pmb{\eta}} = (\pmb{\tilde{\theta}}_1,\dots \pmb{\tilde{\theta}}_i, \pmb{\tilde{\delta}})$ for two-stage procedure;
%$\hat{\pmb{\eta}} =(\pmb{\hat{\theta}}_1,\dots \pmb{\hat{\theta}}_i, \pmb{\hat{\delta}})$ for MLE procedure.
%and compare the asymptotic relative efficiency of $\tilde{\pmb{\eta}}$ and $\hat{\pmb{\eta}}$.
%
%Asymptotics: yet to be done.\\
%~\cite{kim2007comparison} show the estimation of $\pmb{\theta}$ may be seriously affected.
%They compare the two-stage approach and Canonical Maximum Likelihood Method by simulation and
%conclude that Canonical Maximum Likelihood is prefered from a computational statistics and data analysis point of view.
%
%\subsection{Canonical Maximum Likelihood Method}\label{subsec:canonical-maximum-likelihood-method}
%This approach was studied by~\cite{genest1995semiparametric} and~\cite{shih1995inferences}.
%One estimates the margins using empirical CDF
%\begin{align}F_X(x)=\frac{1}{n+1}\sum_{i=1}^n 1(X_i \leq x)\end{align},
%
%we maximize the log-likelihood
%\begin{align}
%    L(\delta) = \sum_{i=1}^n \log [c_\delta \{F_X(X_i), F_Y(Y_i)\}]
%    \end{align}
%
%This procedure does not require specification of marginals.
%
%
%
%
%
%%also by Wang and Ding, 2000; Tsukahara, 2005
%%This is also known as pseudo maximum likelihood (PML) and as canonical maximum likelihood (see Cherubini et al., 2004)
%%
%%Genest and Werker (2002) obtained conditions under which the PMLE is asymptotically efficient.
%%
%%

% ----------------
% --- Estimation of Copula ---
% ----------------


\subsection{Copula Selection}\label{subsec:copula-selection}

\natp{\em [I suggest: 3.3 Calibration and copula selection; 3.3.1 Method of moments; 3.3.2 Comparison with MLE; 3.3.3 Copula selection]}

\natp{\em [Please avoid the word dependency. In probability theory, it is dependence.]}

The dependence structure of price data changes across time, in which
both the dependency parameters and the type of dependence,
dependencies between cryptos and the BTCF are no exception. \natp{\em [?]}
\natp{For this reason, we allow for a flexible choice of the
  best-fitting copula, by re-calibrating periodically and
  re-evaluating performance of the various copulas. (was: In this hedging exercise, we find a best fitting copula to model the dependency for every set of training data.)}
We select the best-fitting copula, characterised by the lowest {\em Akaike Information Criterion (AIC)},
\begin{equation*}
 \text{AIC} = 2k- 2 \log(L),
\end{equation*}
where $k$ is the number of estimated
parameteres and $L$ is the likelihood \citep{Akaike1973}. 

% they tend to suggest the same copula as the best fitting one.
%Simulation studies has also been carried out to compare different copula selection methods, see \cite{}.
\natp{Other model selection criteria, such as the (was: Notice that
  there are other model selection procedure and criteria, e.g.) TIC}
\natp{or } likelihood ratio test \natp{could be used instead.}
For a survey of model selection and inference, see \cite{anderson1998comparison}.
Among various copula selection procedures, AIC is a popular choice for
its applicability, \natp{see e.g.\ \cite{breymann2003dependence} (was:
  for example \cite{breymann2003dependence} use the AIC to select best fitting copulae)}.
In our case, the AICs are calculated only with dependency likelihood
since the marginals are modelled via kernel density estimators.
The selected copula will then be enter the calculation of the optimal
hedge ratio.
% We consider the copula with the lowest AIC for a particular set of data the best fitting one and use it to generate OHR.


%%% Local Variables:
%%% mode: latex
%%% TeX-master: "SRM"
%%% End:



\subsection{Risk Measures}\label{subsec:spectral-risk-measures}
We consider a variety of risk measures: variance, Value-at-Risk (VaR), Expected Shortfall (ES), and Exponential Risk Measure (ERM).
%They are used in many literature about hedging, e.g. ;
%The risk measures are also used by regulatory bodies,
%for example Basel III ....
A summary of risk measures being used in portfolio selection problem can be found in \citet{hardle2008applied}.\medskip
\medskip


Let $Z$ be a random variable of distribution $F_Z$.
\begin{enumerate}
	\item Variance is $\text{Var}(F_Z)$
	\item VaR of a given confidence level $\alpha$ is $\text{VaR}(F_Z) = -F_{Z}^{(-1)}(1-\alpha)$
	\item ES with parameter $\alpha$ is $\text{ES}(F_Z) = -\frac{1}{1-\alpha}\int_0^{1-\alpha}F_Z^{(-1)}(p)dp$
	\item ERM with Arrow-Pratt coeficient of absolute risk aversion $k$ is $\text{ERM}_k(F_Z) = \int_0^{1-\alpha}\phi(p) F_Z^{(-1)}(p)dp$ where $\phi$ is a weight function described in (\ref{eq:phi}) below.
	\end{enumerate}\medskip

VaR, ES, and ERM fall into the class of Spectral Risk Measure (SRM).
SRM has the from \citep{Acerbi2002}%, adam2008spectral,dowd2008spectral}
\begin{align}
	\rho_\phi(r^h) = - \int_0^1 \phi(p) F_{Z}^{(-1)}(p)d p,
	\end{align}

where $p$ is the loss quantile and $\phi(p)$ is a user-defined weighting function defined over $[0,1]$. \medskip
We consider only admissible risk spectra $\phi(p)$ %(named by \citet{Acerbi2002})
\begin{enumerate}[label=\roman*]
	\item $\phi$ is positive
	\item $\phi$ is decreasing
	\item integrates to one.
	\end{enumerate}\medskip

The VaR's $\phi(p)$ gives all its weight on the $1-\alpha$ quantile of $Z$ and zero elsewhere,
i.e. the weighting function is a Dirac delta function, hence violates the ii property of admissible risk spectra.
The ES' $\phi(p)$ gives all tail quantiles the same weight of $\frac{1}{1-\alpha}$ and non-tail quantiles zero weight.
ERM assumes investor's risk preference is in a form of exponential utility function $U(x)=1-e^{kx}$,
its corresponding risk spectrum is defined as

\begin{align}
	\phi(p) =\frac{k e^{-k(1-p)}}{1-e^{-k}} , \label{eq:phi}
	\end{align}
where $k$ is the Arrow-Pratt coefficient of absolute risk aversion. \medskip

The parameter $k$ has an economic interpretation of being the ratio between the second derivative and first derivative
of investor's utility function on an risky asset

\begin{align}
	k = -\frac{U''(x)}{U'(x)},
	\end{align}
for $x$ in all possible outcomes.
In case of the exponential utility, $k$ is the the constant absolute risk aversion (CARA).



%Spectral risk measures can also be written as
%\begin{align}
%	\rho_\phi(r^h) = - \int_\mathbb{R} x f_{r^h}(x) \phi\{F_{r^h}(x)\} d x.
%	\end{align}
% ----------------
% --- Risk measures' definition and properties ---
% ----------------


%In this work, we consider two portfolios: $r^h = R^{\text{BTC}} - h R^{\text{future}}$ and $r^h = R^{\text{BTC}} - h R^{\text{CRIX}}$.
%
%\natp{\em [For now, stick to $r_S$ and $r_F$. This is still the
%  general part, so no need to mention Bitcoin, CRIX, etc. here. Also,
%  wasn't the idea to hedge CRIX with the future? I don't see why CRIX
%  would be used to hedge Bitcoin when futures are readily
%  available. Entering into futures contracts requires no notional,
%  which makes them ideal for hedging.]}

%OLD
%\section{Methodology}\label{sec:methodology}
%Following \citet{barbi2014copula}, we consider the problem of optimal
%hedge ratios by extending the commonly known minimum variance hedge
%ratio to more general risk measures and dependence
%structures.\medskip
%
%Hedge portfolio: $R_t^h = R_t^S - h R_t^F$, involving returns of spot
%and future contract and where $h$ is the hedge ratio.\medskip
%
%The optimal hedge ratio is
%\begin{align}
%    h^\ast = \argmin_h \rho_\phi(s,h),
%    \end{align}
%for given
%confidence level $1-s$ (if applicable, e.g.\ in the case of VaR, ES),
%where $\rho_\phi$ is a spectral risk measure with weighting function
%$\phi$ (see below).
%In other words, our task is to search for the optimal $h$ which can minimize a particular risk measure.
%We call the risk measure being used in search process of $h^\ast$ risk reduction objective.
%This naming is to differentiate the risk objective and risk outcome.
%One can see from result section that the $h^\ast$ which minimize a particular risk measure in training does not
%necessarily minimize the risk measure in testing data.
%For example in table \ref{OOSRHVaR99}, the best performing risk reduction objective to reduce out-of-sample Value-at-Risk 99\% is
%exponential risk measure $k=10$. \medskip
%
%The distribution function of $Z$ can be expressed in terms of the
%copula and the marginal distributions as Proposition \ref{prop:dfrh}
%result shows (this is a corrected version of Corollary 2.1 of
%\citep{barbi2014copula}). For practical applications, it is numerically
%faster and more stable to use additional information about the
%specific copula and marginal distributions. We therefore derive
%semi-analytic formulas for a number of special cases, such as the
%Gaussian-, Student $t$-, normal inverse Gaussian (NIG) and Archimedean
%copulas in Section \ref{sec:dependence}.
%
%\begin{proposition}
%  \label{prop:dfrh}
%  Let $X$ and $Y$ be two real-valued random variables on the same
%  probability space $(\Omega, \mathcal A, p)$ with corresponding
%  absolutely continuous copula $C_{X, Y}(w,\lambda)$ and
%  continuous marginals $F_{X}$ and $F_{Y}$. Then, the distribution
%  of of $Z$ is given by
%  \begin{equation}
%    \label{eq:3}
%    F_{Z}(x) = 1- \int^1_0 D_1 C_{X, Y}
%    \left( u, F_{Y} \left( \frac{F^{(-1)}_{X}(u)-x}{h} \right)
%    \right)\, d u.
%  \end{equation}
%\end{proposition}\medskip
%Here $D_1 C(u,v)=\displaystyle \frac{\partial}{\partial u} C(u,v)$,
%which is easily shown to fulfil, see e.g.\ Equation (5.15) of
%\citep{McNeil2005}:\footnote{%
%  Let $F_X(x)=u$, $F_Y(y)=v$. Then, formally,
%  \begin{align*}
%    \frac{\partial}{\partial F_X(x)} C(F_X(x), F_Y(y)) %
%    &= \frac{\partial}{\partial F_X(x)} \p(U\leq F_X(x),
%      V\leq F_Y(y)) %
%      = \p(U\in d F_X(x), V\leq F_Y(y))\\ %
%    &= \p(V\leq F_Y(y)| U = F_X(x))\cdot \p(U \in d
%      F_X(x)) %
%      = \p(Y\leq y|X=x)\cdot \p(U\in d u)\\ %
%    &= \p(Y\leq y|X=x).
%  \end{align*}}
%\begin{equation}
%  \label{eq:1}
%  D_1 C_{X,Y}(F_X(x), F_Y(y)) = \p(Y\leq y|X=x).
%\end{equation}
%\begin{proof}
%  Using the identity (\ref{eq:1}) gives
%  \begin{align*}
%    F_{Z}(x) &= \p(r^S - h Y\leq x) %
%                 = \E\left[\p\left(Y\geq \frac{X-x}{h}\Big|
%                 X\right)\right]\\[10pt]
%               &= 1-\E\left[\p\left(Y\leq \frac{X-x}{h}\Big|
%                 X\right)\right]% \\[10pt]
%               = 1- \int_0^1 D_1 C_{X, Y}\left(u,
%                 F_{Y}\left(\frac{F^{(-1)}_{X}(u) -
%                 x}{h}\right)\right)\, d u.
%  \end{align*}
%\end{proof}\medskip
%
%In addition to \cite{barbi2014copula} we propose an expression for the density of $Z$
%
%\begin{proposition} With the same setting of the above proposition, the density of $Z$ can be written as
%  \begin{align}
%  f_{Z}(y) &= \left|\frac{1}{h}\right|\int_0^1 c_{X, Y} \left[u,
%  F_{Y}\left\{\frac{F^{(-1)}_{X}(u)-y}{h}\right\}
%  \right]
%   \cdot
%  f_{Y}
%  \left\{\frac{F^{(-1)}_{X}(u)-y}{h}\right\} du \label{eq:density1}
%  \end{align}, or
%    \begin{align}
%      f_{Z}(y)
%      = \int_0^1 c_{X, Y} \left[u,
%      F_{X}\left\{y + h F^{(-1)}_{Y}(u)\right\}
%      \right]
%       \cdot
%      f_{X}
%      \left\{
%      y+ hF^{(-1)}_{Y}(u)
%      \right\} du.\label{eq:density2}
%  \end{align}
%  \end{proposition}
