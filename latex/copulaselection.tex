
\subsubsection{Copula selection}\label{subsec:copula-selection}
\natp{As the (was: The)} dependence structure of price data changes
across time, \natp{(delete: in which
both the dependency parameters and the type of dependence.
For this reason,)} we allow for a flexible choice of the best-fitting
copula, by re-calibrating periodically and re-evaluating performance
of the various copulas. 
\natp{In each re-calibration, we (was: We)} select the best-fitting
copula, characterised by the lowest {\em Akaike Information Criterion
  (AIC)},
\begin{equation*}
 \text{AIC} = 2k- 2 \log(L),
\end{equation*}
where $k$ is the number of estimated
parameteres and $L$ is the likelihood \citep{Akaike1973}. 

% they tend to suggest the same copula as the best fitting one.
%Simulation studies has also been carried out to compare different copula selection methods, see \cite{}.
Other model selection criteria, such as the TIC~\citep{takeuchi1976distribution} or likelihood ratio test could be used instead.
For a survey of model selection and inference, see \cite{anderson1998comparison}.
Among various copula selection procedures, AIC is a popular choice for
its applicability, see e.g. \cite{breymann2003dependence}.
In our case, the AICs are calculated only with dependence likelihood
since the marginals are modelled via kernel density estimators.
The selected copula will then enter the calculation of the optimal
hedge ratio.
% We consider the copula with the lowest AIC for a particular set of data the best fitting one and use it to generate OHR.


%%% Local Variables:
%%% mode: latex
%%% TeX-master: "SRM"
%%% End:
