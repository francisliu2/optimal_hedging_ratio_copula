
\subsection{Copulae}\label{subsec:copulae}
As seen in the last section, the risk measures are all
functionals of the joint distribution of $R^S$ and $R^F$.
\natp{To capture different aspects of the dependence structure, we
  therefore consider a number of different copulas, which are layed
  out in details below: (was: 
The following copulae are considered:)} Gaussian-, $t$-, Frank-,
Gumbel-, Clayton-, Plackett-, mixture, and factor copula. 
As this hedging exercise concerns only portfolios with two assets, we
focus on the bivariate version of copulae and some important
features of a copula, including Kendall's $\tau_K$ and Spearman's $\rho_S$. \medskip

Kendall's $\tau$ and Spearman's $\rho$ are measure of association in terms of concordance, see \cite{kruskal1958ordinal}
Let $(x_i, y_i)$ and $(x_j, y_j)$ denote two observations from a vector $(X, Y)$ of continuous random variables.
A pair of observations is concordant if $x_i<x_j$ and $y_i < y_j$, discordant if
$x_i>x_j$ and $y_i < y_j$ or if $x_i<x_j$ and $y_i>y_j$.
For a bivariate random variable of $n$ observations, there are $\binom{n}{2}$ distinct pairs. \medskip

Let $c$ denote the number of concordant pairs, and $d$ the number of discordant pairs,
Kendall's tau is defined as follows \citep{Nelsen1999}
\begin{align*}
\tau &= \frac{c-d}{c+d} \\
     &= \frac{c-d}{\binom{n}{2}};
\end{align*}

Let $F$ and $G$ be cdfs of $X$ and $Y$ respectively, Spearman's $\rho$ is
\begin{align*}
\rho = 12\E (UV)-3;
\end{align*}

Upper tail dependence is defined as
\begin{equation*}
\lambda_U \overset{\mathrm{def}}{=}  \lim_{q
  \rightarrow 1^-} \p\{X > F_{X}^{(-1)}(q)|Y > F_{Y}^{(-1)}(q)\};
\end{equation*}

Lower tail dependence is defined as
\begin{equation*}
\lambda_L \overset{\mathrm{def}}{=}  \lim_{q
  \rightarrow 0^+} \p\{X \leq F_{X}^{(-1)}(q)|Y \leq
F_{Y}^{(-1)}(q)\}. 
\end{equation*}
Furthermore, we denote the Fr{\'e}chet-Hoeffding lower bound by
$\bm{W}$, the product copula by $\bm{\Pi}$, and the Fr{\'e}chet-Hoeffding
upper bound by $\bm{M}$. They represent cases of perfect negative
dependence, independence, and perfect positive dependence,
respectively. 
For further details, we refer readers to \citet{joe1997multivariate}
and \citet{Nelsen1999}; see also \citet{hardle2010copulis}.

\subsubsection{Gaussian and $t$ Copulae}\label{sec:ellpitical-copulae}

Gaussian and $t$ copulae are dervived from Gaussian and $t$ \textbf{distributions}
Since Gaussian and $t$ distributions are elliptical distributions, Gaussian and $t$ copulae are called elliptical copulae.

Gaussian copula (bivariate) is defined as
\begin{align}
  \bm{C}(u,v) &= \Phi_{2, \rho}\{\Phi^{-1}(u), \Phi^{-1}(v)\} \nonumber \\
              &= \int_{-\infty}^{\Phi^{-1}(u)}
                \int_{-\infty}^{\Phi^{-1}(v)}
                \frac{1}{2\pi\sqrt{1-\rho^2}}
                \exp{\left\{
                \frac{s^2-2\rho st+t^2}{2(1-\rho^2)}
                \right\}} \dd s\, \dd t,
\end{align}
where $\Phi_{2, \rho}$ is the cdf of bivariate Normal distribution
with zero mean, unit variance, and correlation coefficient $\rho$, and
$\Phi^{-1}$ is the quantile function univariate standard normal
distribution. \medskip

Note that we use $\rho$ to denote the correlation parameter as well as a $\rho(\cdot)$ to denote a risk measure. \medskip

The Gaussian copula is fully specified by the correlation parameter $\rho$.
Like all elliptical copulas, it is symmetric.
It has no tail dependence, which, in a finance context, implies that it often underestimates tail risk. \medskip

The Gaussian copula density is
\begin{align}
  \bm{c}_\rho(u,v) &= \frac{\bm{\varphi}_{2,\rho}\{\Phi^{-1}(u), \Phi^{-1}(v)\}}
                     {\varphi\{\Phi^{-1}(u)\} \cdot \varphi\{\Phi^{-1}(v)\}} \nonumber \\[10pt]
                   &= \frac{1}{2\pi\sqrt{1-\rho^2}}\exp\left\{
                     -\frac{u^2 - 2\rho uv + v^2}{2(1-\rho^2)}
                     \right\},
\end{align}
where $\bm{\varphi}_{2,\rho}(\cdot)$ is the pdf corresponding to
$\Phi_{2, \rho}$, and $\varphi(\cdot)$ the standard normal
distribution pdf. \medskip

Kendall's $\tau_K$ and Spearman's $\rho_S$ of a bivariate Gaussian copula are
    \begin{align}
        \tau_K(\rho) = \frac{2}{\pi}\arcsin\rho
        \end{align}
    \begin{align}
        \rho_S(\rho) = \frac{6}{\pi}\arcsin\frac{\rho}{2}.
        \end{align}

The $t$-copula has the form
\begin{align*}
        \bm{C}(u,v) &= \bm{T}_{2, \rho, \nu}\{T^{-1}_\nu(u), T^{-1}_\nu(v)\} \nonumber \\[10pt]
            &= \int_{-\infty}^{T^{-1}_\nu(u)}
               \int_{-\infty}^{T^{-1}_\nu(v)}
            \frac{\Gamma\left(\frac{\nu+2}{2}\right)}
            {\Gamma\left(\frac{\nu}{2}\right)\pi\nu\sqrt{1-\rho^2}}\\[10pt]
           & \left(
        1+\frac{s^2-2st\rho+t^2}{\nu}
        \right)^{-\frac{\nu+2}{2}} ds dt,
    \end{align*}
where $\bm{T}_{2, \rho, \nu}$ denotes the cdf of
bivariate $t$ distribution with scale parameter $\rho$ \natp{\em
  [$\rho$ specifies the dependence, so why is it a scale parameter?
  Are you sure?]} and degrees of freedom parameter $\nu$, 
and where $T^{-1}_\nu(\cdot)$ is the quantile function of a standard
$t$ distribution with degree of freedom $\nu$. 
\natp{Contrary to the Gaussian copula, the $t$-copula has a non-zero
  tail dependence coefficient, which makes it more appropriate for
  dependence modelling in finance. The Gaussian copula arises as
  $\nu\rightarrow\infty$.}

\natp{\em [Please make sure to use equation numbers only if the
  formulas are referenced.]}
The copula density is
\begin{align*}
    \bm{c}(u,v) &= \frac{\bm{t}_{2, \rho, \nu}\{T^{-1}_\nu(u), T^{-1}_\nu(v)\}}
    {t_\nu\{T^{-1}_\nu(u)\}\cdot t_\nu\{T^{-1}_\nu(v)\}},
    \end{align*}
where $\bm{t}_{2,\rho, \nu}$ is the pdf of $\bm{T}_{2, \rho, \nu}$
and $t_\nu$ the density of standard $t$ distribution.\medskip

Like all the other elliptical copulae, the $t$-copula's Kendall's
$\tau$ is identical to that of the Gaussian copula \citep[see][and
references therein]{demarta2005t}. 


\subsubsection{Archimedean Copulae}\label{sec:archimedean-copula}
The family of Archimedean copulae forms a large class of copulae with
many convenient features.
\natp{Contrary to elliptical copulas, which are derived from
  elliptical distributions. Archimedean copulas are determined via a
  simple parametric form of the dependence structure. A promiment
  feature is the ability to model asymmetric dependence structures. }
In general, they take a form
\begin{align*}
    \bm{C}(u,v)= \psi^{-1}\{\psi(u), \psi(v)\},
    \end{align*}
where $\psi:[0,1] \rightarrow [0,\infty)$ is a continuous, strictly
decreasing and convex function such that $\psi(1)=0$ for any
permissible dependence parameter $\theta$. $\psi$ is also called
generator. $\psi^{-1}$ is the inverse of the generator.

\natp{\em [Remove the Frank copula? Or are we still using it? Then
  start with Clayton and Gumbel.]}

The Frank copula (B3 in \citet{joe1997multivariate}) is a radial symmetric copula and cannot produce any tail dependence.
It takes the form
\begin{align*}
    \bm{C}_{\theta}(u,v) &= \frac{1}{\theta}
    \log \left\{
    1 + \frac{(e^{-\theta u}-1)(e^{-\theta v}-1)}{e^{-\theta}-1}
    \right\}
    \end{align*}
where $\theta \in [0, \infty]$ is the dependency parameter.
$\bm{C}_{-\infty} = \bm{M}$, $\bm{C}_1 = \bm{\Pi}$, and $\bm{C}_\infty = \bm{W}$.

The Copula density is
\begin{align}
    \bm{c}_{\theta}(u,v) &= \frac{\theta e^{\theta(u+v)(e^\theta-1)}}
    {\left\{e^\theta-e^{\theta u}-e^{\theta v}+e^{\theta (u+v)}\right\}^2}
    \end{align}\medskip

Frank copula has Kendall's $\tau$ and Spearman's $\rho$ as follow:
\begin{align}
    \tau_K(\theta) = 1-4\frac{D_1\{-\log(\theta)\}}{\log(\theta)},
    \end{align}
and
\begin{align}
    \rho_S(\theta) = 1-12\frac{D_2\{-\log(\theta)\} - D_1\{\log(\theta)\}}{\log(\theta)},
    \end{align}
where $D_1$ and $D_2$ are the Debye function of order 1 and 2.
Debye function is $D_n = \frac{n}{x^n}\int_0^x\frac{t^n}{e^t-1}dt$.\medskip

The Gumbel copula (B6 in \citet{joe1997multivariate}) has upper tail
dependence with the dependence parameter $\lambda^U =
2-2^{\frac{1}{\theta}}$ and displays no lower tail dependence. 
\begin{equation*}
  \bm{C}_{\theta}(u,v) = \exp{-\{ (-\log(u))^\theta +(-\log(v))^\theta 
    \}^{\frac{1}{\theta}}},
\end{equation*}
where $\theta \in [1,\infty)$ is the dependence parameter.
    
While the Gumbel copula cannot model perfect counter-dependence
\citep{Nelsen2002}, $\bm{C}_{1} = \bm{\Pi}$ models the independence, 
and $\lim_\theta^\infty \bm{C}_\theta = \bm{W}$ models the perfect
dependence. The copula density takes the form
%\begin{align}
%        f
%    \end{align}
  \begin{equation*}
    \tau_K(\theta) =\frac{\theta-1}{\theta}. 
   \end{equation*}

The Clayton copula, by contrast to Gumbel copula,
generates lower tail dependence of the form $\lambda^L =
2^{-\frac{1}{\theta}}$, but cannot generate upper tail dependence.
The Clayton copula takes the form
\begin{equation*}
  \bm{C}_{\theta}(u,v) = \left\{
    \max(u^{-\theta}+v^{-\theta}-1,0)\right\}^{-\frac{1}{\theta}},
\end{equation*}
where $\theta \in (-\infty, \infty)$ is the dependence parameter.
Moreover, $\lim_\theta^{-\infty} \bm{C}_\theta = \bm{M}$, $\bm{C}_0 =
\bm{\Pi}$, and $\lim_\theta^\infty \bm{C}_\theta = \bm{W}$. 
Kendall's $\tau$ of the Clayton copula is given by 
\begin{align}
    \tau_K(\theta) =\frac{\theta}{\theta+2}.
    \end{align}

\subsubsection{Mixture Copula}\label{sec:mixture-copula}
The mixture copula is a linear combination of copulae. 
For a 2-dimensional random variable $\bm{X}=(X_1,X_2)^\top$,
its distribution can be written as linear combination of $K$ copulae
\begin{equation}
    \p(X_1 \leq x_1, X_2 \leq x_2) = \sum_{k=1}^K p^{(k)} \cdot
    \bm{C}^{(k)}\{F^{(k)}_{X_1}(x_1;\bm{\gamma}^{(k)}_1), 
    F^{(k)}_{X_2}(x_2;\bm{\gamma}^{(k)}_2); \bm{\theta^{(k)}}\}
    \label{eq:2}
    \end{equation}
where $p^{(k)} \in [0,1]$ is the weight of each component,
$\bm{\gamma}^{(k)}$ \natp{are the parameters (was: is the parameter)}
of the marginal distribution in the $k$-th component, 
and $\bm{\theta^{(k)}}$ \natp{are the (was: is the)} dependence
parameter\natp{s} of the copula of the $k$-th component. 
The weights add up to one $\sum_{k=1}^K p^{(k)}=1$. \medskip

We deploy a simplified version of the above representation by assuming
the margins of $\bm{X}$ are not a mixture. \natp{\em [Is this a
  precise formulation? The margins are not mixtures anyway, just
  specified for each copula component. Perhaps write: ... by
  specifying the same margins for each copula component.]}

\natp{\em [Check notation of quantile function throughout. I think we
  should use $F^{(-1)}$ instead of $F^{-1}$, as the latter can be
  mistaken for $1/F$].}

By Sklar's theorem one may write \natp{\em [Only for the special case
  where the margins are fixed, right? Mention this.]}
\begin{equation*} 
    \bm{C}(u,v)= \sum_{k=1}^K p^{(k)} \cdot \bm{C}^{(k)}\{F^{-1}_{X_1}(u),
    F^{-1}_{X_2}(v); \bm{\theta^{(k)}}\}.
    \end{equation*}
The copula density is again a linear combination of copula densities
\begin{align}
    \bm{c}(u,v)= \sum_{k=1}^K p^{(k)} \cdot \bm{c}^{(k)}\{F^{-1}_{X_1}(u),
    F^{-1}_{X_2}(v); \bm{\theta^{(k)}}\}.
    \end{align}

While Kendall's $\tau$ of mixture copula is not known in closed form,
Spearman's $\rho$ is specified by the following statement. 

\begin{proposition}
  \natp{In the setting of \eqref{eq:2} \em[Please check if this is
    correct! Also, please check if the copula must be continuous.]}, 
  let $\rho_S^{(k)}$ be Spearman's $\rho$ of the $k$-th component
  \natp{(delete: and $\sum_{k=1}^K p^{(k)}=1$ holds,)}. 
  Spearman's $\rho$ of the mixture copula is given by 
  \begin{align}
        \rho_S = \sum_{k=1}^K p^{(k)} \cdot \rho_S^{(k)}
        \end{align}
    \end{proposition}

\begin{proof}
    Since Spearman's $\rho$ is defined as \citep{Nelsen1999}
    \begin{equation*}
      \rho_S = 12 \int_{\mathbb{I}^2} \bm{C}(s,t) ds dt - 3,
    \end{equation*}
    Spearman's $\rho$ of the the mixture copula is given by summation
    of the components 
       \begin{align}
        \rho_S = 12 \int_{\mathbb{I}^2} \sum_{k=1}^K p^{(k)} \cdot
         \bm{C}^{(k)}(s,t) ds dt - 3. 
        \end{align}
    \end{proof}

\natp{\em [If the Fr'echet class is not used below, then I suggest to
  remove the example, and replace it by one sentence wiht a reference,
  i.e.: An example of a mixture copula is the Fr'echet class of
  copulas, which are given by convex combinations of $\bm{W}$,
  $\bm{\Pi}$, and $\bm{M}$ \citep{Nelsen1999}.]} 
    
\begin{example}
    The Fr{\'e}chet class can be seen as an example of mixture copula.
    It is a convex combinations of $\bm{W}$, $\bm{\Pi}$, and $\bm{M}$ \citep{Nelsen1999}
    \begin{align}
        \bm{C}_{\alpha, \beta}(u,v)
        = \alpha \bm{M}(u,v) +
        (1-\alpha-\beta)\bm{\Pi}(u,v)
        +\beta \bm{W}(u,v),
        \end{align}
    where $\alpha$ and $\beta$ are the dependence parameters, with $\alpha, \beta \geq 0$ and
    $\alpha+\beta \leq 1$.
    Its Kendall's $\tau$ and Spearman's $\rho$ are
    \begin{align}
        \tau_K(\alpha, \beta) = \frac{(\alpha - \beta)(\alpha+\beta+2)}{3}
        \end{align}
    , and
    \begin{align}
        \rho_S(\alpha, \beta) = \alpha - \beta
        \end{align}
    \end{example}\medskip
%Example 2 Gumbel-Clayton mixture
%Example 3 Hu 2006.

We use a mixture of Gaussian and independent copulas in our analysis,
i.e., 
\begin{equation*}
  \bm{C}(u,v) = p\cdot \bm{C}^\text{Gaussian}(u,v) + (1-p)(uv),
\end{equation*}
with corresponding density is
\begin{equation*}
  \bm{c}(u,v) = p\cdot \bm{c}^\text{Gaussian}(u,v) + (1-p).
\end{equation*}

This mixture models the amount of ``random noise'' that appears in the
dependence structure. In the hedging exercise, \natp{(delete: the structure of)} the
``random noise'' \natp{adds an unhedgable component to the two-asset
  portfolio, whose weight $(1-p)$ is calibrated from market data (was:
  is not of our concern nor we can hedge the noise by a two-asset portfolio.)}
\natp{(delete: However, the proportion of the ``random noise'' does
  affect the distribution of $r^h$, so as the optimal hedging ratio
  $h^\ast$.) \em[I think this can be deleted, but maybe not?]}
One can consider the mixture copula as a handy tool for stress testing.
Similar to this Gaussian mix Independent copula,
$t$ copula is also a two parameter copula allow us to model the noise,
but its interpretation of parameters is not as intuitive as that of a mixture.
The mixing variable $p$ is the proportion of a manageable (hedgable) Gaussian copula,
while the remaining proportion $1-p$ cannot be managed. \natp{\em [Not
  sure I understand the comparison with the $t$ copula. I think you
  might be thinking of the case where the scaling variable of the
  $t$-copula is large and the correlation is moderate, which produces
  some observations along the negative diagonal. However, this needs
  to be carefully explained -- or left out.]}

\subsubsection{NIG factor copula}

The {\em normal inverse Gaussian (NIG)\/} distribution, introduced by
\citep{BarndorffNielsen1997}, has density function
\begin{equation*}
  g(x; \alpha,\beta, \mu, \delta) = \frac{\alpha}{\pi} \e^{\delta
    \sqrt{\alpha^2-\beta^2} -\beta\mu} \frac{1}{q((x-\mu)/\delta)}
  K_1\left[\delta \alpha q\left(\frac{x-\mu}{\delta}\right) \right]
  \e^{\beta x},\quad x>0,
\end{equation*}
where $q(x) = \sqrt{1+x^2}$ and where $K_1$ is the modified Bessel
function of third order and index $1$. The parameters satisfy $0\leq
|\beta|\leq \alpha$, $\mu\in \R$ and $\delta>0$. The parameters have
the following interpretation: $\mu$ and $\delta$ are location and
scale parameters, respectively, $\alpha$ determines the heaviness of
the tails and $\beta$ determines the degree of asymmetry. If
$\beta=0$, then the distribution is symmetric around $\mu$.

The moment-generating function of the NIG distribution is given by
\begin{equation*}
  M(u; \alpha, \beta, \mu, \delta) = \exp\left( \delta
    \left(\sqrt{\alpha^2-\beta^2} - \sqrt{\alpha^2 - (\beta +
        u)^2}\right) + \mu u\right). 
\end{equation*}
As a direct consequence, moments are easily calculated with the
expectation and variance of the NIG distribution being
\begin{align}
  \label{eq:4}
  \mathbb E X &= \mu + 
                \frac{\delta \beta}{\sqrt{\alpha^2-\beta^2}}\\
  \label{eq:5}
  \text{Var}(X) &= \frac{\alpha^2\delta}{(\alpha^2-\beta^2)^{3/2}}.
\end{align}


The $\text{NIG}(\alpha, \beta, \mu\, \delta)$ distribution belongs to
the class of so-called {\em normal
variance-mean mixture},  (see Section 3.2 of
\citep{McNeil2005}): $X$ follows an
$\text{NIG}(\alpha,\beta,\mu,\delta)$ distribution if $X$ conditional
on $W$ follows a normal distribution with mean $\mu+\beta W$ and
variance $W$, i.e., 
\begin{equation*}
  X|W\stackrel{\mathcal L}\sim \Ncdf(\mu + \beta W, W),
\end{equation*}
where $W$ follows an {\em inverse Gaussian distribution}, denoted by
$\text{IG}(\delta, \sqrt{\alpha^2-\beta^2})$.

It is easily seen from the moment-generating function that the NIG distribution is preserved under linear combinations, provided
the variables share the parameters $\alpha$ and $\beta$. For this
and other reasons, the NIG distribution is popular in many areas of
financial modelling; for example, it gives rise 
to the normal inverse Gaussian L\'evy process, which may be represented
as a Brownian motion with a time change.

In the setting here, we consider the {\em NIG factor copula}. This is
not directly derived from the multivariate NIG distribution, but
determined through a factor structure instead. The factor structure,
which 
was applied e.g.\ in \citep{Kalemanova2007} for calibrating CDO's,
gives additionaly flexibility as it does not force the components to
have a mixing variable $W$.
The following proposition introduces the NIG factor model and some of
its properties.
\begin{proposition}
  \label{prop:NIG}
  Let $Z\sim \text{NIG}(\alpha, \beta, \mu, \delta)$ and
  $Z_i\sim \text{NIG}(\alpha, \beta, \mu_i, \delta_i)$,
  $i=1,\ldots, n$ be independent NIG-distributed random
  variables. Then:
  \begin{enumerate}
  \item  $X_i = Z + Z_i\sim \text{NIG}(\alpha,\beta,\mu+\mu_i,
  \delta+\delta_i)$,
\item and 
  \begin{align}
    \text{Cov}(X_i,X_j) &= \text{Var(Z)},\nonumber\\
    \text{Corr}(X_i,X_j) &= \frac{\delta}{\sqrt{(\delta+\delta_i)
                           (\delta+\delta_j)}}. \label{eq:6}
  \end{align}
\end{enumerate}
\end{proposition}
\begin{proof}
  \begin{enumerate}
  \item This follows directly from the moment-generating function. 
  \item For the covariance,
    \begin{align*}
      \text{Cov}(X_i,X_j)
      &= \E[(Z+Z_i) (Z+Z_j)] - \E[Z+Z_i] \E[Z+Z_j]\\
      &= \E[Z^2] -(\E Z)^2.
    \end{align*}
    The correlation is determined directly from \eqref{eq:5}. 
  \end{enumerate}
\end{proof}

The NIG factor copula is obtained by transforming the margins to
uniforms (see Sklar's Theorem), giving (e.g.\
\citep{krupskii2013factor}):
\begin{equation*}
  C_{r^S, r^F}(F_{r^S}(r^S), F_{r^F}(r^F)) = \int_\mathbb{R}
  F_{Z_1}(F_{X_1}^{-1} \circ F_{r^S}(r^S) -z) \cdot
  F_{Z_2}(F_{X_2}^{-1} \circ F_{r^F}(r^F) -z) \cdot
  f_Z(z) dz
  \end{equation*}
If the margins are continuous, then Spearman's rho of NIG factor
copula is 
\begin{equation*}
  \rho_S = 12 \int \int \int_{\mathbb{R}^3}
  F_{X_1}(x_1) \cdot
  F_{X_2}(x_2) \cdot
  f_{Z_1}(x_1-z) \cdot
  f_{Z_2}(x_2-z) \cdot
  f_Z(z) dx_1 dx_2 dz - \frac{1}{48}.
  \end{equation*}

% \begin{proof}
%   \begin{align}
%   \rho_S(r^S, r^F) &= \rho\{F_{r^S}(r^S), F_{r^F}(r^F)\} \\
%     &= \rho\{F_{X_1}(X_1), F_{X_2}(X_2)\} \\
%     &= 12 \cdot \mathbb{E}\{F_{X_1}(X_1) \cdot F_{X_2}(X_2) \} - \frac{1}{48}\\
%     &= 12 \cdot \int \int_{\mathbb{R}^2} F_{X_1}(X_1) \cdot F_{X_2}(X_2) dF_{X_1,X_2}(x_1,x_2)\\
%     \end{align}
%   Because
%   \begin{align}
%     F_{X_1,X_2}(x_1,x_2) &= \mathbb{P}(X_1 \leq x_1, X_2 \leq x_2)\\
%     &= \mathbb{P}(Z_1 \leq x_1 - Z, Z_2 \leq x_2 - Z) \\
%     &= \int_\mathbb{R}\mathbb{P}(Z_1 \leq x_1 - z) \cdot \mathbb{P}(Z_2 \leq x_2 - z) \cdot f_Z(z) dz,
%     \end{align}
%   so,
%   \begin{align}
%     \rho_S(r^S, r^F) = 12 \cdot \int \int \int_{\mathbb{R}^3} F_{X_1}(x_1) \cdot F_{X_2}(x_2) \cdot f_{Z_1}(x_1 -z) \cdot f_{Z_2}(x_2 -z) \cdot f_{Z}(z) dx_1 dx_2 dz -\frac{1}{48}
%     \end{align}
%   \end{proof}


\subsection{Other Copula}\label{subsec:other-copula}
\natp{\em [Why is there a separate subsection instead of 3.1.4?]}

\natp{\em [Also, the reason to include the Packett copula needs to be
  made more clear; maybe with some evidence of what we see in the
  data? Or with explaining that other copulas do not have the
  property, and what it means that they do not have the property.]}

The Plackett copula has an expression
\begin{align}
    \bm{C}_{\theta}(u,v) &= \frac{1+(\theta-1)(u+v)}{2(\theta-1)}
                         - \frac{\sqrt{\{
    1+(\theta-1)(u+v)\}^2 - 4uv\theta(\theta-1)}}{2(\theta-1)}
    \end{align}
\begin{align}
    \rho_S(\theta) = \frac{\theta+1}{\theta-1} - \frac{2\theta \log \theta}{(\theta-1)^2}
    \end{align}\medskip

We include Placket copula in our analysis as it possesses a special property,
the cross-product ratio is equal to the dependence parameter
\begin{align}
    &\phantom{=} \frac{\p(U \leq u, V \leq v) \cdot \p(U > u, V > v)}
    {\p(U \leq u, V > v) \cdot \p(U > u, V \leq v)}\nonumber\\
    &= \frac{\bm{C}_\theta(u,v)\{1-u-v+\bm{C}_\theta(u,v)\}}{\{u-\bm{C}_\theta(u,v)\}\{v-\bm{C}_\theta(u,v)}\nonumber\\
    &= \theta.
    \end{align}\medskip

That is, the dependence parameter is equal to the ratio between number of concordence pairs and number of discordence pairs of a bivariate random variable.


%%% Local Variables:
%%% mode: latex
%%% TeX-master: "SRM"
%%% End:
