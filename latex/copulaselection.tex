\subsection{Copula Selection}\label{subsec:copula-selection}
The dependency structure of price data changes across time, in which both the dependency parameters and the type of dependency, dependencies between cryptos and the BTCF are no exception.
In this hedging exercise, we find a best fitting copula to model the dependency for every set of training data.
We select the best fitting copula that provider the lowest Akaike Information Criterion (AIC)
$$ \text{AIC} = 2k- 2 \log(L)$$, where $k$ is the number of estimated parameteres and $L$ is the likelihood. \medskip

% they tend to suggest the same copula as the best fitting one.
%Simulation studies has also been carried out to compare different copula selection methods, see \cite{}.
Notice that there are other model selection procedure and criteria, e.g. TIC , likelihood ratio test.
For a survey of model selection and inference, see \cite{anderson1998comparison}.
Among various copula selection procedures, AIC is a popular choice for its applicability, for example \cite{breymann2003dependence} use the AIC to select best fitting copulae.
In our case, the AICs are calculated only with dependency likelihood since the marginals are modelled in kernel density.
The selected copula will then be used to search for OHR. \medskip
%We consider the copula with the lowest AIC for a particular set of data the best fitting one and use it to generate OHR.