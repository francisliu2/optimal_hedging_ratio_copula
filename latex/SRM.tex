\documentclass[11pt,a4paper,english]{article}
\usepackage[titletoc, title]{appendix}
\usepackage{amsmath}
\usepackage{amssymb}
\usepackage{bm}
\usepackage{array}
\usepackage{babel}
\usepackage{bbding}
\usepackage{color}
\usepackage[normal]{caption}
\usepackage{subcaption}
\usepackage{epsfig}
\usepackage{graphicx}
\usepackage{pdflscape}
\usepackage{lipsum}
%\usepackage{multirow}
\usepackage{psfrag}
\usepackage{proofapnd}
\usepackage[round]{natbib}
\usepackage[table,xcdraw]{xcolor}
\usepackage{afterpage}

%\usepackage{enumerate}% http://ctan.org/pkg/enumerate
%\usepackage{cjk}
%\usepackage{CJK}
%\newcommand{\zh}[1]{\begin{CJK}{UTF8}{bsmi}#1\end{CJK}}
%\newcommand{\zhs}[1]{\begin{CJK}{UTF8}{gbsn}#1\end{CJK}}


%% Better roman enumeration list; no conflict with hyperref
\usepackage{enumerate}
\newenvironment{enumeroman} { %
\begin{enumerate}[(i)]
} { %
\end{enumerate}}%

\newenvironment{alphlist} { %
  \begin{enumerate}[(a)]
} { %
\end{enumerate}} %

%\usepackage{bbm}
%\usepackage[T1]{fontenc}
%\usepackage[normal]{caption2} % for caption
\usepackage{rotating}
\usepackage[margin=2cm]{geometry} % for the same margin
\usepackage{latexsym}
%\usepackage{subfig}
\usepackage{float}
\usepackage{setspace}
\usepackage{slashbox}
\usepackage{enumitem}
%\usepackage{tablefootnote}
\usepackage{multirow, tabularx,longtable, ragged2e, booktabs, siunitx}
%\usepackage{footnote}
%\usepackage{tablefootnote}
%\usepackage{footnotehyper}
%\makesavenoteenv{table, tabularx}
%\makesavenoteenv{}

\sisetup{
%    per-mode = symbol,
    output-decimal-marker = {.},
%    group-minimum-digits = 4,
%    range-units = brackets,
%    list-final-separator = { \translate{and} },
%    list-pair-separator = { \translate{and} },
%    range-phrase = { \translate{to (numerical range)} },
}
\newcommand{\ra}[1]{\renewcommand{\arraystretch}{#1}}
\newcolumntype{Y}{>{\centering\arraybackslash}X}
\newcolumntype{C}{>{\Centering\arraybackslash}X}
\newcolumntype{s}{>{\hsize=.2\hsize \Centering\arraybackslash}X}

\usepackage{authblk}
\usepackage{hyperref}
\usepackage{indentfirst} % Macht eine Einrückung nach der Section
\bibliographystyle{ecta}

\definecolor{markergreen}{rgb}{0.6, 1.0, 0}
\definecolor{darkgreen}{rgb}{0, .5, 0}
\definecolor{darkred}{rgb}{.7,0,0}
\definecolor{markergreen}{rgb}{0.6, 1.0, 0}
\definecolor{darkgreen}{rgb}{0, .5, 0}
\definecolor{darkorange}{rgb}{1,0.3,0}
\definecolor{darkred}{RGB}{.7,0,0}
\definecolor{darkblue}{RGB}{0,29,245}
\definecolor{orange}{RGB}{239, 133, 54}
\definecolor{lightblue}{RGB}{59, 188, 175}

%https://stackoverflow.com/questions/64369710/what-are-the-hex-codes-of-matplotlib-tab10-palette
\definecolor{plt1}{RGB}{31, 119, 180}
\definecolor{plt2}{RGB}{255, 127, 14}
\definecolor{plt3}{RGB}{44, 160, 44}
\definecolor{plt4}{RGB}{214, 39, 40}
\definecolor{plt5}{RGB}{148, 103, 189}
\definecolor{plt6}{RGB}{140, 86, 75}
\definecolor{plt7}{RGB}{227, 119, 194}
\definecolor{plt8}{RGB}{127, 127, 127}
\definecolor{plt9}{RGB}{188, 189, 34}
\definecolor{plt10}{RGB}{23, 190, 207}

\providecommand{\marker}[1]{\fcolorbox{markergreen}{markergreen}{{#1}}}
\providecommand{\mj}[1]{\textcolor{darkred}{#1}}
\providecommand{\francis}[1]{\textcolor{darkgreen}{#1}}
\providecommand{\natp}[1]{\textcolor{darkorange}{#1}}

\setlist[itemize]{leftmargin=*}
\setlist[description]{leftmargin=*}

%\setlength{\topmargin}{0.0 in} \setlength{\textwidth}{6in}
%\setlength{\oddsidemargin}{0.5in}
%\setlength{\evensidemargin}{-0.01in} \setlength{\textheight}{9in}
\captionsetup{font={onehalfspacing,small}, labelfont=bf}

\title{\LARGE \bf Hedging Cryptos with Bitcoin Futures}
%\title{\natp{\large\bf Hedging Cryptocurrencies with Bitcoin Futures}}
\author{
	\begin{tabular}[t]{ccc}
		\and
        Francis Liu\thanks{
			Department of Business and Economics, Berlin School of Economics and Law, Badensche Str. 52, 10825 Berlin, Germany.
            Blockchain Research Center, Humboldt-Universität zu Berlin, Germany.
            International Research Training Group
1792, Humboldt-Universität zu Berlin, Germany.
     E-mail: \texttt{Francis.Liu@hwr-berlin.de}.}

		 \and
        Natalie Packham\thanks{
			Department of Business and Economics, Berlin School of Economics and Law, Badensche Str. 52, 10825 Berlin, Germany.
            International Research Training Group 1792, Humboldt-Universität zu Berlin, Germany.
     E-mail: \texttt{packham@hwr-berlin.de}.}

        \and
		Meng-Jou Lu
        \thanks{
             Department of Finance, Asia University, 500, Lioufeng Rd., Wufeng, Taichung 41354, Taiwan
             Department of Finance, Asia University, 500, Lioufeng Rd., Wufeng, Taichung 41354, Taiwan
     E-mail: \texttt{mangrou@gmail.com}.}

		 \and
         Wolfgang Karl H\"ardle\thanks{
			Blockchain Research Center, Humboldt-Universit\"at zu Berlin, Germany. Wang Yanan Institute for Studies in Economics, Xiamen University, China. Sim Kee Boon Institute for Financial Economics, Singapore Management University, Singapore. Faculty of Mathematics and Physics, Charles University, Czech Republic. National Yang Ming Chiao Tung University, Taiwan.
     E-mail: \texttt{haerdle@wiwi.hu-berlin.de}.}
        \thanks{ Financial support of the European Union's Horizon 2020 research and innovation program ``FIN-TECH: A Financial supervision and Technology compliance training programme" under the grant agreement No 825215 (Topic: ICT-35-2018, Type of action: CSA), the European Cooperation in Science \& Technology COST Action grant CA19130 - Fintech and Artificial Intelligence in Finance - Towards a transparent financial industry, the Deutsche Forschungsgemeinschaft's IRTG 1792 grant,
                 the Yushan Scholar Program of Taiwan %(\begin{CJK}{UTF8}{gbsn}\zh{台灣玉山學者計劃}\end{CJK}),
                 the Czech Science Foundation's grant no. 19-28231X / CAS: XDA 23020303, as well as support by Ansar Aynetdinov (\texttt{ansar.aynetdinov@hu-berlin.de}) are greatly acknowledged.
     }
	\end{tabular}
}
\date{This version: \today}
%%%%%%%%%%%%%%%%%%%%%%%%%%%%%%%%%%%%%%%%%%%%%%%%%%%%%%%%%%%%%%%%%%%%%%%%%%%%%%%%%%%%%%%%%%%%%%%%
\renewcommand{\baselinestretch}{1.2}
%\newcommand{\indicator}{$1{\hskip -2.5 pt}\hbox{I}$}
\newcommand{\indicator}{I}
%%
%% $Id: Definitions.tex,v 1.6 2008/07/26 14:55:50 natalie Exp $
%% $Source: /Users/natalie/cvs/tex/dynamics/Definitions.tex,v $
%% $Date: 2008/07/26 14:55:50 $
%% $Revision: 1.6 $
%%

\usepackage{mathrsfs}

%% GENERAL DEFINITIONS
\unitlength1cm

%% COMMAND DEFINITIONS
\newcommand{\E}{{\mathbb{E}}}
%%\renewcommand{\E}{{\mathds E}}
%%\renewcommand{\E}{{\varmathbb{E}}}
%%\renewcommand{\E}{{\mathrm{I\!E}}}
\providecommand{\R}{{\mathbb{R}}}
\newcommand{\T}{{\mathbb{T}}}
\newcommand{\Fb}{{\mathbb{F}}}
\newcommand{\Eqn}{{\mathbb{E}}_{{\bf Q}_N}}
\newcommand{\Eq}{{\mathbb{E}}_{{\bf Q}}}
\newcommand{\Eqm}{{\mathbb{E}}_{{\bf Q}_M}}
\newcommand{\EqT}{{\mathbb{E}}_{{\bf Q}_T}}
\newcommand{\EqTz}{{\mathbb{E}}_{{\bf Q}_{T_2}}}
\newcommand{\EqTe}{{\mathbb{E}}_{{\bf Q}_{T_1}}}
\newcommand{\EqSe}{{\mathbb{E}}_{{\bf Q}_{S^1}}}
\newcommand{\EqSz}{{\mathbb{E}}_{{\bf Q}_{S^2}}}
\newcommand{\p}{{\bf P}}
%%\renewcommand{\p}{{\mathds{P}}}
%%\renewcommand{\p}{{\varmathbb{P}}}
%%\renewcommand{\p}{{\mathrm{I\!P}}}
\newcommand{\pas}{\text{{\bf P}--a.s.}}
\newcommand{\paa}{\text{{\bf P}--a.a.}}
\newcommand{\qas}{\text{{\bf Q}--a.s.}}
\newcommand{\e}{{\bf e}}
\newcommand{\q}{{\bf Q}}
\newcommand{\qn}{{\bf Q}_N}
\newcommand{\qm}{{\bf Q}_M}
\newcommand{\qT}{{\bf Q}_T}
\newcommand{\qTz}{{\bf Q}_{T_2}}
\newcommand{\qTe}{{\bf Q}_{T_1}}
\newcommand{\qS}{{\bf Q}_S}
\newcommand{\qSe}{{\bf Q}_{S^1}}
\newcommand{\qSz}{{\bf Q}_{S^2}}
\newcommand{\F}{{\cal F}}
\newcommand{\G}{{\cal G}}
\newcommand{\A}{{\cal A}}
\newcommand{\Hc}{{\cal H}}
\newcommand{\dP}{{\rm d}{\bf P}}
\newcommand{\du}{{\rm d}u}
%%\newcommand{\dt}{{\rm d}t}
\newcommand{\dd}{{\rm d}}
\newcommand{\df}{{\rm \bf DF}}
\providecommand{\N}{{\mathbb N}}
\providecommand{\Ncdf}{{\rm N}}
%\renewcommand{\Ncdf}{{\Phi}}
\newcommand{\n}{{\rm n}}
\newcommand{\emb}{\bf \em}
\newcommand{\1}{\textbf{1}}
\newcommand{\qs}{{\q_{\rm Swap}}}
\newcommand{\fx}{{\rm fx}}
\newcommand{\V}{{\rm Var}}
%\newcommand{\C}{{\bf C}}
\newcommand{\Om}{{\Omega}}
\providecommand{\limn}{\ensuremath{\lim_{n\rightarrow\infty}}}
\providecommand{\qv}[2]{\ensuremath{\langle #1,#1\rangle_{#2}}}

%% ENVIRONMENT DEFINITIONS
%\newtheorem{prop}{Proposition}[section]
%\newtheorem{theo}{Theorem}[section]
%\newtheorem{lem}{Lemma}[section]
%\newtheorem{ass}{Assumption}[section]
%\newtheorem{cor}{Corollary}[section]
%\newtheorem{aufg}{Exercise}[section]
%\newtheorem{defi}{Definition}[section]

\ifx\prop\undefined
\newtheorem{prop}{Proposition}[section]
\fi
\newtheorem{theo}[prop]{Theorem}
\newtheorem{lem}[prop]{Lemma}
\newtheorem{cor}[prop]{Corollary}
\newtheorem{defi}[prop]{Definition}

%% enumeration in lists
\providecommand{\labelenumi}{{\rm (\roman{enumi})}}
   %\setlength{\topsep}{0cm}
    \setlength{\labelsep}{0.3cm}
    %\setlength{\itemindent}{0cm}
   \setlength{\leftmargin}{10cm}
    \setlength{\labelwidth}{5cm}

\providecommand{\cadlag}{c\`adl\`ag }
\providecommand{\cadlagns}{c\`adl\`ag}
\providecommand{\caglad}{c\`agl\`ad }
\providecommand{\cad}{c\`ad}
\providecommand{\cag}{c\`ag}
\providecommand{\levy}{L\'evy\ }
\providecommand{\levyns}{L\'evy}
\providecommand{\levyito}{L\'evy-It\^o\ } 
\providecommand{\levykhinchin}{L\'evy-Khinchin\ }
\providecommand{\D}{\ensuremath{D(\R_+,\R)}}
\providecommand{\Dsig}{\ensuremath{D(\R_+, \R_+\setminus\{0\}})}
\providecommand{\Dd}{\ensuremath{D(\R_+,\R^d)}}
\providecommand{\C}{\ensuremath{C(\R_+,\R)}}
\providecommand{\Cd}{\ensuremath{C(\R_+,\R^d)}}
\providecommand{\rpos}{\ensuremath{{[0,\infty)}}}

\def\Z{{\mathbb Z}}
%\def\N{{\mathbb N}}
%\def\R{{\mathbb R}}
%\def\C{{\mathbb C}}
%\def\H{{\mathbb H}}
\def\P{{\mathbb P}}
\def\Q{{\mathbb Q}}
%\def\E{{\mathbb E}}
\def\I{{\mathbb I}}
%\def\T{{\mathbb T}}
%\def\F{{\mathbb F}}
\def\M{{\mathbb M}}
%\def\Hc{{\mathcal H}}
\def\Mc{{\mathcal M}}
\def\filtration#1{{\ensuremath\mathcal{#1}}}
%\def\filt{{\mathcal F}}
\def\tp{\tilde{\p}}
\providecommand{\vec}[1]{\ensuremath{\bm #1}}
\providecommand{\vecb}[1]{\ensuremath{\bm #1}}
\providecommand{\abs}[1]{\ensuremath{\lvert#1\rvert}}
\providecommand{\norm}[1]{\ensuremath{\lVert#1\rVert}}
\providecommand{\var}{\ensuremath{\text{Var}}}
\providecommand{\cov}{\ensuremath{\text{Cov}}}
\providecommand{\borel}[0]{\ensuremath{\mathcal{B}}}
\providecommand{\intinf}[0]{\ensuremath{\int_{-\infty}^\infty}}
\providecommand{\intpos}[0]{\ensuremath{\int_0^\infty}}
\providecommand{\intneg}[0]{\ensuremath{\int_{-\infty}^0}}
\providecommand{\todo}[1]{\footnote{#1}}
\providecommand{\dynkin}[0]{\ensuremath{\mathcal D}}
\providecommand{\ce}[2]{\ensuremath{\E(#1|\filtration{#2})}}
\providecommand{\inv}[1]{\ensuremath{#1^{(-1)}}}
\providecommand{\os}[2]{\ensuremath{#1^{(#2)}}}
\providecommand{\pos}[2]{\ensuremath{h_{#1}(#2)}}
%\providecommand{\poslong}[2]{\ensuremath{h(#1, #2)}}
\providecommand{\poslong}[3]{\ensuremath{h_{#1, #2}(#3)}}

%% Class of finite variation processes
\providecommand{\classfv}{\ensuremath{\mathscr V}}
\providecommand{\classv}{\ensuremath{\mathscr V}}
%% Stochastic integral operator
\providecommand{\stint}{\ensuremath{\cdotp}}
\providecommand{\classh}{\ensuremath{\mathscr H^2}}
\providecommand{\classhloc}{\ensuremath{\mathscr H^2_{\rm loc}}}
\providecommand{\classm}{\ensuremath{\mathscr M}}
\providecommand{\classmloc}{\ensuremath{\mathscr M_{\rm loc}}}
\providecommand{\classl}{\ensuremath{L^2}}
\providecommand{\classlloc}{\ensuremath{L^2_{\rm loc}}}
\providecommand{\classa}{\ensuremath{\mathscr A}}
\providecommand{\classaloc}{\ensuremath{\mathscr A_{\rm loc}}}
\providecommand{\classalocpos}{\ensuremath{\mathscr A_{\rm loc}^+}}
\providecommand{\classp}{\ensuremath{\mathscr P}}
\providecommand{\classo}{\ensuremath{\mathscr O}}
\providecommand{\classs}{\ensuremath{\mathscr S}}
\providecommand{\classsp}{\ensuremath{\mathscr S_p}}
\providecommand{\nullset}{\ensuremath{\mathscr N}}

\providecommand{\ito}{It\^o }
\providecommand{\itos}{It\^o's\, }

\providecommand{\variation}[2]{\ensuremath{\rm V_{#1}(#2)}}
\renewcommand{\H}{\ensuremath{\mathcal H}}
%% CPO distribution
\providecommand{\cpo}{\ensuremath{{\rm CPO}}}
\providecommand{\Fsigma}{\ensuremath{\mathcal \F_\infty^\sigma}}
\providecommand{\sigd}{\ensuremath{\mathscr D}}

%% Credit spreads
\providecommand{\s}{{\bf s}}
\providecommand{\classu}{\ensuremath{\mathscr U}}

\providecommand{\sX}{\ensuremath{\mathcal X}}
\providecommand{\sY}{\ensuremath{\mathcal Y}}
\providecommand{\dx}{\ensuremath{\frac{\partial}{\partial x}}} %%
\providecommand{\dt}{\ensuremath{\frac{\partial}{\partial t}}} %%
\providecommand{\dy}{\ensuremath{\frac{\partial}{\partial y}}} %%
%\newcommand{\argmax}{\operatornamewithlimits{argmax}}
%\newcommand{\argmin}{\operatornamewithlimits{argmin}}


\begin{document}

\newtheorem{lemma}{Lemma}
\newtheorem {proposition}[lemma]{Proposition}
\newtheorem {corollary}{Corollary}
\newtheorem {theorem}{Theorem}
\newtheorem{claim}[lemma]{Claim}
\newtheorem{comment}[lemma]{Comment}
\newtheorem{example}[lemma]{Example}
\newtheorem{fact}[lemma]{Fact}
\newtheorem{defn}[lemma]{Definition}
\newtheorem{exercise}{Exercise}[section]

\newtheorem{programming}[exercise]{Programming assignment}
\newenvironment{proof}{{\flushleft\textbf{\textsl{Proof.\ \ }}}}{\hfill{\hfill\rule{2mm}{2mm}}}
\pagenumbering{arabic}

\maketitle

%%%%%%%%%%%%%%%%%%%%%%%%%%%%%%%%%%%%%%%%%%%%%%%%%%%%%%%%%%%%%%%%%%%%%%%%%%%%%%%%%%%%%%%%%%%%%%%
\begin{abstract}
  \footnotesize{

The introduction of derivatives on Bitcoin enables investors to hedge risk exposures in cryptocurrencies.
Because of volatility swings and jumps in cryptocurrency prices, the traditional variance-based approach to obtain hedge ratios is infeasible.
As a consequence, we consider two extensions of the traditional approach: first, different dependence structures are modelled by different copulae, such as the Gaussian, Student-t, Normal Inverse Gaussian and Archimedean copulae;
second, different risk measures, such as value-at-risk, expected shortfall and spectral risk measures are employed to find the optimal hedge ratio.
Extensive out-of-sample tests  in the time period December 2017 until May 2021 give insights in the practice of hedging various cryptos and crypto indices, including Bitcoin,  Ethereum, Cardano, the CRIX index and a number of
crypto-portfolios.
Evidence shows that BTC futures can effectively hedge BTC and BTC-involved indices.
This promising result is consistent across different risk measures and
copulae except for the Frank copula.
On the other hand, we observe complex and diverse dependence
structures between non-BTC-related cryptoassets and the BTC futures. 
As a consequence, the hedge performance of non-BTC-related
cryptoassets is mixed and even infeasible for some assets.
%For other cryptos,
%The choice of the optimal copulae  different dependence structure, some of the cryptos cannot be hedged by BTC futures.
%The choice of the risk measures or copulae;

%Different dependence structures.

%Results of other cryptos and indices are inconclusive.

% Move Meng Jou to third.

% what is the new stuffs

% \natp{\em Old abstract:}\\
% The introduction of derivatives on Bitcoin \natp{(suggest to delete in
%   abstract:, in particular the launch of futures contracts on CME in
%   December 2017 and introduction of cryptocurrency indices like the
%   CRIX or the Bloomberg Galaxy Crypto Index)} 
% enables investors to hedge risk exposures in Bitcoin\natp{, other
%   cryptocurrencies or crypto-portfolios} with futures or
% contingent claims \natp{(delete: on indices)}.
% We investigate \natp{different} methods of \natp{determining the (was: finding an)} optimal hedge ratio
% \natp{(delete: $h^*$)}\natp{, taking into account (was: under)}
% different dependence structures modeled by copulae and \natp{(delete: employing)}
% different optimality definitions based on a range of risk measures. 
% Because of volatility swings and jumps in Bitcoin prices, the
% traditional variance-based approach to obtain the hedge ratios is
% infeasible. 
% The techniques \natp{\em [Which techniques?]} are therefore
% generalised to various risk measures, such as value-at-risk, expected
% shortfall and more generally, spectral risk measures.
% In addition, we deploy different copulae for capturing the dependency between spot and futuresreturns, such as the Gaussian, Student-$t$,
% NIG and Archimedean copulae. Various measures of hedge effectiveness in out-of-sample tests give insights in the practice of hedging Bitcoin.
% We find that across copulae and risk measures, the hedge effectiveness are very similar with the exception of the Frank copula, Expected Shortfall 99\% and Value-at-Risk 99\%.
% Our findings are based on an analysis for the time span from 29/12/2017 to 27/05/2021.
% The results allow investors to construct a stable portfolio with digital assets.
% \\

\noindent {\bf JEL classification: G11, G13}  \\ %G11 and G13 in QF
%\noindent {\bf JEL classification: % C38,
%  C53, % F34,
%  G11, G17}  \\

\noindent {\bf Keywords:} Cryptocurrencies, risk management, hedging,
copulas}
\pagestyle{empty}
\end{abstract}

%\natp{{\bf TODO}:
%  \begin{itemize}
%  \item Please generate all graphics as pdf and possibly eps. pdf is a
%    vector graphic format, so it scales well. eps may be
%    required during the publishing process.
%  \item Plackett copula: This is a bivariate copula only, which is
%    probably one of the reasons it is not commonly found in finance
%    applications. It does not have tail dependence, which is one of
%    the things we typically look for in finance. We need a compelling
%    reason why it is of interest, otherwise I suggest to remove it.
%  \item We need to discuss if we use copulas or copulae. I have a
%    preference for copulas, which is the terminology used by Nelsen,
%    McNeil et al., Joe.
%  \item The introduction needs to be revised, see comments
%    below. Think about what needs to go into the introduction and what
%    does not, and then stick to this in a concise way. Also, re-read
%    every sentence 2-3 times and think carefully about what it says
%    and what it is supposed to say.
%  \end{itemize}
%}


%\natp{add
%  \url{https://onlinelibrary.wiley.com/doi/pdfdirect/10.1002/fut.22050?casa_token=l6mOC3XtpSUAAAAA\%3Aa4ikFS_PgHmAFWRzYq1LqDH3Go2esWWO_2wmNI6Ql7B166Cn1eouEiwBLxkMqt__5sCEP47JDjWodPQ}}

\clearpage

\tableofcontents

\clearpage
%%%%%%%%%%%%%%%%%%%%%%%%%%%%%%%%%%%%%%%%%%%%%%%%%%%%%%%%%%%%%%%%%%%%%%%%%%%%%%%%%%%%%%%%%%%%%%%%%%%%%%%%%%%%%%%%%%%%%%%%%%%%%%%%%%%%%%%%%%%%%%%%%%%%%%
%%%%%%%%%%%%%%%%%%%%%%%%%%%%%%%%%%%%%%%%%%%%%%%%%%%%%%%%%%%%%%%%%%%%%%%%%%%%%%%%%%%%%%%%%%%%%%%%%%%%%%%%%%%%%%%%%%%%%%%%%%%%%%%%%%%%%%%%%%%%%%%%%%%%%%
\section{Introduction}\label{sec:introduction}
Cryptocurrencies (CC's) are a fast-growing asset class,
with many more CCs now available on the market since the first
cryptocurrency Bitcoin (BTC) surfaced \citep{nakamoto2009}.
In response to the rapid development of the CC market, the CME Group
launched exchange-traded BTC futures contracts in December
2017.
At the time of writing, the CME is the only exchanged offering
regulated crypto futures. 
The average daily volume and open interest of the CME BTC futures are
\$2,518 M and \$2,836 M respectively.\natp{\em [Add an approximate
  date. Where are these figures from?]}
% \natp{\em [I think the following sentence can be deleted. Readers of
%   Quantitative Finance know the advantages of futures contracts.]}
Because it is regulated, the CME BTC derivatives market  
is an attractive way for institutional
investors to participate in or manage their exposure in the crypto
market.
%The price
%of BTC even surged to \$ 64,500 in mid-April 2021 up by 460\% from
% from \$ 11,500 six months earlier in October 2020 and up by 850\%
% from a year earlier. Just a month later, by mid-May 2021, the price
% had fallen to \$ 50,000, a one-month return of -22.5\%.
 % see tiingo_btc.csv 
As more individual and institutional investors are adding CCs and CC
derivatives to their portfolios, the need to understand
downside risks and find suitable ways to hedge against extreme risks
is created. 
Fom a risk management perspective, the roller-coaster ride
of crypto prices may create significant basis risk, even when using
simple hedges involving crypto portfolios and BTC futures. This
requires analysing the dependence structure of cryptos and
futures beyond linear correlation. 

In this paper, we examine static hedges of crypto portfolios
with Bitcoin futures. Owing to the asymmetry of crypto returns as
well as the occurrence of extreme events, we consider different 
dependence structures via a variety of copula models. We then optimise
the hedge ratio using different risk measures. A similar study was
conducted by \citep{barbi2014copula} for equity and FX portfolios.
\citet{barbi2014copula}'s work is based on \citet{cherubini2011copula}
to derive the distribution of linear combinations of margins with
copulae describing the dependence structure. We slightly extend their
results and come up with a formula for the linear combination of
random variables for our purpose.

The hedge ratio is the appropriate amount of futures contracts to hold
in order to eliminate the risk exposure in the underlying security.   
The determination of the optimal hedge ratio relies primarily
on the dependence between BTC and futures prices.
%In this paper, we investigate the performance of different copulae and
%risk measures in hedging Bitcoin and CRIX with Bitcoin
%futures. {\color{blue} only this portfolio?} \natp{\em [$\leftarrow$ Make this the first sentence of the paper!]}
Financial asset returns have long known to be non-Gaussian, see e.g.\
\citep{fama1963mandelbrot,Cont2001}. Specifically, Gaussian models
cannot produce the heavy tails and the asymmetry observed in 
asset returns, which in turn implies a consistent underestimation of
financial risks. 
%\natp{\cite{Cont2001}}. \natp{\em [non-Gaussian is not
  %very specific, a uniform r.v.\ is also non-Gaussian. Try to express
 % this in a more meaningful way, e.g.\ Financial data are known to
  %exhibit more extreme events than a normal distribution can capture.]}
Therefore, to minimize downside risk, one cannot solely rely on
second-order moment calculations. Moreover, variance as a risk measure
does not account for the variety of investors' utility functions. In
particular, it is known that  
investors are tail-risk averse, see \cite{menezes1980increasing}.
Copulae provide the flexibility to model multivariate random variables
separately by their margins and dependence structure.
The concept of copulae was originally developed (but not under this
name) by Wassily Hoeffding \citep{hoeffding1940masstabinvariante}
and later popularised by the work of Abe Sklar \citep{Sklar1959}.

Different risk measures account for investors' risk attitudes.
They serve as loss functions in the search process of the optimal
hedge ratio. Of the vast literature discussing the relationship between
risk measures and investors' risk attitudes, we refer readers to
\citet{artzner1999coherent} for an axiomatic 
approach of risk measure construction;
\citet{embrechts2002correlation} for reasoning of using expected
shortfall (ES) and spectral risk measures (SRM) in addition to
value-at-risk (VaR);
\citet{Acerbi2002} for direct linkages between risk measures and
investor's risk attitudes using the concept of a ``risk aversion
function''. 


In order to capture a variety of risk preferences, in addition to
variance, we include the risk measures value-at-risk (VaR), expected
shortfall (ES), and spectral risk measures (SRM). 
%Coherency is a very natural property that suggest diversification will reduce risk.
VaR is widely used by the finance industry and easy to understand. 
ES and SRM are chosen because of their coherence property, in
particular, they recognize diversification benefits.
SRM can also be directly related to an individual's utility function.
Examples are the exponential SRM and power SRM introduced by
\citet{dowd2008spectral}. 
%\natp{\em [Careful with wording! The
 %hedge is not optimal. The optimal hedge ratio that minimises a risk measure is chosen.]}
%In particular, the SRM proposed by \citet{Acerbi2002} accounts for investors' utility (i.e. risk preference).
%SRM is a weighted average of the quantities of a loss distribution, the weights of which depend on the investor's risk-aversion. In other words, 

% \natp{\em [This needs not go in the introduction. Just briefly mention the risk measures, and possibly that ES and SRM are chosen because of their coherence.]}
%\medskip
%
%\natp{The introduction should go along the lines:
%  \begin{itemize}
%  \item 1-2 sentence: Which problem is solved in this paper?
%  \item Background Bitcoin: Growth, but roller-coaster ride, institutional investment,
%    exchange-traded futures (the exchange is important!) (5 sentences,
%    with references!)
%  \item Significant tail risks and basis risk lead to the need to
%    investigate even static hedges (=futures) with more refined
%    methods than minimum-variance based (reference to Ederington
%    here; this uses variance as risk measure and correlation as
%    dependence measure).
%  \item To capture empirical properties, extend to other risk measures
%    and dependence models. 
%  \end{itemize}
%  }

In this work, we study the effectiveness of hedging various CC's and
crypto indices using Bitcoin futures under copula models and different
risk preferences. 
In an extensive back-test,\footnote{We thank the data provider
  Tiingo (\href{https://www.tiingo.com/}{https://www.tiingo.com/}) for
  providing the crypto price data.}
 we find the ability of the BTC futures to hedge BTC and BTC-related
 indices promising, regardless of the choices of the copula (with the
 exception of the Frank copula) and risk measure. 
On the other hand, the ability of BTC futures to hedge other cryptos
\natp{depends on various factors, such as ... (was: is inconclusive)
  {\em [Something more useful should be said here. ``is inconclusive''
  means: ``stop reading here'']}}. 
Instead of suggesting a particular copula or risk measures, we discuss
the characteristics of different settings.  


The paper is organized as follows. Section 2 introduces the notion of
an optimal hedge ratio; Section 3 decribes the method of estimation of
copulae; Section 4 provides the empirical result; Section 5
concludes. 
All calculations in this work can be reproduced with the data and code
available at \href{http://www.quantlet.com/}{www.quantlet.com
  {\includegraphics[height=\baselineskip]{_pics/qletlogo_tr.png}}}. 
%%%%%%%%%%%%%%%%%%%%%%%%%%%%%%%%%%%%%%%%%%%%%%%%%%%%%%%%%%%%%%%%%%%%%%%%%%%%%%

\section{Optimal hedge ratio}\label{sec:optimal-hedge-ratio}

\subsection{Distribution of hedge portfolio}\label{subsec:DHP}
We form a portfolio with two assets, consisting of one unit in the
spot asset and a short position of $h$ units of a futures contract,
for example one Bitcoin and a short position in a CME Bitcoin
futures contract. 
The objective is to minimize the risk of the exposure in the spot. 
Let $R^S$ and $R^F$ be the (discrete) returns of the spot and
futures price. The (discrete) return of the portfolio is\footnote{%
In practice, as the nominal investment in the futures is zero, $R^F$
is understood as the return on the notional amount underlying the
futures contract. In other words, if both the spot price $S_{t-1}$
and the futures price $F_{t-1}$ are 
normalised to $1$, then the portfolio return will be identical to the
portfolio value change $\Delta V = \Delta S - h \Delta F$, where $\Delta S =
S_t-S_{t-1}$, etc.}
\begin{equation*}
R^h = R^S -h R^F.
\end{equation*}
%\natp{\em [I fixed this, please check.] [We need to discuss the
%  footnote. Generally, the portfolio return is $R_p = \sum_{i=1}^n w_i
%  R_i$. With the futures contract, the notional investment in the
%  futures is zero, so the portfolio return is $(S_0 (1+R^S) -h F_0
%  R^F)/S_0-1 = R^S-h R^F$, if $S_0=F_0$.]}

To measure risk, we define a risk measure $\rho$ to be a mapping from
a financial position or its return, such as $R^h$, to a real number, which is often
interpreted as the amount of money to make the position acceptable
(e.g.\ to a regulator), see e.g.\ \citep{Foellmer2002}.
For example, a widely used risk measure is value-at-risk (VaR), which,
at the confidence level $\alpha$,
is derived from the $1-\alpha$ quantile of the return distribution. %  at the confidence level $\alpha$ is
% the absolute value of the $1-\alpha$-quantile of $R^h$, i.e., $\text{VaR}_{1-\alpha} =
% -F_{R^h}^{(-1)}(1-\alpha) = -\inf\{x \in \mathbb{R}: 1-\alpha \leq
% F_{R^h}(x) \}$, where $F_{R^h}$ is the distribution function of
% $R^h$.

If the portfolio reduces the risk of the spot position, then
we call this a hedge portfolio.
An optimal hedge ratio $h^*$ is a parameter that
minimizes the risk of the aforementioned portfolio
\begin{equation*}
h^* = \argmin_h \rho(R^h).
\end{equation*}

Obviously the cdf and pdf of $R^h$ and the risk measure depend on the
joint distribution of $R^S$ and $-hR^F$. However, optimising $h$
according to $f_{R^S,-hR^F}$ is unfavorable since one would need to
calibrate the joint pdf $f_{R^S,-hR^F}$ whenever updating $h$.
Another problem of using the joint pdf is that one lacks the
flexibility to model the margins separately from the dependence
structure. Copulae allow to overcome both of these problems. 

The advantage of using copulae is two-fold.
First, copulae are invariant under strictly
monotone increasing function \citep{schweizer1981nonparametric}, a
property used in Lemma \ref{lemma:copula} below. 
Second, copulae allow us to model the margins and dependence structure 
separately, a result known as Sklar's Theorem \citep{Sklar1959}, which
is given as Theorem \ref{theorem:sklar} below. 
See also \citep{Nelsen1999, joe1997multivariate, McNeil2005} for
Sklar's Theorem and more properties of copulae.

We adapt the definition of a two-dimensional copula from \citep{Nelsen1999} as follows.

\begin{defi} [A two-dimensional copula]
  A two-dimensional copula is a function $C: [0,1]^2 \mapsto [0,1]$ with following properties:
  \begin{enumerate}
    \item For every $u,v$ in $[0,1]$,
      \[C(u,0)= C(0,v)=0, \]
    \[C(u,1)= u \text{, and}\]
    \[C(1,v)= v;\]
    \item For every $u_1,u_2, v_1, v_2$ in $[0,1]$ such that $u_1 \leq u_2$ and $v_1 \leq v_2$,
    \[C(u_2,v_2)-C(u_2,v_1)-C(u_1, v_2)+C(u_1,v_1) \geq 0\].
    \end{enumerate}
  \end{defi}

The second property is called 2-non-decreasing.
In other words, the two-dimensional copula is a joint cdf of a two-dimensional random vector
on a unit square with uniform marginals.

The following Hoeffding-Sklar-Theorem (usually known as Sklar theorem) ensures the existence of copula.

\begin{theorem}[Hoeffding-Sklar-Theorem]
  \label{theorem:sklar}
  Let $F$ be a joint distribution function with marginal distributions
  $F_X$ and $F_Y$. Then, there exists a copula $C:[0,1]^2 \mapsto
  [0,1]$ such that, for all $x,y\in \R$
  \begin{equation}
    \label{eq:4}
    F(x,y)=C\{F_X(x), F_Y(y)\}.
  \end{equation}
  If the margins are continuous, then $C$ is unique; otherwise $C$ is
  unique on the range of the margins.

  Conversely, if $C$ is a copula and $F_X, F_Y$ are univariate
  distribution functions, then the function $F$ defined by (\ref{eq:4})
  is a joint distribution function with margins $F_X, F_Y$.
\end{theorem}

Indeed, many basic results about copulae can be traced back to early
works of Wassily Hoeffding \citep{hoedffding1940, hoedffding1941}. 
The works aimed to derive a measure of relationship of variables,
which is invariant under change of scale. 
See also \citet{hoeffding2012collected} for English translations of
the original papers written in German. 
%The following Lemma is not hard to prove.

\begin{lemma}
  \label{lemma:copula}
  Let $h>0$ and let $X$ and $Y$ be continuous random variables. Then,
  the joint distribution of the portfolio positions 
  can be expressed via the joint distribution of the securities as
  follows:
  \begin{align}
  C_{X, hY}\left(F_X(s),F_{hY}(t)\right) = C_{X,
    Y}\left(F_X(s),F_{Y}(t/h)\right), \quad s,t\in \R.
    \end{align}
  \end{lemma}

\begin{proof}
  Since copulae are invariant under strictly monotone increasing
  function \cite[Theorem 3 (i)]{schweizer1981nonparametric} or
  \cite[Theorem 2.4.3]{Nelsen1999}, 
  \begin{equation*}
    C_{X, hY}\left(F_X(s),F_{hY}(t)\right) = C_{X, Y}\left(F_X(s),F_{hY}(t)\right).
    \end{equation*}
Re-writing the second argument of the copula gives
\begin{equation*}
  F_{hY}(t) = \mathbb{P}(hY \leq t)
  = \mathbb{P}(Y \leq t/h)
  = F_Y(t/h).
\end{equation*}
\end{proof}

%The optimal hedge ratio is $h^\ast = \argmin_h \rho(Z)$, that is the best ratio that can minimize the risk of a hedged portfolio measured in terms of $\rho$ .
Leveraging these two features of copulae, \citet{barbi2014copula}
introduce the distribution of linear combinations of random variables
using copulae. 
We slightly edit their Corollary 2.1 of their work and yield the 
following expression of the distribution. 

\begin{proposition}
  \label{prop:dfrh}
  Let $X$ and $Y$ be two real-valued continuous random
  variables on a
  probability space $(\Omega, \F, \p)$ with
  absolutely continuous copula $C_{X, Y}$ and marginal distribution functions $F_{X}$
  and $F_{Y}$. Then, the distribution function of $Z=X-hY$, $h >0$,  is given by
  \begin{equation}
    \label{eq:3}
    F_{Z}(z) = 1- \int^1_0 D_1 C_{X, Y}
    \left[ u, F_{Y} \left\{ \frac{F^{(-1)}_{X}(u)-z}{h} \right\}
    \right]\, d u,   
  \end{equation}
  where, $F^{(-1)}$ denotes the inverse of $F$, i.e., the quantile
function.
\end{proposition}
Here, $D_1 C(u,v)=\displaystyle \frac{\partial}{\partial u}
C(u,v)$ and, see e.g.\ Equation (5.15) of \citep{McNeil2005},
\begin{equation}
  \label{eq:1}
  D_1 C_{X,Y}\{F_X(x), F_Y(y)\} = \p(Y\leq y|X=x).
\end{equation}
\begin{proof}
  Using the identity (\ref{eq:1}) gives
  \begin{align*}
    F_{Z}(z) &= \p(X - h Y\leq z) %
                 = \E\left\{\p\left(Y\geq \frac{X-z}{h}\Big|
                 X\right)\right\}\\[10pt]
               &= 1-\E\left\{\p\left(Y\leq \frac{X-z}{h}\Big|
                 X\right)\right\}% \\[10pt]
               = 1- \int_0^1 D_1 C_{X, Y}\left[u,
                 F_{Y}\left\{\frac{F^{(-1)}_{X}(u) -
                 z}{h}\right\}\right]\, d u.
  \end{align*}
  \end{proof}

%In addition to~\cite{barbi2014copula} we propose a more handy
%expression for the pdf of $Z$.
%\natp{\em [Please double-check the ``+'' signs in the second equation.]}\ \francis{ \em [the + sign is correct.]}

\begin{corollary} The pdf of $Z$ can be written as
  \begin{align}
  f_{Z}(z) &= h^{-1}\int_0^1 c_{X, Y} \left[
  F_{Y}\left\{\frac{F^{(-1)}_{X}(u)-z}{h}\right\}, u
  \right]
   \cdot
  f_{Y}
  \left\{\frac{F^{(-1)}_{X}(u)-z}{h}\right\} du, \label{eq:density1}
  \end{align}
  \end{corollary}
Note that the pdf of $Z$ in the above proposition can be assessed via numerical integration
as long as we have the copula density and the marginal
densities.
A multivariate generalised of the expression above and its proof can be found in the
appendix \ref{sec:appendix}.

\subsection{Backtesting Procedure}\label{sec:empirical-procedure}
First, we take the earliest 300 data points from the dataset 
as training data to obtain the optimal hedge ratio via the following steps:

\begin{enumerate}
\item \textbf{Construct univariate kernel density function (KDE)}:
  Construct the spot and futures' univariate kernel density functions seperately
  using the Gaussian kernel. The bandwidths are determined seperately by the refined plug-in method \citep[section
  3.3.3]{hardle2004nonparametric}.
\item \textbf{Calibrate copulae}:
  Calibrate the copulae outlined in section \ref{subsec:copulae} by the
  method of moments described in section \ref{subsec:simulated-method-of-moments}.
\item \textbf{Select copula}:
  Compute the Akaike Information Criterion (AIC). The copula with the
  best (i.e., lowest) AIC is used for the next step. 
  A discussion of this step is found in \ref{subsec:copula-selection}.
\item \textbf{Determine optimal hedge ratio}:
  Determine the optimal hedge ratios with respect to different
  risk measures numerically. 
  To do so, we draw samples from the calibrated copulae and KDEs 
  and search for the hedge ratio that gives the lowest risk measure. 
  The risk measures are outlined in \ref{subsec:spectral-risk-measures} 
  The minimisation algorithm \textit{scipy.optimize.minimize} from a Python package Scipy \citep{2020SciPy-NMeth} is used for the search of optimal hedge ratio.
\end{enumerate}

Next, we apply the optimal hedge ratio to the test data to obtain out-of-sample hedged portfolio returns.
The test data is the 5 data points subsequent to the last training data point. 
The out-of-sample portfolio returns is also 5 data points in length.

Finally, we roll forward by 5 data points and repeat the steps until the test data reach the end of the dataset. 
The collection of out-of-sample portfolio returns forms a non-overlapping time serie (rolling step size is equal to test data length) that represents the performance of 
the hedging methodology. See Section \ref{subsec:HP2} for results and discussions.  

The backtesting procedure without the copula selection step is also carried out to examine the effects of deploying different copula. 
Section \ref{subsec:HP1} discusses the effects. 

\section{Copulae and risk measures}\label{sec:crm}

Recall the definitions given 

\subsection{Dependence measures in copula terms}
This section introduces the dependence measures in Copula terms that are relevant to this work, 
they are the Kendall's tau, Spearman's rho, and quantile dependence. 
The sample, population versions, as well as the version written in copula, 
of the depednece measures are introduced as they will be used in the method of moments calibration described in Section \ref{subsec:simulated-method-of-moments}. 

The following definitions are adapted from \cite{Nelsen1999}. 

\begin{defi}[Concordance]
  Let $(x_i, y_i)$ and $(x_j, y_j)$ denote two realisations of a
  vector $(X, Y)$ of continuous random variables. 
  A pair of observations is concordant if $x_i<x_j$ and $y_i < y_j$, discordant if
  $x_i>x_j$ and $y_i < y_j$ or if $x_i<x_j$ and $y_i>y_j$. 
\end{defi}

The index of observations $i$ and $j$ are interchangable, so the case
$x_i>x_j$ and $y_i>y_j$ is covered.

\begin{defi}[Sample version of Kendall's tau]
  Let $\{(x_1, y_1), ..., (x_n, y_n)\}$ be realisations of a random vector $(X, Y)$,
  let $c$ denote the number of concordant pairs, and $d$ the number of discordant pairs. 
  The {\em{sample version of Kendall's tau}} is defined as:
  \begin{equation*}
  \hat \tau_K = \frac{c-d}{c+d} = \frac{c-d}{\binom{n}{2}}. 
  \end{equation*}
  \end{defi} % put all endall's tau together

  The second equality holds because there are $\binom{n}{2}$ distinct pairs for $n$ observations of a bivariate random variable. 

\begin{defi}[Population Kendall's tau]
  Let $(X_1, Y_1)$ and $(X_2, Y_2)$ be independent and identically distributed random vectors, each with joint distribution function $H$.
  The population Kandall's tau is defined as the difference between probability of concordance and the probability of discordance. 
  That is  
  \begin{equation*}
  \tau_K = P\left\{(X_1- X_2)(Y_1-Y_2) >0\right\} - P\left\{(X_1- X_2)(Y_1-Y_2) < 0\right\}. 
  \end{equation*}
  \end{defi}

\begin{prop}
  Let $X$ and $Y$ be continuous random variables whose copula is $C$. 
  Then the population Kendall's tau for $X$ and $Y$ is 
  \[\tau_K = 1-4\int_{[0,1]^2}
  \frac{\partial C(u,v)}{\partial u}
  \frac{\partial C(u,v)}{\partial v}
  dudv. \]
\end{prop}

We refer readers to \cite{Nelsen1999}[section 5.1.1] for the proof. 

\begin{defi}[Rank]
  Let $x_1,...,x_n$ be realisations of a one dimensional random variable $X$. 
  The rank of $x_i$ is $r_i = k$ if $x_i$ is the $k$-th smallest among $x_1,...,x_n$.
  \end{defi}

\begin{defi}[Sample Spearman's rho]
  Let $\{(x_1, y_1),..., (x_n, y_n)\}$ be realisations of a vector $(X, Y)$ of random variables,
  $r_{ix}$ and $r_{iy}$ be the rank of $x_i$ and $y_i$ respectively, 
  $r_x = (r_{1x},..., r_{nx})$, and $r_y = (r_{1y},..., r_{ny})$. 

  The sample Spreaman's rho is defined as
  \begin{equation*}
  \hat \rho_S = \hat \rho(r_x, r_y), 
  \end{equation*}
  where $\hat \rho$ is the sample Pearson correlation.
  \end{defi}

\begin{defi}[Population Spearman's rho]
  Let $F_X$ and $F_Y$ be the cdfs of random variable $X$ and $Y$ respectively, 
  The population Spreaman's rho is defined as follows:
  \begin{equation*}
  \rho_S = \rho(F_X(X), F_Y(Y)),  
  \end{equation*}
  where $\rho$ is the population Pearson correlation. 
  \end{defi}

  \begin{theo}
    Let $X$ and $Y$ be continuous random variables whose copula is $C$. 
    Then the population Spearman's rho for $X$ and $Y$ is 
    \[\rho_S = 12\int_{[0,1]^2}C(u,v)dudv-3. \]
  \end{theo}

We refer readers to \cite{joe1997multivariate}[section 2.12.2] for the proof. 

Quantile dependence measures the probability of two variables that is higher or below a given quantile of their univariate distributions.

(simulated base paper from Oh and Patton)
\begin{defi}[Sample quantile dependence] 
  Let $\hat F_X$ and $\hat F_Y$ be the empirical cdfs of random variable $X$ and $Y$ respectively.
  Let $(x_1, y_1),...,(x_n, y_n)$ be $n$ realisations of $X$ and $Y$. 
  The sample quantile dependence of $X$ and $Y$ at the $q$-th quantile is 

  \begin{equation*}
    \lambda_q = \begin{cases}
      (nq)^{-1}\sum_{i=1}^n 1\left( 
        \hat F_X(x_i) \leq q, \hat F_Y(y_i) \leq q 
      \right) & q \leq 0.5 \\
      (n(1-q))^{-1}\sum_{i=1}^n 1\left( 
        \hat F_X(x_i) > q, \hat  F_Y(y_i) > q 
      \right) & q > 0.5 
    \end{cases},
  \end{equation*}
  where $1(\cdot)$ is the indicator function. 

\end{defi}

\subsection{Copulae}\label{sec:ellpitical-copulae}

To capture different aspects of the dependence structure, we consider
a set of different copulas, which are layed
  out in detail below. These are the Gaussian-, $t$-, Frank-,
Gumbel-, Clayton-, mixture, NIG factor, and Plackett-copula. 

Figure~\ref{fig:copulaeScatterPlot} shows scatter plots of random
samples of each of the copulae treated. 
\begin{figure}[t]
    \centering
  \includegraphics[width=\textwidth]{_pics/copulas_scatterplots.pdf}
  \caption{Scatterplots of samples drawn from various copulae. All
    copulae are calibrated to Spearman's $\rho$ of 0.75 before
    sampling.}\label{fig:copulaeScatterPlot} 
\end{figure}

As this hedging backtest concerns only portfolios with two assets, we
focus on the bivariate version of each copula. 

\subsubsection{Gaussian and $t$ Copulae}\label{sec:ellpitical-copulae}

The Gaussian and $t$ copulae are dervived from Gaussian and $t$
distributions. 

The bivariate Gaussian copula is defined as
\begin{align*}
  \bm{C}(u,v) &= \Phi_{2, \rho}\{\Phi^{(-1)}(u), \Phi^{(-1)}(v)\} \nonumber \\
              &= \int_{-\infty}^{\Phi^{(-1)}(u)}
                \int_{-\infty}^{\Phi^{(-1)}(v)}
                \frac{1}{2\pi\sqrt{1-\rho^2}}
                \exp{\left\{
                \frac{s^2-2\rho st+t^2}{2(1-\rho^2)}
                \right\}} \dd s\, \dd t,\quad, u,v\in [0,1],
\end{align*}
where $\Phi_{2, \rho}$ is the bivariate Normal cdf
with zero mean, unit variance, and correlation coefficient $\rho$, and
$\Phi^{(-1)}$ is the quantile function of the univariate standard normal
distribution.
The Gaussian copula is fully specified by the correlation parameter $\rho$. \footnote{
The symbol $\rho$ is used to denote both the correlation parameter as
well as a general risk measure. However, it will be clear from the
context, what $\rho$ refers to.}
It has no tail dependence, which, in a finance context, implies that
it often underestimates tail risk.  

% The Gaussian copula density is
% \begin{equation*}
%   \bm{c}_\rho(u,v) = \frac{\bm{\varphi}_{2,\rho}\{\Phi^{(-1)}(u), \Phi^{(-1)}(v)\}}
%                      {\varphi\{\Phi^{(-1)}(u)\} \cdot \varphi\{\Phi^{(-1)}(v)\}} 
%                   = \frac{1}{2\pi\sqrt{1-\rho^2}}\exp\left\{
%                      -\frac{u^2 - 2\rho uv + v^2}{2(1-\rho^2)}
%                      \right\},
% \end{equation*}
% where $\bm{\varphi}_{2,\rho}(\cdot)$ is the pdf corresponding to
% $\Phi_{2, \rho}$, and $\varphi(\cdot)$ the standard normal
% pdf. \natp{\em [I think the abbreviations cdf and pdf where not
%   introduced. Please double-check.]}

Kendall's $\tau_K$ and Spearman's $\rho_S$ of the bivariate Gaussian copula are
    \begin{align*}
        \tau_K(\rho) = \frac{2}{\pi}\arcsin\rho
        \end{align*}
    \begin{align*}
        \rho_S(\rho) = \frac{6}{\pi}\arcsin\frac{\rho}{2}.
        \end{align*}

The $t$-copula has the form
\begin{multline*}
        \bm{C}(u,v) = \bm{T}_{2, \rho, \nu}\{T^{(-1)}_\nu(u), T^{(-1)}_\nu(v)\}\\
        = \int_{-\infty}^{T^{(-1)}_\nu(u)}
               \int_{-\infty}^{T^{(-1)}_\nu(v)}
            \frac{\Gamma\left(\frac{\nu+2}{2}\right)}
            {\Gamma\left(\frac{\nu}{2}\right)\pi\nu\sqrt{1-\rho^2}}
             \left(
        1+\frac{s^2-2st\rho+t^2}{\nu}
        \right)^{-\frac{\nu+2}{2}}\, \dd s\, \dd t,
    \end{multline*}
where $\bm{T}_{2, \rho, \nu}$ denotes the 
bivariate $t$ cdf with dependence parameter $\rho$ and degrees of
freedom parameter $\nu$, $\nu>2$,
and where $T^{(-1)}_\nu(\cdot)$ is the quantile function of a standard
$t$ distribution with parameter $\nu$. 

The $t$-copula and Gaussian copula with parameter $\rho$ have equal Kendall's $\tau$, \citep[see][and references therein]{demarta2005t}.

On the other hand, the $t$-copula has a non-zero tail dependence coefficient,
 which makes it more appropriate for dependence modelling in finance. (ref)
% The copula density is
% \begin{align*}
%     \bm{c}(u,v) &= \frac{\bm{t}_{2, \rho, \nu}\{T^{(-1)}_\nu(u), T^{(-1)}_\nu(v)\}}
%     {t_\nu\{T^{(-1)}_\nu(u)\}\cdot t_\nu\{T^{(-1)}_\nu(v)\}},
%     \end{align*}
% where $\bm{t}_{2,\rho, \nu}$ is the pdf of $\bm{T}_{2, \rho, \nu}$
% and $t_\nu$ the density of standard $t$ distribution.

\subsubsection{Archimedean copulae}\label{sec:archimedean-copula}
The family of Archimedean copulae forms a large class of copulae with
many convenient features.
% Contrary to elliptical copulas, which are derived from
% elliptical distributions.
Archimedean copulas are determined via a simple parametric form of the
dependence structure. A prominent feature is the ability to model
asymmetric dependence structures.  

In general, an Archimedean copula takes the form
\begin{align*}
  \bm{C}_\theta(u,v) = \psi^{(-1)}\{\psi(u; \theta), \psi(v; \theta); \theta\},\quad u,v\in [0,1],
    \end{align*}
where $\psi:[0,1] \rightarrow [0,\infty)$ is a continuous, strictly
decreasing and convex function such that $\psi(1)=0$ for any
permissible dependence parameter $\theta$. The function $\psi$ is 
called the generator, with $\psi^{(-1)}$ its inverse.

The {\em Frank copula\/} (B3 in \citet{joe1997multivariate}) takes the form
\begin{align*}
    \bm{C}_{\theta}(u,v) &= \frac{1}{\theta}
    \log \left\{
    1 + \frac{(e^{-\theta u}-1)(e^{-\theta v}-1)}{e^{-\theta}-1}
    \right\}, \quad u,v\in [0,1],
    \end{align*}
    with $\theta \in [0, \infty]$ the dependence parameter. 
    It is a symmetric copula and cannot produce any tail
    dependence. The following parameters correspond perfect dependence
    and independence: $\bm{C}_{-\infty} = \bm{M}$, $\bm{C}_1 = \bm{\Pi}$,
    and $\bm{C}_\infty = \bm{W}$. 
    % The copula density is
    % \begin{align*}
    %   \bm{c}_{\theta}(u,v)
    %   &= \frac{\theta e^{\theta(u+v)(e^\theta-1)}}
    %     {\left\{e^\theta-e^{\theta u}-e^{\theta v}+e^{\theta (u+v)}\right\}^2}.
    % \end{align*}
    The Frank copula has Kendall's $\tau$ :
\begin{align*}
    \tau_K(\theta) = 1-4\frac{D_1\{-\log(\theta)\}}{\log(\theta)},
    \end{align*}
% and
% \begin{align*}
%     \rho_S(\theta) = 1-12\frac{D_2\{-\log(\theta)\} - D_1\{\log(\theta)\}}{\log(\theta)},
%     \end{align*}
where $D_1$ and $D_2$ are the Debye function of order 1 and 2, with
the Debye function defined as $D_n =
\frac{n}{x^n}\int_0^x\frac{t^n}{e^t-1}dt$.
We refer readers to \cite{abramowitz1972handbook}[p.998] for definition of the Debye function. 

The {\em Gumbel copula\/} (B6 in \citet{joe1997multivariate}) has
distribution function
\begin{equation*}
  \bm{C}_{\theta}(u,v) = \exp{-\{ (-\log(u))^\theta +(-\log(v))^\theta 
    \}^{\frac{1}{\theta}}},
\end{equation*}
where $\theta \in [1,\infty)$ is the dependence parameter.
Its  Kendall's tau takes the form \begin{equation*}
  \tau_K(\theta) =\frac{\theta-1}{\theta}. 
 \end{equation*}
It has upper tail dependence with dependence parameter $\lambda^U
= 2-2^{\frac{1}{\theta}}$ and displays no lower tail dependence. 
    
While the Gumbel copula cannot model perfect counter-dependence
\citep{Nelsen2002}, $\bm{C}_{1} = \bm{\Pi}$ models independence, 
and $\lim_{\theta\rightarrow\infty} \bm{C}_\theta = \bm{W}$ models
perfect dependence. 


The {\em Clayton copula\/} takes the form
\begin{equation*}
  \bm{C}_{\theta}(u,v) = \left\{
    \max(u^{-\theta}+v^{-\theta}-1,0)\right\}^{-\frac{1}{\theta}},
\end{equation*}
where $\theta \in (-\infty, \infty)$ is the dependence parameter.
The Clayton copula, by contrast to Gumbel copula,
generates lower tail dependence with $\lambda^L =
2^{-\frac{1}{\theta}}$, but cannot generate upper tail dependence.
Moreover, $\lim_{\theta\rightarrow -\infty} \bm{C}_\theta = \bm{M}$, $\bm{C}_0 =
\bm{\Pi}$, and $\lim_{\theta\rightarrow\infty} \bm{C}_\theta = \bm{W}$. 
Kendall's $\tau$ of the Clayton copula is given by 
\begin{align*}
    \tau_K(\theta) =\frac{\theta}{\theta+2}.
    \end{align*}

\subsubsection{Mixture Copula}\label{sec:mixture-copula}
The mixture copula is a linear combination of copulae. 
The distribution of a 2-dimensional random variable
$\bm{X}=(X_1,X_2)^\top$ is written as linear combination of $K$
copulae 
\begin{equation*} 
    \bm{C}(u,v)= \sum_{k=1}^K p^{(k)} \cdot \bm{C}^{(k)}\{F^{(-1)}_{X_1}(u),
    F^{(-1)}_{X_2}(v); \bm{\theta^{(k)}}\}, \quad u,v\in [0,1].
  \end{equation*}
  Here, $\bm{\theta^{(k)}}$ refers to the parameters of the
    $k$-th copula.
%     Likewise, the copula density is a linear
%     combination of copula densities 
% \begin{align*}
%     \bm{c}(u,v)= \sum_{k=1}^K p^{(k)} \cdot \bm{c}^{(k)}\{F^{(-1)}_{X_1}(u),
%     F^{(-1)}_{X_2}(v); \bm{\theta^{(k)}}\}.
%     \end{align*}
   
While Kendall's $\tau$ of the mixture copula is not known in closed form,
Spearman's $\rho$ is easily derived as 
\begin{equation*}
  \rho_S = \sum_{k=1}^K p^{(k)} \cdot \rho_S^{(k)}. 
\end{equation*}

% \natp{\em [Old text below.]}

% While Kendall's $\tau$ of the mixture copula is not known in closed form,
% Spearman's $\rho$ is specified by the following statement. 
% \begin{proposition}
%   Let $\rho_S^{(k)}$ be Spearman's $\rho$ of the $k$-th component
%   Spearman's $\rho$ of the mixture copula is given by 
%   \begin{align*}
%         \rho_S = \sum_{k=1}^K p^{(k)} \cdot \rho_S^{(k)}.
%         \end{align*}
%     \end{proposition}

% \begin{proof}
%   Since Spearman's $\rho$ is defined as \citep{Nelsen1999}
%   \begin{equation*}
%     \rho_S = 12 \int_{\mathbb{I}^2} \bm{C}(s,t) ds dt - 3,
%   \end{equation*}
%   Spearman's $\rho$ of the the mixture copula is given by summation
%   of the components 
%   \begin{align*}
%     \rho_S = 12 \int_{\mathbb{I}^2} \sum_{k=1}^K p^{(k)} \cdot
%     \bm{C}^{(k)}(s,t) ds dt - 3. 
%   \end{align*}
% \end{proof}
% \natp{\em [Continue here.]}

An example of a mixture copula is the Fr\'echet class of copulae, which
are given by convex combinations of $\bm{W}$, $\bm{\Pi}$, and $\bm{M}$
\citep{Nelsen1999}.  

We use a mixture of Gaussian and independence copulae in our analysis,
i.e., 
\begin{equation*}
  \bm{C}(u,v) = p\, \bm{C}^\text{Gaussian}(u,v) + (1-p)(uv),\quad p\in (0,1).
\end{equation*}
% with corresponding density 
% \begin{equation*}
%   \bm{c}(u,v) = p\, \bm{c}^\text{Gaussian}(u,v) + (1-p).
% \end{equation*}

This mixture models the amount of ``random noise'' that appears in the
off-diagonal region of the dependence structure where the Gaussian copula has no control.
In the hedging exercise, the structure of the off-diagonal ``random noise'' is not our main concern, 
but the amount of it might affect the hedging effectiveness. 

\subsubsection{NIG factor copula}

Normal Inverse Gaussian (NIG) distribution is a flexible and yet analytical tractable distribution introduced by
\citep{BarndorffNielsen1997}.
The {\em NIG factor copula} is constructed based on the characteristics of the NIG disribution. 
We present the reparameterised version of NIG factor copula in this section.

The NIG distribution has density function
\begin{equation*}
  g(x; \alpha,\beta, \mu, \delta) = \frac{\alpha}{\pi} \e^{\delta
    \sqrt{\alpha^2-\beta^2} -\beta\mu} \frac{1}{q((x-\mu)/\delta)}
  K_1\left[\delta \alpha q\left(\frac{x-\mu}{\delta}\right) \right]
  \e^{\beta x},\quad x>0,
\end{equation*}
where $q(x) = \sqrt{1+x^2}$ and where $K_1$ is the modified Bessel
function of third order and index $1$. The parameters satisfy $0\leq
|\beta|\leq \alpha$, $\mu\in \R$ and $\delta>0$. The parameters have
the following interpretation: $\mu$ and $\delta$ are location and
scale parameters, respectively, $\alpha$ determines the heaviness of
the tails and $\beta$ determines the degree of asymmetry. If
$\beta=0$, then the distribution is symmetric around $\mu$.

The cdf and quantile function of NIG distribution, denoted by $G(x; \alpha, \beta, \mu, \delta)$ and $G^{(-1)}(x; \alpha, \beta, \mu, \delta)$,
 have no known analytical form.
 In this work, they are computed via numerical integration of the density and by simulation.

The NIG distribution belongs to
the class of so-called {\em normal
variance-mean mixture distributions},  (see Section 3.2 of 
\citep{McNeil2005}): $X$ follows an
$\text{NIG}(\alpha,\beta,\mu,\delta)$ distribution if $X$ conditional
on $W$ follows a normal distribution with mean $\mu+\beta W$ and
variance $W$, i.e., 
\begin{equation*}
  X|W\stackrel{\mathcal L}\sim \Ncdf(\mu + \beta W, W),
\end{equation*}
where $W$ follows an {\em inverse Gaussian distribution}, denoted by
$\text{IG}(\delta, \sqrt{\alpha^2-\beta^2})$.

Simulation procedure of NIG$(\alpha, \beta, \mu, \delta)$ distribution is a natural result of the above decomposition. 
To simulate the NIG distribution, first simulate a random variable $w \sim IG(\delta, \sqrt{\alpha^2-\beta^2})$, 
then simluate $x \sim N(\mu+ \beta w, w)$ given $w$.

The moment-generating function of the NIG distribution is given by
\begin{equation*}
  M(u; \alpha, \beta, \mu, \delta) = \exp\left( \delta
    \left(\sqrt{\alpha^2-\beta^2} - \sqrt{\alpha^2 - (\beta +
        u)^2}\right) + \mu u\right). 
\end{equation*}
As a direct consequence, moments are easily calculated with the
expectation and variance of the NIG distribution being
\begin{align*}
  \mathbb E X &= \mu +
                \frac{\delta \beta}{\sqrt{\alpha^2-\beta^2}}
  \end{align*}
\begin{align} \label{eq:5}
  \text{Var}(X) &= \frac{\alpha^2\delta}{(\alpha^2-\beta^2)^{3/2}}.
\end{align}

It is easily seen from the moment-generating function that the NIG distribution is preserved under linear combinations, provided
the variables share the parameters $\alpha$ and $\beta$. 
\begin{proposition}
  \label{prop:NIG}
  Let $Z\sim \text{NIG}(\alpha, \beta, \mu, \delta)$ and
  $Z_i\sim \text{NIG}(\alpha, \beta, \mu_i, \delta_i)$,
  $i=1,\ldots, n$ be independent NIG-distributed random
  variables. Then:
  \begin{enumerate}
  \item  $X_i = Z + Z_i\sim \text{NIG}(\alpha,\beta,\mu+\mu_i,
  \delta+\delta_i)$,
\item and 
  \begin{align}
    \text{Cov}(X_i,X_j) &= \text{Var(Z)},\nonumber\\
    \text{Corr}(X_i,X_j) &= \frac{\delta}{\sqrt{(\delta+\delta_i)
                           (\delta+\delta_j)}}. \label{eq:6}
  \end{align}
\end{enumerate}
\end{proposition}
\begin{proof}
  \begin{enumerate}
  \item This follows directly from the moment-generating function. 
  \item For the covariance,
    \begin{equation*}
      \text{Cov}(X_i,X_j)
      = \E[(Z+Z_i) (Z+Z_j)] - \E[Z+Z_i] \E[Z+Z_j]
      = \E[Z^2] -(\E Z)^2.
    \end{equation*}
    The correlation is determined directly from \ref{eq:5}.
  \end{enumerate}
\end{proof}

The NIG distribution is popular in many areas of
financial modelling; for example, it gives rise 
to the normal inverse Gaussian L\'evy process, which may be represented
as a Brownian motion with a time change.
In the setting here, we consider the {\em NIG factor copula}, which is
not directly derived from the multivariate NIG distribution, but
determined through a factor structure instead. \footnote{The factor structure,
which was applied e.g.\ in \citep{Kalemanova2007} for calibrating CDO's,
gives additionaly flexibility as it does not force the components to
have a mixing variable $W$.}

Denote 
\begin{align*}
  X &= Z + Z_1 \\ 
  Y &= Z + Z_2,
  \end{align*}
where $Z \sim \text{NIG}(\alpha, \beta, \mu, \delta)$, $Z_1 \sim \text{NIG}(\alpha, \beta, \mu_1, \delta_1)$, 
$Z_2 \sim \text{NIG}(\alpha, \beta, \mu_2, \delta_2)$, and $Z, Z_1, Z_2$ are mutually independent. 

The following reparameterisation steps reduce the number of parameters to three:
\begin{enumerate}
  \item Set $\mu = \mu_1= \mu_2 = 0$ . Location parameter does not affect the correlation structure.
  \item Set $\delta = \frac{(\alpha^2-\beta^2)^{3/2}}{\alpha^2}$, $\delta_1 = \delta_2$, $\tilde \delta = \delta_1 = \delta_2$. 
  The correlation between X and Y is fully captured by $\alpha, \beta$, and $\tilde \delta$.  
\end{enumerate}
  % Denote the margins by $u,v \sim U(0,1)$. 
% The NIG factor model is obtained by transforming the unifrom margins to standardised NIG distriutions.

\begin{prop}
  Let $u,v \in [0,1]$, $f(\cdot) = g\left(\cdot; \alpha, \beta, 0, \frac{(\alpha^2-\beta^2)^{3/2}}{\alpha^2}
  \right)$ and $F (\cdot) = G(\cdot; \alpha, \beta, 0, \tilde \delta)$, 
  the {\em NIG factor copula} is 
  \begin{equation*}
    C(u,v) = \int_\mathbb{R} F(u-z)F(v-z)f(z)dz,
  \end{equation*}
  where $\alpha, \beta \in \mathbb{R}$ satisfying $0 \leq |\beta| \leq \alpha$, and $\tilde \delta >0 $.
\end{prop}

The parameters $\alpha, \beta, \tilde \delta$ fully control the dependence between $u$ and $v$, $u, v \sim U(0,1)$, captured by the NIG factor copula.
We refer readers to \cite{krupskii2013factor} for the methodology of constructing a factor copula. 

The quantile dependence and Spearman rho of NIG factor copula have no known analystical form.
In this work, the quantile dependence is computed numerically; 
the Spearman rho is approximated by the Spearman rho of the bivariate Gaussian copula. 
When $\beta \rightarrow 0$ and $\alpha \rightarrow \infty$, the NIG distribution behaves similarly to Gaussian distribution, 
making the NIG factor copula (bivariate) behaves similarly to the Gaussian copula (bivariate), and therefore, 
the NIG factor copula's Spreaman rho is well approximated by the Spearman rho of the bivariate Gaussian copula.




% \natp{\em [Please clarify that $\circ$ refers to composition. Clean
%   up notation, e.g.\ marginals can be denoted $F_F$ and $F_S$, Use
%   just $C$ for the copula. What are $Z_1$ and $Z_2$? I don't find the
%   formula in the paper mentioned. Also, where is the formula for
%   Kendall's tau taken from?]}
%  The NIG factor copula is obtained by transforming the margins to
% uniforms (see Sklar's Theorem), giving (e.g.\
% \citep{krupskii2013factor}):
% \begin{equation*}
%   C_{r^S, r^F}(F_{r^S}(r^S), F_{r^F}(r^F)) = \int_\mathbb{R}
%   F_{Z_1}(F_{X_1}^{(-1)} \circ F_{r^S}(r^S) -z) \cdot
%   F_{Z_2}(F_{X_2}^{(-1)} \circ F_{r^F}(r^F) -z) \cdot
%   f_Z(z) dz.
%   \end{equation*}
% If the margins are continuous, then Spearman's rho of NIG factor
% copula is 
% \begin{equation*}
%   \rho_S = 12 \int \int \int_{\mathbb{R}^3}
%   F_{X_1}(x_1) \cdot
%   F_{X_2}(x_2) \cdot
%   f_{Z_1}(x_1-z) \cdot
%   f_{Z_2}(x_2-z) \cdot
%   f_Z(z) dx_1 dx_2 dz - \frac{1}{48}.x
%   \end{equation*}

% \begin{proof}
%   \begin{align}
%   \rho_S(r^S, r^F) &= \rho\{F_{r^S}(r^S), F_{r^F}(r^F)\} \\
%     &= \rho\{F_{X_1}(X_1), F_{X_2}(X_2)\} \\
%     &= 12 \cdot \mathbb{E}\{F_{X_1}(X_1) \cdot F_{X_2}(X_2) \} - \frac{1}{48}\\
%     &= 12 \cdot \int \int_{\mathbb{R}^2} F_{X_1}(X_1) \cdot F_{X_2}(X_2) dF_{X_1,X_2}(x_1,x_2)\\
%     \end{align}
%   Because
%   \begin{align}
%     F_{X_1,X_2}(x_1,x_2) &= \mathbb{P}(X_1 \leq x_1, X_2 \leq x_2)\\
%     &= \mathbb{P}(Z_1 \leq x_1 - Z, Z_2 \leq x_2 - Z) \\
%     &= \int_\mathbb{R}\mathbb{P}(Z_1 \leq x_1 - z) \cdot \mathbb{P}(Z_2 \leq x_2 - z) \cdot f_Z(z) dz,
%     \end{align}
%   so,
%   \begin{align}
%     \rho_S(r^S, r^F) = 12 \cdot \int \int \int_{\mathbb{R}^3} F_{X_1}(x_1) \cdot F_{X_2}(x_2) \cdot f_{Z_1}(x_1 -z) \cdot f_{Z_2}(x_2 -z) \cdot f_{Z}(z) dx_1 dx_2 dz -\frac{1}{48}
%     \end{align}
%   \end{proof}


\subsubsection{Plackett copula}\label{subsec:other-copula}
The Plackett copula has distribution function
\begin{align*}
    \bm{C}_{\theta}(u,v) &= \frac{1+(\theta-1)(u+v)}{2(\theta-1)}
                         - \frac{\sqrt{\{
    1+(\theta-1)(u+v)\}^2 - 4uv\theta(\theta-1)}}{2(\theta-1)},
\end{align*} where $0 \leq \theta < \infty$.
Spearman's Rho is given by 
\begin{align*}
    \rho_S(\theta) = \frac{\theta+1}{\theta-1} - \frac{2\theta \log
  \theta}{(\theta-1)^2}. 
    \end{align*}

The Placket copula possesses a special property:
the cross-product ratio is equal to the dependence parameter
\begin{equation} \label{eq:PlackettCrossProduct}
    % &\phantom{=}
    \frac{\p(U \leq u, V \leq v) \cdot \p(U > u, V > v)}
    {\p(U \leq u, V > v) \cdot \p(U > u, V \leq v)}\nonumber
    =
      \frac{\bm{C}_\theta(u,v)\{1-u-v+\bm{C}_\theta(u,v)\}}{\{u-\bm{C}_\theta(u,v)\}\{v-\bm{C}_\theta(u,v)}\nonumber 
    = \theta.
\end{equation}
In words, the dependence parameter is equal to the ratio of the 
number of concordance pairs and the number of discordance pairs of a 
bivariate random variable. 

%! Author = francis
%! Date = 30.10.20


\subsection{Simulated Method of Moments}\label{subsec:simulated-method-of-moments}
This method is suggested by Oh and Patton (2013).
In this setting, rank correlation e.g. Spearman's $\rho$ or Kendall's $\tau$,
and quantile dependence measures at different levels $\lambda_q$
are calibrated against their empirical counterparts.\medskip

Spearman's rho, Kendall's tau, and quantile dependence of a pair $(X,Y)$
with copula $C$ are defined as
\begin{align}
  \rho_S &= 12 \int\int_{I^2} C_{\bm{\theta}}(u,v)\, \dd u\, \dd v-3\label{eq:rho_S}\\
  \tau_K &= 4\mathbb{E}[C_{\bm{\theta}}\{F_X(x), F_Y(y)\}]-1,\\
  \lambda_q &=
  \begin{cases}
    \p(F_X(X)\leq q| F_Y(Y)\leq q) = \displaystyle \frac{C_{\bm{\theta}}(q,q)}{q},
    &\text{ if } q\in (0,0.5],\\
    \p(F_X(X)>q|F_Y(Y)>q) =\displaystyle \frac{1-2q+C_{\bm{\theta}}(q,q)} {1-q},
    &\text{ if } q\in (0.5,1).
  \end{cases}
\end{align}\medskip
The empirical counterparts are
\begin{align*}
  \hat\rho_S &= \frac{12}{n} \sum_{k=1}^n \hat F_X(x_k) \hat F_Y(y_k)
               -3,\\
  \hat\tau_K &= \frac{4}{n}\sum_{k=1}^n \hat{C}\{\hat{F}_X(x_i),\hat{F}_X(y_i)\} -1 ,\\
  \hat\lambda_q &=
                  \begin{cases}
                    \displaystyle\frac{1}{n} \sum_{k=1}^n \frac{\1_{\{\hat
                        F_X(x_k)\leq q, \hat F_Y(y_k)\leq q\}}} {q},
                    &\text { if } q\in (0, 0.5],\\
                    \displaystyle \frac{1}{n} \sum_{k=1}^n
                    \frac{\1_{\{\hat F_X(x_k)>q, \hat F_Y(y_k)>q\}}}
                    {1-q}, &\text { if } q\in (0.5,1).
                  \end{cases},
\end{align*}
where $\hat{F}(x) := \frac{1}{n}\sum_{k=1}^n 1_{\{x_i\leq x\}}$ and
$\hat{C}(u,v) := \frac{1}{n}\sum_{k=1}^n 1_{\{u_i\leq u, v_i\leq v\}}$.\medskip

We denote $\tilde{\bm{m}}(\bm{\theta})$ be a $m$-dimensional vector of dependence measures according the the
dependence parameters $\bm{\theta}$,and  $\hat{\bm{m}}$ be the corresponding empirical counterpart.
The difference between dependence measures and their counterpart is denoted by
\begin{align*}
    \bm{g}(\bm{\theta}) = \hat{\bm{m}} - \tilde{\bm{m}}(\bm{\theta}).
\end{align*}\medskip

The SMM estimator is
\begin{align*}
    \hat{\bm{\theta}} = \argmin_{\bm{\theta}\in \bm{\Theta}} \bm{g}(\bm{\theta})^\intercal
    \hat{\bm{W}}
     \bm{g}(\bm{\theta}),
\end{align*}
where $\hat{W}$ is some positive definite weigh matrix.\medskip

In this work, we use $\tilde{\bm{m}}(\bm{\theta}) = (\rho_S, \lambda_{0.05}, \lambda_{0.1},
\lambda_{0.9}, \lambda_{0.95})^\intercal$
for calibration of Bitcoin price and CME Bitcoin future.

\subsection{Maximum Likelihood Estimation}\label{subsec:maximum-likelihood-estimation}
By Sklar's theorem, the joint density of a $d$-dimensional random variable $\bm{X}$ with sample size $n$ can be written as
\begin{align}
    \bm{f}_{\bm{X}}(x_1, ..., x_d) = \bm{c}\{F_{X_1}(x_1), ..., F_{X_d}(x_d)\} \prod_{j=1}^d f_{X_i}(x_i).
    \end{align}
We follow the treatment of MLE documented in section 10.1 of \citet{joe1997multivariate}, namely the inference functions for margins or IFM method.
The log-likelihood $\sum^n_{i=1}f_{\bm{X}}(X_{i,1}, ..., X_{i,d})$ can be decomposed into dependence part and marginal part,
\begin{align}
    L(\bm{\theta}) &= \sum_{i=1}^n \bm{c}\{F_{X_1}(x_{i,1};\bm{\delta}_1), ..., F_{X_d}(x_{i,d}; \bm{\delta}_d);\bm{\gamma}\}
    + \sum_{i=1}^n \sum_{j=1}^d f_{X_j}(x_{i,j};\bm{\delta}_j)
    &= L_C(\bm{\delta}_1, ..., \bm{\delta}_d, \bm{\gamma}) + \sum_{j=1}^d L_j(\bm{\delta}_j)
    \end{align}
where $\bm{\delta}_j$ is the parameter of the $j$-th margin, $\bm{\gamma}$ is the parameter of the parametric copula, and
$\bm{\theta} = (\bm{\delta}_1,..., \bm{\delta}_d, \bm{\gamma})$.

Instead of searching the $\bm{\theta}$ is a high dimensional space, \citet{joe1997multivariate} suggests to
search for $\hat{\bm{\delta}_1},..., \hat{\bm{\delta}_d}$ that maximize $L_1(\bm{\delta}_1), ..., L_d(\bm{\delta}_d)$,
then search for $\hat{\bm{\gamma}}$ that maximize $L_C(\hat{\bm{\delta}_1},..., \hat{\bm{\delta}_d}, \bm{\gamma})$.

That is, under regularity conditions, $(\hat{\bm{\delta}_1},..., \hat{\bm{\delta}_d}, \hat{\bm{\gamma}})$ is the solution of
\begin{align}
    \left( \frac{\partial L_1}{\partial \bm{\delta}_1}, ..., \frac{\partial L_d}{\partial \bm{\delta}_d},
    \frac{\partial L_C}{\partial \bm{\gamma}}\right) = \bm{0}.
    \end{align}

However, the IFM requires making assumption to the distribution of of the margins.
\citet{genest1995semiparametric} suggests to replace the estimation of marginals parameters estimation by non-parameteric estimation.
Given non-parametric estimator $\hat{F}_i$ of the margins $F_i$, the estimator of the dependence parameters $\bm{\gamma}$ is
\begin{align}
    \hat{\bm{\gamma}} = \argmax_{\bm{\gamma}} \sum_{i=1}^n \bm{c}\{ \hat{F}_{X_1}(x_{i,1}), ..., \hat{F}_{X_d}(x_{i,d});\bm{\gamma}\}.
    \end{align}



%With the decomposition, the MLE estimator for a bivariate parametric copula is
%\begin{align}
%    \hat{\bm{\theta}} = \argmax_{\bm{\theta} \in \bm{\Theta}} l(X_1,X_2; \bm{\theta}), \label{eq:EMLE}
%    \end{align}
%where
%\begin{align}
%    l(X_1,X_2; \bm{\theta}) = \sum_{i=1}^n \log c(x_{i,1}, x_{i,2};\bm{\theta}). \label{eq:Likelihood}
%    \end{align}\medskip

%Procedure of maximising equation~\ref{eq:EMLE} as a whole is called exact maximum likelihood method.
%Leveraging the attractive feature of copula that one can model the dependence structure and marginals separately,
%we rewrite~\ref{eq:Likelihood} into canonical expression
%\begin{align}
%    l(X,Y; \bm{\theta}) = \sum_{k=1}^n \log c\{F_X(x_i; \delta_X), F_Y(y_i; \delta_Y); \bm{\gamma}\}
%    + \sum_{k=1}^n \log f_X(x_i; \bm{\delta}_X) + \sum_{k=1}^n \log f_X(y_i; \bm{\delta}_Y),
%    \end{align}
%where the $\bm{\gamma}$ is the dependence parameter in the copula and $\bm{\delta}$ is the parameters in the margins.\medskip
%
%The inference-functions for margins (IFM) approach by Joe is a two step procedure of maximising~\ref{eq:EMLE}.
%The approach calibrate first the $\bm{\delta}$s and then the  $\bm{\gamma}$.\medskip
%
%Similar to the IFM approach, pseudo-maximum likelihood approach by Genest and Rivest (1993) replace the parametric margins by
%empirical estimates, we rewrite \ref{eq:Likelihood} again with
%\begin{align}
%    l(X,Y; \bm{\theta}) = \sum_{k=1}^n \log c(u_i, v_i;\bm{\gamma}),
%    \end{align}
%where $u_i = \hat{F}_X(x_i)$ and $v_i = \hat{F}_Y(y_i)$.

\subsection{Comparison}
Both the simulated method of moments and the maximum likelihood estimation are unbiased and
proven to give good fits.
The problem remain is which procedure is more suitable for hedging.
%Cryptocurrencies are known to be very volatile.
Sample and fitted quantile dependence for Bitcoin and CME future.

%\begin{figure}[th]
%\includegraphics[width=\textwidth]{_pics/t Copula quantile dependence.png}
%\includegraphics[width=\textwidth]{_pics/Gumbel Copula quantile dependence.png}
%\includegraphics[width=\textwidth]{_pics/Clayton Copula quantile dependence.png}
%  \caption{}
%\label{fig:quantile dependence1}
%\end{figure}


The MM estimation perform just as we decided: match the upper and lower quantile dependence.




%
%
%\subsection{Two-Stage Estimation}\label{subsec:two-stage-estimation}
%~\cite{joe2005asymptotic} study the efficiency of a two-stage estimation procedure of copula estimation.
%The authors also call this method inference function for margins IFM.
%
%\textbf{Pros}
%\begin{enumerate}
%    \item Almost as efficient as MLE methods but easier to be implemented
%    \item Yields an asymptotically Gaussian, unbiased estimate
%\end{enumerate}
%
%\textbf{Cons}
%\begin{enumerate}
%    \item Subject to specification of marginals \cite{kim2007comparison}
%\end{enumerate}
%
%Our data
%\begin{align}
%    \pmb{y} = \begin{bmatrix}
%                  y_{11} & \cdots & y_{1i}\\
%                  \vdots & \ddots & \vdots \\
%                  y_{n1} & \cdots & y_{ni}
%                  \end{bmatrix}
%    \end{align}
%Let $F$ and $f$ be the joint cdf and joint density of $\pmb{y}$ with parameters $\pmb{\delta}$,
%and let $F_i$ and $f_i$ be the marginal cdf and marginal density for the $i^\text{th}$ random variable with parameters $\pmb{\theta}_i$, we have
%\begin{align}
%    f(\pmb{y}; \pmb{\theta}_1, \pmb{\theta}_2,\dots \pmb{\theta}_i, \pmb{\delta}) =
%    c\{F_1(\pmb{y}_1;\pmb{\theta}_1), F_2(\pmb{y}_2; \pmb{\theta}_2), \dots, F_i(\pmb{y}_1;\pmb{\theta}_i); \pmb{\delta}\}
%    \prod^i_{j=1}f_i(\pmb{y}_j;\pmb{\theta}_j)
%    \end{align}
%
%For a sample of size $n$, the log-likelihood of functions of the $i^\text{th}$ univariate margin is
%\begin{align}
%    L_i(\theta_i) = \sum^n_{m=1} \log f_i(y_{mi}; \theta_i),
%    \end{align}
%
%and the log-likelihood function for the joint distribution is
%\begin{align}
%    L(\delta, \theta_1, \theta_2, \dots, \theta_i) = \sum^n_{m=1}\sum^i_{j=1} \log f(y_{mj}; \delta, \theta_1, \theta_2, ..., \theta_i)
%    \end{align}
%
%In most cases, one does not have closed form estimators and numerical techniques are needed.
%Numerical ML estimation difficulty increase when the total number of parameters increases.
%The two-stage estimation is designed to overcome this problem.
%
%The two-stage procedure is
%\begin{enumerate}
%    \item estimate the univariate parameters from separate univariate likelihoods to get $\tilde{\pmb{\theta}_1}, ..., \tilde{\pmb{\theta}_i}$
%    \item maximize $L(\pmb{\delta}, \tilde{\pmb{\theta}_1}, \dots, \tilde{\pmb{\theta}_i})$ over $\pmb{\delta}$ to get $\tilde{\pmb{\delta}}$
%    \end{enumerate}
%
%Under regularity conditions
%\footnote{Regularity conditions include
%1. $\exists \frac{\partial \log f(x;\theta)}{\partial \theta}, \frac{\partial^2 \log f(x;\theta)}{\partial \theta^2}, \frac{\partial^3 \log f(x;\theta)}{\partial \theta^3}$ for all $x$;
%2. $\exists g(x), h(x) and H(x)$ such that for $\theta$ in a neighborhood $N(\theta_0)$ the relations
%$\left|\frac{\partial f(x;\theta)}{\partial theta}\right| \leq g(x)$,
%$\left|\frac{\partial^2 f(x;\theta)}{\partial \theta^2}\right| \leq h(x)$,
%$\left|\frac{\partial^3 f(x;\theta)}{\partial \theta^3}\right| \leq H(x)$ hold for all $x$, and
%$\int g(x) dx < \infty$, $\int h(x) dx < \infty$, $\mathbb{E}_\theta \{H(X)\} < \infty$ for $\theta \in N(\theta_0)$;
%3. For each $\theta \in \Theta$, $0< \mathbb{E}_\theta \left\{
%\left(
%\frac{\partial \log f(X;\theta)}{\partial \theta}
%\right)^2
%\right\}$. For detail see section 4.2.2 of~\cite{serfling2009approximation}}
%, $(\pmb{\tilde{\theta}}_1,\dots \pmb{\tilde{\theta}}_i, \pmb{\tilde{\delta}})$ is the solution of
%\begin{align}
%    (\partial L_1 / \partial \pmb{\theta}^\intercal_1,
%    \dots, \partial L_i / \partial \pmb{\theta}^\intercal_i, \partial L / \partial \pmb{\pmb{\delta}}^\intercal_1) = \pmb{0}
%    \end{align}
%
%For comparison, if we optimize $L$ directly without the two-stage procedure (i.e.~MLE), we solve for
%\begin{align}
%    (\partial L / \partial \pmb{\theta}^\intercal_1,
%    \dots, \partial L / \partial \pmb{\theta}^\intercal_i, \partial L / \partial \pmb{\pmb{\delta}}^\intercal_1) = \pmb{0}
%    \end{align}
%
%We denote the two solutions as
%$\tilde{\pmb{\eta}} = (\pmb{\tilde{\theta}}_1,\dots \pmb{\tilde{\theta}}_i, \pmb{\tilde{\delta}})$ for two-stage procedure;
%$\hat{\pmb{\eta}} =(\pmb{\hat{\theta}}_1,\dots \pmb{\hat{\theta}}_i, \pmb{\hat{\delta}})$ for MLE procedure.
%and compare the asymptotic relative efficiency of $\tilde{\pmb{\eta}}$ and $\hat{\pmb{\eta}}$.
%
%Asymptotics: yet to be done.\\
%~\cite{kim2007comparison} show the estimation of $\pmb{\theta}$ may be seriously affected.
%They compare the two-stage approach and Canonical Maximum Likelihood Method by simulation and
%conclude that Canonical Maximum Likelihood is prefered from a computational statistics and data analysis point of view.
%
%\subsection{Canonical Maximum Likelihood Method}\label{subsec:canonical-maximum-likelihood-method}
%This approach was studied by~\cite{genest1995semiparametric} and~\cite{shih1995inferences}.
%One estimates the margins using empirical CDF
%\begin{align}F_X(x)=\frac{1}{n+1}\sum_{i=1}^n 1(X_i \leq x)\end{align},
%
%we maximize the log-likelihood
%\begin{align}
%    L(\delta) = \sum_{i=1}^n \log [c_\delta \{F_X(X_i), F_Y(Y_i)\}]
%    \end{align}
%
%This procedure does not require specification of marginals.
%
%
%
%
%
%%also by Wang and Ding, 2000; Tsukahara, 2005
%%This is also known as pseudo maximum likelihood (PML) and as canonical maximum likelihood (see Cherubini et al., 2004)
%%
%%Genest and Werker (2002) obtained conditions under which the PMLE is asymptotically efficient.
%%
%%

% ----------------
% --- Estimation of Copula ---
% ----------------

\subsubsection{Copula selection}\label{subsec:copula-selection}
As the dependence structure of price data changes
across time, we allow for a flexible choice of the best-fitting
copula, by re-calibrating periodically and re-evaluating performance
of the various copulas. 
In each re-calibration, we select the best-fitting
copula, characterised by the lowest {\em Akaike Information Criterion
  (AIC)},
\begin{equation*}
 \text{AIC} = 2k- 2 \log(L),
\end{equation*}
where $k$ is the number of estimated
parameteres and $L$ is the likelihood \citep{Akaike1973}. 

% they tend to suggest the same copula as the best fitting one.
%Simulation studies has also been carried out to compare different copula selection methods, see \cite{}.
Other model selection criteria, such as the TIC~\citep{takeuchi1976distribution} or likelihood ratio test could be used instead.
For a survey of model selection and inference, see \cite{anderson1998comparison}.
Among various copula selection procedures, AIC is a popular choice for
its applicability, see e.g. \cite{breymann2003dependence}.
In our case, the AICs are calculated only with dependence likelihood
since the marginals are modelled via kernel density estimators.
The selected copula will then enter the calculation of the optimal
hedge ratio.
% We consider the copula with the lowest AIC for a particular set of data the best fitting one and use it to generate OHR.

\subsection{Risk measures}\label{subsec:spectral-risk-measures}
The optimal hedge ratio is determined for the following variety of risk measures: variance, Value-at-Risk (VaR), Expected Shortfall (ES), and Exponential Risk Measure (ERM).
A summary of risk measures being used in portfolio selection problem
can be found in \citet{hardle2008applied}. 
The risk measures here serve as risk minimization objectives, i.e. loss functions for searching the optimal hedge ratio. 
%They are used in many literature about hedging, e.g. ;
%The risk measures are also used by regulatory bodies,
%for example Basel III ....

The risk measures are defined as follows.
Let $Z$ be a random
variable with distribution function $F_Z$.
\begin{enumerate}
\item Variance: $\text{Var}(Z) = \E[(Z-\E Z)^2]$. 
\item VaR at confidence level $\alpha$: $\text{VaR}_\alpha(Z) = -F_{Z}^{(-1)}(1-\alpha)$
\item ES at confidence level $\alpha$: $\text{ES}(F_Z) = -\frac{1}{1-\alpha}\int_0^{1-\alpha}F_Z^{(-1)}(p)dp$
\item ERM with Arrow-Pratt coefficient of absolute risk
  aversion $k$:
  \begin{equation*}
    \text{ERM}_k(F_Z) = \int_0^{1-\alpha}\phi(p) F_Z^{(-1)}(p)dp,
  \end{equation*}
  where $\phi$ is a weight function described in (\ref{eq:phi}) below.
\end{enumerate}

VaR, ES, and ERM fall into the class of spectral risk measures (SRM),
which have the form \citep{Acerbi2002}%, adam2008spectral,dowd2008spectral}
\begin{equation*}
  \rho_\phi(r^h) = - \int_0^1 \phi(p) F_{Z}^{(-1)}(p)d p,
\end{equation*}
where $p$ is the loss quantile and $\phi(p)$ is a user-defined
weighting function defined on $[0,1]$.
We consider only so-called admissible risk spectra $\phi(p)$, i.e.,
fulfilling %(named by \citet{Acerbi2002})
\begin{enumerate}[label=(\roman*)]
\item $\phi$ is positive,
\item $\phi$ is decreasing,
\item and $\int\phi=1$. 
\end{enumerate}

The VaR's $\phi(p)$ gives all its weight on the $1-\alpha$ quantile of
$Z$ and zero elsewhere, i.e., the weighting function is a Dirac delta
function, and hence it violates the (ii) property of admissible risk
spectra.  
The ES' $\phi(p)$ gives all tail quantiles the same weight of
$\displaystyle\frac{1}{1-\alpha}$ and non-tail quantiles zero weight. 
The ERM assumes investors' risk preference are in the form of an
exponential utility function $U(x)=1-e^{kx}$, so its corresponding
risk spectrum is defined as
\begin{equation*}
  \phi(p) =\frac{k e^{-k(1-p)}}{1-e^{-k}} , \label{eq:phi}
\end{equation*}
where $k$ is the Arrow-Pratt coefficient of absolute risk aversion. 
The parameter $k$ has an economic interpretation as being the ratio
between the second derivative and first derivative 
of investor's utility function on an risky asset,
\begin{equation*}
  k = -\frac{U''(x)}{U'(x)},
\end{equation*}
for $x$ in all possible outcomes.
In case of the exponential utility, $k$ is the the constant absolute risk aversion (CARA).


% ----------------
% --- Describe the methodology of finding the optimal h ---
% ----------------

%\section{Results}\label{sec:results}

% \ra{1.1}
    {\begin{tabularx}{\textwidth}{lYYYYY} \toprule
         Spot/ Copula & $t$ & Plackett & GMI & rotGumbel & NIG \\ \midrule
     \multicolumn{6}{l}{Individual Cryptos}                                                                                 \\
        \ \ \ BTC          & 65.18      & \phantom{0}3.57              & \phantom{0}0.89                     & \phantom{0}0.89               & 27.68                  \\
        \ \ \ ETH          & \phantom{0}2.68       & \phantom{0}5.36              & \phantom{0}7.14                     & 83.93              & \phantom{0}0.89                   \\
        \ \ \ ADA          & \phantom{0}0.00       & \phantom{0}0.00              & \phantom{0}0.00                     & \phantom{0}0.00               & 100.00\phantom{0}                 \\
        \ \ \ LTC          & 11.61      & \phantom{0}0.00              & \phantom{0}2.68                     & 28.57              & 57.14                  \\
        \ \ \ XRP          & \phantom{0}0.00       & 27.68             & \phantom{0}2.68                     & 69.64              & \phantom{0}0.00                   \\
   \multicolumn{6}{l}{Crypto Indices with BTC Constituent}                                                                  \\
        \ \ \ BITX         & 48.15      & \phantom{0}0.00               & 17.28                    & 19.75              & 14.81                  \\
        \ \ \ CRIX         & 53.41      & \phantom{0}0.00               & 12.50                    & \phantom{0}3.41               & 30.68                  \\
        \ \ \ BITW100      & 51.85      & \phantom{0}0.00               & 9.88                     & 35.80              & \phantom{0}2.47                   \\
    \multicolumn{6}{l}{Crypto Indices without BTC Constituent}                                                              \\
        \ \ \ BITW20       & \phantom{0}0.00        & \phantom{0}0.00               & \phantom{0}0.00                      & 96.30              & \phantom{0}3.70                   \\
        \ \ \ BITW70       & \phantom{0}0.00        & \phantom{0}0.00               & \phantom{0}0.00                        & 98.77              & \phantom{0}1.23                  \\
    \bottomrule
    \end{tabularx}
       }










\subsection{Copula Selection Results}\label{sec: copula results}
The copula selection result is provides insight that help us understand the crypto market better, so
we illustrate the copula selection result in this section.
Decisions of the AIC procedure are summarised in table \ref{tab:copulasection}. \medskip

Overall, $t$-copula, rotated Gumbel (rotGumbel), and the NIG factor copula are the most frequently chosen copulae by the AIC procedure. \medskip

The $t$-copula is frequently chosen by AIC to model the dependency between the BTC and BTC involved indices, CRIX, BITX, BITW100, and the BTC future.
BTC and BTC involved indices exhibit strong tail dependence (both upper and lower tail) with BTCF.
We could interpret tail dependence much more of a tendency for one asset to be extreme when another is extreme and vice versa \citep{McNeil2015}.
In fact, the $t$ copula has been suggested in various empirical studies to model financial data, such as \cite{zeevi2002beyond} and \cite{breymann2003dependence}.
Those studies suggest $t$-copula is a better model over the Gaussian copula which financial data often seem to exhibit tail dependence. \medskip

On the other hand, the radially symmetric feature makes the $t$-copula not a good choice to model the other hedging pairs.
Demarta and McNeil describe the symmetry feature "strong", because if $(U_1, ..., U_d)$ is a vector distributed in $t$-copula,
then $(U_1, ..., U_d) \overset{\mathcal{L}}= (1-U_1, ..., 1-U_d)$.
This symmetry can be justified in the dependence structure between a future and its underlying by the theory of future pricing,
which suggests the price of a future is a function of the underlying price \citep{hull2003options}.
However, there is no such relationship between a future and an asset which is not the underlying, and so the radial symmetry becomes a drawback to model other hedging pairs e.g. ETH and BITX70.
Another drawback of the $t$-copula is the lack of flexibility to model off-diagonal region since Rho and nu jointly control the density of the off-diagonal region.
%The off-diagonal region (HF paper breymann2003dependence)
This is why sometimes the Gaussian Mix Independence (GMI) better model the dependence.  \medskip

Among the three popular copulae, rotGumbel copula shows its ability to model the dependency between ETH and BTCF,
94 out of 112 training sets are best fitted with the rotated Gumbel.
rotGumbel also performs well when modelling dependency between XRP, BITW20, BITW70, and the BTCF.
In particular, the whole time series of the two indices, BITW20 and BITW70, are best fitted solely with the rotated Gumbel copula.
The frequently chosen rotated Gumbel indicates the styled fact of financial data: prices tends to drop together.  \medskip
%  (ref) The reason for that

%The rotated Gumbel model the lower tail dependency very well, it has a lower quantile dependence controlled by its parameter

%BTC upper tail dependence BTC lower tail dependence (with fitted copula (t and rotGumbel) tail dependence)

%ETH upper tail dependence ETH lower tail dependence (with fitted copula (t and rotGumbel) tail dependence)

In fact, Clayton's AIC in many of the training sets is the second lowest, just higher than that of rotated Gumbel.
This is because the Clayton copula has the same ability to model the lower quantile dependence.
However, Clayton's radial like feature does not match the behaviour of the financial data. \medskip

It is worth to mention that although the NIG factor copula is penalised heavily due to its three parameters setup, it is frequently chosen to be the best copula to model the dependency between individual cryptos and the BTC future.
An extreme case would be ADA, only NIG factor is chosen in our dataset.
Another dependency structure being best described by the NIG factor copula is the pair of LTC-BTC future.
64 out of 112 training sets are best fitted by the NIG factor copula.
Indices like BITX and CRIX are sometimes best fitted with the NIG factor copula as well, accounting for modelling 12 and 27 training sets respectively.
The popularity of the NIG factor reflects the ability of the copula to model very complex dependency structure.
NIG factor copula is able to model the tail, radial asymmetry, and off diagonal behaviour.  \medskip %(ADA samples and fitted NIG samples)

Frank copula is generally not a good choice to model financial data just like what \cite{barbi2014copula} has reported.
Plackett is known for its dependence parameter can be written as the cross-product ratio \citep{joe1997multivariate}.
However, this feature does not bring the Plackett Copula advantage over other copulae to model the dependence structure between cryptos and BTCF. \medskip

\begin{table}[H]
 \ra{1.1}
    {\begin{tabularx}{\textwidth}{lYYYYY} \toprule
         Spot/ Copula & $t$ & Plackett & GMI & rotGumbel & NIG \\ \midrule
     \multicolumn{6}{l}{Individual Cryptos}                                                                                 \\
        \ \ \ BTC          & 65.18      & \phantom{0}3.57              & \phantom{0}0.89                     & \phantom{0}0.89               & 27.68                  \\
        \ \ \ ETH          & \phantom{0}2.68       & \phantom{0}5.36              & \phantom{0}7.14                     & 83.93              & \phantom{0}0.89                   \\
        \ \ \ ADA          & \phantom{0}0.00       & \phantom{0}0.00              & \phantom{0}0.00                     & \phantom{0}0.00               & 100.00\phantom{0}                 \\
        \ \ \ LTC          & 11.61      & \phantom{0}0.00              & \phantom{0}2.68                     & 28.57              & 57.14                  \\
        \ \ \ XRP          & \phantom{0}0.00       & 27.68             & \phantom{0}2.68                     & 69.64              & \phantom{0}0.00                   \\
   \multicolumn{6}{l}{Crypto Indices with BTC Constituent}                                                                  \\
        \ \ \ BITX         & 48.15      & \phantom{0}0.00               & 17.28                    & 19.75              & 14.81                  \\
        \ \ \ CRIX         & 53.41      & \phantom{0}0.00               & 12.50                    & \phantom{0}3.41               & 30.68                  \\
        \ \ \ BITW100      & 51.85      & \phantom{0}0.00               & 9.88                     & 35.80              & \phantom{0}2.47                   \\
    \multicolumn{6}{l}{Crypto Indices without BTC Constituent}                                                              \\
        \ \ \ BITW20       & \phantom{0}0.00        & \phantom{0}0.00               & \phantom{0}0.00                      & 96.30              & \phantom{0}3.70                   \\
        \ \ \ BITW70       & \phantom{0}0.00        & \phantom{0}0.00               & \phantom{0}0.00                        & 98.77              & \phantom{0}1.23                  \\
    \bottomrule
    \end{tabularx}
       }









\label{tab:copulasection}}
\end{table}

\subsection{Hedging Effectiveness Results}\label {sec: HE results}
In this section, we analyse the out-of-sample hedging effectiveness (HE) of BTCF as hedging.
HE is defined as $$\text{HE} = 1-\frac{\rho_h}{\rho_s},$$
a measure of the percentage reduction of portfolio risk attribute, in our case the spot $\rho_s$,
to hedged portfolio risk attribute $\rho_h$.
A higher HE indicates a higher hedging effectiveness and larger risk reduction. \medskip

The HE above is a generalisation of Ederington measure of hedging performance, where we,
in addition to variance, include other risk measures: Expected Shortfall 5\% and 1\% (ES5 and ES1), Value-at-Risk 5\% and 1\% (VaR5 and VaR1), and ERM.
In particular, ES5 is recommended by the Basel Committee on Banking Supervision (BCBS) to replace VaR as a quantitative risk metrics system.
The proposed reform aimed at enhancing the risk metric system's ability to capture tail risk. \medskip
%Discussions of the issue can be found in literatures.
%
We obtain a time series of out-of-sample $r^h$ of each hedging pair and each risk reduction objective by concatenating the out-of-sample results.
Then, we apply stationary block bootstrapping (SB) to the time series introduced by \cite{Politis1994} in our analysis in order to preserve the temporal structure of the data while sampling.
The SB procedure is as follow.
Assume a time series with $N$ observations $\{X_t\}_{t \in [1,N]}$ is a strong stationary, weakly dependence time series of interest,
we form blocks of samples $B = \{X_i, ..., X_{i+j-1}\}$.
Index $i$ is a random variable uniformly distributed over $[1,2,...,N]$ and $j$ is geometric distributed random variable with parameter .
The block index $i$ and block length $j$ are independent.
For any index $k$ which is greater than $N$, the sample $X_k$ is defined to be $X_{k(\mod N)}$.
For each block, we calculate the hedging effectiveness with different risk measures mentioned above.
We choose p=0.005, implying the expected block length is 200.
100 blocks are drawn for each risk minimising objective and spot. \medskip

From figure \ref{fig:HEboxplot}, we can see, as expected, the BTC involving spots, the BTC, CRIX, BITX and BITW100, are well hedged by the BTCF.
Surprisingly, the performances are consistent across different risk reduction objectives and different HE evaluation.
The median HE to BTC generated by various risk reduction objectives is ranging from 89.45\% to 99.31\%, median HE to CRIX is ranging from 81.13\% to 95.22\%,
median HE to BITX is ranging from 79.06\% to 94.84\%, median HE to BITW100 Is ranging from 71.07\% to 92.98\%. \medskip

The HE of BTCF to other cryptos and indices are substantially lower than to the BTC involving spots, but the consistency the performances across different risk reduction objectives and HE evaluation remains.
The median HE to BITW20 generated by various risk reduction objectives is ranging from 24.67\% to 47.02\%, median HE to BITW70 is ranging from 23.61\% to 49.30\%,
median HE to ADA is ranging from 9.01\% to 29.30\%, median HE to ETH Is ranging from 30.07\% to 36.18\%, median HE to LTC Is ranging from 37.74\% to 51.30\%,
median HE to XRP Is ranging from 0.46\% to 30.89\%.
\begin{figure}[h]
\includegraphics[width=\textwidth]{_pics/ES5_HE_boxplot.pdf}
  \caption{HE evaluated in different risk measures to various assets generated by minimizing ES5.
  \href{http://www.quantlet.com/}{\includegraphics[width=20pt]{_pics/qletlogo_tr.png}} }
\label{fig:HEboxplot}
\end{figure}

\section{Empirical Results}\label{sec:results}

\begin{figure}[t]
\centering
\begin{minipage}[t]{.475\textwidth}
    \centering
    \includegraphics[width=\textwidth]{_pics/MSE_BTC.pdf}
  \caption{Out-of-sample mean square errors of BTC-BTCF portfolios constructed with different copula and risk minimization objectives.
    The Frank copula is inferior in the BTC-involved portfolios.}
%    \href{http://www.quantlet.com/}{\includegraphics[height=\baselineskip]{_pics/qletlogo_tr.png}} }
\label{fig:MSE_BTC}
\end{minipage}
\hfill
\begin{minipage}[t]{.475\textwidth}
    \centering
    \includegraphics[width=\textwidth]{_pics/semiLowerVariance_BTC.pdf}
  \caption{Out-of-sample lower semivariance of BTC-BTCF portfolios constructed with different copula and risk minimization objectives.
  The Frank copula is obviously inferior.}
%  \href{http://www.quantlet.com/}{\includegraphics[height=\baselineskip]{_pics/qletlogo_tr.png}} }
\label{fig:SLV_BTC}
\end{minipage}
\end{figure}
\subsection{Data}\label{subsec:data}
In the empirical analysis, we consider the risk reduction capability
of CME Bitcoin Futures (BTCF) on five cryptos, namely Bitcoin (BTC), Ethereum
(ETH), Cardano (ADA), Litecoin (LTC) and Ripple (XRP), as well as five
crypto indexes, namely BITX, BITW100, CRIX, BITW20, and BITW70.
ETH, ADA, LTC, and XRP are popular cryptos tradable in various
exchanges and have large market capitalization. 
BITX, BITW100, and CRIX are market-cap weighted crypto indexes with
BTC as constituent. 
BITX and BITW100 track the total return of the 10 and 100 cryptos
with largest market-cap, respectively. 
CRIX decides the number of constituents by AIC and tracks that number
of cryptos with largest market-cap. In our case, the number of
constituents in CRIX is 5. 
BITW20 is also a market-cap weighted crypto index but with the 20
largest market-cap cryptos outside the constituents of BITX.
BITW70 has the same construction as BITW20 but with the 70 largest
market-cap cryptos outside BITX and BITW20. 
Therefore, BTC is excluded as a constituent in BITW20 and BITW70.

For each of the 10 hedging portfolios, a crypto or index is considered
as the spot and held in a unit size long position, while 
the front BTCF is held in a short position with units corresponding to
the OHR in order to reduce the risk of the spot. 
Except for the hedge of BTC, all hedging portfolios are considered to
be cross-asset hedges. 

We collect the spots' and BTCF's daily prices at 15:00 US Central Time
(CT). The reason for choosing this particular time is that the CME
group determines the daily settlements for BTCF's based on the trading
activities on CME Globex between 14:59 and 15:00 CT. This is also the
reporting time of the daily closing price by Bloomberg. 
The crypto spot data is collected from the data provider called
Tiingo (\href{https://www.tiingo.com/}{https://www.tiingo.com/}).
\natp{\em [thanks somewhere.]}
Tiingo aggregates crypto OHLC (open, high, low, and close) prices fed
by APIs from various exchanges. It covers major exchanges, such as
Binance, Gemini, Poloniex etc., so Tiingo's aggregated OHLC price is a
good representation a tradable market price. 
For each crypto, we match the opening price at 15:00 CT from Tiingo
with the daily BTCF closing price from Bloomberg.
Since CRIX is not available at 15:00 CT, we recalculated an hourly
CRIX using the monthly constituents weights and the hourly OHLC price
data collected from Tiingo. 
BITX, BITW20, BITW70, and BITW100 are collected from the official
website of their publisher Bitwise.com. 
The daily reporting time of the Bitwise indexes is 15:00 CT.

At the time of writing, the CRIX is undergoing the listing process on
the S\&P Dow Jones Indices, the official CRIX data will then be
calculated with Lukka Prime Data and available to the public via S\&P.

\subsection{Overview of the out-of-sample data}\label{subsec:oosdata}

For every asset and hedge portfolio, we concatenate the out-of-sample data to form a time series for analysis.
The date range of the out-of-sample time series is from 2019-10-21 to 2021-05-27, in total of 405 data points in each time series.
We analyse these time series throughout the whole result section. \medskip

We introduce the out-of-sample data in this subsection before we proceed to analysing the hedged portfolio results.
Figure~\ref{fig:BTC_price} presents the BTC and BTCF price in USD in the first panel and the arithmetic difference between the daily return of BTC and BTCF, i.e. $R_s - R_f$, in the
second panel.
In the first panel, the black vertical lines with capital letter labels indicate the days of the five most negative daily return of BTC during out-pf-sample period.
Table \ref{tab:BTC_5min} summarizes the relevant news headlines and events of those days. \medskip

Figures~\ref{fig:index_price} and \ref{fig:individualCoins_price}
\natp{\em [swap references to figures? So it's Figures 3 and 4 ...]} are the cumulative returns of the indices and individual cryptos respectively.
The black vertical lines labeled by assets name are the largest daily price drop of the assets in the out-of-sample data. \medskip

The out-of-sample data covers the pre-COVID19 period, 2019-10-21 to 2020-03-09, as well as the COVID19 period, 2019-03-19 onwards.
We can observe an overall upward trend of crypto prices in both periods.
Nonetheless, the volatilities of assets are high (annualized around 100\%) regardless of COVID19.

%\newpage
\begin{figure}[!t]
\includegraphics[width=\textwidth]{_pics/BTC_price.pdf}
  \caption{Out-of-sample BTC and BTCF price. The first panel presents the price of BTC in blue line and that of BTCF in orange line.
  The black vertical lines with capital letter labels indicate the five most negative daily return of BTC in the out-of-sample data.
  The second panel presents the difference between the \% return of BTC and BTCF.
  The black vertical lines with lowercase letter label indicate the five most negative returns.
  The crosses locate the level the returns.
  \href{http://www.quantlet.com/}{\includegraphics[height=\baselineskip]{_pics/qletlogo_tr.png}} }
\label{fig:BTC_price}
\end{figure}

\begin{table}[!]
    \centering
      \begin{tabularx}{.8\textwidth}{cccX}
        \toprule
        Label &   Date & \% Drop in Price &  Summary\\
        \midrule
        A &  2020-03-09 & 13.83 &  Coronavirus outbreak that affect the global markets; BTC as potential safe-haven was questioned.$^1$\\
        B &  2020-03-12 & 22.89 &  Continuation of the 2020-03-09 drop.  \\
        C &  2020-05-11 & 12.11 &  Price correction (from \$10,000 to \$8,100) after BTC price surge because of the third supply halving.$^{2,3}$ \\
        D &  2021-01-11 & 14.41 &  Short term correction of BTC hits the \$40,000 mark.$^4$\\
        E &  2021-05-17 & 11.86 &  Tesla stopped taking BTC as payment due to environmental concerns about energy use to process transaction.$^5$\\
        \bottomrule
      \end{tabularx}
        \caption{Summary of events that associated with the five most negative daily price drops in out-of-sample BTC price data.
        The capital letter labels in the first column are the labels in the first panel of figure~\ref{fig:BTC_price}.
        $^1$ is reported by the CNBC news \url{https://cnb.cx/3HZ2x7K}; $^2$ is from Forbes \url{https://bit.ly/3rdJPmP};
        $^3$ is from livemint.com \url{https://bit.ly/3FRi6Na};
        $^4$ is from CNBC \url{https://cnb.cx/3nU0ppO};
        $^5$ is from Reuters \url{https://reut.rs/3leCiAv}.
        }
        \label{tab:BTC_5min}
  \end{table}

%\clearpage

%\newpage
\begin{figure}[t]
\includegraphics[width=\textwidth]{_pics/index_price.pdf}
  \caption{Out-of-sample cumulative return of crypto indices.
  The black vertical lines indicate largest price drop of indices indicated by the labels.
  \href{http://www.quantlet.com/}{\includegraphics[height=\baselineskip]{_pics/qletlogo_tr.png}} }
\label{fig:index_price}
\end{figure}

\begin{figure}[!]
\includegraphics[width=\textwidth]{_pics/individualCoins_price.pdf}
  \caption{Out-of-sample cumulative return of individual cyptos.
  The black vertical lines indicate largest price drop of cryptos indicated by the labels.
  \href{http://www.quantlet.com/}{\includegraphics[height=\baselineskip]{_pics/qletlogo_tr.png}} }
\label{fig:individualCoins_price}
\end{figure}

\begin{table}[!]
    \centering
      \begin{tabularx}{.8\textwidth}{cccX}
        \toprule
        Label &  Date & \% Drop in Price &  Summary\\
        \midrule
        CRIX    &2020-03-09 & 23.77 & \multirow[t]{6}{\hsize}{Coronavirus outbreak that affect the global markets including the crpyto market.}\\
        BITX    & & 23.68 &  \\
        BITW100 & & 23.87 &  \\
        BITW20  & & 26.66 &  \\
        ADA     & &23.55 &  \\
        ETH     & &27.40 &  \\
        BITW70  & 2021-05-19& 27.64 & The spillover of the BTC shock on 2021-05-17 (label A in figure~\ref{fig:BTC_price} and table~\ref{tab:BTC_5min})\\
        XRP     & 2020-12-23 & 41.00 & Top executives were sued by the SEC of misleading investors$^1$. \\
        \bottomrule
      \end{tabularx}
        \caption{Summary of events that associated with largest price drops in out-of-sample data.
        The labels in the first column are the labels in figure \ref{fig:individualCoins_price} and figure \ref{fig:index_price}.
        CRIX, BITX, BITW100, BITW20, ADA and ETH have the same date the reason of the largest drop. $^1$ is reported by Bloomberg \url{https://bloom.bg/3cWdita}.}
        \label{tab:All_min}
  \end{table}
%\clearpage


%%% Local Variables:
%%% mode: latex
%%% TeX-master: "SRM"
%%% End:



\begin{landscape}
\begin{figure}[h]
  \begin{minipage}[t]{0.475\linewidth}
    \centering
    \includegraphics[height=.5\linewidth]{_pics/MSE_indices.png}
    \label{MSE_indices}
    \caption{Out-of-sample mean square errors of indices' hedge portfolios. Plots in a row share the same colour scale for comparison.}
  \end{minipage}
  \hfill
  \begin{minipage}[t]{0.475\linewidth}
    \centering
    \includegraphics[height=.5\linewidth]{_pics/MSE_cryptos.png}
    \label{MSE_cryptos}
    \caption{Out-of-sample mean square errors of cryptos' hedge portfolios. Each plot has its own colour scale.}
  \end{minipage}
  \begin{minipage}[b]{0.475\linewidth}
    \centering
    \includegraphics[height=.5\linewidth]{_pics/semiVariance_indices.png}
    \caption{Out-of-sample lower semi variance of indices' hedge portfolios. Plots in a row share the same colour scale for comparison.}
  \end{minipage}
    \hfill
  \begin{minipage}[b]{0.475\linewidth}
    \centering
    \includegraphics[height=.5\linewidth]{_pics/semiVariance_cryptos.png}

    \caption{Out-of-sample lower semi variance of cryptos' hedge portfolios. Each plot has its own colour scale.}
  \end{minipage}
\end{figure}
\end{landscape}

\subsection{An overview of the hedged portfolios without the copula
  selection step}
\label{subsec:HP1}
First, we analyse the results of the hedged portfolios without the
copula selection step in order to get a better understanding of how a
copula affects the hedged portfolio with various risk minimization
objectives.
To do so, we inspect the hedge performance of copulas by
the mean square error and lower semi-variance.
The mean square error
is the distance between a perfect hedge and the hedged portfolio
returns $\operatorname{MSE}= \E(R^2)$.
The lower semi-variance is defined as
$\operatorname{LSV}=\E \left( (R-\E(R))^2 \1_{\{R\leq \E(R)\}} \right)$.
All results presentedd here are out-of-sample results obtained without
the copula selection step in order to compare the performances across
copulae.

%\textcolor{darkblue}{As presented in Fig 3 and 4, either individual cryptos or indice, their cumulative returns dropped in Mar 2020. it's due to the result of COVID19. we can explain this for these two plots.}\\
%\textcolor{darkblue}{Here i think it should insert a paragraph to interpret how you enter the copulae, otherwise it's weird that comes to Fig 5 and 6.}\\

\natp{\em [Please fix order of figures and references. Figures 3 and 4
  are referenced here for the first time, when Figures 5 etc. have
  already been discussed in the previous section.]}

Figure \ref{fig:MSE_BTC} and \ref{fig:SLV_BTC} are the mean square
error and lower semivariance of BTC-BTCF. By far, the Frank copula
is the worst performing copula.
%\natp{\em [Can you swap the plots so that they are ordered
%  according to the scales?]}
In Figures \ref{fig:MSE_indices} and \ref{fig:SLV_indices}, the
phenomenom of the Frank copula being inferior to its counterparts can
be observed for the CRIX, BITX, BITW100, and
BITW20-BTCF portfolios.
Interestingly, in all of these portfolios, the spot has a
strong dependence with the BTCF.
In contrast, the inferiority of the Frank copula is less prominent in
the BITW70, ADA, ETH, LTC and XRP-BTCF portfolios.
As a consequence, it appears that the Frank copula is not an
appropriate choice to model assets with strong dependence.

A further observation from Figures \ref{fig:MSE_cryptos} and
\ref{fig:SLV_cryptos} is that the Gumbel copula does not perform as
well as other copulas in the ETH, LTC, and XRP-BTCF portfolios.
The reason is that the Gumbel copula has only upper tail dependence,
while ETH, LTC, and XRP exhibit lower tail dependence with BTCF.
We shall discuss this in the following section.

\natp{\em [Please also reference the lower semi-variance figures. All
  figures and tables must be reference. Is it possible to increase the
  font sizes in the graphs, at least for the titles. They will not be
  readable in the final version otherwise.  Also -- as mentioned
  before -- please create .eps or .pdf graphics. These are scalable,
  wheres png does not scale well.]}

\subsection{Copula Selection Results}\label{subsec:-copula-results}
\begin{table}[t]
 \ra{1.1}
    {\begin{tabularx}{\textwidth}{lYYYYY} \toprule
         Spot/ Copula & $t$ & Plackett & GMI & rotGumbel & NIG \\ \midrule
     \multicolumn{6}{l}{Individual Cryptos}                                                                                 \\
        \ \ \ BTC          & 65.18      & \phantom{0}3.57              & \phantom{0}0.89                     & \phantom{0}0.89               & 27.68                  \\
        \ \ \ ETH          & \phantom{0}2.68       & \phantom{0}5.36              & \phantom{0}7.14                     & 83.93              & \phantom{0}0.89                   \\
        \ \ \ ADA          & \phantom{0}0.00       & \phantom{0}0.00              & \phantom{0}0.00                     & \phantom{0}0.00               & 100.00\phantom{0}                 \\
        \ \ \ LTC          & 11.61      & \phantom{0}0.00              & \phantom{0}2.68                     & 28.57              & 57.14                  \\
        \ \ \ XRP          & \phantom{0}0.00       & 27.68             & \phantom{0}2.68                     & 69.64              & \phantom{0}0.00                   \\
   \multicolumn{6}{l}{Crypto Indices with BTC Constituent}                                                                  \\
        \ \ \ BITX         & 48.15      & \phantom{0}0.00               & 17.28                    & 19.75              & 14.81                  \\
        \ \ \ CRIX         & 53.41      & \phantom{0}0.00               & 12.50                    & \phantom{0}3.41               & 30.68                  \\
        \ \ \ BITW100      & 51.85      & \phantom{0}0.00               & 9.88                     & 35.80              & \phantom{0}2.47                   \\
    \multicolumn{6}{l}{Crypto Indices without BTC Constituent}                                                              \\
        \ \ \ BITW20       & \phantom{0}0.00        & \phantom{0}0.00               & \phantom{0}0.00                      & 96.30              & \phantom{0}3.70                   \\
        \ \ \ BITW70       & \phantom{0}0.00        & \phantom{0}0.00               & \phantom{0}0.00                        & 98.77              & \phantom{0}1.23                  \\
    \bottomrule
    \end{tabularx}
       }









 \caption{Copula selection results (shortened).
        The values are the absolute frequencies of a copula chosen by
        the AIC procedure during the out-of-sample period. 
        Each frequenc represents five trading days, which corresponds
        to the recalibration interval.
        The table show the frequently chosen copulas, which are
        $t$, Plackett, Gaussian Mix Independent (GMI), rotated Gumbel
        (rotGumbel) and Normal Inverse Gaussian factor copula (NIG). 
        }
    \label{tab:copulasection}
\end{table}
Next, we inspect the copula selection results by the AIC procedure
described in section~\ref{subsec:copula-selection}. 
Although the copula selection is only an intermediate step to obtain
the optimal hedge ratios,
the result of this step can help us better understand the dependence
feature between BTCF and the assets we study in this work.
This provides valuable information for modeling the assets in the future.
The decisions of the AIC procedure are summarised in Table
\ref{tab:copulasection}. Overall, the $t$-copula, rotated Gumbel
(rotGumbel), and the NIG factor copula are the most frequently chosen
copulae by the AIC procedure.

The $t$-copula is predominantly to model the dependence between 
the BTC and BTC-involving-indices, CRIX, BITX, BITW100, and the BTC
future.
BTC and BTC-involving-indices exhibit strong (upper and lower) tail
dependence with BTCF.  We interpret tail dependence as a strong
tendency for one asset to be extreme when the other is extreme and
vice versa \citep{McNeil2015}.
In fact, the $t$ copula has been recommended in various empirical
studies to model financial data, such as~\cite{zeevi2002beyond} and~
\cite{breymann2003dependence}.
Those studies suggest that the $t$-copula is a better model compared
to the Gaussian copula as financial data typically exhibit heavy tails
and tail dependence. 

\natp{\em [I do not understand the argument with the radial
  symmetry. Specifying the definition of radial symmetry is also not
  helpful here. To sharpen the argument, it should be something like
  this: On the other hand, the symmetry of the $t$-copula appears to
  be a poor choice to model the remaining hedging pairs. Here, the AIC
  criterion predominantly selects copulas that allow for asymmetry
  between the spot and the underlying. This reflects that overall
  dependence between a 
  non-BTC-related spot asset and the BTCF may be low, but tail risk,
  especially on the down-side is present, as crashes in the crypto
  market, which occur frequently, do not differentiate between
  assets.]} 
On the other hand, the \natp{\em (delete: radial)} symmetry of the
$t$-copula appears to be a poor choice to model the remaining hedging
pairs. 
\cite{demarta2005t} describes the radial symmetry feature of the $t$-copula ``strong'' as it is a radially symmetric distribution.
To be specific, if $(U_1, ..., U_d)$ is a vector distributed in $t$-copula,
then $(U_1, ..., U_d) \overset{\mathcal{L}}= (1-U_1, ...,
1-U_d)$.
This symmetry can be justified in the dependence structure between a
futures and its underlying by the theory of futures pricing,
which suggests the price of a futures is a function of the underlying
price \citep{hull2003options}. However, there is no such relationship
between a futures and an asset which is not the underlying. Besides,
asset prices tend to crash simultaneously whereas positive development
tends to be idiosyncratic.   

Among the three popular copulae, rotGumbel copula shows its ability to
model the dependence between ETH and BTCF. rotGumbel also performs
well when modelling dependence between XRP, BITW20, BITW70, and the
BTCF. In particular, the whole time series of the two indices, BITW20
and BITW70, are best fitted solely with the rotated Gumbel
copula.

In fact, Clayton's AIC in many of the training sets is the second
lowest, just higher than that of rotated Gumbel. This is because the
Clayton copula has the same ability to model the lower quantile
dependence. However, Clayton's radial like feature does not match the
behaviour of the financial data. 

It is worth to mention that although the NIG factor copula is
penalised heavily due to its three parameters setup, it is frequently
chosen to be the best copula to model the dependence between
individual cryptos and the BTC future. An extreme case would be ADA,
where only the NIG factor is chosen in our dataset. 
Another dependence structure best described by the NIG factor
copula is the pair of LTC-BTCF, with 64 out of 112 training sets best
fitted by the NIG factor copula. Indices like BITX and CRIX are
sometimes best fitted with the NIG factor copula as well, accounting
for modelling 12 and 27 training sets, respectively. 
\natp{The popularity of the NIG factor copula reflects the ability of the
copula to model complex dependence structure, involving heavier tails
than the Gaussian as well as asymmetric distributions. (was: the
NIG factor copula is able to model the tail, radial asymmetry.)}

%and
%off-diagonal (the region away from the diagonal line $(0,0)$ to
%$(1,1)$, see figure \ref{fig:copulaeScatterPlot}) behaviour. \natp{\em
%  [I think you did not understand my comment. It is clear what the
%  ``off-diagonal'' is. It is unclear what is meant by ``off-diagonal
%  behaviour''. Do you mean extremes on the other diagonal? Do you mean
%  a cloud of points away from the diagonal? Not sure what the NIG
%  factor copula is capable of modelling in this respect. Perhaps you
%  have a reference?]}

The Frank copula turn out to generally be a poor choice to model financial
data (as also reported by \cite{barbi2014copula}).
The Plackett copula is characterised by its dependence parameter being
equal to the cross-product ratio, see
eq.~\ref{eq:PlackettCrossProduct}. However, apparently, this property
does not capture the dependence structure of cryptos and BTCF.


\subsection{Hedged portfolios with the copula selection step}\label{subsec:HP2}

\afterpage{
  \begin{landscape}
  \begin{table}[!] \centering
\resizebox{.8\paperheight}{!}{%
\begin{tabular}{l*{10}{r}}
\toprule
{} &   BTC & ETH & ADA & LTC & XRP & BITX & CRIX & BITW100 & BITW20 & BITW70\\
\midrule
  \multicolumn{10}{l}{Spots assets only}   \\
\ \ \ Mean \%  &      0.3915 &      0.6819 &      0.9467 &      0.3227 &      0.2987 &      0.4308 &      0.4602 &      0.4683 &      0.6249 &      0.6353 \\
\ \ \ Std \%   &      4.4023 &      6.0103 &       6.699 &      6.4781 &      7.9843 &      4.5676 &       4.542 &      4.6174 &      5.5021 &      5.8155 \\
\ \ \ MD \%    &    -25.9965 &    -32.0144 &    -26.8528 &    -37.5913 &    -52.7652 &     -27.022 &    -27.1385 &    -27.2694 &    -31.0092 &    -32.3453 \\
\ \ \ MD date &  2020-03-12 &  2020-03-12 &  2020-03-12 &  2021-05-19 &  2020-12-23 &  2020-03-12 &  2020-03-12 &  2020-03-12 &  2020-03-12 &  2021-05-19 \\
  \multicolumn{10}{l}{Variance minimizing portfolios}   \\
\ \ \ Mean \%  &      0.0215 &      0.2823 &      0.5617 &     -0.0871 &     -0.0123 &      0.0561 &      0.0812 &      0.0855 &      0.2429 &      0.2706 \\
\ \ \ Std \%   &      0.3221 &      3.8741 &      5.2722 &      3.9052 &      7.1537 &      0.9954 &      0.9183 &      1.1986 &      3.5846 &      3.8838 \\
\ \ \ MD \%    &     -1.4393 &    -17.7421 &    -13.8687 &    -28.3029 &    -52.5236 &     -7.7567 &     -7.1025 &    -11.3866 &     -21.468 &    -23.9984 \\
\ \ \ MD date &  2020-11-30 &  2021-05-19 &  2021-01-08 &  2021-05-19 &  2020-12-23 &  2021-05-19 &  2021-05-19 &  2021-05-19 &  2021-05-19 &  2021-05-19 \\
  \multicolumn{10}{l}{VaR 95\% minimizing portfolios}   \\
\ \ \ Mean \%  &      0.0253 &      0.3084 &      0.5726 &     -0.0742 &      0.0208 &      0.0562 &      0.0863 &      0.0846 &      0.2728 &      0.2847 \\
\ \ \ Std \%   &      0.3294 &      3.8944 &      5.2204 &      3.9145 &       7.152 &       0.993 &      0.9151 &       1.198 &       3.594 &      3.9133 \\
\ \ \ MD \%    &     -1.5347 &     -19.175 &    -14.6974 &    -28.3672 &    -52.5667 &     -7.5639 &     -6.9744 &    -11.2582 &    -22.0733 &    -24.6513 \\
\ \ \ MD date &  2020-11-30 &  2021-05-19 &  2021-05-19 &  2021-05-19 &  2020-12-23 &  2021-05-19 &  2021-05-19 &  2021-05-19 &  2021-05-19 &  2021-05-19 \\
  \multicolumn{10}{l}{VaR 99\% minimizing portfolios}   \\
\ \ \ Mean \%  &      0.0176 &      0.2977 &      0.5562 &     -0.0852 &      0.0352 &      0.0593 &      0.0738 &      0.0823 &      0.2499 &      0.2788 \\
\ \ \ Std \%   &       0.327 &      3.9132 &      5.3466 &      4.1503 &      7.1658 &      1.0178 &      0.9695 &      1.2338 &       3.621 &      3.9257 \\
\ \ \ MD \%    &     -1.5689 &    -18.6061 &    -15.4795 &    -29.0915 &    -52.5727 &     -8.0299 &     -7.0185 &    -11.8752 &    -21.6634 &    -24.5294 \\
\ \ \ MD date &  2020-11-30 &  2021-05-19 &  2021-05-19 &  2021-05-19 &  2020-12-23 &  2021-05-19 &  2021-05-19 &  2021-05-19 &  2021-05-19 &  2021-05-19 \\
  \multicolumn{10}{l}{ES 95\% minimizing portfolios}   \\
\ \ \ Mean \%  &      0.0204 &      0.3082 &      0.5525 &     -0.0808 &      0.0176 &      0.0591 &      0.0777 &      0.0848 &      0.2608 &      0.2785 \\
\ \ \ Std \%   &      0.3234 &       3.889 &      5.2673 &      3.9829 &      7.1533 &      1.0065 &      0.9207 &      1.2125 &      3.6115 &      3.9157 \\
\ \ \ MD \%    &     -1.5629 &    -18.7819 &    -14.9647 &    -28.4608 &    -52.5698 &     -7.6211 &     -6.9894 &    -11.1357 &     -21.543 &    -24.3474 \\
\ \ \ MD date &  2020-11-30 &  2021-05-19 &  2021-05-19 &  2021-05-19 &  2020-12-23 &  2021-05-19 &  2021-05-19 &  2021-05-19 &  2021-05-19 &  2021-05-19 \\
  \multicolumn{10}{l}{ES 99\% minimizing portfolios}   \\
\ \ \ Mean \%  &      0.0148 &       0.308 &      0.5016 &     -0.1029 &       -0.02 &      0.0598 &      0.0835 &      0.0781 &      0.2538 &       0.266 \\
\ \ \ Std \%   &      0.3476 &      3.8954 &       5.404 &      4.1581 &      7.2887 &      1.0312 &      0.9461 &       1.264 &      3.6323 &       3.932 \\
\ \ \ MD \%    &     -1.6225 &    -18.7625 &    -15.4481 &    -29.1727 &      -52.57 &     -7.7424 &     -7.0203 &    -11.9263 &    -21.9866 &    -24.4764 \\
\ \ \ MD date &  2020-11-30 &  2021-05-19 &  2021-05-19 &  2021-05-19 &  2020-12-23 &  2021-05-19 &  2021-05-19 &  2021-05-19 &  2021-05-19 &  2021-05-19 \\
  \multicolumn{10}{l}{ERM $k=10$ minimizing portfolios}   \\
\ \ \ Mean \%  &      0.0223 &      0.3117 &      0.5722 &     -0.0512 &      0.0155 &       0.059 &       0.084 &      0.0853 &      0.2564 &      0.2818 \\
\ \ \ Std \%   &      0.3221 &      3.8679 &       5.359 &      3.8812 &      7.1579 &      1.0078 &      0.9087 &      1.2032 &      3.6009 &      3.9074 \\
\ \ \ MD \%    &     -1.5242 &    -18.8729 &    -14.3885 &    -28.0879 &    -52.5689 &     -7.8581 &      -7.053 &    -11.1846 &     -21.592 &     -24.525 \\
\ \ \ MD date &  2020-11-30 &  2021-05-19 &  2021-01-08 &  2021-05-19 &  2020-12-23 &  2021-05-19 &  2021-05-19 &  2021-05-19 &  2021-05-19 &  2021-05-19 \\
\bottomrule
\end{tabular}}
\end{table}
\end{landscape}
}

We now turn to the hedge performance. Table~\ref{tab:bigTable}
presents the first two moments, maximum drawdown (MD) and the date of
MD of the hedge portfolios. \natp{An interesting observation is the
  similarity of the statistics when minimising with respect to
  different risk measures. (was: 
We observe that the statistics of the
portfolios with different objectives are similar to each
other. \natp{\em [Unclear what is meant by ``different objectives''?
  Different risk measures?]})} Detailed statistics are found in
Tables~\ref{tab:var_rh} to \ref{tab:ERM_rh} in Appendix
\ref{sec:SSHP}. 
 
Unsurprisingly, the BTC-involved spots, i.e., BTC, CRIX, BITX, and
BITW100, are well hedged by the BTCF regardless of risk minimization
objective. 
The BTC-not-involved spots, on the contrary, are less promising. Those
hedge portfolios' returns are as volatile or nearly as volatile as the
assets themselves, see for example ADA and XRP. 
We shall further discuss the effectiveness of hedge in the next
section. %\ref{sec: HE results}. 

\subsection{Hedging Effectiveness Results}\label{sec: HE results}
\begin{figure}[t]
\includegraphics[width=\textwidth]{_pics/HE_boxplot.pdf}
  \caption{Hedging effectiveness (HE) of portfolios with different risk minimization objectives evaluated by the corresponding risk minimization objectives.
            The boxplots indicate the the median, upper quartile, lower quartile, minimium and maximum of the bootstrapped HE.
            The HE of BTC-involved spots are significantly higher than that of BTC-not-involved spots.
  \href{http://www.quantlet.com/}{\includegraphics[height=\baselineskip]{_pics/qletlogo_tr.png}} }
\label{fig:HEboxplot}
\end{figure}
In this section, we analyse the out-of-sample hedging effectiveness
(HE) of BTCF as a hedge instrument. 
HE is defined as $$\text{HE} = 1-\frac{\rho_h}{\rho_s},$$
i.e., it measures the percentage reduction of risk of the hedge
portfolio $\rho_h $ relative to the risk of the spot position $\rho_s$.
A higher HE indicates a greater risk reduction and thus the hedge is
more effective.  
The HE above is a generalisation of how \citet{ederington1979hedging}
evaluates hedge performance, which focusses on variance as the risk
measure. 
Aside from variance, we include the risk measures which act as
loss function while searching for the optimal hedge ratios: ES 95\%
and 99\%, VaR 95\% and 99\% and ERM.

\natp{\em [A description is missing of how the hedge actually
  works. How much data enters the calibration (I believe it's 300
  days). Then what happens? On each day we have a hedge ratio. Do we
  construct a new hedge each day? How long is the hedge period? So,
  after which time period is P\&L calculated. 
  This could also be added to Section 2.2.\\
  What is the motivation behind the bootstrapping?]}

The formulation above gives a point estimate per \natp{test data
  point (was: testing data)}.
However, each of our test data contain only 5 data points, the length
is not sufficient to draw meaningful risk measure results. \natp{\em
  [Last sentence unclear, because information is missing, see above.]}
To address this issue, we apply bootstrapping method on the
concatenated test data time series as described in the beginning of
the result section. 

\natp{Bootstrapping refers to sampling from the empirical distribution of a
given data sample (e.g.\ a time series of fiancial returns). The
principal idea underlying bootstrapping is to provide statistical
information about estimators that cannot be derived from just one
realisation of the data. The method was introduced by
\cite{Efron1979}; see also \citep{efron1994introduction,
  davison1997bootstrap}. (was: 
The bootstrapping method is known to be a powerful nonparametric tool 
for approximating complicated statistics~\citep{efron1994introduction,
  davison1997bootstrap}.)}

We apply the stationary block bootstrap of \cite{Politis1994} in our
analysis in order to account for the time-dependence of the data while 
sampling.
The sampling method of the stationary bootstrapping procedure is as
follows. 
Assuming a time series $\{X_t\}_{t \in [1,N]}$ that is
a stationary strong, weakly dependent time series. 
Blocks of samples $\{X_i, ..., X_{i+j-1}\}$, where the index $i$ is a
random variable uniformly distributed over 
$[1,2,...,N]$ and $j$ is geometric distributed random variable with
parameter $p$ independent of $i$. 
For any index $k$ which is greater than $N$, the sample $X_k$ is
defined to be $X_{k(\mod N)}$. \natp{\em [Fix notation with $\mod$
  operator.]} 
For each block, we calculate the hedging effectiveness as outlined above.
We choose $p=1/250$, implying the average block length is
250. \natp{\em [I think this was changed, right?]}
This average block length is chosen to reasonably calculate ES and VaR.
100 blocks are drawn for each risk minimising objective and
spot. 
%\natp{\em [You need to go over this again. There are many
%  grammatical errors in there that need to be fixed. Also explain that
%  expected block length is 200 so that ES99\% and VaR99\% can be
%  calculated. Is the hedge ratio calculated for each sample drawn,
%  etc.? ]}

\natp{\em [The previous part still needs quite a bit of work as it
  remains unclear to the reader what is actually going on. The
  principal idea is to create an out-of-sample distribution of HE, but
  with one sample path this is not possible. Therefore, the bootstrap,
  right?]}

Figure \ref{fig:HEboxplot} report the bootstrapped HE samples from the
concatenated out-of-sample hedge portfolio return. As expected, the
BTC involving spots, the BTC, CRIX, BITX and BITW100, are well hedged
by the BTCF.  The HEs of the other cryptos and indices are
substantially lower than to the BTC-related instruments, but 
exhibit a consistent performances across different risk measures. 
As it turns out, some HE bootstrapping samples are even negative,
which means the ``hedge'' portfolio actually increases the risk. 
\natp{This is, of course, counter-productive to hedging indicating
  that BTC futures may not be suitable for cross-asset hedges (was:
  This is an unfavorable situation for investors if they want to hedge 
cryptos with BTC futures. 
We do not recommend BTC futures being used to cross hedge cryptos.)}

%\natp{\em[It does not
%  make sense to compare across different risk measures, so I suggest
%  to delete the last statement. }

%\natp{[How can HE be negative??? Especially
%  for variance this is impossible.]}

%\natp{\em [We need some proper conclusions here.
%  \begin{itemize}
%  \item It is noticable that the HE of variance is highest.
%  \item Comparing ES95\% and ES99\%, it is seen that ES95\%
%    performs better. This is an indication that tail risk remmains
%    despite hedging.
%  \end{itemize}
%  ]}



%\natp{\em [A general conclusion is needed. That ES95\% / Var95\%
%  perform better than their 99\% counterparts is also evident from
%  Figures 7 and 8. So focussing on tail risk by choosing an extreme tail risk
%  measure does not lead to a promising hedge. Also variance performs
%  well, especially for the lower correlated instruments /
%  indices. Var95\% also performs well across all copulas. Please check
%  if the tables in the appendix confirm this.\\
%  Main conclusion is that the choice of copula highlights the
%  differences in the instruments / indices and their relationships
%  with the futures contract. Unfortunately, we see that the copula
%  choice does not always lead to the best hedge (see e.g.\ ADA), where
%  Frank, Gaussian, Plackett and rotGumbel perform much better in
%  terms of MSE. I suspect that has to do with the fact that hedging is
%  necessarily linear, so a better copula model may not necessarily
%  attain a better (linear) hedge. This means that the copula selection
%  results should be taken as interesting by themselves. Also, they
%  might be useful if using more than one hedge instrument, e.g.\ by
%  including options as hedge instruments.]}

%%% Local Variables:
%%% mode: latex
%%% TeX-master: "SRM"
%%% End:

% ----------------
% --- Results ---
% ----------------

\section{Conclusion and Outlook}\label{sec:conclusion-and-outlook}
We study the hedging effectiveness of Bitcoin futures to cryptos and crypto indices.
We consider commonly used risk measures with different configurations to cater different risk appetite and scenarios, they are variance, Value-at-Risk 95\% and 99\%,
Expected Shortfall 95\% and 99\%, and Exponential Risk Measure $k=10$. Dependence between cryptos and the futures plays an important role in hedging as it determines the distribution of the portfolio returns.
However, in the time of writing, the crypto market is a vibrant and fast-developing market, causing cryptos to have complex and, possibly,
time changing dependence structures with the Bitcoin futures.
We use copulae, a flexible statistical tool that model marginals and dependence separately to capture complexity of the dependence structure;
and periodically re-calibrate copulae and find the best-fitting copula via AIC to address the potential time changing dependence. \medskip

An extensive out-of-sample backtest suggests that the Bitcoin futures is consistently capable of hedging BTC and BTC-involved indices,
BITX, CRIX, and BITW100, under different risk minimisation objectives and copula models.
The MSEs and LSVs of the resulting portfolios are indistinguishably at a low level except the Frank copula.
On the other hand, the AIC procedure flavours t-copula because it captures the tail dependence feature of the data.
Unsurprisingly, the portfolios' out-of-sample maximum drawdowns are significantly reduced. \medskip

Contrarily, we observe a diverse result of the capability of BTC futures to hedge other cryptos and BTC-not-involved indices.
In general, the ES 95\% and VaR 95\% perform better than their 99\% counterparts.
In particular minimising ES 99\% leads to relatively high MSE and LSV disregard of which copula is in use.
The ES 99\% and VaR 99\% even result in out-of-sample maximum drawdowns which are higher than that of the 95\% counterparts in some portfolios,
for example in the ETH- and LTC-BTCF portfolio.
Therefore, we conclude that overly emphasising tail risks by choosing extreme tail risk measures does not lead to a promising hedge in a cross hedging setting. \medskip

The AIC procedure mainly flavours rotGumbel and NIG factor copula in modelling other cryptos and BTC-not-involved indices.
This reflects the idiosyncratic nature of downward movements in crypto market.
However, the best-fitting copula does not necessary leads to best performing portfolio in MSE or LSV, see e.g. ADA. \medskip

This is what we call the discrepancy between the copula selection result and the MSE LSV results.
We suspect the discrepancy is the result of our hedge being a linear, long and short only with assets and derivative with linear payoffs i.e. futures.
Although copulae are flexible to model complex dependence structures by emphasising numbers of important features
e.g. lower tail dependence and radial symmetry, the simple linear hedge is very limited in flexibility to address to complex dependence.
Including derivatives with non-linear payoffs e.g. options might be a way out. \medskip

%A possible research direction would be hedging the linear part with futures and non-linear part with options, e.g. the BTC options.
% ----------------
% --- Conclusion and Discussion ---
% ----------------

\clearpage
%
\bibliography{finance} %
\newpage
\section{Appendix}\label{sec:appendix}

\begin{proposition}
   Let $\bm{X} = (X_1, ..., X_d)^\top$ be real-valued random variables with corresponding
   copula density $\bm{c}_{X_1, ..., X_d}$, and continuous marginals $F_{X_1}, ..., F_{X_d}$.
   Then,
   density of the linear combination of marginals $Z = n_1 \cdot X_1 + ... +  n_d \cdot X_d $ is

   \begin{align}
   f_Z(z) &= \left| n_1^{-1} \right| \int_{[0,1]^{d-1}} \left[ \bm{c}_{X_1,...,X_d}
      \{F_{X_1} \circ S(z), u_2, ..., u_d \} \cdot
      f_{X_1} \circ S(z) \right] du_2 ... du_d \label{density} \\
      S(z) &= \frac{1}{n_1}\cdot z - \frac{n_2}{n_1} \cdot F^{-1}_{X_2}(u_2) - ... -  \frac{n_d}{n_1} \cdot F^{-1}_{X_d}(u_d)
      \end{align}
   \end{proposition}

\begin{proof} \medskip
   Rewrite $Z = n_1 \cdot X_1 + ... +  n_d \cdot X_d $ in matrix form
   \begin{align}
      \begin{bmatrix}
      Y \\ X_2 \\ \vdots \\ X_d
      \end{bmatrix}
      = \begin{bmatrix}
      n_1    & n_2   & \cdots & n_d     \\
      0      & 1     &  \cdots & 0       \\
      \vdots &       & \ddots & \vdots \\
      0      & \cdots &       & 1  \\
      \end{bmatrix}
      \begin{bmatrix}
         X_1 \\ X_2 \\ \vdots \\ X_d
      \end{bmatrix}
      = \bm{A}
      \begin{bmatrix}
         X_1 \\ X_2 \\ \vdots \\ X_d
      \end{bmatrix}.
      \end{align} \medskip

   By transformation variables
   \begin{align}
      \bm{f}_{Z,X_2,...,X_d}(z, x_2, ...,x_d) &= \bm{f}_{X_1,...,X_d}\left( \bm{A}^{-1}
      \begin{bmatrix}
         z \\ x_2 \\ \vdots \\ x_d
      \end{bmatrix}
      \right)  \cdot |\det \bm{A}^{-1}| \\
      &= \left| n_1^{-1} \right| \bm{f}_{X_1,...,X_d}\{S(z), x_2,...,x_d\}
      \end{align} \medskip

   Let $u_i = F_{X_i}(x_i)$ and
   use the relationship
   \begin{align}
      \bm{c}_{X_1,...,X_d}(u_1, ..., u_d)=\frac{\bm{f}_{X_1,...,X_d}(x_1,...,x_d)}{\prod_{i=1}^d f_{X_i}(x_i)},
   \end{align}
   we have
   \begin{align}
     & \bm{f}_{Z,X_2,...,X_d}(z, x_2, ...,x_d) = \\
      & \left| n_1^{-1} \right| \cdot
      \bm{c}_{X_1,...,X_d}\{F_{X_1} \circ S(z), u_2, ...,  u_d\}  \cdot
      f_{X_1} \{ S(z) \} \cdot
      \prod_{i=2}^d f_{X_i}(x_i)
      \end{align}

   The claim \ref{density} is obtained by integrating out $x_2, ... x_d$ by substituting $dx_i = \frac{1}{f_{X_i}(x_i)}du_i$.
   \end{proof}

\clearpage
\section{Supplementary Material: Intraday Hedging}

\natp{\em [This section must be shortened significantly. It is not
  necessary to repeat everything, just point out the differences.]}

This supplementary material extends the study in the main body to an
intraday rebalancing setting. \natp{\em [Do we really have intraay
  rebalancing? Or just intraday P\&L? If rebalancing, then clarify.]}
The idea is to infer if the findings from the daily setting extend to
intraday data. 

\natp{\em [From my point of view there is no need to motivate intraday
  risk management.]}
Studying the intraday hedge is familiar to academia, 
e.g. \citet{harris2010limits}, \citet{dungey2013impact},
\citet{tse2013does}, and \citet{sheu2014incremental}.

Numerous studies on the crypto market are associated with or motivated
by the presence of intraday traders as well, 
e.g. \citet{petukhina2021rise}, \citet{meshcheryakov2020ethereum},
\citet{alexander2022role}, \citet{zhang2022data} and
\citet{katsiampa2022high}. 

\natp{(delete: We have in view to robustify the results from the main body.)}
The methodology in this supplementary material is similar to the main
body, except we 
\begin{enumerate}
\item form two hedging portfolios, BTC-BTCF and ETH-BTCF,
\item simulate trades using Deribit hourly data, and
\item rebalance every four hours.
\end{enumerate}

\subsection{Data}
The intraday analysis is built upon a dataset of date range from
2020-06-01 00:00 UTC to 2020-08-01 00:00 UTC. 
All the price data are sampled hourly.
The Deribit contract \natp{BTC-25SEP20 (was: 20BTCUSD25SEP20)} represents the BTCF;
the Deribit BTC and ETH index represent the spot of BTC and ETH, respectively.
We take the hourly closing mid-price of the BTCUSD25SEP20 as the futures price and the last value of the BTC and ETH index in every hourly bucket as the spot prices.
Since the date range of the data is fully covered by the lifetime of BTCUSD25SEP20, this study does not require rolling procedure to roll over futures contract near expiry.

\subsubsection{Procedure}
Starting from oldest data:
\begin{enumerate}
    \item Calibrate a copula by a training data of 336 datapoints, equivalent to 14 days data
    \item Draw samples $(\tilde r_s, \tilde r_f)$ from the calibrated copula
    \item Numerically search for $h^* = \arg \max \phi(\tilde r_s - h\tilde r_f)$ according to a risk measure $\phi$
    \item Apply $h^*$ to testing data to yield $r_h$; the testing data is the consecutive 4 data points to training data, i.e. the 4 hours data consecutive to the last training data
    \item Repeat the procedure for the next 4 datapoints
    \item Concatenate $r_h$s the and sort chronologically to form a full length out-of-sample hedging portfolio returns
\end{enumerate}

The procedure is further repeated for all the combinations of risk measures and copulae.
The full-length out-of-sample returns represent the corresponding performance of a particular risk measure-copula combination.
They are used in computation mean square error (MSE) and lower-semi-variance (LSV) shown in the following section.

The AIC selection step is performed between Steps 1 and 2 of the procedure above.
The resulting out-of-sample returns are a mix of results from the copula that has the lowest AIC on the training data.
We keep a record of how many times a copula is choosen by this step.
To yield robust HE measures, we apply stationary bootstrapping to the
full-length AIC selected out-of-sample returns with the following
parameters: $p=1/4, T=300, N=1000$. 

\subsection{Results}
\begin{figure}[t]
\includegraphics[width=\textwidth]{_pics/hourly_HE.png}
  \caption{Intraday HEs of BTC-BTCF and ETH-BTCF portfolio. The HEs of BTC-BTCF portfolios are significantly higher than $0$,
which suggests involving the BTCF in the portfolio can effectively reduce market risk.
The HEs of ETH-BTCF portfolios are lower than that of BTC-BTCF portfolios, and sometimes go below $0$.
We nullify the hedging ability of BTCF in this intraday ETH-BTCF setting.
  \href{http://www.quantlet.com/}{\includegraphics[height=\baselineskip]{_pics/qletlogo_tr.png}} }
\label{fig:HEboxplot_intraday}
\end{figure}

\textit{Bootstrapped out-of-sample HEs}: The analysis begins with the
boxplot in figure \ref{fig:HEboxplot_intraday} of the bootstrapped out-of-sample
HEs. 
In general, most of the daily rebalancing results of BTC-BTCF carry over to the intraday rebalancing schedule;
\natp{\em [This can be {\bf significantly} shortened!]} The intraday rebalancing ETH-BTCF is different from its daily rebalancing counterpart.
Note that the exact values of HEs from the two rebalancing schedules should not be directly compared for two reasons: 1. The data are from different date ranges; 2. Various factors contribute to the difference between results from different sampling frequencies, e.g. Epps effect, microstructure noise,
and asynchronous trading.
However, we compare the patterns and conclusions to get a general understanding of the hedging issue for future use.

The main difference between the intraday rebalancing and daily
rebalancing ETH-BTCF portfolio is that negative HEs appear in the
intraday results in all the risk measures we consider. \natp{This
  implies that hedgers cannot draw on potential intraday dependence
  between BTC and ETH to hedge an ETH crypto position. (delete: 
The negative HEs suggests that BTCF should not be used to hedge against ETH in an intraday setting.
Consider also the fact that BTCF is written on BTC instead of ETH, hedgers have no ground to assume they can take advantage of the intraday dependency structure between ETH and BTCF for hedging.
Therefore, we mainly focus on the BTC-BTCF portfolios and mentioning
the results of the ETH-BTCF portfolio is needed.)} 


\natp{Turning to BTC-BTCF, among the (delete: \textit{BTC-BTCF HEs}: The HEs of BTC-BTCF are
significantly higher than zero across different risk measures,
suggesting that adding BTCF to a BTC portfolio can effectively reduce
the risk measured by selected measures. Among)} 
risk measures, HE of variance is the highest \natp{(delete: for the
  BTC-BTCF portfolio)}, ranging between 72\% to 98\%, while the
  HE's other risk measures range between 25\% to 80\%.
The finding that reducing variance is a well-achievable objective is
consisten with the findings of the daily rebalancing schedule.
\natp{\em [We can consider deleting the discussion about 99\% risk measures...]}
On the other hand, the HEs of ES99\% and VaR99\% are relatively more dispersed and skewed to the left.
Both risk measures consider only 1\% of the data from the left for deciding the hedge ratio and computing the HEs.
Considering only a few data points naturally leads to a less reliable hedge ratio and lower consistency HEs.
Evidence also shows that ES99\% VaR99\% minimising portfolios have higher MSE and LSV.
This result is again consistent with the daily rebalancing setting.


\begin{figure}[t]
  \begin{center}
    \includegraphics[width=.65\textwidth]{_pics/revision_BTC_MSEs.png}
  \end{center}
  \caption{Intraday out-of-sample MSEs of the BTC-BTCF portfolio constructed by combinations of copula and ris minimization objectives.
    The Frank copula is again inferior. Minimising ES99\% results in higher MSEs disregard of which copula is in use.
  \href{http://www.quantlet.com/}{\includegraphics[height=\baselineskip]{_pics/qletlogo_tr.png}}
\natp{\em [Graph is cut off. Make pdf or eps.]}}
\label{fig:BTC_MSE}
\end{figure}

\begin{figure}[t]
  \begin{center}
    \includegraphics[width=.65\textwidth]{_pics/revision_BTC_LSVs.png}
  \end{center}
  \caption{Intraday out-of-sample LSVs of the BTC-BTCF portfolio constructed by combinations of copula and ris minimization objectives.
    The Frank copula is again inferior. Minimising ES99\% results in higher MSEs disregard of which copula is in use.
  \href{http://www.quantlet.com/}{\includegraphics[height=\baselineskip]{_pics/qletlogo_tr.png}} }
\label{fig:BTC_LSV}
\end{figure}

Figures \ref{fig:BTC_MSE} and \ref{fig:BTC_LSV} report the
out-of-sample MSE and LSV of the BTC-BTCF when different copulae and
risk measures are in use. The MSE and LSV are in a similar magnitude
\natp{\em [than what?]} with a few exceptions, a similar pattern to the daily rebalancing
setting. \natp{\em [Fix sentence.]}

Across various copulae, the BTC-BTCF portfolio that minimises VaR95\%
provides the lowest MSE and LSV. 
Portfolios that minimise variance and ERM with $k=10$ result in similar
magnitudes of MSEs and LSVs \natp{\em [than what?]}, which are
slightly greater than VaR95\%, especially when Gumbel and NIG copulae
are in use to model the dependence. ES99\% generates the highest MSEs
and LSVs, regardless of the copula. \natp{(delete: Notice that the
  portfolios that 
minimise ES99\% and VaR99\% are generally riskier than their 95\%
counterparts in terms of MSEs and LSVs.)}

Across various risk measures, Gumbel and NIG copulae perform
well in the resulting portfolios’ MSE and LSV, except for
ES99\%. 
The Frank copula performs worst, regardless of the risk measure. 
These results are consistent with the daily rebalancing setting and
results in other literature. \natp{\em [Other literature? Perhaps
  better to delete this. Or be more specific.]}
As Gumbel and NIG are the only copulae that can model upper
tail dependence, this suggests that the upper tail dependence is an
essential feature of the dependence structure for hedging.
\natp{\em [Delete?]} The conclusion is further supported by comparing the Gumbel and the
rotated Gumbel copula. 
The rotated Gumbel copula is the 180-degree rotated version of the Gumbel copula, sharing all the features of the Gumbel copula but switching from modelling the lower tail dependence to upper tail dependence.
The rotated Gumbel copula results in portfolios with higher MSEs and
LSVs consistently across risk measures. 

\begin{table}[t]
 \ra{1.1}
    {\begin{tabularx}{\textwidth}{lYYYYY} \toprule
         Spot/ Copula & $t$ & Plackett & GMI & rotGumbel & NIG \\ \midrule
        \ \ \ BTC          & 60.00          & 1.11              & 3.33                        & 8.89                  & 26.67                  \\
        \ \ \ ETH          & 35.14          & 0                 & 24.32                       & 15.68                 & 24.86                   \\
    \bottomrule
    \end{tabularx}
       }









 \caption{Intraday copula selection results (shortened).
        The values are the percentage counts of a copula chosen by the AIC procedure during the out-of-sample period.
        The table shows only the frequently chosen copula, i.e. $t$, Plackett, Gaussian Mix Independent (GMI), rotated Gumbel (rotGumbel), and
        Normal Inverse Gaussian factor copula (NIG).
        }
    \label{tab:copulasection_intraday}
\end{table}


Table \ref{tab:copulasection_intraday} shows the relative frequencies
of the best fitting copula according to AIC.

\natp{\em [No need to repeat all the numbers in the text, as they are
  given in the table. Draw conclusions directly and mention only
  exceptional numbers.]}
They are $t$-, Plackett, Gaussian Mix Independent, rotated Gumbel and NIG.
Similar to the result in the daily rebalancing schedule, most of the time, 60\% in this case,
the AIC procedure chooses t-Copula to model the dependence structure of BTC-BTCF in the intraday setting.
For the rest of the time, the NIG copula is mainly chosen, accounting for around 26\% of the time.
Rotated Gumbel, Gaussian Mix Independent, and Plackett are spontaneously chosen.
On the other hand, the intraday ETH-BTCF’s AIC selection result is very different from that of the daily rebalancing.
There are three copulae: $t$-, Gaussian Mix Independent, and NIG copula, that are closely chosen, instead of a single copula, rotated Gumbel copula, dominating the list in the daily rebalancing setting.
In the intraday setting, the three copulae are chosen 35.1\%, 24.9\%, and 24.3\% of the time, respectively.



\end{document}
