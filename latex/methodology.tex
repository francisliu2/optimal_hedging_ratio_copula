\section{Optimal hedge ratio}
\label{sec:optimal-hedge-ratio}
We form a portfolio with two assets, a spot asset and a future contract, for example Bitcoin spot and CME Bitcoin future.
Our objective is to minimize the risk of the exposure in the spot.
To keep a simple portfolio setting, we long one unit of the spot and short $h$ unit of the future with $h \in [0, \infty)$.
Let $r^S$ and $r^F$ be the log returns of the spot and future price, the log return of the portfolio is
\[r^h = r^S -h r^F.\]
We call this portfolio a hedged portfolio: the price movement of spot is hedged by the price movement of future.
\medskip

%\footnote{The log return is simply the log of geometric difference between price at time $t$ and $t-1$ $\log \frac{\text{Price}_t}{\text{Price}_{t-1}}$.
%The index $t$ follow the trading calendar.
%The handling of calendar is important, we will discuss that in the data section.}


Risk is measured by risk measures.
Assume the payoff $r^h$ of a portfolio lives in a probability space, $r^h \in L(\Omega,\F, \P)$,
and there is a risk measure on $r^h$ $\rho: r^h \mapsto \mathbb{R}$.
We are looking for an optimal hedge ratio $h^*$ which minimizes risk measure

\[h^* = \argmin_h \rho(r^h).\]

Most risk measures are defined as functionals of the portfolio loss distribution $F_{r^h}$, i.e. $\rho: F_{r^h} \mapsto \mathbb{R}$.
For example, Value-at-Risk (VaR) is simply the quantile of $r^h$ multiply with negative one $\text{VaR}_{1-\alpha} = -F_{r^h}^{(-1)}(1-\alpha) = -\inf\{x \in \mathbb{R}: 1-\alpha \leq F_{r^h}(x) \}$, where $\alpha$ is a parameter chosen by investor.
We need the knowledge of $F_{r^h}$ in order to measure risk.
By convolution of random variables~\citep{WKHMVA}, $f_{r^h}(z) = \int_{-\infty}^{\infty}f_{r^S, -hr^F}(x, z-x)dx$, where
$f_{r^S, -hr^F}$ is the joint pdf of $r^S$ and $-hr^F$.
Obviously the cdf of $r^h$ and risk measure depend on the joint distribution of $r^S$ and $-hr^F$.\medskip

Optimising $h$ according to $f_{r^S,-hr^F}$ is unfavorable in a sense that one needs to calibrate a new joint pdf $f_{r^S, -hr^F}$ when updating $h$.
This is too time consuming and unnecessary.
Another problem of using joint pdf is that one lacks of flexibility to model the margins.
A joint pdf completely determine the form of its marginals, for example, margins of a bivariate $t$-distribution are
univariate $t$-distributions.\medskip

To overcome the problems, we use copulae.
The benefit of using copulae is two folded.
First, copulae allow us to model the margins and dependence structure separately, see Sklar's Theorem.
Second, copulae are invariance under strictly monotone increasing function \citep{schweizer1981nonparametric}, see lemma below.

\begin{theorem}[Hoeffding Sklar Theorem]
  Let $F$ be a joint distribution function with margins $F_X,
  F_Y$. Then, there exists a copula $C:[0,1]^2 \mapsto [0,1]$ such
  that, for all $x,y\in \R$
  \begin{equation}
    \label{eq:4}
    F(x,y)=C\{F_X(x), F_Y(y)\}.
  \end{equation}
  If the margins are continuous, then $C$ is unique; otherwise $C$ is
  unique on the range of the margins.

  Conversely, if $C$ is a copula and $F_X, F_Y$ are univariate
  distribution functions, then the function $F$ defined by (\ref{eq:4})
  is a joint distribution function with margins $F_X, F_Y$.
\end{theorem}

Indeed, many basic results about copulae can be traced back to early works of Wassily Hoeffding \citep{hoedffding1940, hoedffding1941}.
The works aimed to derive a measure of relationship of variables which is invariant under change of scale.
Readers can refer to \citet{hoeffding2012collected} for English translations of the works.

\begin{lemma}
  \begin{align}
  C_{X, hY}\{F_X(s),F_{hY}(t)\} = C_{X, Y}\{F_X(s),F_{Y}(t/h)\}.
    \end{align}
  \end{lemma}


%The optimal hedge ratio is $h^\ast = \argmin_h \rho(Z)$, that is the best ratio that can minimize the risk of a hedged portfolio measured in terms of $\rho$ . \medskip
Leveraging the two features of copulae, \citet{barbi2014copula} introduces the distribution of linear combination of random variables using copulae.
We slightly edit the Corollary 2.1 of their work and yield the following correct expression of the distribution.

\begin{proposition}
  \label{prop:dfrh}
  Let $X$ and $Y$ be two real-valued continuous random
  variables on a
  probability space $(\Omega, \F, \p)$ with
  absolutely continuous copula $C_{X, Y}$ and marginal distribution functions $F_{X}$
  and $F_{Y}$. Then, the distribution function of of $Z$ is given by 
  \begin{equation}
    \label{eq:3}
    F_{Z}(z) = 1- \int^1_0 D_1 C_{X, Y}
    \left[ u, F_{Y} \left\{ \frac{F^{(-1)}_{X}(u)-z}{h} \right\}
    \right]\, d u.
  \end{equation}
\end{proposition}
Here, $F^{(-1)}$ denotes the inverse of $F$, i.e., the quantile
function. \medskip

Here $D_1 C(u,v)=\displaystyle \frac{\partial}{\partial u} C(u,v)$ see e.g.\ Equation (5.15) of
\citep{McNeil2005}:
\begin{equation}
  \label{eq:1}
  D_1 C_{X,Y}(F_X(x), F_Y(y)) = \p(Y\leq y|X=x).
\end{equation}
\begin{proof}
  Using the identity (\ref{eq:1}) gives
  \begin{align*}
    F_{Z}(z) &= \p(X - h Y\leq z) %
                 = \mbox{\sf E}\left\{\p\left(Y\geq \frac{X-z}{h}\Big|
                 X\right)\right\}\\[10pt]
               &= 1-\mbox{\sf E}\left\{\p\left(Y\leq \frac{X-z}{h}\Big|
                 X\right)\right\}% \\[10pt]
               = 1- \int_0^1 D_1 C_{X, Y}\left[u,
                 F_{Y}\left\{\frac{F^{(-1)}_{X}(u) -
                 z}{h}\right\}\right]\, d u.
  \end{align*}
  \end{proof}

%In addition to~\cite{barbi2014copula} we propose a more handy
%expression for the pdf of $Z$.
%\natp{\em [Please double-check the ``+'' signs in the second equation.]}\ \francis{ \em [the + sign is correct.]}

\begin{corollary} Given the formulation of the above portfolio, the pdf of $Z$ can be written as
  \begin{align}
  f_{Z}(z) &= \left|\frac{1}{h}\right|\int_0^1 c_{X, Y} \left[
  F_{Y}\left\{\frac{F^{(-1)}_{X}(u)-z}{h}\right\}, u
  \right]
   \cdot
  f_{Y}
  \left\{\frac{F^{(-1)}_{X}(u)-z}{h}\right\} du \label{eq:density1}
  \end{align}, or
    \begin{align}
      f_{Z}(z)
      = \int_0^1 c_{X, Y} \left[
      F_{X}\left\{z + h F^{(-1)}_{Y}(u)\right\}, u
      \right]
       \cdot
      f_{X}
      \left\{
      z+ hF^{(-1)}_{Y}(u)
      \right\} du. \label{eq:density2}
  \end{align}
  \end{corollary}
The two expression are equivalent.
Notice that the pdf of $Z$ in the above proposition is readily accessible as long as we have
the copula density and the marginal densities.
The proof and a generic expression can be found in the
appendix. \medskip

%In this work, we consider two portfolios: $r^h = R^{\text{BTC}} - h R^{\text{future}}$ and $r^h = R^{\text{BTC}} - h R^{\text{CRIX}}$.
%
%\natp{\em [For now, stick to $r_S$ and $r_F$. This is still the
%  general part, so no need to mention Bitcoin, CRIX, etc. here. Also,
%  wasn't the idea to hedge CRIX with the future? I don't see why CRIX
%  would be used to hedge Bitcoin when futures are readily
%  available. Entering into futures contracts requires no notional,
%  which makes them ideal for hedging.]}

%OLD
%\section{Methodology}\label{sec:methodology}
%Following \citet{barbi2014copula}, we consider the problem of optimal
%hedge ratios by extending the commonly known minimum variance hedge
%ratio to more general risk measures and dependence
%structures.\medskip
%
%Hedge portfolio: $R_t^h = R_t^S - h R_t^F$, involving returns of spot
%and future contract and where $h$ is the hedge ratio.\medskip
%
%The optimal hedge ratio is
%\begin{align}
%    h^\ast = \argmin_h \rho_\phi(s,h),
%    \end{align}
%for given
%confidence level $1-s$ (if applicable, e.g.\ in the case of VaR, ES),
%where $\rho_\phi$ is a spectral risk measure with weighting function
%$\phi$ (see below).
%In other words, our task is to search for the optimal $h$ which can minimize a particular risk measure.
%We call the risk measure being used in search process of $h^\ast$ risk reduction objective.
%This naming is to differentiate the risk objective and risk outcome.
%One can see from result section that the $h^\ast$ which minimize a particular risk measure in training does not
%necessarily minimize the risk measure in testing data.
%For example in table \ref{OOSRHVaR99}, the best performing risk reduction objective to reduce out-of-sample Value-at-Risk 99\% is
%exponential risk measure $k=10$. \medskip
%
%The distribution function of $Z$ can be expressed in terms of the
%copula and the marginal distributions as Proposition \ref{prop:dfrh}
%result shows (this is a corrected version of Corollary 2.1 of
%\citep{barbi2014copula}). For practical applications, it is numerically
%faster and more stable to use additional information about the
%specific copula and marginal distributions. We therefore derive
%semi-analytic formulas for a number of special cases, such as the
%Gaussian-, Student $t$-, normal inverse Gaussian (NIG) and Archimedean
%copulas in Section \ref{sec:dependence}.
%
%\begin{proposition}
%  \label{prop:dfrh}
%  Let $X$ and $Y$ be two real-valued random variables on the same
%  probability space $(\Omega, \mathcal A, p)$ with corresponding
%  absolutely continuous copula $C_{X, Y}(w,\lambda)$ and
%  continuous marginals $F_{X}$ and $F_{Y}$. Then, the distribution
%  of of $Z$ is given by
%  \begin{equation}
%    \label{eq:3}
%    F_{Z}(x) = 1- \int^1_0 D_1 C_{X, Y}
%    \left( u, F_{Y} \left( \frac{F^{(-1)}_{X}(u)-x}{h} \right)
%    \right)\, d u.
%  \end{equation}
%\end{proposition}\medskip
%Here $D_1 C(u,v)=\displaystyle \frac{\partial}{\partial u} C(u,v)$,
%which is easily shown to fulfil, see e.g.\ Equation (5.15) of
%\citep{McNeil2005}:\footnote{%
%  Let $F_X(x)=u$, $F_Y(y)=v$. Then, formally,
%  \begin{align*}
%    \frac{\partial}{\partial F_X(x)} C(F_X(x), F_Y(y)) %
%    &= \frac{\partial}{\partial F_X(x)} \p(U\leq F_X(x),
%      V\leq F_Y(y)) %
%      = \p(U\in d F_X(x), V\leq F_Y(y))\\ %
%    &= \p(V\leq F_Y(y)| U = F_X(x))\cdot \p(U \in d
%      F_X(x)) %
%      = \p(Y\leq y|X=x)\cdot \p(U\in d u)\\ %
%    &= \p(Y\leq y|X=x).
%  \end{align*}}
%\begin{equation}
%  \label{eq:1}
%  D_1 C_{X,Y}(F_X(x), F_Y(y)) = \p(Y\leq y|X=x).
%\end{equation}
%\begin{proof}
%  Using the identity (\ref{eq:1}) gives
%  \begin{align*}
%    F_{Z}(x) &= \p(r^S - h Y\leq x) %
%                 = \E\left[\p\left(Y\geq \frac{X-x}{h}\Big|
%                 X\right)\right]\\[10pt]
%               &= 1-\E\left[\p\left(Y\leq \frac{X-x}{h}\Big|
%                 X\right)\right]% \\[10pt]
%               = 1- \int_0^1 D_1 C_{X, Y}\left(u,
%                 F_{Y}\left(\frac{F^{(-1)}_{X}(u) -
%                 x}{h}\right)\right)\, d u.
%  \end{align*}
%\end{proof}\medskip
%
%In addition to \cite{barbi2014copula} we propose an expression for the density of $Z$
%
%\begin{proposition} With the same setting of the above proposition, the density of $Z$ can be written as
%  \begin{align}
%  f_{Z}(y) &= \left|\frac{1}{h}\right|\int_0^1 c_{X, Y} \left[u,
%  F_{Y}\left\{\frac{F^{(-1)}_{X}(u)-y}{h}\right\}
%  \right]
%   \cdot
%  f_{Y}
%  \left\{\frac{F^{(-1)}_{X}(u)-y}{h}\right\} du \label{eq:density1}
%  \end{align}, or
%    \begin{align}
%      f_{Z}(y)
%      = \int_0^1 c_{X, Y} \left[u,
%      F_{X}\left\{y + h F^{(-1)}_{Y}(u)\right\}
%      \right]
%       \cdot
%      f_{X}
%      \left\{
%      y+ hF^{(-1)}_{Y}(u)
%      \right\} du.\label{eq:density2}
%  \end{align}
%  \end{proposition}
