\section{Empirical Results}\label{sec:results}

\subsection{Data}\label{subsec:data}
In the empirical analysis, we consider the risk reduction capability
of CME Bitcoin Futures (BTCF) on five cryptos, namely Bitcoin (BTC), Ethereum
(ETH), Cardano (ADA), Litecoin (LTC) and Ripple (XRP), as well as five
crypto indexes, namely BITX, BITW100, CRIX, BITW20, and BITW70.
ETH, ADA, LTC, and XRP are popular cryptos tradable in various
exchanges and have large market capitalization. 
BITX, BITW100, and CRIX are market-cap weighted crypto indexes with
BTC as constituent. 
BITX and BITW100 track the total return of the 10 and 100 cryptos
with largest market-cap, respectively. 
CRIX decides the number of constituents by AIC and tracks that number
of cryptos with largest market-cap. In our case, the number of
constituents in CRIX is 5. 
BITW20 is also a market-cap weighted crypto index but with the 20
largest market-cap cryptos outside the constituents of BITX.
BITW70 has the same construction as BITW20 but with the 70 largest
market-cap cryptos outside BITX and BITW20. 
Therefore, BTC is excluded as a constituent in BITW20 and BITW70.

For each of the 10 hedging portfolios, a crypto or index is considered
as the spot and held in a unit size long position, while 
the front BTCF is held in a short position with units corresponding to
the OHR in order to reduce the risk of the spot. 
Except for the hedge of BTC, all hedging portfolios are considered to
be cross-asset hedges. 

We collect the spots' and BTCF's daily prices at 15:00 US Central Time
(CT). The reason for choosing this particular time is that the CME
group determines the daily settlements for BTCF's based on the trading
activities on CME Globex between 14:59 and 15:00 CT. This is also the
reporting time of the daily closing price by Bloomberg. 
The crypto spot data is collected from the data provider called
Tiingo (\href{https://www.tiingo.com/}{https://www.tiingo.com/}).
\natp{\em [thanks somewhere.]}
Tiingo aggregates crypto OHLC (open, high, low, and close) prices fed
by APIs from various exchanges. It covers major exchanges, such as
Binance, Gemini, Poloniex etc., so Tiingo's aggregated OHLC price is a
good representation a tradable market price. 
For each crypto, we match the opening price at 15:00 CT from Tiingo
with the daily BTCF closing price from Bloomberg.
Since CRIX is not available at 15:00 CT, we recalculated an hourly
CRIX using the monthly constituents weights and the hourly OHLC price
data collected from Tiingo. 
BITX, BITW20, BITW70, and BITW100 are collected from the official
website of their publisher Bitwise.com. 
The daily reporting time of the Bitwise indexes is 15:00 CT.

At the time of writing, the CRIX is undergoing the listing process on
the S\&P Dow Jones Indices, the official CRIX data will then be
calculated with Lukka Prime Data and available to the public via S\&P.

\subsection{Overview of the out-of-sample data}\label{subsec:oosdata}

For every asset and hedge portfolio, we concatenate the out-of-sample data to form a time series for analysis.
The date range of the out-of-sample time series is from 2019-10-21 to 2021-05-27, in total of 405 data points in each time series.
We analyse these time series throughout the whole result section. \medskip

We introduce the out-of-sample data in this subsection before we proceed to analysing the hedged portfolio results.
Figure~\ref{fig:BTC_price} presents the BTC and BTCF price in USD in the first panel and the arithmetic difference between the daily return of BTC and BTCF, i.e. $R_s - R_f$, in the
second panel.
In the first panel, the black vertical lines with capital letter labels indicate the days of the five most negative daily return of BTC during out-pf-sample period.
Table \ref{tab:BTC_5min} summarizes the relevant news headlines and events of those days. \medskip

Figures~\ref{fig:index_price} and \ref{fig:individualCoins_price}
\natp{\em [swap references to figures? So it's Figures 3 and 4 ...]} are the cumulative returns of the indices and individual cryptos respectively.
The black vertical lines labeled by assets name are the largest daily price drop of the assets in the out-of-sample data. \medskip

The out-of-sample data covers the pre-COVID19 period, 2019-10-21 to 2020-03-09, as well as the COVID19 period, 2019-03-19 onwards.
We can observe an overall upward trend of crypto prices in both periods.
Nonetheless, the volatilities of assets are high (annualized around 100\%) regardless of COVID19.

%\newpage
\begin{figure}[!t]
\includegraphics[width=\textwidth]{_pics/BTC_price.pdf}
  \caption{Out-of-sample BTC and BTCF price. The first panel presents the price of BTC in blue line and that of BTCF in orange line.
  The black vertical lines with capital letter labels indicate the five most negative daily return of BTC in the out-of-sample data.
  The second panel presents the difference between the \% return of BTC and BTCF.
  The black vertical lines with lowercase letter label indicate the five most negative returns.
  The crosses locate the level the returns.
  \href{http://www.quantlet.com/}{\includegraphics[height=\baselineskip]{_pics/qletlogo_tr.png}} }
\label{fig:BTC_price}
\end{figure}

\begin{table}[!]
    \centering
      \begin{tabularx}{.8\textwidth}{cccX}
        \toprule
        Label &   Date & \% Drop in Price &  Summary\\
        \midrule
        A &  2020-03-09 & 13.83 &  Coronavirus outbreak that affect the global markets; BTC as potential safe-haven was questioned.$^1$\\
        B &  2020-03-12 & 22.89 &  Continuation of the 2020-03-09 drop.  \\
        C &  2020-05-11 & 12.11 &  Price correction (from \$10,000 to \$8,100) after BTC price surge because of the third supply halving.$^{2,3}$ \\
        D &  2021-01-11 & 14.41 &  Short term correction of BTC hits the \$40,000 mark.$^4$\\
        E &  2021-05-17 & 11.86 &  Tesla stopped taking BTC as payment due to environmental concerns about energy use to process transaction.$^5$\\
        \bottomrule
      \end{tabularx}
        \caption{Summary of events that associated with the five most negative daily price drops in out-of-sample BTC price data.
        The capital letter labels in the first column are the labels in the first panel of figure~\ref{fig:BTC_price}.
        $^1$ is reported by the CNBC news \url{https://cnb.cx/3HZ2x7K}; $^2$ is from Forbes \url{https://bit.ly/3rdJPmP};
        $^3$ is from livemint.com \url{https://bit.ly/3FRi6Na};
        $^4$ is from CNBC \url{https://cnb.cx/3nU0ppO};
        $^5$ is from Reuters \url{https://reut.rs/3leCiAv}.
        }
        \label{tab:BTC_5min}
  \end{table}

%\clearpage

%\newpage
\begin{figure}[t]
\includegraphics[width=\textwidth]{_pics/index_price.pdf}
  \caption{Out-of-sample cumulative return of crypto indices.
  The black vertical lines indicate largest price drop of indices indicated by the labels.
  \href{http://www.quantlet.com/}{\includegraphics[height=\baselineskip]{_pics/qletlogo_tr.png}} }
\label{fig:index_price}
\end{figure}

\begin{figure}[!]
\includegraphics[width=\textwidth]{_pics/individualCoins_price.pdf}
  \caption{Out-of-sample cumulative return of individual cyptos.
  The black vertical lines indicate largest price drop of cryptos indicated by the labels.
  \href{http://www.quantlet.com/}{\includegraphics[height=\baselineskip]{_pics/qletlogo_tr.png}} }
\label{fig:individualCoins_price}
\end{figure}

\begin{table}[!]
    \centering
      \begin{tabularx}{.8\textwidth}{cccX}
        \toprule
        Label &  Date & \% Drop in Price &  Summary\\
        \midrule
        CRIX    &2020-03-09 & 23.77 & \multirow[t]{6}{\hsize}{Coronavirus outbreak that affect the global markets including the crpyto market.}\\
        BITX    & & 23.68 &  \\
        BITW100 & & 23.87 &  \\
        BITW20  & & 26.66 &  \\
        ADA     & &23.55 &  \\
        ETH     & &27.40 &  \\
        BITW70  & 2021-05-19& 27.64 & The spillover of the BTC shock on 2021-05-17 (label A in figure~\ref{fig:BTC_price} and table~\ref{tab:BTC_5min})\\
        XRP     & 2020-12-23 & 41.00 & Top executives were sued by the SEC of misleading investors$^1$. \\
        \bottomrule
      \end{tabularx}
        \caption{Summary of events that associated with largest price drops in out-of-sample data.
        The labels in the first column are the labels in figure \ref{fig:individualCoins_price} and figure \ref{fig:index_price}.
        CRIX, BITX, BITW100, BITW20, ADA and ETH have the same date the reason of the largest drop. $^1$ is reported by Bloomberg \url{https://bloom.bg/3cWdita}.}
        \label{tab:All_min}
  \end{table}
%\clearpage


%%% Local Variables:
%%% mode: latex
%%% TeX-master: "SRM"
%%% End:


\subsection{An overview of the hedged portfolios without the copula selection step}\label{subsec:HP1}
First, we analyse the result of hedged portfolios without the copula selection step in order to get a better understanding of how a copula affect the hedged portfolio with various risk minimization objectives.
To do so, we inspect the performance of copula in hedging by the mean square error and lowersemi variance.
Mean square error is the distance between a perfect hedge and the hedged portfolio returns $\operatorname{MSE}=\frac{1}{n}\sum_{i=1}^n(r^h_i)^2$.
Lower semi variance is $\operatorname{LSV}=\frac{1}{n}\sum_{i=1}^n\{\operatorname{E}(r^h)-r^h_i\}^2$.
All results here are out-of-sample results obtained without the copula selection step in order to compare the performances across copulae.  \medskip

\begin{figure}[th]
    \centering
    \includegraphics[width=\textwidth]{_pics/MSE_BTC.pdf}
  \caption{Mean square errors of BTC-BTCF portfolios constructed with different copula and risk minimization objectives.
    The Frank copula is inferior in the BTC-involved portfolios.
    \href{http://www.quantlet.com/}{\includegraphics[height=\baselineskip]{_pics/qletlogo_tr.png}} }
\label{fig:MSE_BTC}
\end{figure}

\begin{figure}[th]
    \centering
    \includegraphics[width=\textwidth]{_pics/semiLowerVariance_BTC.pdf}
  \caption{Lower semivariance of BTC-BTCF portfolios constructed with different copula and risk minimization objectives.
  The Frank copula is obviously inferior.
  \href{http://www.quantlet.com/}{\includegraphics[height=\baselineskip]{_pics/qletlogo_tr.png}} }
\label{fig:SLV_BTC}
\end{figure}

\begin{figure}[th]
    \centering
    \includegraphics[width=\textwidth]{_pics/MSE_other.pdf}
  \caption{Mean square errors of portfolios constructed with different copula and risk minimization objectives.
  \href{http://www.quantlet.com/}{\includegraphics[height=\baselineskip]{_pics/qletlogo_tr.png}} }
\label{fig:MSE_other}
\end{figure}

\begin{figure}[th]
    \centering
    \includegraphics[width=\textwidth]{_pics/semiLowerVariance_other.pdf}
  \caption{Lower semivariance of portfolios constructed with different copula and risk minimization objectives.
  \href{http://www.quantlet.com/}{\includegraphics[height=\baselineskip]{_pics/qletlogo_tr.png}} }
\label{fig:SLV_other}
\end{figure}

\textcolor{darkblue}{As presented in Fig 3 and 4, either individual cryptos or indice, their cumulative returns dropped in Mar 2020. it's due to the result of COVID19. we can explain this for these two plots.}\\
\textcolor{darkblue}{Here i think it should insert a paragraph to interpret how you enter the copulae, otherwise it's weird that comes to Fig 5 and 6.}\\

Figure \ref{fig:MSE_BTC} and \ref{fig:SLV_BTC} are the mean square error and lower semivariance of BTC-BTCF, we can see the Frank copula is the worst performing copula:
the resulting hedged portfolio returns is far away from a perfect hedge.
In figure \ref{fig:MSE_other} and \ref{fig:SLV_other}, the phenomena of Frank copula being inferior to its counterparts can be observed from the results of the CRIX, BITX, BITW100, and BITW20-BTCF portfolios.
Interestingly, the spot in those portfolios usually have a strong dependency to the BTCF.
In contrast, the inferiority of the Frank copula is less prominent in the BITW70, ADA, ETH, LTC and XRP-BTCF portfolios.
We suspect that the Frank copula is not a choice to model assets with strong dependency. \textcolor{darkblue}{Frank copula is not appropriate for data that exhibit asymmetric and heavy tails.}  \medskip

We can also observe from figure \ref{fig:MSE_other} and \ref{fig:SLV_other} that Gumbel copula is not performing as good as other copula in the ETH, LTC, and XRP-BTCF portfolios.
The reason is the Gumbel copula has only the upper tail dependence, while the ETH, LTC, and XRP exhibit lower tail dependency with BTCF.
We will discuss this in the following section. \medskip

\subsection{Copula Selection Results}\label{sec: copula results}

\natp{\em [Table 8 on the last page is ``dangling''. Why is it not in
  the text? And why is it not referenced? Also, I suggest to resize it
  to a normal font size.]}

\textcolor{darkblue}{Interpret the steps of copula selection.} \\
Next, we inspect the copula selection result.\textcolor{darkblue}{(I delete "in this section"...)}
Although the copula selection is only an intermediate step to obtain the OHRs,
the result of this step can help us better understand the dependence feature between BTCF and the assets we study in this work.
This gives us valuable information to model the assets in the future.
Decisions of the AIC procedure are summarised in table \ref{tab:copulasection}. \medskip

Overall, $t$-copula, rotated Gumbel (rotGumbel), and the NIG factor copula are the most frequently chosen copulae by the AIC procedure. \medskip

The $t$-copula is frequently chosen by AIC to model the dependency between the BTC and BTC involved indices, CRIX, BITX, BITW100, and the BTC future.
BTC and BTC involved indices exhibit strong tail dependence (both upper and lower tail) with BTCF.
We could interpret tail dependence much more of a tendency for one asset to be extreme when another is extreme and vice versa \citep{McNeil2015}.
In fact, the $t$ copula has been suggested in various empirical studies to model financial data, such as \cite{zeevi2002beyond} and \cite{breymann2003dependence}.
Those studies suggest $t$-copula is a better model over the Gaussian copula which financial data often seem to exhibit tail dependence. \textcolor{darkblue}{(because the thick and left skew properties of financial data tail distribution are documented. )} \medskip

On the other hand, the radially symmetric feature makes the $t$-copula not a good choice to model the other hedging pairs.
\cite{demarta2005t} describe the symmetry feature "strong", because if $(U_1, ..., U_d)$ is a vector distributed in $t$-copula,
then $(U_1, ..., U_d) \overset{\mathcal{L}}= (1-U_1, ..., 1-U_d)$.
This symmetry can be justified in the dependence structure between a futures and its underlying by the theory of futures pricing,
which suggests the price of a futures is a function of the underlying price \citep{hull2003options}.
However, there is no such relationship between a futures and an asset which is not the underlying, and so the radial symmetry becomes a drawback to model other hedging pairs e.g. ETH and BITX70.
Another drawback of the $t$-copula is the lack of flexibility to model off-diagonal region since $\rho$ and $\nu$ jointly control the density of the off-diagonal region.
%The off-diagonal region (HF paper breymann2003dependence)
This is why sometimes the Gaussian Mix Independence (GMI) better model the dependence.  \medskip

Among the three popular copulae, rotGumbel copula shows its ability to model the dependency between ETH and BTCF,
94 out of 112 training sets are best fitted with the rotated Gumbel.
rotGumbel also performs well when modelling dependency between XRP, BITW20, BITW70, and the BTCF.
In particular, the whole time series of the two indices, BITW20 and BITW70, are best fitted solely with the rotated Gumbel copula.
The frequently chosen rotated Gumbel indicates the styled fact of financial data: prices tends to drop together.  \medskip

In fact, Clayton's AIC in many of the training sets is the second lowest, just higher than that of rotated Gumbel.
This is because the Clayton copula has the same ability to model the lower quantile dependence.
However, Clayton's radial like feature does not match the behaviour of the financial data. \medskip

It is worth to mention that although the NIG factor copula is penalised heavily due to its three parameters setup,
it is frequently chosen to be the best copula to model the dependency between individual cryptos and the BTC future.
An extreme case would be ADA, only NIG factor is chosen in our dataset.
Another dependency structure being best described by the NIG factor copula is the pair of LTC-BTCF.
64 out of 112 training sets are best fitted by the NIG factor copula.
Indices like BITX and CRIX are sometimes best fitted with the NIG factor copula as well, accounting for modelling 12 and 27 training sets respectively.
The popularity of the NIG factor reflects the ability of the copula to model very complex dependency structure.
NIG factor copula is able to model the tail, radial asymmetry, and off diagonal behaviour.  \medskip %(ADA samples and fitted NIG samples)

Frank copula is generally not a good choice to model financial data just like what \cite{barbi2014copula} has reported.
Plackett is known for its dependence parameter can be written as the cross-product ratio \citep{joe1997multivariate}.
However, this feature does not bring the Plackett Copula advantage over other copulae to model the dependence structure between cryptos and BTCF. \medskip

\begin{table}[t!]
 \ra{1.1}
    {\begin{tabularx}{\textwidth}{lYYYYY} \toprule
         Spot/ Copula & $t$ & Plackett & GMI & rotGumbel & NIG \\ \midrule
     \multicolumn{6}{l}{Individual Cryptos}                                                                                 \\
        \ \ \ BTC          & 65.18      & \phantom{0}3.57              & \phantom{0}0.89                     & \phantom{0}0.89               & 27.68                  \\
        \ \ \ ETH          & \phantom{0}2.68       & \phantom{0}5.36              & \phantom{0}7.14                     & 83.93              & \phantom{0}0.89                   \\
        \ \ \ ADA          & \phantom{0}0.00       & \phantom{0}0.00              & \phantom{0}0.00                     & \phantom{0}0.00               & 100.00\phantom{0}                 \\
        \ \ \ LTC          & 11.61      & \phantom{0}0.00              & \phantom{0}2.68                     & 28.57              & 57.14                  \\
        \ \ \ XRP          & \phantom{0}0.00       & 27.68             & \phantom{0}2.68                     & 69.64              & \phantom{0}0.00                   \\
   \multicolumn{6}{l}{Crypto Indices with BTC Constituent}                                                                  \\
        \ \ \ BITX         & 48.15      & \phantom{0}0.00               & 17.28                    & 19.75              & 14.81                  \\
        \ \ \ CRIX         & 53.41      & \phantom{0}0.00               & 12.50                    & \phantom{0}3.41               & 30.68                  \\
        \ \ \ BITW100      & 51.85      & \phantom{0}0.00               & 9.88                     & 35.80              & \phantom{0}2.47                   \\
    \multicolumn{6}{l}{Crypto Indices without BTC Constituent}                                                              \\
        \ \ \ BITW20       & \phantom{0}0.00        & \phantom{0}0.00               & \phantom{0}0.00                      & 96.30              & \phantom{0}3.70                   \\
        \ \ \ BITW70       & \phantom{0}0.00        & \phantom{0}0.00               & \phantom{0}0.00                        & 98.77              & \phantom{0}1.23                  \\
    \bottomrule
    \end{tabularx}
       }









%\begin{table}[th] \centering
\resizebox{\textwidth}{!}{%
\begin{tabular}{lrrrrrr}
\toprule
         {} &    Mean \% &     Std \% &      Skew &       Kurt &         MD \% &     MD date \\
\ \ \ BTCF    &    0.39 &   4.63 & -0.51 &   4.42 & -26.99 &  2020-03-12 \\
     \multicolumn{7}{l}{Hedging Instrument} \\
\ \ \ BTC     &    0.39 &   4.40 & -0.59 &   4.66 & -26.00 &  2020-03-12 \\
\ \ \ ETH     &    0.68 &   6.01 & -0.26 &   5.26 & -32.01 &  2020-03-12 \\
\ \ \ ADA     &    0.95 &   6.70 &  0.17 &   2.31 & -26.85 &  2020-03-12 \\
\ \ \ LTC     &    0.32 &   6.48 & -0.99 &   5.30 & -37.59 &  2021-05-19 \\
\ \ \ XRP     &    0.30 &   7.98 &  0.55 &  12.49 & -52.77 &  2020-12-23 \\
\ \ \ LTC     &    0.32 &   6.48 & -0.99 &   5.30 & -37.59 &  2021-05-19\\
\ \ \ BITX    &    0.21 &   4.74 & -0.55 &   4.04 & -27.02 &  2020-03-12 \\
\ \ \ CRIX    &    0.22 &   4.70 & -0.49 &   4.16 & -27.14 &  2020-03-12 \\
\ \ \ BITW100 &    0.22 &   4.76 & -0.63 &   4.03 & -27.27 &  2020-03-12 \\
\ \ \ CRIX    &    0.22 &   4.70 & -0.49 &   4.16 & -27.14 &  2020-03-12\\
\ \ \ BITW20  &    0.18 &   5.50 & -0.90 &   3.72 & -31.01 &  2020-03-12 \\
\ \ \ BITW70  &    0.17 &   5.85 & -0.96 &   4.11 & -32.35 &  2021-05-19 \\
\ \ \ BITW20  &    0.18 &   5.50 & -0.90 &   3.72 & -31.01 &  2020-03-12\\
\ \ \ BITW70  &    0.17 &   5.85 & -0.96 &   4.11 & -32.35 &  2021-05-19\\
\bottomrule
\end{tabular}}
\caption{BTCF's Hedging Effectiveness to various assets generated by variance as risk reduction objectives.}
\label{tab:varHE}

\end{table}
    \label{tab:copulasection}}
\end{table}

\subsection{Hedged portfolios with the copula selection step}\label{subsec:HP2}
We analyse hedged portfolios in this section.
Table~\ref{tab:ERM_rh} to table~\ref{tab:var_rh} summarise the statistics of daily returns of hedged portfolios. \medskip

The tables look repeating: for each hedged portfolios,
the first four moments as well as the maximum drawdown (MD) and the date of MD are very similar across different risk minimization objectives.
This is because the optimal hedge ratios of different risk minimization objectives fall into a small range. \medskip

On the other hand, the statistics vary across portfolios.
Unsurprisingly, the BTC-involved spots, i.e. BTC, CRIX, BITX, and BITW100, are well hedged by the BTCF regardless of risk minimization objective. \medskip

Contrarily, BTC-not-involved spots' portfolios are less promising.
Th hedged portfolio returns are as volatile as the spots.
We will further discuss the effectiveness of hedge in the next section. %\ref{sec: HE results}.

%\begin{table}[t]
\resizebox{\textwidth}{!}{%
%\tiny
\begin{tabular}{lrrrrrr}
\toprule
{} &    Mean ($10^{-3}$) &      Std ($10^{-3}$) &    Skew &    Kurt &       LQ ($10^{-3}$)&        MD ($10^{-3}$)\\
\midrule
BTC     &  0.1948 &   3.9138 & -0.8949 &  11.867 &  -1.2785 &  31.8969 \\
CRIX    &  0.8048 &   8.8261 &  0.0677 &  14.939 &  -3.6619 &  70.5299 \\
BITX    &  0.5896 &  10.0776 & -0.4427 &  13.084 &  -3.5729 &  78.5815 \\
BITW100 &  0.8526 &  12.0317 & -1.6522 &  20.556 &  -4.2111 & 111.8459 \\
BITW20  &  2.5642 &  36.0091 & -0.3446 &   4.215 & -13.5096 & 215.9201 \\
BITW70  &  2.8181 &  39.0745 & -0.6952 &   4.874 & -16.0984 & 245.2498 \\
ADA     &  1.3983 &  54.0242 &  1.0344 &   4.201 & -26.5629 & 194.0610 \\
ETH     &  1.0541 &  37.9113 &  0.8743 &   6.759 & -17.0680 & 188.7288 \\
LTC     & -2.0647 &  42.7759 & -0.0729 &   5.628 & -22.2628 & 280.8793 \\
XRP     & -1.7083 &  64.9295 &  1.1923 &  21.932 & -22.8506 & 525.6895 \\
\bottomrule
\end{tabular}}

%\end{table}
\begin{table}[!] \centering \resizebox{\textwidth}{!}{%
\begin{tabular}{lrrrrrrr} \toprule
         {} &    Mean \% &     Std \% &      Skew &       Kurt &         MD \% &     MD date & ES 5\% \\
\midrule
     \multicolumn{7}{l}{Individual Cryptos}                                                                                 \\
\ \ \ BTC     &  0.0204 &  0.3234 & -1.0150 &   3.4423 &  -1.5629 &  2020-11-30 &  0.0101 \\
\ \ \ ETH     &  0.3082 &  3.8890 &  1.0119 &   7.4077 & -18.7819 &  2021-05-19 &  0.0782 \\
\ \ \ ADA     &  0.5525 &  5.2673 &  1.2557 &   4.2423 & -14.9647 &  2021-05-19 &  0.0984 \\
\ \ \ LTC     & -0.0808 &  3.9829 & -0.4957 &   7.2302 & -28.4608 &  2021-05-19 &  0.0962 \\
\ \ \ XRP     &  0.0176 &  7.1533 &  1.1411 &  19.9176 & -52.5698 &  2020-12-23 &  0.1354 \\
   \multicolumn{7}{l}{Crypto Indices with BTC Constituent}                                                                  \\
\ \ \ BITX    &  0.0591 &  1.0065 & -0.3453 &  12.1335 &  -7.6211 &  2021-05-19 &  0.0215 \\
\ \ \ CRIX    &  0.0777 &  0.9207 &  0.0164 &  13.5608 &  -6.9894 &  2021-05-19 &  0.0173 \\
\ \ \ BITW100 &  0.0848 &  1.2125 & -1.6397 &  19.7472 & -11.1357 &  2021-05-19 &  0.0274 \\
    \multicolumn{7}{l}{Crypto Indices without BTC Constituent}                                                              \\
\ \ \ BITW20  &  0.2608 &  3.6115 & -0.3555 &   4.2016 & -21.5430 &  2021-05-19 &  0.0804 \\
\ \ \ BITW70  &  0.2785 &  3.9157 & -0.6949 &   4.8047 & -24.3474 &  2021-05-19 &  0.0908 \\
\bottomrule
\end{tabular}}
\caption{Summary statistics of out-of-sample daily returns of hedged portfolios that minimize ES 5\%.}
\label{tab:ES5_rh}

\end{table}
\begin{table}[th] \centering \resizebox{\textwidth}{!}{%
\begin{tabular}{lrrrrrrr} \toprule
         {} &    Mean \% &     Std \% &      Skew &       Kurt &         MD \% &     MD date & ES 1\% \\
\midrule
     \multicolumn{7}{l}{Individual Cryptos}                                                                                 \\
\ \ \ BTC     &  0.014808 &  0.347559 & -0.835432 &   3.305439 &  -1.622527 &  2020-11-30 &  0.023384 \\
\ \ \ ETH     &  0.307977 &  3.895367 &  0.984009 &   7.494711 & -18.762453 &  2021-05-19 &  0.129950 \\
\ \ \ ADA     &  0.501591 &  5.404018 &  1.100767 &   3.960739 & -15.448083 &  2021-05-19 &  0.146283 \\
\ \ \ LTC     & -0.102863 &  4.158086 & -0.775700 &   7.437497 & -29.172721 &  2021-05-19 &  0.164734 \\
\ \ \ XRP     & -0.019965 &  7.288724 &  1.112146 &  18.873229 & -52.569956 &  2020-12-23 &  0.251630 \\
   \multicolumn{7}{l}{Crypto Indices with BTC Constituent}                                                                  \\
\ \ \ BITX    &  0.059801 &  1.031183 & -0.440995 &  11.586333 &  -7.742446 &  2021-05-19 &  0.041121 \\
\ \ \ CRIX    &  0.083524 &  0.946138 & -0.036149 &  12.404693 &  -7.020300 &  2021-05-19 &  0.034979 \\
\ \ \ BITW100 &  0.078145 &  1.263996 & -1.964529 &  21.883557 & -11.926340 &  2021-05-19 &  0.059302 \\
    \multicolumn{7}{l}{Crypto Indices without BTC Constituent}                                                              \\
\ \ \ BITW20  &  0.253793 &  3.632266 & -0.408622 &   4.446241 & -21.986552 &  2021-05-19 &  0.128208 \\
\ \ \ BITW70  &  0.266023 &  3.932037 & -0.759783 &   5.004985 & -24.476390 &  2021-05-19 &  0.153457 \\
\bottomrule
\end{tabular}}
\caption{Summary Statistics of Hedged Portfolios that minimize ES 1\%.}
\label{tab:ES1_rh}

\end{table}
\begin{table}[th] \centering \resizebox{\textwidth}{!}{%
\begin{tabular}{lrrrrrrr} \toprule
         {} &    Mean \% &     Std \% &      Skew &       Kurt &         MD \% &     MD date & VaR 5\% \\
\midrule
     \multicolumn{7}{l}{Individual Cryptos}                                                                                 \\
\ \ \ BTC     &  0.0253 &  0.3294 & -0.9725 &   3.4373 &  -1.5347 &  2020-11-30 &  0.0063 \\
\ \ \ ETH     &  0.3084 &  3.8944 &  1.0243 &   7.4297 & -19.1750 &  2021-05-19 &  0.0514 \\
\ \ \ ADA     &  0.5726 &  5.2204 &  1.2981 &   4.2544 & -14.6974 &  2021-05-19 &  0.0769 \\
\ \ \ LTC     & -0.0742 &  3.9145 & -0.3836 &   7.5384 & -28.3672 &  2021-05-19 &  0.0622 \\
\ \ \ XRP     &  0.0208 &  7.1520 &  1.1269 &  19.8930 & -52.5667 &  2020-12-23 &  0.0683 \\
   \multicolumn{7}{l}{Crypto Indices with BTC Constituent}                                                                  \\
\ \ \ BITX    &  0.0562 &  0.9930 & -0.3117 &  12.4780 &  -7.5639 &  2021-05-19 &  0.0128 \\
\ \ \ CRIX    &  0.0863 &  0.9151 &  0.0718 &  13.7915 &  -6.9744 &  2021-05-19 &  0.0092 \\
\ \ \ BITW100 &  0.0846 &  1.1980 & -1.6592 &  21.3725 & -11.2582 &  2021-05-19 &  0.0164 \\
    \multicolumn{7}{l}{Crypto Indices without BTC Constituent}                                                              \\
\ \ \ BITW20  &  0.2728 &  3.5940 & -0.3721 &   4.4896 & -22.0733 &  2021-05-19 &  0.0546 \\
\ \ \ BITW70  &  0.2847 &  3.9133 & -0.6580 &   4.7874 & -24.6513 &  2021-05-19 &  0.0626 \\
\bottomrule
\end{tabular}}
\caption{Summary Statistics of Hedged Portfolios that minimize VaR 5\%.}
\label{tab:VaR5_rh}

\end{table}
\begin{table}[th] \centering \resizebox{\textwidth}{!}{%
\begin{tabular}{lrrrrrrr} \toprule
         {} &    Mean \% &     Std \% &      Skew &       Kurt &         MD \% &     MD date & VaR 5\% \\
\midrule
     \multicolumn{7}{l}{Individual Cryptos}                                                                                 \\
\ \ \ BTC     &  0.0176 &  0.3270 & -1.0405 &   3.3742 &  -1.5689 &  2020-11-30 &  0.0134 \\
\ \ \ ETH     &  0.2977 &  3.9132 &  0.9547 &   7.2414 & -18.6061 &  2021-05-19 &  0.1026 \\
\ \ \ ADA     &  0.5562 &  5.3466 &  1.1362 &   3.9334 & -15.4795 &  2021-05-19 &  0.1106 \\
\ \ \ LTC     & -0.0852 &  4.1503 & -0.7234 &   7.3208 & -29.0915 &  2021-05-19 &  0.1030 \\
\ \ \ XRP     &  0.0352 &  7.1658 &  1.1582 &  19.8506 & -52.5727 &  2020-12-23 &  0.1387 \\
   \multicolumn{7}{l}{Crypto Indices with BTC Constituent}                                                                  \\
\ \ \ BITX    &  0.0593 &  1.0178 & -0.5331 &  13.3100 &  -8.0299 &  2021-05-19 &  0.0247 \\
\ \ \ CRIX    &  0.0738 &  0.9695 & -0.4729 &  13.6500 &  -7.0185 &  2021-05-19 &  0.0245 \\
\ \ \ BITW100 &  0.0823 &  1.2338 & -1.9365 &  23.1938 & -11.8752 &  2021-05-19 &  0.0347 \\
    \multicolumn{7}{l}{Crypto Indices without BTC Constituent}                                                              \\
\ \ \ BITW20  &  0.2499 &  3.6210 & -0.3866 &   4.3396 & -21.6634 &  2021-05-19 &  0.0988 \\
\ \ \ BITW70  &  0.2788 &  3.9257 & -0.7635 &   5.1288 & -24.5294 &  2021-05-19 &  0.1147 \\
\bottomrule
\end{tabular}}
\caption{Summary statistics of out-of-sample daily returns of hedged portfolios that minimize VaR 1\%.}
\label{tab:VaR1_rh}

\end{table}
\begin{table}[th] \centering \resizebox{\textwidth}{!}{%
\begin{tabular}{lrrrrrrr} \toprule
         {} &    Mean \% &     Std \% &      Skew &       Kurt &         MD \% &     MD date & Variance \\
\midrule
     \multicolumn{7}{l}{Individual Cryptos}                                                                                 \\
\ \ \ BTC     &  0.0215 &  0.3221 & -1.0119 &   3.1929 &  -1.4393 &  2020-11-30 &    0.0000 \\
\ \ \ ETH     &  0.2823 &  3.8741 &  0.9469 &   7.1064 & -17.7421 &  2021-05-19 &    0.0015 \\
\ \ \ ADA     &  0.5617 &  5.2722 &  1.3634 &   4.4818 & -13.8687 &  2021-01-08 &    0.0028 \\
\ \ \ LTC     & -0.0871 &  3.9052 & -0.3617 &   7.6239 & -28.3029 &  2021-05-19 &    0.0018 \\
\ \ \ XRP     & -0.0123 &  7.1537 &  1.1451 &  20.0236 & -52.5236 &  2020-12-23 &    0.0043 \\
   \multicolumn{7}{l}{Crypto Indices with BTC Constituent}                                                                  \\
\ \ \ BITX    &  0.0561 &  0.9954 & -0.4204 &  13.2487 &  -7.7567 &  2021-05-19 &    0.0001 \\
\ \ \ CRIX    &  0.0812 &  0.9183 & -0.0027 &  14.3136 &  -7.1025 &  2021-05-19 &    0.0001 \\
\ \ \ BITW100 &  0.0855 &  1.1986 & -1.7440 &  22.2644 & -11.3866 &  2021-05-19 &    0.0001 \\
    \multicolumn{7}{l}{Crypto Indices without BTC Constituent}                                                              \\
\ \ \ BITW20  &  0.2429 &  3.5846 & -0.3063 &   4.1622 & -21.4680 &  2021-05-19 &    0.0013 \\
\ \ \ BITW70  &  0.2706 &  3.8838 & -0.6490 &   4.6312 & -23.9984 &  2021-05-19 &    0.0015 \\
\bottomrule
\end{tabular}}
\caption{Summary statistics of out-of-sample daily returns of hedged portfolios that minimize variance.}
\label{tab:var_rh}

\end{table}

\subsection{Hedging Effectiveness Results}\label{sec: HE results}
In this section, we analyse the out-of-sample hedging effectiveness (HE) of BTCF as hedging.
HE is defined as $$\text{HE} = 1-\frac{\rho_h}{\rho_s},$$
a measure of the percentage reduction of portfolio risk attribute, in our case the spot $\rho_s$,
to hedged portfolio risk attribute $\rho_h$.
A higher HE indicates a higher hedging effectiveness and larger risk reduction. \medskip

The HE above is a generalisation of Ederington measure of hedging performance, where we,
in addition to variance, include other risk measures: Expected Shortfall 5\% and 1\% (ES5 and ES1), Value-at-Risk 5\% and 1\% (VaR5 and VaR1), and ERM.
In particular, ES5 is recommended by the Basel Committee on Banking Supervision (BCBS) to replace VaR as a quantitative risk metrics system.
The proposed reform aimed at enhancing the risk metric system's ability to capture tail risk. \medskip
%Discussions of the issue can be found in literatures.
%
We obtain a time series of out-of-sample $r^h$ of each hedging pair and each risk reduction objective by concatenating the out-of-sample results.
Then, we apply stationary block bootstrapping (SB) to the time series introduced by \cite{Politis1994} in our analysis in order to preserve the temporal structure of the data while sampling.
The SB procedure is as follow.
Assume a time series with $N$ observations $\{X_t\}_{t \in [1,N]}$ is a strong stationary, weakly dependence time series of interest,
we form blocks of samples $B = \{X_i, ..., X_{i+j-1}\}$.
Index $i$ is a random variable uniformly distributed over $[1,2,...,N]$ and $j$ is geometric distributed random variable with parameter .
The block index $i$ and block length $j$ are independent.
For any index $k$ which is greater than $N$, the sample $X_k$ is defined to be $X_{k(\mod N)}$.
For each block, we calculate the hedging effectiveness with different risk measures mentioned above.
We choose $p=0.005$, implying the expected block length is 200.
100 blocks are drawn for each risk minimising objective and spot. \medskip

From figure \ref{fig:HEboxplot}, we report, as expected, the BTC involving spots, the BTC, CRIX, BITX and BITW100, are well hedged by the BTCF.
The performances are consistent across different risk reduction objectives and different HE evaluation.
The median HE to BTC generated by various risk reduction objectives is ranging from 89.45\% to 99.31\%, median HE to CRIX is ranging from 81.13\% to 95.22\%,
median HE to BITX is ranging from 79.06\% to 94.84\%, median HE to BITW100 Is ranging from 71.07\% to 92.98\%. \medskip

The HE of BTCF to other cryptos and indices are substantially lower than to the BTC involving spots, but the consistency the performances across different risk reduction objectives and HE evaluation remains.
The median HE to BITW20 generated by various risk reduction objectives is ranging from 24.67\% to 47.02\%, median HE to BITW70 is ranging from 23.61\% to 49.30\%,
median HE to ADA is ranging from 9.01\% to 29.30\%, median HE to ETH Is ranging from 30.07\% to 36.18\%, median HE to LTC Is ranging from 37.74\% to 51.30\%,
median HE to XRP Is ranging from 0.46\% to 30.89\%.
\begin{figure}[h]
\includegraphics[width=\textwidth]{_pics/ES5_HE_boxplot.pdf}
  \caption{Hedging effectiveness (HE) of portfolios with different risk minimizatio objectives evaluated by the corresponding risk minimization objectives.
            The boxplots indicate the the median, upper quartile, lower quartile, minimium and maximum of the bootstrapped HE.
            The HE of BTC-involved spots are significantly higher than that of BTC-not-involved spots.
  \href{http://www.quantlet.com/}{\includegraphics[height=\baselineskip]{_pics/qletlogo_tr.png}} }
\label{fig:HEboxplot}
\end{figure}