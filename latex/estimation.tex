%! Author = francis
%! Date = 30.10.20

\subsection{Calibration and selection of copulae}\label{sec:estimation}
We introduce the method to calibrate copulae to our data in this section.
In general, there are two types of calibration procedures to calibrate copulae:
maximum likelihood (MLE) and method of moments (MM).
We decide to deploy the latter since it calibrates according to the moments desired. 

In the following subsection, we present the configuration of the method of moments procedures in this work.
In the subsection after, we argue that MM is more suitable to this work by comparing MM with MLE.

\subsubsection{Method of moments}
\label{subsec:simulated-method-of-moments}

We trace back the usage of MM to calibrate copulae to \citet{Genest1987, genest1993statistical}.
The moments mainly refer to Kendall's $\tau$ or Spearman's $\rho$.
We extend MM to quantile dependence measures denoted by $\lambda_q$ for quantile level $q$.

Spearman's $\rho$, Kendall's $\tau$, and quantile dependence of the copula $C$ are defined as
\begin{align*}
  \rho_S &= 12 \int\int_{I^2} C_{\bm{\theta}}(u,v)\, \dd u\, \dd v-3\label{eq:rho_S}\\
  \tau_K &= 4\E[C_{\bm{\theta}}\{F_X(x), F_Y(y)\}]-1,\\
  \lambda_q &=
  \begin{cases}
    \p(F_X(X)\leq q| F_Y(Y)\leq q) = \displaystyle \frac{C_{\bm{\theta}}(q,q)}{q},
    &\text{ if } q\in (0,0.5],\\
    \p(F_X(X)>q|F_Y(Y)>q) =\displaystyle \frac{1-2q+C_{\bm{\theta}}(q,q)} {1-q},
    &\text{ if } q\in (0.5,1).
  \end{cases}
\end{align*}
The empirical counterparts are
\begin{align*}
  \hat\rho_S &= \frac{12}{n} \sum_{k=1}^n \hat F_X(x_k) \hat F_Y(y_k)
               -3,\\
  \hat\tau_K &= \frac{4}{n}\sum_{k=1}^n \hat{C}\{\hat{F}_X(x_i),\hat{F}_X(y_i)\} -1 ,\\
  \hat\lambda_q &=
                  \begin{cases}
                    \displaystyle\frac{1}{n} \sum_{k=1}^n \frac{\1_{\{\hat
                        F_X(x_k)\leq q, \hat F_Y(y_k)\leq q\}}} {q},
                    &\text { if } q\in (0, 0.5],\\
                    \displaystyle \frac{1}{n} \sum_{k=1}^n
                    \frac{\1_{\{\hat F_X(x_k)>q, \hat F_Y(y_k)>q\}}}
                    {1-q}, &\text { if } q\in (0.5,1),
                  \end{cases}
\end{align*}
where $\displaystyle \hat{F}(x) =
  \frac{1}{n}\sum_{k=1}^n 1_{\{x_i\leq x\}}$ and
$\displaystyle \hat{C}(u,v) = \frac{1}{n}\sum_{k=1}^n 1_{\{u_i\leq u, v_i\leq v\}}$. 

Denote by $\bm{m}(\bm{\theta})$ the $m$-dimensional vector of
dependence measures according the dependence parameters
$\bm{\theta}$,and let $\hat{\bm{m}}$ be the corresponding empirical
counterpart. 
The difference between dependence measures and their counterpart is denoted by
\begin{align*}
    \bm{g}(\bm{\theta}) = \hat{\bm{m}} - \bm{m}(\bm{\theta}).
\end{align*}

The MM estimator is
\begin{align*}
    \hat{\bm{\theta}} = \argmin_{\bm{\theta}\in \bm{\Theta}} \bm{g}(\bm{\theta})^\top
    \hat{\bm{W}}
     \bm{g}(\bm{\theta}),
\end{align*}
where $\hat{W}$ is some positive definite weight matrix.
In this work, we use
$\bm{m}(\bm{\theta}) = (\rho_S, \lambda_{0.05}, \lambda_{0.1}, 
\lambda_{0.9}, \lambda_{0.95})^\top$
for calibration.
$\hat{W}$ is set to identity matrix.

\subsubsection{Comparison between method of moments and maximum likelihood}
\label{subsec:maximum-likelihood-estimation}
By the Hoeffding-Sklar theorem, the joint density of a $d$-dimensional random variable $\bm{X}$ with sample size $n$ can be written as
\begin{equation*}
    \bm{f}_{\bm{X}}(x_1, ..., x_d) = \bm{c}\{F_{X_1}(x_1), ..., F_{X_d}(x_d)\} \prod_{j=1}^d f_{X_i}(x_i).
    \end{equation*}
We follow the treatment of MLE documented in section 10.1 of
\citet{joe1997multivariate}, namely the {\em inference functions for
margins (IFM) method}.
The log-likelihood $\sum^n_{i=1}f_{\bm{X}}(X_{i,1}, ..., X_{i,d})$ can be decomposed into a dependence part and a marginal part,
\begin{align*}
    L(\bm{\theta}) &= \sum_{i=1}^n \bm{c}\{F_{X_1}(x_{i,1};\bm{\delta}_1), ..., F_{X_d}(x_{i,d}; \bm{\delta}_d);\bm{\gamma}\}
    + \sum_{i=1}^n \sum_{j=1}^d f_{X_j}(x_{i,j};\bm{\delta}_j)\\
    &= L_C(\bm{\delta}_1, ..., \bm{\delta}_d, \bm{\gamma}) + \sum_{j=1}^d L_j(\bm{\delta}_j)
    \end{align*}
where $\bm{\delta}_j$ are the parameters of the $j$-th margin, $\bm{\gamma}$ is the parameter of the parametric copula, and
$\bm{\theta} = (\bm{\delta}_1,..., \bm{\delta}_d, \bm{\gamma})$.
Instead of searching the $\bm{\theta}$ in a high dimensional space, \citet{joe1997multivariate} suggests to
search for $\hat{\bm{\delta}_1},..., \hat{\bm{\delta}_d}$ that maximize $L_1(\bm{\delta}_1), ..., L_d(\bm{\delta}_d)$,
then search for $\hat{\bm{\gamma}}$ that maximize $L_C(\hat{\bm{\delta}_1},..., \hat{\bm{\delta}_d}, \bm{\gamma})$.

%\natp{\em [I suggest to delete the next part, as the regularity
%  conditions are unclear, and it is just a first-order condition,
%  which is a-priori not clear to hold in a two-step procedure.]}
%That is, under regularity conditions, $(\hat{\bm{\delta}_1},..., \hat{\bm{\delta}_d}, \hat{\bm{\gamma}})$ is the solution of
%\begin{align}
%    \left( \frac{\partial L_1}{\partial \bm{\delta}_1}, ..., \frac{\partial L_d}{\partial \bm{\delta}_d},
%    \frac{\partial L_C}{\partial \bm{\gamma}}\right) = \bm{0}.
%    \end{align}

%However, the IFM requires making assumption on the distribution of the
%margins.\natp{\em [delete until here.]}

We follow \citet{genest1995semiparametric} who suggest to replace the estimation of marginals parameters estimation by non-parameteric estimation.
Given non-parametric estimator $\hat{F}_i$ of the margins $F_i$, the estimator of the dependence parameters $\bm{\gamma}$ is
\begin{equation*}
    \hat{\bm{\gamma}} = \argmax_{\bm{\gamma}} \sum_{i=1}^n \bm{c}\{ \hat{F}_{X_1}(x_{i,1}), ..., \hat{F}_{X_d}(x_{i,d});\bm{\gamma}\}.
    \end{equation*}

Both the simulated method of moments and the maximum likelihood estimation are unbiased.
The question though which procedure is more suitable for hedging.

\begin{figure}[h]
%\includegraphics[width=\textwidth]{_pics/t Copula quantile dependence.png}
\includegraphics[width=\textwidth]{_pics/Gumbel Copula quantile dependence.pdf}
\includegraphics[width=\textwidth]{_pics/Clayton Copula quantile dependence.pdf}
  \caption{Quantile dependences of Gumbel, and Clayton Copula. The \textcolor{darkblue}{blue circle dots} are
  the quantile dependence estimates of Bitcoin and CME future, \textcolor{darkblue}{blue dotted lines} are the estimates' 90\% confidence interval.
  \textcolor{orange}{Orange dotted line} is the copula implied quantile dependence by MM estimation.
  \textcolor{lightblue}{Light blue dotted line} is the copula implied quantile dependence by MLE estimation.
%  \href{http://www.quantlet.com/}{\includegraphics[height=\baselineskip]{_pics/qletlogo_tr.png}}
  }
\label{fig:quantile dependence1}
\end{figure}

Figure~\ref{fig:quantile dependence1} shows the empirical quantile dependence of Bitcoin and CME future and the copula implied
quantile dependence of the MLE and MM calibration procedures.
Although the MLE is a better fit to a range of quantile dependence in the middle, it fails to address the situation in the tails.
We find that our data empirically has low quantile dependence in the lower ends ($q<10\%$).
We argue that MM is preferred as it produces a better fit to the dependence
structure in the tail behaviour, contrary to MLE. 

Therefore, we deploy the method of moments throughout the
analysis.
We choose the $5^\text{th}$-, $10^\text{th}$-, $90^\text{th}$-, $95^\text{th}$-quantile, and Spearman's $\rho$ as the moments.


%
%
%\subsection{Two-Stage Estimation}\label{subsec:two-stage-estimation}
%~\cite{joe2005asymptotic} study the efficiency of a two-stage estimation procedure of copula estimation.
%The authors also call this method inference function for margins IFM.
%
%\textbf{Pros}
%\begin{enumerate}
%    \item Almost as efficient as MLE methods but easier to be implemented
%    \item Yields an asymptotically Gaussian, unbiased estimate
%\end{enumerate}
%
%\textbf{Cons}
%\begin{enumerate}
%    \item Subject to specification of marginals \cite{kim2007comparison}
%\end{enumerate}
%
%Our data
%\begin{align}
%    \pmb{y} = \begin{bmatrix}
%                  y_{11} & \cdots & y_{1i}\\
%                  \vdots & \ddots & \vdots \\
%                  y_{n1} & \cdots & y_{ni}
%                  \end{bmatrix}
%    \end{align}
%Let $F$ and $f$ be the joint cdf and joint density of $\pmb{y}$ with parameters $\pmb{\delta}$,
%and let $F_i$ and $f_i$ be the marginal cdf and marginal density for the $i^\text{th}$ random variable with parameters $\pmb{\theta}_i$, we have
%\begin{align}
%    f(\pmb{y}; \pmb{\theta}_1, \pmb{\theta}_2,\dots \pmb{\theta}_i, \pmb{\delta}) =
%    c\{F_1(\pmb{y}_1;\pmb{\theta}_1), F_2(\pmb{y}_2; \pmb{\theta}_2), \dots, F_i(\pmb{y}_1;\pmb{\theta}_i); \pmb{\delta}\}
%    \prod^i_{j=1}f_i(\pmb{y}_j;\pmb{\theta}_j)
%    \end{align}
%
%For a sample of size $n$, the log-likelihood of functions of the $i^\text{th}$ univariate margin is
%\begin{align}
%    L_i(\theta_i) = \sum^n_{m=1} \log f_i(y_{mi}; \theta_i),
%    \end{align}
%
%and the log-likelihood function for the joint distribution is
%\begin{align}
%    L(\delta, \theta_1, \theta_2, \dots, \theta_i) = \sum^n_{m=1}\sum^i_{j=1} \log f(y_{mj}; \delta, \theta_1, \theta_2, ..., \theta_i)
%    \end{align}
%
%In most cases, one does not have closed form estimators and numerical techniques are needed.
%Numerical ML estimation difficulty increase when the total number of parameters increases.
%The two-stage estimation is designed to overcome this problem.
%
%The two-stage procedure is
%\begin{enumerate}
%    \item estimate the univariate parameters from separate univariate likelihoods to get $\tilde{\pmb{\theta}_1}, ..., \tilde{\pmb{\theta}_i}$
%    \item maximize $L(\pmb{\delta}, \tilde{\pmb{\theta}_1}, \dots, \tilde{\pmb{\theta}_i})$ over $\pmb{\delta}$ to get $\tilde{\pmb{\delta}}$
%    \end{enumerate}
%
%Under regularity conditions
%\footnote{Regularity conditions include
%1. $\exists \frac{\partial \log f(x;\theta)}{\partial \theta}, \frac{\partial^2 \log f(x;\theta)}{\partial \theta^2}, \frac{\partial^3 \log f(x;\theta)}{\partial \theta^3}$ for all $x$;
%2. $\exists g(x), h(x) and H(x)$ such that for $\theta$ in a neighborhood $N(\theta_0)$ the relations
%$\left|\frac{\partial f(x;\theta)}{\partial theta}\right| \leq g(x)$,
%$\left|\frac{\partial^2 f(x;\theta)}{\partial \theta^2}\right| \leq h(x)$,
%$\left|\frac{\partial^3 f(x;\theta)}{\partial \theta^3}\right| \leq H(x)$ hold for all $x$, and
%$\int g(x) dx < \infty$, $\int h(x) dx < \infty$, $\mathbb{E}_\theta \{H(X)\} < \infty$ for $\theta \in N(\theta_0)$;
%3. For each $\theta \in \Theta$, $0< \mathbb{E}_\theta \left\{
%\left(
%\frac{\partial \log f(X;\theta)}{\partial \theta}
%\right)^2
%\right\}$. For detail see section 4.2.2 of~\cite{serfling2009approximation}}
%, $(\pmb{\tilde{\theta}}_1,\dots \pmb{\tilde{\theta}}_i, \pmb{\tilde{\delta}})$ is the solution of
%\begin{align}
%    (\partial L_1 / \partial \pmb{\theta}^\intercal_1,
%    \dots, \partial L_i / \partial \pmb{\theta}^\intercal_i, \partial L / \partial \pmb{\pmb{\delta}}^\intercal_1) = \pmb{0}
%    \end{align}
%
%For comparison, if we optimize $L$ directly without the two-stage procedure (i.e.~MLE), we solve for
%\begin{align}
%    (\partial L / \partial \pmb{\theta}^\intercal_1,
%    \dots, \partial L / \partial \pmb{\theta}^\intercal_i, \partial L / \partial \pmb{\pmb{\delta}}^\intercal_1) = \pmb{0}
%    \end{align}
%
%We denote the two solutions as
%$\tilde{\pmb{\eta}} = (\pmb{\tilde{\theta}}_1,\dots \pmb{\tilde{\theta}}_i, \pmb{\tilde{\delta}})$ for two-stage procedure;
%$\hat{\pmb{\eta}} =(\pmb{\hat{\theta}}_1,\dots \pmb{\hat{\theta}}_i, \pmb{\hat{\delta}})$ for MLE procedure.
%and compare the asymptotic relative efficiency of $\tilde{\pmb{\eta}}$ and $\hat{\pmb{\eta}}$.
%
%Asymptotics: yet to be done.\\
%~\cite{kim2007comparison} show the estimation of $\pmb{\theta}$ may be seriously affected.
%They compare the two-stage approach and Canonical Maximum Likelihood Method by simulation and
%conclude that Canonical Maximum Likelihood is prefered from a computational statistics and data analysis point of view.
%
%\subsection{Canonical Maximum Likelihood Method}\label{subsec:canonical-maximum-likelihood-method}
%This approach was studied by~\cite{genest1995semiparametric} and~\cite{shih1995inferences}.
%One estimates the margins using empirical CDF
%\begin{align}F_X(x)=\frac{1}{n+1}\sum_{i=1}^n 1(X_i \leq x)\end{align},
%
%we maximize the log-likelihood
%\begin{align}
%    L(\delta) = \sum_{i=1}^n \log [c_\delta \{F_X(X_i), F_Y(Y_i)\}]
%    \end{align}
%
%This procedure does not require specification of marginals.
%
%
%
%
%
%%also by Wang and Ding, 2000; Tsukahara, 2005
%%This is also known as pseudo maximum likelihood (PML) and as canonical maximum likelihood (see Cherubini et al., 2004)
%%
%%Genest and Werker (2002) obtained conditions under which the PMLE is asymptotically efficient.
%%
%%


%%% Local Variables:
%%% mode: latex
%%% TeX-master: "SRM"
%%% End:
